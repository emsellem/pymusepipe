%% Generated by Sphinx.
\def\sphinxdocclass{report}
\documentclass[letterpaper,10pt,english]{sphinxmanual}
\ifdefined\pdfpxdimen
   \let\sphinxpxdimen\pdfpxdimen\else\newdimen\sphinxpxdimen
\fi \sphinxpxdimen=.75bp\relax
\ifdefined\pdfimageresolution
    \pdfimageresolution= \numexpr \dimexpr1in\relax/\sphinxpxdimen\relax
\fi
%% let collapsible pdf bookmarks panel have high depth per default
\PassOptionsToPackage{bookmarksdepth=5}{hyperref}

\PassOptionsToPackage{booktabs}{sphinx}
\PassOptionsToPackage{colorrows}{sphinx}

\PassOptionsToPackage{warn}{textcomp}
\usepackage[utf8]{inputenc}
\ifdefined\DeclareUnicodeCharacter
% support both utf8 and utf8x syntaxes
  \ifdefined\DeclareUnicodeCharacterAsOptional
    \def\sphinxDUC#1{\DeclareUnicodeCharacter{"#1}}
  \else
    \let\sphinxDUC\DeclareUnicodeCharacter
  \fi
  \sphinxDUC{00A0}{\nobreakspace}
  \sphinxDUC{2500}{\sphinxunichar{2500}}
  \sphinxDUC{2502}{\sphinxunichar{2502}}
  \sphinxDUC{2514}{\sphinxunichar{2514}}
  \sphinxDUC{251C}{\sphinxunichar{251C}}
  \sphinxDUC{2572}{\textbackslash}
\fi
\usepackage{cmap}
\usepackage[T1]{fontenc}
\usepackage{amsmath,amssymb,amstext}
\usepackage{babel}



\usepackage{tgtermes}
\usepackage{tgheros}
\renewcommand{\ttdefault}{txtt}



\usepackage[Bjarne]{fncychap}
\usepackage{sphinx}

\fvset{fontsize=auto}
\usepackage{geometry}


% Include hyperref last.
\usepackage{hyperref}
% Fix anchor placement for figures with captions.
\usepackage{hypcap}% it must be loaded after hyperref.
% Set up styles of URL: it should be placed after hyperref.
\urlstyle{same}


\usepackage{sphinxmessages}
\setcounter{tocdepth}{1}



\title{pymusepipe Documentation}
\date{Jul 24, 2023}
\release{2.25.4}
\author{Eric Emsellem}
\newcommand{\sphinxlogo}{\vbox{}}
\renewcommand{\releasename}{Release}
\makeindex
\begin{document}

\ifdefined\shorthandoff
  \ifnum\catcode`\=\string=\active\shorthandoff{=}\fi
  \ifnum\catcode`\"=\active\shorthandoff{"}\fi
\fi

\pagestyle{empty}
\sphinxmaketitle
\pagestyle{plain}
\sphinxtableofcontents
\pagestyle{normal}
\phantomsection\label{\detokenize{index::doc}}


\sphinxAtStartPar
This is the documentation of \sphinxstylestrong{pymusepipe}.


\chapter{Contents}
\label{\detokenize{index:contents}}
\sphinxstepscope


\section{Welcome}
\label{\detokenize{welcome:welcome}}\label{\detokenize{welcome::doc}}
\sphinxAtStartPar
Pymusepipe is a Python package which serves as a wrapper around the main processing steps of the MUSE data reduction pipeline (Weilbacher et al. \sphinxhref{https://ui.adsabs.harvard.edu/abs/2020A\%26A...641A..28W/abstract}{2019}). Pymusepipe includes a simple data organiser and prescriptions for the structure of the data files (but no database per se), a wrapper around the main functionalities of MUSE data reduction pipeline, accessed via \sphinxhref{https://www.eso.org/sci/software/pipelines/muse/}{EsoRex} command\sphinxhyphen{}line recipes, to remove the instrumental signatures. Pymusepipe additionally provides a set of modules supporting the alignment, mosaicking, two\sphinxhyphen{}dimensional and three\sphinxhyphen{}dimensional convolution.

\sphinxAtStartPar
pymusepipe is also made for multi\sphinxhyphen{}pointing mosaics and multi\sphinxhyphen{}targets surveys
as it will process targets automatically when provided a specific dictionary
of which target and which pointings to consider.

\sphinxAtStartPar
A description of the pipeline and its usage to reduce data from the PHANGS\sphinxhyphen{}MUSE survey are presented in Emsellem et al. (\sphinxhref{https://ui.adsabs.harvard.edu/abs/2022A\%26A...659A.191E/abstract/}{2022})

\begin{sphinxadmonition}{note}{Contact}

\sphinxAtStartPar
The pymusepipe module is maintained by Eric Emsellem.
contact via 
\end{sphinxadmonition}

\begin{sphinxadmonition}{attention}{Attention:}
\sphinxAtStartPar
Please do not forget to cite Emsellem et al. (\sphinxhref{https://ui.adsabs.harvard.edu/abs/2022A\%26A...659A.191E/abstract/}{2022})
and the MUSE data reduction pipeline paper (Weilbacher et al. \sphinxhref{https://ui.adsabs.harvard.edu/abs/2020A\%26A...641A..28W/abstract}{2019}) if you make use of pymusepipe in your work. In particular, we suggest you add the following text (or equivalent) to the data reduction section of your work.

\sphinxAtStartPar
\sphinxstyleemphasis{The dataset was reduced using recipes the MUSE data processing pipeline software (Weilbacher et al. 2019). All recipes were executed with ESOREX commands, wrapped around using the dedicated python package pymusepipe (Emsellem et al. 2022).}
\end{sphinxadmonition}


\subsection{GitHub Repository}
\label{\detokenize{welcome:github-repository}}
\sphinxAtStartPar
You can access the source code of pymusepipe and its previous releases directly in its official GitHub repository \sphinxurl{https://github.com/emsellem/pymusepipe}.

\begin{DUlineblock}{0em}
\item[] 
\end{DUlineblock}

\sphinxstepscope


\section{Install}
\label{\detokenize{install:install}}\label{\detokenize{install::doc}}

\subsection{Prerequisites}
\label{\detokenize{install:prerequisites}}
\sphinxAtStartPar
Pymusepipe assumes you have a working installation of ESOREX (i.e. that you have a working
MUSE data reduction pipeline installation) and likwid\sphinxhyphen{}pin. The installation of these components is not covered here but can be found in the \sphinxhref{https://ftp.eso.org/pub/dfs/pipelines/instruments/muse/muse-pipeline-manual-2.8.7.pdf}{MUSE Pipeline User Manual}.

\sphinxAtStartPar
Pymusepipe uses \sphinxhref{https://www.python.org}{Python 3} and is not compatible with Python 2.
It requires a number of standard python packages including:
\begin{itemize}
\item {} 
\sphinxAtStartPar
\sphinxstylestrong{numpy}

\item {} 
\sphinxAtStartPar
\sphinxstylestrong{scipy}

\item {} 
\sphinxAtStartPar
\sphinxstylestrong{matplotlib}

\item {} 
\sphinxAtStartPar
\sphinxstylestrong{astropy}

\item {} 
\sphinxAtStartPar
\sphinxstylestrong{mpdaf} a utility package to process and analyse datacubes, and more specifically
MUSE cubes, images and spectra developed by the MUSE GTO\sphinxhyphen{}CRAL Team.

\end{itemize}

\sphinxAtStartPar
In addition some packages are needed to access specific functionality:
\begin{itemize}
\item {} 
\sphinxAtStartPar
\sphinxstylestrong{pypher} to use the convolution package of pymusepipe

\item {} 
\sphinxAtStartPar
\sphinxstylestrong{spacepylot} to use the automatic alignment module

\end{itemize}


\subsection{Installation}
\label{\detokenize{install:installation}}
\sphinxAtStartPar
You can install this package via pypi via a simple:

\begin{sphinxVerbatim}[commandchars=\\\{\}]
\PYG{n}{pip} \PYG{n}{install} \PYG{n}{pymusepipe}
\end{sphinxVerbatim}

\sphinxAtStartPar
You can obviously also install it by cloning it from github, or
downloading the source (from github) and do something like:

\begin{sphinxVerbatim}[commandchars=\\\{\}]
\PYG{n}{python} \PYG{n}{setup}\PYG{o}{.}\PYG{n}{py} \PYG{n}{develop}
\end{sphinxVerbatim}

\sphinxAtStartPar
The “develop” option is recommended as it actually does not copy
the files in your system but just creates a link.
In that way you can easily update the source software without
reinstalling it. The link will directly use the source which has been udpated.

\sphinxAtStartPar
The other option is to use the standard “install” option:

\begin{sphinxVerbatim}[commandchars=\\\{\}]
\PYG{n}{python} \PYG{n}{setup}\PYG{o}{.}\PYG{n}{py} \PYG{n}{install}
\end{sphinxVerbatim}

\sphinxstepscope


\section{Getting Started}
\label{\detokenize{getting_started:getting-started}}\label{\detokenize{getting_started::doc}}

\subsection{Basic Usage \sphinxhyphen{} Dealing with OBs, individually}
\label{\detokenize{getting_started:basic-usage-dealing-with-obs-individually}}
\sphinxAtStartPar
The pymusepipe wrapper is meant to provide the user with a simple way to
run the MUSE pipeline.

\sphinxAtStartPar
Only three steps are needed:
\begin{enumerate}
\sphinxsetlistlabels{\arabic}{enumi}{enumii}{}{.}%
\item {} 
\sphinxAtStartPar
preparing the data (download),

\item {} 
\sphinxAtStartPar
preparing the configuration files (templates are provided),

\item {} 
\sphinxAtStartPar
using the code (a few lines at most).

\end{enumerate}

\sphinxAtStartPar
I recommend to use \sphinxcode{\sphinxupquote{Ipython}} as an environment, possibly via
a \sphinxcode{\sphinxupquote{screen}} or \sphinxcode{\sphinxupquote{tmux}} which would allow you to disconnect from the server that
actually runs the commands.


\subsection{Step 1: Preparing your data}
\label{\detokenize{getting_started:step-1-preparing-your-data}}
\sphinxAtStartPar
The first thing to do is to prepare the folder structure to welcome your
MUSE datasets.

\sphinxAtStartPar
Imagine you have:
\begin{itemize}
\item {} 
\sphinxAtStartPar
a target or field named e.g. \sphinxcode{\sphinxupquote{NGC1000}}.

\item {} 
\sphinxAtStartPar
several \sphinxstylestrong{datasets}. In the simplest cases this corresponds to data from one
MUSE Observing Block (OB), including all the calibration and science object raw data files, as
downloaded from the ESO archive. In practice pymusepipe will also reduce several OBs, provided
all the necessary calibrations are avaiable. Some functionality requires a distinction between
\sphinxstylestrong{dataset} and \sphinxstylestrong{pointing}. This distinction is described in {\hyperref[\detokenize{mosaicking::doc}]{\sphinxcrossref{\DUrole{doc}{Targets and Mosaicking}}}}.

\end{itemize}

\sphinxAtStartPar
Then under your data root folder \textless{}my\_data\_folder\textgreater{} create the following folder structure:

\begin{sphinxVerbatim}[commandchars=\\\{\}]
\PYG{o}{\PYGZlt{}}\PYG{n}{my\PYGZus{}data\PYGZus{}folder}\PYG{o}{\PYGZgt{}}\PYG{o}{/}\PYG{n}{NGC1000}
   \PYG{o}{.}\PYG{o}{/}\PYG{n}{OB001}
      \PYG{o}{.}\PYG{o}{/}\PYG{n}{Raw}
   \PYG{o}{.}\PYG{o}{/}\PYG{n}{OB002}
     \PYG{o}{.}\PYG{o}{/}\PYG{n}{Raw}
   \PYG{o}{.}\PYG{o}{/}\PYG{n}{OB003}
     \PYG{o}{.}\PYG{o}{/}\PYG{n}{Raw}
\end{sphinxVerbatim}

\sphinxAtStartPar
Each dataset, or OB for short, has a \sphinxcode{\sphinxupquote{Raw}} folder.

\sphinxAtStartPar
The next step is to download your MUSE data (including raw calibrations) from the
ESO web site, and put all the raw files (in fitsor fits.gz format) into each individual
\sphinxcode{\sphinxupquote{Raw}} folders, associated with the right dataset.


\subsection{Step 2: Preparing your configuration files}
\label{\detokenize{getting_started:step-2-preparing-your-configuration-files}}
\sphinxAtStartPar
pymusepipe only needs two configurations ascii files:
\begin{enumerate}
\sphinxsetlistlabels{\arabic}{enumi}{enumii}{}{.}%
\item {} 
\sphinxAtStartPar
\sphinxcode{\sphinxupquote{calib\_tables.dic}}, which contains a series of file names associated
with muse static calibration files and other configuration files (e.g. fitler lists). Most
names are self\sphinxhyphen{}explantory. These include:
\begin{itemize}
\item {} 
\sphinxAtStartPar
\sphinxstyleemphasis{geo\_table} and \sphinxstyleemphasis{astro\_table}: static files, time dependent geometry files can be specified
(see \sphinxcode{\sphinxupquote{rc.dic}}).

\item {} 
\sphinxAtStartPar
\sphinxstyleemphasis{badpix\_table}, \sphinxstyleemphasis{vignetting\_mask}, \sphinxstyleemphasis{std\_flux\_table}, \sphinxstyleemphasis{extinct\_table}, \sphinxstyleemphasis{line\_catalog},
statical calibrations provided with the MUSE pipeline, no need to change these.

\item {} 
\sphinxAtStartPar
\sphinxstyleemphasis{filter\_list} : used in case you wish to provide your own. Note that the file it needs
to follow the MUSE standard for such a table.

\end{itemize}

\item {} 
\sphinxAtStartPar
\sphinxcode{\sphinxupquote{rc.dic}}, which provides the root folders for the static calibration files and for your datasets.
\begin{itemize}
\item {} 
\sphinxAtStartPar
\sphinxstyleemphasis{root} provides the root folder for your data. For Target NGC1000, and OB 1,
the Raw data will be looked for in \sphinxstyleemphasis{root}/NGC1000/OB001/Raw.

\item {} 
\sphinxAtStartPar
\sphinxstyleemphasis{musecalib} should contain the standard MUSE calibration files.
These are distributed in the MUSE pipeline installation in a “muse\sphinxhyphen{}calib\sphinxhyphen{}x.x.x/cal” folder.

\item {} 
\sphinxAtStartPar
\sphinxstyleemphasis{musecalib\_time}: time dependent geometry and astrometry files (the correspondence
between observing run dates and specific files are hard\sphinxhyphen{}coded into pymusepipe).

\end{itemize}

\end{enumerate}

\sphinxAtStartPar
Examples of such files are provided in the \sphinxcode{\sphinxupquote{config\_templates}} folder of the pymusepipe package.


\subsection{Step 3: Running the pipeline}
\label{\detokenize{getting_started:step-3-running-the-pipeline}}
\sphinxAtStartPar
Here is an example of how to run the pipeline to reduce a single OB (dataset):

\begin{sphinxVerbatim}[commandchars=\\\{\}]
\PYG{c+c1}{\PYGZsh{} Import the modules}
\PYG{k+kn}{import} \PYG{n+nn}{pymusepipe} \PYG{k}{as} \PYG{n+nn}{pmp}
\PYG{k+kn}{from} \PYG{n+nn}{pymusepipe} \PYG{k+kn}{import} \PYG{n}{musepipe}

\PYG{c+c1}{\PYGZsh{} define the paths to the two configuration files}
\PYG{n}{rcfile} \PYG{o}{=} \PYG{l+s+s2}{\PYGZdq{}}\PYG{l+s+s2}{my\PYGZus{}data/MUSE/Config/rc.dic}\PYG{l+s+s2}{\PYGZdq{}}
\PYG{n}{calfile} \PYG{o}{=} \PYG{l+s+s2}{\PYGZdq{}}\PYG{l+s+s2}{my\PYGZus{}data\PYGZus{}MUSE/Config/calib\PYGZus{}tables.dic}\PYG{l+s+s2}{\PYGZdq{}}

\PYG{c+c1}{\PYGZsh{} Initialisation of the python \PYGZhy{} MusePipe Class \PYGZhy{} structure}
\PYG{n}{mypipe} \PYG{o}{=} \PYG{n}{musepipe}\PYG{o}{.}\PYG{n}{MusePipe}\PYG{p}{(}\PYG{n}{targetname}\PYG{o}{=}\PYG{l+s+s2}{\PYGZdq{}}\PYG{l+s+s2}{NGC1000}\PYG{l+s+s2}{\PYGZdq{}}\PYG{p}{,} \PYG{n}{dataset}\PYG{o}{=}\PYG{l+m+mi}{1}\PYG{p}{,} \PYG{n}{rc\PYGZus{}filename}\PYG{o}{=}\PYG{n}{rcfile}\PYG{p}{,}
                        \PYG{n}{cal\PYGZus{}filename}\PYG{o}{=}\PYG{n}{calfile}\PYG{p}{,} \PYG{n}{log\PYGZus{}filename}\PYG{o}{=}\PYG{l+s+s2}{\PYGZdq{}}\PYG{l+s+s2}{NGC1000\PYGZus{}version01.log}\PYG{l+s+s2}{\PYGZdq{}}\PYG{p}{,}
                        \PYG{n}{fakemode}\PYG{o}{=}\PYG{k+kc}{False}\PYG{p}{,} \PYG{n}{overwrite\PYGZus{}astropy\PYGZus{}table}\PYG{o}{=}\PYG{k+kc}{True}\PYG{p}{,}
                        \PYG{n}{filter\PYGZus{}list}\PYG{o}{=}\PYG{l+s+s2}{\PYGZdq{}}\PYG{l+s+s2}{white,Cousins\PYGZus{}R}\PYG{l+s+s2}{\PYGZdq{}}\PYG{p}{,}
                        \PYG{n}{filter\PYGZus{}for\PYGZus{}alignment}\PYG{o}{=}\PYG{l+s+s2}{\PYGZdq{}}\PYG{l+s+s2}{Cousins\PYGZus{}R}\PYG{l+s+s2}{\PYGZdq{}}\PYG{p}{)}

\PYG{c+c1}{\PYGZsh{} Launching the pipeline}
\PYG{n}{mypipe}\PYG{o}{.}\PYG{n}{run\PYGZus{}recipes}\PYG{p}{(}\PYG{p}{)}
\end{sphinxVerbatim}

\sphinxAtStartPar
That’s it! Your data has now been reduced!

\sphinxAtStartPar
Some explanation may be needed to understand what is happening:
\begin{itemize}
\item {} 
\sphinxAtStartPar
\sphinxcode{\sphinxupquote{targetname}}: is just the name of the target, used to decided where the data will be

\item {} 
\sphinxAtStartPar
\sphinxcode{\sphinxupquote{dataset}}: the number of the OB that will be used, namely “OB001” etc.

\item {} 
\sphinxAtStartPar
\sphinxcode{\sphinxupquote{logfile}}: name of the logging file.

\item {} 
\sphinxAtStartPar
\sphinxcode{\sphinxupquote{fakemode}}: you can set this to True if you just wish to initialise things without
actually running any recipes. The pipeline will only set things up but if you run any recipes
will only “fake” them (not launch any esorex command, only spitting the log out)

\item {} 
\sphinxAtStartPar
\sphinxcode{\sphinxupquote{filter\_list}}: list of filter names to use to reconstruct images when building
up cubes. This should be part of the filter\_list fits table provided (see \sphinxcode{\sphinxupquote{calib\_tables}}
config file).

\item {} 
\sphinxAtStartPar
\sphinxcode{\sphinxupquote{filter\_for\_alignment}}: specific filter name used for alignment between exposures.

\end{itemize}

\sphinxAtStartPar
Other options can be useful:
\begin{itemize}
\item {} 
\sphinxAtStartPar
\sphinxcode{\sphinxupquote{musemode}}: this is by default \sphinxcode{\sphinxupquote{WFM\_NOAO\_N}} which is the most often used MUSE mode. This will filter out exposures not compatible with the given mode.

\item {} 
\sphinxAtStartPar
\sphinxcode{\sphinxupquote{reset\_log}}: will reset the log file. By default it is False, hence new runs will be appended.

\item {} 
\sphinxAtStartPar
\sphinxcode{\sphinxupquote{overwrite\_astropy\_table}}: by default this is False. If True, new runs will rewrite the Astropy output tables.

\item {} 
\sphinxAtStartPar
\sphinxcode{\sphinxupquote{time\_astrometry\textasciigrave{}}}: by default it is False, meaning the pipeline will try to detect a GEOMETRY and ASTROMETRY Files delivered with the Rawfiles by ESO. If set to True, it will use the time dependent astro/geo files provided by the GTO Team but you would need to make these available on your system.Hence I would recommend to keep the default (False).

\end{itemize}


\subsection{Under the hood of run\_recipes}
\label{\detokenize{getting_started:under-the-hood-of-run-recipes}}
\sphinxAtStartPar
\sphinxcode{\sphinxupquote{run\_recipes()}} launches a default set of functions listed below:

\begin{sphinxVerbatim}[commandchars=\\\{\}]
\PYG{c+c1}{\PYGZsh{} generate the master bias using the muse\PYGZus{}bias esorex recipe}
\PYG{n}{mypipe}\PYG{o}{.}\PYG{n}{run\PYGZus{}bias}\PYG{p}{(}\PYG{p}{)}
\PYG{c+c1}{\PYGZsh{} generate the master flat using the muse\PYGZus{}flat esorex recipe}
\PYG{n}{mypipe}\PYG{o}{.}\PYG{n}{run\PYGZus{}flat}\PYG{p}{(}\PYG{p}{)}
\PYG{c+c1}{\PYGZsh{} generate the wavelength calibration using the muse\PYGZus{}wavecal esorex recipe}
\PYG{n}{mypipe}\PYG{o}{.}\PYG{n}{run\PYGZus{}wave}\PYG{p}{(}\PYG{p}{)}
\PYG{c+c1}{\PYGZsh{} generate the lsf using the muse\PYGZus{}lsf esorex recipe}
\PYG{n}{mypipe}\PYG{o}{.}\PYG{n}{run\PYGZus{}lsf}\PYG{p}{(}\PYG{p}{)}
\PYG{c+c1}{\PYGZsh{} generate the illumination correction using the muse\PYGZus{}lsf esorex recipe}
\PYG{n}{mypipe}\PYG{o}{.}\PYG{n}{run\PYGZus{}twilight}\PYG{p}{(}\PYG{n}{illum}\PYG{o}{=}\PYG{k+kc}{True}\PYG{p}{)}
\PYG{c+c1}{\PYGZsh{} process individual exposures to remove the instrumental signature usign the muse\PYGZus{}scibasic}
\PYG{c+c1}{\PYGZsh{} esorex recipes. It runs on both the object, standard star and sky exposures}
\PYG{n}{mypipe}\PYG{o}{.}\PYG{n}{run\PYGZus{}scibasic\PYGZus{}all}\PYG{p}{(}\PYG{n}{illum}\PYG{o}{=}\PYG{k+kc}{True}\PYG{p}{)}
\PYG{c+c1}{\PYGZsh{} generates the response function using the standard star observations and the muse\PYGZus{}standard}
\PYG{c+c1}{\PYGZsh{} esorex recipe}
\PYG{n}{mypipe}\PYG{o}{.}\PYG{n}{run\PYGZus{}standard}\PYG{p}{(}\PYG{p}{)}
\PYG{c+c1}{\PYGZsh{} uses the sky exposures to generate a sky spectrum}
\PYG{n}{mypipe}\PYG{o}{.}\PYG{n}{run\PYGZus{}sky}\PYG{p}{(}\PYG{n}{fraction}\PYG{o}{=}\PYG{l+m+mf}{0.8}\PYG{p}{)}
\PYG{c+c1}{\PYGZsh{} runs the esosex muse\PYGZus{}scipost recipe individually on each object exposures generating}
\PYG{c+c1}{\PYGZsh{} a datacubes and image in the requested filter for each exposure.}
\PYG{c+c1}{\PYGZsh{} These images are then used for aligment.}
\PYG{n}{mypipe}\PYG{o}{.}\PYG{n}{run\PYGZus{}prep\PYGZus{}align}\PYG{p}{(}\PYG{p}{)}
\PYG{c+c1}{\PYGZsh{} runs the muse\PYGZus{}exp\PYGZus{}align recipe to generate an OFFSET\PYGZus{}TABLE files containing the astrometric}
\PYG{c+c1}{\PYGZsh{} shifts between individual exposures. Pymusepipe provides more refined options for this}
\PYG{n}{mypipe}\PYG{o}{.}\PYG{n}{run\PYGZus{}align\PYGZus{}bydataset}\PYG{p}{(}\PYG{p}{)}
\PYG{c+c1}{\PYGZsh{} ??}
\PYG{n}{mypipe}\PYG{o}{.}\PYG{n}{run\PYGZus{}align\PYGZus{}bygroup}\PYG{p}{(}\PYG{p}{)}
\PYG{c+c1}{\PYGZsh{} generates the final aligned datacubes for individual exposures using muse\PYGZus{}scipost}
\PYG{n}{mypipe}\PYG{o}{.}\PYG{n}{run\PYGZus{}scipost\PYGZus{}perexpo}\PYG{p}{(}\PYG{p}{)}
\PYG{c+c1}{\PYGZsh{} generates the sky datacube}
\PYG{n}{mypipe}\PYG{o}{.}\PYG{n}{run\PYGZus{}scipost\PYGZus{}sky}\PYG{p}{(}\PYG{p}{)}
\PYG{c+c1}{\PYGZsh{} merge exposures in the final datacube}
\PYG{n}{mypipe}\PYG{o}{.}\PYG{n}{combine\PYGZus{}dataset}\PYG{p}{(}\PYG{p}{)}
\end{sphinxVerbatim}

\sphinxAtStartPar
Individual pipeline stages can be (re)run by calling any of the individual functions
above. The order is important, as in any data reduction process

\begin{sphinxadmonition}{attention}{Attention:}
\sphinxAtStartPar
This pipeline flow closely mirrors the standard data reduction for MUSE data implemented
by the e.g. EsoReflex workflow. Pymusepipe offers alternative recipes to perform alignment
({\hyperref[\detokenize{alignment::doc}]{\sphinxcrossref{\DUrole{doc}{Alignment}}}}) and mosaicking ({\hyperref[\detokenize{mosaicking::doc}]{\sphinxcrossref{\DUrole{doc}{Targets and Mosaicking}}}}). For best results, therefore, we do not
recommend running the above workflow. Examples workflows are presented in {\hyperref[\detokenize{phangs_example::doc}]{\sphinxcrossref{\DUrole{doc}{PHANGS pipeline example}}}}.
\end{sphinxadmonition}


\subsection{Structure of the output}
\label{\detokenize{getting_started:structure-of-the-output}}

\subsubsection{Folders}
\label{\detokenize{getting_started:folders}}
\sphinxAtStartPar
The structure of the output is driven by a set of folder names described in
{\hyperref[\detokenize{api/pymusepipe:module-pymusepipe.init_musepipe}]{\sphinxcrossref{\sphinxcode{\sphinxupquote{pymusepipe.init\_musepipe()}}}}} in a few dictionaries (\sphinxcode{\sphinxupquote{dic\_input\_folders()}},
\sphinxcode{\sphinxupquote{dic\_folders()}}, \sphinxcode{\sphinxupquote{dic\_folders\_target()}}). You can in principle change the names
of the folders themselves, although it is not advisable.

\sphinxAtStartPar
The pipeline will create the folder structure automatically, checking whether the folders exist or not.


\subsubsection{Log files}
\label{\detokenize{getting_started:log-files}}
\sphinxAtStartPar
Two basic log files are produced: one is the Esorex output which will be stored in the
“Esorex\_log” folder. The other one will be in the “Log” folder with the name provided at start:
that one is like a shell script which can be used to rerun things directly via the command line.
In the “Log” folder, there will also be, for each log file, a file “.out” and one with “.err”
extensions, respectively including all the stdout and stderr messages. This may be useful to trace
details in the data reduction and problems.


\subsubsection{Astropy Tables}
\label{\detokenize{getting_started:astropy-tables}}
\sphinxAtStartPar
Each recipe will trigger the creation of a astropy Table.
These are stored under “Astro\_Tables”. You can use these to monitor which files have been
processed or used.


\subsubsection{Sof files}
\label{\detokenize{getting_started:sof-files}}
\sphinxAtStartPar
Sof files are saved under the “Sof” directory for each esorex recipes used in the pipeline.
These are useful to see exactly which files are processed by each esorex recipe.


\subsubsection{Python structure}
\label{\detokenize{getting_started:python-structure}}
\sphinxAtStartPar
Most of the information you may need is actually stored in the python
{\hyperref[\detokenize{api/pymusepipe:pymusepipe.musepipe.MusePipe}]{\sphinxcrossref{\sphinxcode{\sphinxupquote{pymusepipe.musepipe.MusePipe}}}}} class structure. More details to come.

\sphinxstepscope


\section{Alignment}
\label{\detokenize{alignment:alignment}}\label{\detokenize{alignment::doc}}

\subsection{Limitations of run\_align recipes}
\label{\detokenize{alignment:limitations-of-run-align-recipes}}
\sphinxAtStartPar
The esorex implementation of the alignment procedure (implemented in
the {\hyperref[\detokenize{api/pymusepipe:pymusepipe.prep_recipes_pipe.PipePrep.run_align_bydataset}]{\sphinxcrossref{\sphinxcode{\sphinxupquote{pymusepipe.prep\_recipes\_pipe.PipePrep.run\_align\_bydataset()}}}}})
for multiple object exposures suffers from some severe limitations:
\begin{itemize}
\item {} 
\sphinxAtStartPar
It does not perform absolute astrometry, but merely fixes the astrometry of
subsequent exposures to the WCS of the first one. This is problematic for comparison of MUSE
data with external datasets.

\item {} 
\sphinxAtStartPar
It works by finding and matching point sources across white light images from
multiple exposures. This requires the images to contain a sufficient number of point sources.
Moreover, in case of mosaics, is requires the \sphinxstyleemphasis{overlap region} between different MUSE pointings
to contain a sufficient number of point sources. In practice, this requirement is very limiting.

\end{itemize}

\sphinxAtStartPar
Pymusepipe provides the {\hyperref[\detokenize{api/pymusepipe:module-pymusepipe.align_pipe}]{\sphinxcrossref{\sphinxcode{\sphinxupquote{pymusepipe.align\_pipe()}}}}} module to overcome both these limitations.

\sphinxstepscope


\section{Targets and Mosaicking}
\label{\detokenize{mosaicking:targets-and-mosaicking}}\label{\detokenize{mosaicking::doc}}
\sphinxAtStartPar
tbw

\sphinxstepscope


\section{Convolution}
\label{\detokenize{convolution:convolution}}\label{\detokenize{convolution::doc}}
\sphinxAtStartPar
tbw

\sphinxstepscope


\section{PHANGS pipeline example}
\label{\detokenize{phangs_example:phangs-pipeline-example}}\label{\detokenize{phangs_example::doc}}
\sphinxAtStartPar
tbw

\sphinxstepscope


\section{License}
\label{\detokenize{license:license}}\label{\detokenize{license:id1}}\label{\detokenize{license::doc}}
\sphinxAtStartPar
MIT License

\sphinxAtStartPar
Copyright (c) {[}2019{]} {[}Eric Emsellem{]}

\sphinxAtStartPar
Permission is hereby granted, free of charge, to any person obtaining a copy
of this software and associated documentation files (the “Software”), to deal
in the Software without restriction, including without limitation the rights
to use, copy, modify, merge, publish, distribute, sublicense, and/or sell
copies of the Software, and to permit persons to whom the Software is
furnished to do so, subject to the following conditions:

\sphinxAtStartPar
The above copyright notice and this permission notice shall be included in all
copies or substantial portions of the Software.

\sphinxAtStartPar
THE SOFTWARE IS PROVIDED “AS IS”, WITHOUT WARRANTY OF ANY KIND, EXPRESS OR
IMPLIED, INCLUDING BUT NOT LIMITED TO THE WARRANTIES OF MERCHANTABILITY,
FITNESS FOR A PARTICULAR PURPOSE AND NONINFRINGEMENT. IN NO EVENT SHALL THE
AUTHORS OR COPYRIGHT HOLDERS BE LIABLE FOR ANY CLAIM, DAMAGES OR OTHER
LIABILITY, WHETHER IN AN ACTION OF CONTRACT, TORT OR OTHERWISE, ARISING FROM,
OUT OF OR IN CONNECTION WITH THE SOFTWARE OR THE USE OR OTHER DEALINGS IN THE
SOFTWARE.

\sphinxstepscope


\section{Contributors}
\label{\detokenize{authors:contributors}}\label{\detokenize{authors:authors}}\label{\detokenize{authors::doc}}\begin{itemize}
\item {} 
\sphinxAtStartPar
Eric Emsellem \textless{}\sphinxhref{mailto:eric.emsellem@eso.org}{eric.emsellem@eso.org}\textgreater{}

\end{itemize}

\sphinxstepscope


\section{Changelog}
\label{\detokenize{changelog:changelog}}\label{\detokenize{changelog:changes}}\label{\detokenize{changelog::doc}}

\subsection{Version 2.9.9}
\label{\detokenize{changelog:version-2-9-9}}\begin{itemize}
\item {} 
\sphinxAtStartPar
Cleaning the full version including Target Sample

\end{itemize}


\subsection{Version 1.0}
\label{\detokenize{changelog:version-1-0}}\begin{itemize}
\item {} 
\sphinxAtStartPar
First structure with MusePointing

\end{itemize}

\sphinxstepscope


\section{pymusepipe}
\label{\detokenize{api/modules:pymusepipe}}\label{\detokenize{api/modules::doc}}
\sphinxstepscope


\subsection{pymusepipe package}
\label{\detokenize{api/pymusepipe:pymusepipe-package}}\label{\detokenize{api/pymusepipe::doc}}

\subsubsection{Submodules}
\label{\detokenize{api/pymusepipe:submodules}}

\subsubsection{pymusepipe.align\_pipe module}
\label{\detokenize{api/pymusepipe:module-pymusepipe.align_pipe}}\label{\detokenize{api/pymusepipe:pymusepipe-align-pipe-module}}\index{module@\spxentry{module}!pymusepipe.align\_pipe@\spxentry{pymusepipe.align\_pipe}}\index{pymusepipe.align\_pipe@\spxentry{pymusepipe.align\_pipe}!module@\spxentry{module}}
\sphinxAtStartPar
MUSE\sphinxhyphen{}PHANGS alignement module. This module can be used to align MUSE
reconstructed images either with each others or using a reference background
image. It spits the results out in a Fits table which can then be used
to process and mosaic Muse PIXTABLES via the MUSE ESO pipeline.
It includes a normalisation factor, an estimate of the background,
as well as any potential rotation. Fine tuning
can be done by hand by the user, using a set of reference plots.
\index{AlignMuseDataset (class in pymusepipe.align\_pipe)@\spxentry{AlignMuseDataset}\spxextra{class in pymusepipe.align\_pipe}}

\begin{fulllineitems}
\phantomsection\label{\detokenize{api/pymusepipe:pymusepipe.align_pipe.AlignMuseDataset}}
\pysigstartsignatures
\pysiglinewithargsret{\sphinxbfcode{\sphinxupquote{class\DUrole{w,w}{  }}}\sphinxcode{\sphinxupquote{pymusepipe.align\_pipe.}}\sphinxbfcode{\sphinxupquote{AlignMuseDataset}}}{\sphinxparam{\DUrole{n,n}{name\_reference}}, \sphinxparam{\DUrole{n,n}{folder\_reference}\DUrole{o,o}{=}\DUrole{default_value}{None}}, \sphinxparam{\DUrole{n,n}{folder\_muse\_images}\DUrole{o,o}{=}\DUrole{default_value}{None}}, \sphinxparam{\DUrole{n,n}{name\_muse\_images}\DUrole{o,o}{=}\DUrole{default_value}{None}}, \sphinxparam{\DUrole{n,n}{sel\_indices\_images}\DUrole{o,o}{=}\DUrole{default_value}{None}}, \sphinxparam{\DUrole{n,n}{median\_window}\DUrole{o,o}{=}\DUrole{default_value}{10}}, \sphinxparam{\DUrole{n,n}{subim\_window}\DUrole{o,o}{=}\DUrole{default_value}{10}}, \sphinxparam{\DUrole{n,n}{dynamic\_range}\DUrole{o,o}{=}\DUrole{default_value}{10}}, \sphinxparam{\DUrole{n,n}{border}\DUrole{o,o}{=}\DUrole{default_value}{10}}, \sphinxparam{\DUrole{n,n}{hdu\_ext}\DUrole{o,o}{=}\DUrole{default_value}{(0, 1)}}, \sphinxparam{\DUrole{n,n}{chunk\_size}\DUrole{o,o}{=}\DUrole{default_value}{15}}, \sphinxparam{\DUrole{n,n}{firstguess}\DUrole{o,o}{=}\DUrole{default_value}{\textquotesingle{}crosscorr\textquotesingle{}}}, \sphinxparam{\DUrole{o,o}{**}\DUrole{n,n}{kwargs}}}{}
\pysigstopsignatures
\sphinxAtStartPar
Bases: \sphinxhref{https://docs.python.org/3.10/library/functions.html\#object}{\sphinxcode{\sphinxupquote{object}}}

\sphinxAtStartPar
Class to align MUSE images onto a reference image.
\index{apply\_extra\_offset\_ima() (pymusepipe.align\_pipe.AlignMuseDataset method)@\spxentry{apply\_extra\_offset\_ima()}\spxextra{pymusepipe.align\_pipe.AlignMuseDataset method}}

\begin{fulllineitems}
\phantomsection\label{\detokenize{api/pymusepipe:pymusepipe.align_pipe.AlignMuseDataset.apply_extra_offset_ima}}
\pysigstartsignatures
\pysiglinewithargsret{\sphinxbfcode{\sphinxupquote{apply\_extra\_offset\_ima}}}{\sphinxparam{\DUrole{n,n}{nima}\DUrole{o,o}{=}\DUrole{default_value}{0}}, \sphinxparam{\DUrole{n,n}{extra\_pixel}\DUrole{o,o}{=}\DUrole{default_value}{None}}, \sphinxparam{\DUrole{n,n}{extra\_arcsec}\DUrole{o,o}{=}\DUrole{default_value}{None}}, \sphinxparam{\DUrole{n,n}{extra\_rotation}\DUrole{o,o}{=}\DUrole{default_value}{None}}, \sphinxparam{\DUrole{o,o}{**}\DUrole{n,n}{kwargs}}}{}
\pysigstopsignatures
\sphinxAtStartPar
Shift image with index nima with the total offset
after adding any extra given offset
This does not return anything but could in principle
if using the output of the self.shift


\paragraph{Input}
\label{\detokenize{api/pymusepipe:input}}\begin{description}
\sphinxlineitem{nima: int}
\sphinxAtStartPar
Index of image to consider

\sphinxlineitem{extra\_pixel: list of 2 floats}
\sphinxAtStartPar
Extra offsets (x,y) in pixels. If None, nothing is applied

\sphinxlineitem{extra\_arcsec: list of 2 floats}
\sphinxAtStartPar
Extra offsets (x,y) in arcsec if extra\_pixel is not provided
If None, nothing is applied

\sphinxlineitem{extra\_rotation: float}
\sphinxAtStartPar
Rotation in degrees. If None, no new extra offset is applied

\end{description}

\end{fulllineitems}

\index{apply\_optical\_flow\_offset\_ima() (pymusepipe.align\_pipe.AlignMuseDataset method)@\spxentry{apply\_optical\_flow\_offset\_ima()}\spxextra{pymusepipe.align\_pipe.AlignMuseDataset method}}

\begin{fulllineitems}
\phantomsection\label{\detokenize{api/pymusepipe:pymusepipe.align_pipe.AlignMuseDataset.apply_optical_flow_offset_ima}}
\pysigstartsignatures
\pysiglinewithargsret{\sphinxbfcode{\sphinxupquote{apply\_optical\_flow\_offset\_ima}}}{\sphinxparam{\DUrole{n,n}{nima}\DUrole{o,o}{=}\DUrole{default_value}{0}}}{}
\pysigstopsignatures
\sphinxAtStartPar
Transfer the value of the optical flow into the extra pixel

\end{fulllineitems}

\index{apply\_optical\_flow\_offset\_listima() (pymusepipe.align\_pipe.AlignMuseDataset method)@\spxentry{apply\_optical\_flow\_offset\_listima()}\spxextra{pymusepipe.align\_pipe.AlignMuseDataset method}}

\begin{fulllineitems}
\phantomsection\label{\detokenize{api/pymusepipe:pymusepipe.align_pipe.AlignMuseDataset.apply_optical_flow_offset_listima}}
\pysigstartsignatures
\pysiglinewithargsret{\sphinxbfcode{\sphinxupquote{apply\_optical\_flow\_offset\_listima}}}{\sphinxparam{\DUrole{n,n}{list\_nima}\DUrole{o,o}{=}\DUrole{default_value}{None}}}{}
\pysigstopsignatures
\sphinxAtStartPar
Apply the optical flow offset as extra pixels offsets and rotation


\paragraph{Input}
\label{\detokenize{api/pymusepipe:id1}}\begin{description}
\sphinxlineitem{list\_nima: list}
\sphinxAtStartPar
If None, will be initiliased to the default list of indices

\end{description}

\end{fulllineitems}

\index{compare() (pymusepipe.align\_pipe.AlignMuseDataset method)@\spxentry{compare()}\spxextra{pymusepipe.align\_pipe.AlignMuseDataset method}}

\begin{fulllineitems}
\phantomsection\label{\detokenize{api/pymusepipe:pymusepipe.align_pipe.AlignMuseDataset.compare}}
\pysigstartsignatures
\pysiglinewithargsret{\sphinxbfcode{\sphinxupquote{compare}}}{\sphinxparam{\DUrole{n,n}{data1}}, \sphinxparam{\DUrole{n,n}{data2}}, \sphinxparam{\DUrole{n,n}{header}\DUrole{o,o}{=}\DUrole{default_value}{None}}, \sphinxparam{\DUrole{n,n}{start\_nfig}\DUrole{o,o}{=}\DUrole{default_value}{1}}, \sphinxparam{\DUrole{n,n}{nlevels}\DUrole{o,o}{=}\DUrole{default_value}{10}}, \sphinxparam{\DUrole{n,n}{levels}\DUrole{o,o}{=}\DUrole{default_value}{None}}, \sphinxparam{\DUrole{n,n}{convolve\_data1}\DUrole{o,o}{=}\DUrole{default_value}{0.0}}, \sphinxparam{\DUrole{n,n}{convolve\_data2}\DUrole{o,o}{=}\DUrole{default_value}{0.0}}, \sphinxparam{\DUrole{n,n}{showcontours}\DUrole{o,o}{=}\DUrole{default_value}{True}}, \sphinxparam{\DUrole{n,n}{showcuts}\DUrole{o,o}{=}\DUrole{default_value}{True}}, \sphinxparam{\DUrole{n,n}{shownormalise}\DUrole{o,o}{=}\DUrole{default_value}{True}}, \sphinxparam{\DUrole{n,n}{showdiff}\DUrole{o,o}{=}\DUrole{default_value}{True}}, \sphinxparam{\DUrole{n,n}{normalise}\DUrole{o,o}{=}\DUrole{default_value}{True}}, \sphinxparam{\DUrole{n,n}{median\_filter}\DUrole{o,o}{=}\DUrole{default_value}{True}}, \sphinxparam{\DUrole{n,n}{ncuts}\DUrole{o,o}{=}\DUrole{default_value}{5}}, \sphinxparam{\DUrole{n,n}{percentage}\DUrole{o,o}{=}\DUrole{default_value}{5.0}}, \sphinxparam{\DUrole{n,n}{suffix\_fig}\DUrole{o,o}{=}\DUrole{default_value}{\textquotesingle{}\textquotesingle{}}}, \sphinxparam{\DUrole{o,o}{**}\DUrole{n,n}{kwargs}}}{}
\pysigstopsignatures
\sphinxAtStartPar
Compare the projected reference and MUSE image
by plotting the contours, the difference and vertical/horizontal cuts.
\begin{quote}\begin{description}
\sphinxlineitem{Parameters}\begin{itemize}
\item {} 
\sphinxAtStartPar
\sphinxstyleliteralstrong{\sphinxupquote{data1}} \textendash{} 

\item {} 
\sphinxAtStartPar
\sphinxstyleliteralstrong{\sphinxupquote{data2}} (\sphinxstyleliteralemphasis{\sphinxupquote{2d np.arrays}}) \textendash{} Array to compare

\item {} 
\sphinxAtStartPar
\sphinxstyleliteralstrong{\sphinxupquote{header}} (\sphinxstyleliteralemphasis{\sphinxupquote{Header}}) \textendash{} If provided, will be use to get the WCS in the plots. Default is None (ignored).

\item {} 
\sphinxAtStartPar
\sphinxstyleliteralstrong{\sphinxupquote{polypar}} (\sphinxstyleliteralemphasis{\sphinxupquote{ODR result}}) \textendash{} If None, it will be recalculated

\item {} 
\sphinxAtStartPar
\sphinxstyleliteralstrong{\sphinxupquote{showcontours}} (\sphinxhref{https://docs.python.org/3.10/library/functions.html\#bool}{\sphinxstyleliteralemphasis{\sphinxupquote{bool}}}\sphinxstyleliteralemphasis{\sphinxupquote{ {[}}}\sphinxstyleliteralemphasis{\sphinxupquote{True}}\sphinxstyleliteralemphasis{\sphinxupquote{{]}}}) \textendash{} 

\item {} 
\sphinxAtStartPar
\sphinxstyleliteralstrong{\sphinxupquote{showcuts}} (\sphinxhref{https://docs.python.org/3.10/library/functions.html\#bool}{\sphinxstyleliteralemphasis{\sphinxupquote{bool}}}\sphinxstyleliteralemphasis{\sphinxupquote{ {[}}}\sphinxstyleliteralemphasis{\sphinxupquote{True}}\sphinxstyleliteralemphasis{\sphinxupquote{{]}}}) \textendash{} 

\item {} 
\sphinxAtStartPar
\sphinxstyleliteralstrong{\sphinxupquote{shownormalise}} (\sphinxhref{https://docs.python.org/3.10/library/functions.html\#bool}{\sphinxstyleliteralemphasis{\sphinxupquote{bool}}}\sphinxstyleliteralemphasis{\sphinxupquote{ {[}}}\sphinxstyleliteralemphasis{\sphinxupquote{True}}\sphinxstyleliteralemphasis{\sphinxupquote{{]}}}) \textendash{} 

\item {} 
\sphinxAtStartPar
\sphinxstyleliteralstrong{\sphinxupquote{showdiff}} (\sphinxhref{https://docs.python.org/3.10/library/functions.html\#bool}{\sphinxstyleliteralemphasis{\sphinxupquote{bool}}}\sphinxstyleliteralemphasis{\sphinxupquote{ {[}}}\sphinxstyleliteralemphasis{\sphinxupquote{True}}\sphinxstyleliteralemphasis{\sphinxupquote{{]}}}) \textendash{} All options corresponding to 1 specific plot. By default
show them all (all True)

\item {} 
\sphinxAtStartPar
\sphinxstyleliteralstrong{\sphinxupquote{ncuts}} (\sphinxhref{https://docs.python.org/3.10/library/functions.html\#int}{\sphinxstyleliteralemphasis{\sphinxupquote{int}}}\sphinxstyleliteralemphasis{\sphinxupquote{ {[}}}\sphinxstyleliteralemphasis{\sphinxupquote{5}}\sphinxstyleliteralemphasis{\sphinxupquote{{]}}}) \textendash{} Number of vertical / horizontal cuts along the ratio
between the 2 maps to be shown (“cuts”)

\item {} 
\sphinxAtStartPar
\sphinxstyleliteralstrong{\sphinxupquote{percentage}} (\sphinxhref{https://docs.python.org/3.10/library/functions.html\#float}{\sphinxstyleliteralemphasis{\sphinxupquote{float}}}\sphinxstyleliteralemphasis{\sphinxupquote{ {[}}}\sphinxstyleliteralemphasis{\sphinxupquote{5}}\sphinxstyleliteralemphasis{\sphinxupquote{{]}}}) \textendash{} Used to compute which percentile to show

\item {} 
\sphinxAtStartPar
\sphinxstyleliteralstrong{\sphinxupquote{start\_nfig}} (\sphinxhref{https://docs.python.org/3.10/library/functions.html\#int}{\sphinxstyleliteralemphasis{\sphinxupquote{int}}}\sphinxstyleliteralemphasis{\sphinxupquote{ {[}}}\sphinxstyleliteralemphasis{\sphinxupquote{1}}\sphinxstyleliteralemphasis{\sphinxupquote{{]}}}) \textendash{} Number of the matplotlib Figure to start with

\item {} 
\sphinxAtStartPar
\sphinxstyleliteralstrong{\sphinxupquote{nlevels}} (\sphinxhref{https://docs.python.org/3.10/library/functions.html\#int}{\sphinxstyleliteralemphasis{\sphinxupquote{int}}}\sphinxstyleliteralemphasis{\sphinxupquote{ {[}}}\sphinxstyleliteralemphasis{\sphinxupquote{10}}\sphinxstyleliteralemphasis{\sphinxupquote{{]}}}) \textendash{} Number of levels for the contour plots

\item {} 
\sphinxAtStartPar
\sphinxstyleliteralstrong{\sphinxupquote{levels}} (\sphinxhref{https://docs.python.org/3.10/library/stdtypes.html\#list}{\sphinxstyleliteralemphasis{\sphinxupquote{list}}}\sphinxstyleliteralemphasis{\sphinxupquote{ of }}\sphinxhref{https://docs.python.org/3.10/library/functions.html\#float}{\sphinxstyleliteralemphasis{\sphinxupquote{float}}}\sphinxstyleliteralemphasis{\sphinxupquote{ {[}}}\sphinxstyleliteralemphasis{\sphinxupquote{None}}\sphinxstyleliteralemphasis{\sphinxupquote{{]}}}) \textendash{} Specific list of levels if any (default is None)

\item {} 
\sphinxAtStartPar
\sphinxstyleliteralstrong{\sphinxupquote{convolve\_data1}} (\sphinxhref{https://docs.python.org/3.10/library/functions.html\#float}{\sphinxstyleliteralemphasis{\sphinxupquote{float}}}\sphinxstyleliteralemphasis{\sphinxupquote{ {[}}}\sphinxstyleliteralemphasis{\sphinxupquote{0}}\sphinxstyleliteralemphasis{\sphinxupquote{{]}}}) \textendash{} If not 0, will convolve with a gaussian of that sigma

\item {} 
\sphinxAtStartPar
\sphinxstyleliteralstrong{\sphinxupquote{convolve\_data2}} (\sphinxhref{https://docs.python.org/3.10/library/functions.html\#float}{\sphinxstyleliteralemphasis{\sphinxupquote{float}}}\sphinxstyleliteralemphasis{\sphinxupquote{ {[}}}\sphinxstyleliteralemphasis{\sphinxupquote{0}}\sphinxstyleliteralemphasis{\sphinxupquote{{]}}}) \textendash{} If not 0, will convolve the reference image
with a gaussian of that sigma

\item {} 
\sphinxAtStartPar
\sphinxstyleliteralstrong{\sphinxupquote{(}}\sphinxstyleliteralstrong{\sphinxupquote{bool}}\sphinxstyleliteralstrong{\sphinxupquote{)}} (\sphinxstyleliteralemphasis{\sphinxupquote{savefig}}) \textendash{} If True, will save the figure into a png

\item {} 
\sphinxAtStartPar
\sphinxstyleliteralstrong{\sphinxupquote{suffix\_fig}} (\sphinxhref{https://docs.python.org/3.10/library/stdtypes.html\#str}{\sphinxstyleliteralemphasis{\sphinxupquote{str}}}) \textendash{} Suffix name to add to the figure filenames

\item {} 
\sphinxAtStartPar
\sphinxstyleliteralstrong{\sphinxupquote{figures}} (\sphinxstyleliteralemphasis{\sphinxupquote{Makes a maximum}}\sphinxstyleliteralemphasis{\sphinxupquote{ of }}\sphinxstyleliteralemphasis{\sphinxupquote{4}}) \textendash{} 

\end{itemize}

\end{description}\end{quote}

\end{fulllineitems}

\index{compare\_ima() (pymusepipe.align\_pipe.AlignMuseDataset method)@\spxentry{compare\_ima()}\spxextra{pymusepipe.align\_pipe.AlignMuseDataset method}}

\begin{fulllineitems}
\phantomsection\label{\detokenize{api/pymusepipe:pymusepipe.align_pipe.AlignMuseDataset.compare_ima}}
\pysigstartsignatures
\pysiglinewithargsret{\sphinxbfcode{\sphinxupquote{compare\_ima}}}{\sphinxparam{\DUrole{n,n}{nima}\DUrole{o,o}{=}\DUrole{default_value}{0}}, \sphinxparam{\DUrole{n,n}{nima\_museref}\DUrole{o,o}{=}\DUrole{default_value}{None}}, \sphinxparam{\DUrole{n,n}{convolve\_muse}\DUrole{o,o}{=}\DUrole{default_value}{0}}, \sphinxparam{\DUrole{n,n}{convolve\_reference}\DUrole{o,o}{=}\DUrole{default_value}{0.0}}, \sphinxparam{\DUrole{o,o}{**}\DUrole{n,n}{kwargs}}}{}
\pysigstopsignatures

\paragraph{Input}
\label{\detokenize{api/pymusepipe:id2}}\begin{description}
\sphinxlineitem{nima: int}
\sphinxAtStartPar
Index of input image

\sphinxlineitem{nima\_museref: int}
\sphinxAtStartPar
Index of second input image for the reference. Default is None, hence ignored
and the default reference image will be used.

\sphinxlineitem{convolve\_muse: float}
\sphinxAtStartPar
Sigma of the gaussian to convolve the input images. Default is 0 (no convolution)

\sphinxlineitem{convolve\_reference: float}
\sphinxAtStartPar
Sigma of the gaussian to convolve the reference. Default is 0 (no convolution)

\sphinxlineitem{threshold\_muse: float}
\sphinxAtStartPar
Minimum value to consider in the input images

\end{description}


\paragraph{Creates}
\label{\detokenize{api/pymusepipe:creates}}
\sphinxAtStartPar
Plots which compare the two input datasets as defined by the indices

\end{fulllineitems}

\index{find\_cross\_peak() (pymusepipe.align\_pipe.AlignMuseDataset method)@\spxentry{find\_cross\_peak()}\spxextra{pymusepipe.align\_pipe.AlignMuseDataset method}}

\begin{fulllineitems}
\phantomsection\label{\detokenize{api/pymusepipe:pymusepipe.align_pipe.AlignMuseDataset.find_cross_peak}}
\pysigstartsignatures
\pysiglinewithargsret{\sphinxbfcode{\sphinxupquote{find\_cross\_peak}}}{\sphinxparam{\DUrole{n,n}{muse\_hdu}}, \sphinxparam{\DUrole{n,n}{rotation}\DUrole{o,o}{=}\DUrole{default_value}{0.0}}, \sphinxparam{\DUrole{n,n}{threshold}\DUrole{o,o}{=}\DUrole{default_value}{None}}, \sphinxparam{\DUrole{o,o}{**}\DUrole{n,n}{kwargs}}}{}
\pysigstopsignatures
\sphinxAtStartPar
Aligns the MUSE HDU to a reference HDU


\paragraph{Input}
\label{\detokenize{api/pymusepipe:id3}}\begin{description}
\sphinxlineitem{muse\_hdu: astropy.io.fits hdu}
\sphinxAtStartPar
MUSE hdu file

\sphinxlineitem{name\_musehdr: str}
\sphinxAtStartPar
name of the muse hdr to save

\sphinxlineitem{rotation: float}
\sphinxAtStartPar
Angle in degrees (0).

\sphinxlineitem{threshold: minimum flux to be used in the cross\sphinxhyphen{}correlation}
\sphinxAtStartPar
Flux below that value will be set to 0.
Default is 0.

\end{description}
\begin{quote}\begin{description}
\sphinxlineitem{returns}\begin{itemize}
\item {} 
\sphinxAtStartPar
\sphinxstyleemphasis{xpix\_cross}

\item {} 
\sphinxAtStartPar
\sphinxstylestrong{ypix\_cross} (\sphinxstyleemphasis{x and y pixel coordinates of the cross\sphinxhyphen{}correlation peak})

\end{itemize}

\end{description}\end{quote}

\end{fulllineitems}

\index{find\_cross\_peak\_ima() (pymusepipe.align\_pipe.AlignMuseDataset method)@\spxentry{find\_cross\_peak\_ima()}\spxextra{pymusepipe.align\_pipe.AlignMuseDataset method}}

\begin{fulllineitems}
\phantomsection\label{\detokenize{api/pymusepipe:pymusepipe.align_pipe.AlignMuseDataset.find_cross_peak_ima}}
\pysigstartsignatures
\pysiglinewithargsret{\sphinxbfcode{\sphinxupquote{find\_cross\_peak\_ima}}}{\sphinxparam{\DUrole{n,n}{nima}\DUrole{o,o}{=}\DUrole{default_value}{0}}, \sphinxparam{\DUrole{n,n}{threshold}\DUrole{o,o}{=}\DUrole{default_value}{None}}}{}
\pysigstopsignatures
\sphinxAtStartPar
Find the cross correlation peak and get the x and y shifts
for a given image, given its index nima


\paragraph{Input}
\label{\detokenize{api/pymusepipe:id4}}\begin{description}
\sphinxlineitem{nima: int}
\sphinxAtStartPar
Index of the image

\sphinxlineitem{threshold: float}
\sphinxAtStartPar
Minimum flux for the cross\sphinxhyphen{}correlation

\end{description}

\end{fulllineitems}

\index{find\_cross\_peak\_listima() (pymusepipe.align\_pipe.AlignMuseDataset method)@\spxentry{find\_cross\_peak\_listima()}\spxextra{pymusepipe.align\_pipe.AlignMuseDataset method}}

\begin{fulllineitems}
\phantomsection\label{\detokenize{api/pymusepipe:pymusepipe.align_pipe.AlignMuseDataset.find_cross_peak_listima}}
\pysigstartsignatures
\pysiglinewithargsret{\sphinxbfcode{\sphinxupquote{find\_cross\_peak\_listima}}}{\sphinxparam{\DUrole{n,n}{list\_nima}\DUrole{o,o}{=}\DUrole{default_value}{None}}, \sphinxparam{\DUrole{n,n}{threshold}\DUrole{o,o}{=}\DUrole{default_value}{None}}}{}
\pysigstopsignatures
\sphinxAtStartPar
Run the cross correlation peaks on all MUSE images
Derive the self.cross\_off\_pixel/arcsec parameters


\paragraph{Input}
\label{\detokenize{api/pymusepipe:id5}}\begin{description}
\sphinxlineitem{list\_nima: list}
\sphinxAtStartPar
list of indices for images to process Should be a list. Default is None
and all images are processed

\sphinxlineitem{threshold: float {[}None{]}}
\sphinxAtStartPar
minimum flux to be used in the cross\sphinxhyphen{}correlation
Flux below that value will be set to 0.

\end{description}

\end{fulllineitems}

\index{get\_imaref\_muse() (pymusepipe.align\_pipe.AlignMuseDataset method)@\spxentry{get\_imaref\_muse()}\spxextra{pymusepipe.align\_pipe.AlignMuseDataset method}}

\begin{fulllineitems}
\phantomsection\label{\detokenize{api/pymusepipe:pymusepipe.align_pipe.AlignMuseDataset.get_imaref_muse}}
\pysigstartsignatures
\pysiglinewithargsret{\sphinxbfcode{\sphinxupquote{get\_imaref\_muse}}}{\sphinxparam{\DUrole{n,n}{muse\_hdu}}, \sphinxparam{\DUrole{n,n}{rotation}\DUrole{o,o}{=}\DUrole{default_value}{0.0}}, \sphinxparam{\DUrole{o,o}{**}\DUrole{n,n}{kwargs}}}{}
\pysigstopsignatures
\sphinxAtStartPar
Returns the ref and input images on the same grid as the given
input hdu assuming a given rotation


\paragraph{Input}
\label{\detokenize{api/pymusepipe:id6}}\begin{description}
\sphinxlineitem{muse\_hdu: HDU}
\sphinxAtStartPar
MUSE hdu file

\sphinxlineitem{name\_musehdr: str}
\sphinxAtStartPar
name of the muse hdr to save

\sphinxlineitem{rotation: float}
\sphinxAtStartPar
Angle in degrees (0).

\sphinxlineitem{threshold: float}
\sphinxAtStartPar
Minimum flux to prepare the image (0).

\end{description}
\begin{quote}\begin{description}
\sphinxlineitem{returns}\begin{itemize}
\item {} 
\sphinxAtStartPar
\sphinxstylestrong{ima\_ref, ima\_muse} (\sphinxstyleemphasis{arrays}) \textendash{} Reprojected images

\item {} 
\sphinxAtStartPar
\sphinxstyleemphasis{Note that the original images are saved in self.\_temp\_input\_origmuse and}

\item {} 
\sphinxAtStartPar
\sphinxstyleemphasis{self.\_temp\_input\_origref when debug mode is on (self.\_debug)}

\end{itemize}

\end{description}\end{quote}

\end{fulllineitems}

\index{get\_normfactor\_ima() (pymusepipe.align\_pipe.AlignMuseDataset method)@\spxentry{get\_normfactor\_ima()}\spxextra{pymusepipe.align\_pipe.AlignMuseDataset method}}

\begin{fulllineitems}
\phantomsection\label{\detokenize{api/pymusepipe:pymusepipe.align_pipe.AlignMuseDataset.get_normfactor_ima}}
\pysigstartsignatures
\pysiglinewithargsret{\sphinxbfcode{\sphinxupquote{get\_normfactor\_ima}}}{\sphinxparam{\DUrole{n,n}{nima}\DUrole{o,o}{=}\DUrole{default_value}{0}}, \sphinxparam{\DUrole{n,n}{median\_filter}\DUrole{o,o}{=}\DUrole{default_value}{True}}, \sphinxparam{\DUrole{n,n}{border}\DUrole{o,o}{=}\DUrole{default_value}{0}}, \sphinxparam{\DUrole{n,n}{convolve\_muse}\DUrole{o,o}{=}\DUrole{default_value}{0.0}}, \sphinxparam{\DUrole{n,n}{convolve\_reference}\DUrole{o,o}{=}\DUrole{default_value}{0.0}}, \sphinxparam{\DUrole{n,n}{chunk\_size}\DUrole{o,o}{=}\DUrole{default_value}{10}}, \sphinxparam{\DUrole{o,o}{**}\DUrole{n,n}{kwargs}}}{}
\pysigstopsignatures
\sphinxAtStartPar
Get the normalisation factor for shifted and projected images. This function only
consider the input image given by index nima and the reference image (after
projection).


\paragraph{Input}
\label{\detokenize{api/pymusepipe:id7}}\begin{description}
\sphinxlineitem{nima: int}
\sphinxAtStartPar
Index of image to consider

\sphinxlineitem{median\_filter: bool}
\sphinxAtStartPar
If True, will median filter

\sphinxlineitem{convolve\_muse: float {[}0{]}}
\sphinxAtStartPar
Will convolve the image with index nima
with a gaussian with that sigma. 0 means no convolution

\sphinxlineitem{convolve\_reference: float {[}0{]}}
\sphinxAtStartPar
Will convolve the reference image
with a gaussian with that sigma. 0 means no convolution

\sphinxlineitem{border: int}
\sphinxAtStartPar
Number of pixels to crop

\sphinxlineitem{threshold: float {[}None{]}}
\sphinxAtStartPar
Threshold for the input image flux to consider

\sphinxlineitem{chunk\_size: int}
\sphinxAtStartPar
Size of chunks to consider for chunk statistics (polynomial normalisation)

\end{description}
\begin{quote}\begin{description}
\sphinxlineitem{returns}\begin{itemize}
\item {} 
\sphinxAtStartPar
\sphinxstylestrong{data} (\sphinxstyleemphasis{2d array})

\item {} 
\sphinxAtStartPar
\sphinxstylestrong{refdata} (\sphinxstyleemphasis{2d array}) \textendash{} The 2 arrays (input, reference) after processing

\end{itemize}

\end{description}\end{quote}

\end{fulllineitems}

\index{get\_shift\_from\_pcc() (pymusepipe.align\_pipe.AlignMuseDataset method)@\spxentry{get\_shift\_from\_pcc()}\spxextra{pymusepipe.align\_pipe.AlignMuseDataset method}}

\begin{fulllineitems}
\phantomsection\label{\detokenize{api/pymusepipe:pymusepipe.align_pipe.AlignMuseDataset.get_shift_from_pcc}}
\pysigstartsignatures
\pysiglinewithargsret{\sphinxbfcode{\sphinxupquote{get\_shift\_from\_pcc}}}{\sphinxparam{\DUrole{n,n}{muse\_hdu}}, \sphinxparam{\DUrole{n,n}{rotation}\DUrole{o,o}{=}\DUrole{default_value}{0.0}}, \sphinxparam{\DUrole{n,n}{threshold}\DUrole{o,o}{=}\DUrole{default_value}{0.0}}, \sphinxparam{\DUrole{n,n}{verbose}\DUrole{o,o}{=}\DUrole{default_value}{False}}, \sphinxparam{\DUrole{o,o}{**}\DUrole{n,n}{kwargs}}}{}
\pysigstopsignatures
\sphinxAtStartPar
Find a guess translation using PCC


\paragraph{Input}
\label{\detokenize{api/pymusepipe:id8}}\begin{description}
\sphinxlineitem{muse\_hdu: HDU}
\sphinxAtStartPar
MUSE hdu file

\sphinxlineitem{rotation: float}
\sphinxAtStartPar
Angle in degrees (0).

\sphinxlineitem{threshold: float}
\sphinxAtStartPar
Minimum flux to prepare the image (0).

\sphinxlineitem{name\_musehdr: str}
\sphinxAtStartPar
Name of the muse hdr to save. Optional. Only operational if self.save\_hdr is True

\end{description}
\begin{quote}\begin{description}
\sphinxlineitem{returns}\begin{itemize}
\item {} 
\sphinxAtStartPar
\sphinxstyleemphasis{xpix\_pcc}

\item {} 
\sphinxAtStartPar
\sphinxstyleemphasis{ypix\_pcc x and y pixel coordinates of the cross\sphinxhyphen{}correlation peak}

\end{itemize}

\end{description}\end{quote}

\end{fulllineitems}

\index{get\_shift\_from\_pcc\_ima() (pymusepipe.align\_pipe.AlignMuseDataset method)@\spxentry{get\_shift\_from\_pcc\_ima()}\spxextra{pymusepipe.align\_pipe.AlignMuseDataset method}}

\begin{fulllineitems}
\phantomsection\label{\detokenize{api/pymusepipe:pymusepipe.align_pipe.AlignMuseDataset.get_shift_from_pcc_ima}}
\pysigstartsignatures
\pysiglinewithargsret{\sphinxbfcode{\sphinxupquote{get\_shift\_from\_pcc\_ima}}}{\sphinxparam{\DUrole{n,n}{nima}\DUrole{o,o}{=}\DUrole{default_value}{None}}, \sphinxparam{\DUrole{n,n}{threshold}\DUrole{o,o}{=}\DUrole{default_value}{None}}, \sphinxparam{\DUrole{n,n}{rotation}\DUrole{o,o}{=}\DUrole{default_value}{None}}, \sphinxparam{\DUrole{n,n}{verbose}\DUrole{o,o}{=}\DUrole{default_value}{False}}}{}
\pysigstopsignatures
\sphinxAtStartPar
Run the PCC shift guess for image nima


\paragraph{Input}
\label{\detokenize{api/pymusepipe:id9}}\begin{description}
\sphinxlineitem{nima: int}
\sphinxAtStartPar
Index of image

\sphinxlineitem{threshold: float {[}None{]}}
\sphinxAtStartPar
minimum value to be used in the phase cross\sphinxhyphen{}correlation
Flux below that value will be set to 0.

\sphinxlineitem{rotation: float}
\sphinxAtStartPar
If None, will take the init\_rotangle. Otherwise it will take the input value

\end{description}

\end{fulllineitems}

\index{get\_shift\_from\_pcc\_listima() (pymusepipe.align\_pipe.AlignMuseDataset method)@\spxentry{get\_shift\_from\_pcc\_listima()}\spxextra{pymusepipe.align\_pipe.AlignMuseDataset method}}

\begin{fulllineitems}
\phantomsection\label{\detokenize{api/pymusepipe:pymusepipe.align_pipe.AlignMuseDataset.get_shift_from_pcc_listima}}
\pysigstartsignatures
\pysiglinewithargsret{\sphinxbfcode{\sphinxupquote{get\_shift\_from\_pcc\_listima}}}{\sphinxparam{\DUrole{n,n}{list\_nima}\DUrole{o,o}{=}\DUrole{default_value}{None}}, \sphinxparam{\DUrole{n,n}{threshold}\DUrole{o,o}{=}\DUrole{default_value}{None}}, \sphinxparam{\DUrole{n,n}{verbose}\DUrole{o,o}{=}\DUrole{default_value}{False}}}{}
\pysigstopsignatures
\sphinxAtStartPar
Run the PCC shift guess on a list of images given by a list
of indices


\paragraph{Input}
\label{\detokenize{api/pymusepipe:id10}}\begin{description}
\sphinxlineitem{list\_nima: list of indices for images to process}
\sphinxAtStartPar
Should be a list. Default is None
and all images are processed

\sphinxlineitem{thhreshold: float {[}None{]}}
\sphinxAtStartPar
minimum value to be used in the phase cross\sphinxhyphen{}correlation
Flux below that value will be set to 0.

\end{description}

\end{fulllineitems}

\index{init\_guess\_offset() (pymusepipe.align\_pipe.AlignMuseDataset method)@\spxentry{init\_guess\_offset()}\spxextra{pymusepipe.align\_pipe.AlignMuseDataset method}}

\begin{fulllineitems}
\phantomsection\label{\detokenize{api/pymusepipe:pymusepipe.align_pipe.AlignMuseDataset.init_guess_offset}}
\pysigstartsignatures
\pysiglinewithargsret{\sphinxbfcode{\sphinxupquote{init\_guess\_offset}}}{\sphinxparam{\DUrole{o,o}{**}\DUrole{n,n}{kwargs}}}{}
\pysigstopsignatures
\sphinxAtStartPar
Initialise first guess, either from cross\sphinxhyphen{}correlation (default)
or from an Offset FITS Table


\paragraph{Input}
\label{\detokenize{api/pymusepipe:id11}}\begin{description}
\sphinxlineitem{firstguess: str}
\sphinxAtStartPar
If “crosscorr” uses cross\sphinxhyphen{}correlation to get the first guess
of the offsets. If “fits” uses the input fits table.

\end{description}

\end{fulllineitems}

\index{init\_optical\_flow\_hdu() (pymusepipe.align\_pipe.AlignMuseDataset method)@\spxentry{init\_optical\_flow\_hdu()}\spxextra{pymusepipe.align\_pipe.AlignMuseDataset method}}

\begin{fulllineitems}
\phantomsection\label{\detokenize{api/pymusepipe:pymusepipe.align_pipe.AlignMuseDataset.init_optical_flow_hdu}}
\pysigstartsignatures
\pysiglinewithargsret{\sphinxbfcode{\sphinxupquote{init\_optical\_flow\_hdu}}}{\sphinxparam{\DUrole{n,n}{muse\_hdu}}, \sphinxparam{\DUrole{n,n}{rotation}\DUrole{o,o}{=}\DUrole{default_value}{0.0}}, \sphinxparam{\DUrole{n,n}{threshold}\DUrole{o,o}{=}\DUrole{default_value}{None}}, \sphinxparam{\DUrole{n,n}{guess\_translation}\DUrole{o,o}{=}\DUrole{default_value}{(0.0, 0.0)}}, \sphinxparam{\DUrole{n,n}{header}\DUrole{o,o}{=}\DUrole{default_value}{None}}, \sphinxparam{\DUrole{n,n}{verbose}\DUrole{o,o}{=}\DUrole{default_value}{False}}, \sphinxparam{\DUrole{o,o}{**}\DUrole{n,n}{kwargs}}}{}
\pysigstopsignatures
\sphinxAtStartPar
Get the optical flow for this hdu


\paragraph{Input}
\label{\detokenize{api/pymusepipe:id12}}\begin{description}
\sphinxlineitem{muse\_hdu: HDU}
\sphinxAtStartPar
Muse HDU input

\sphinxlineitem{rotation: float}
\sphinxAtStartPar
Input rotation

\sphinxlineitem{threshold: float}
\sphinxAtStartPar
Minimum flux to consider in the image

\sphinxlineitem{guess\_translation: tuple of 2 floats}
\sphinxAtStartPar
Guess offset in X and Y, e.g., (0., 0.)

\sphinxlineitem{name\_musehdr: str}
\sphinxAtStartPar
Name of hdr in case those are saved (self.save\_hdr is True)

\end{description}

\end{fulllineitems}

\index{init\_optical\_flow\_ima() (pymusepipe.align\_pipe.AlignMuseDataset method)@\spxentry{init\_optical\_flow\_ima()}\spxextra{pymusepipe.align\_pipe.AlignMuseDataset method}}

\begin{fulllineitems}
\phantomsection\label{\detokenize{api/pymusepipe:pymusepipe.align_pipe.AlignMuseDataset.init_optical_flow_ima}}
\pysigstartsignatures
\pysiglinewithargsret{\sphinxbfcode{\sphinxupquote{init\_optical\_flow\_ima}}}{\sphinxparam{\DUrole{n,n}{nima}\DUrole{o,o}{=}\DUrole{default_value}{0}}, \sphinxparam{\DUrole{n,n}{threshold}\DUrole{o,o}{=}\DUrole{default_value}{None}}, \sphinxparam{\DUrole{n,n}{guess\_offset\_pixel}\DUrole{o,o}{=}\DUrole{default_value}{None}}, \sphinxparam{\DUrole{n,n}{guess\_offset\_arcsec}\DUrole{o,o}{=}\DUrole{default_value}{None}}, \sphinxparam{\DUrole{n,n}{guess\_rotation}\DUrole{o,o}{=}\DUrole{default_value}{None}}, \sphinxparam{\DUrole{n,n}{force\_pcc\_guess}\DUrole{o,o}{=}\DUrole{default_value}{False}}, \sphinxparam{\DUrole{n,n}{verbose}\DUrole{o,o}{=}\DUrole{default_value}{False}}, \sphinxparam{\DUrole{n,n}{provide\_header}\DUrole{o,o}{=}\DUrole{default_value}{True}}, \sphinxparam{\DUrole{o,o}{**}\DUrole{n,n}{kwargs}}}{}
\pysigstopsignatures
\sphinxAtStartPar
Initialise the optical flow using the current image with index nima


\paragraph{Input}
\label{\detokenize{api/pymusepipe:id13}}\begin{description}
\sphinxlineitem{nima: int}
\sphinxAtStartPar
Index of image

\sphinxlineitem{threshold: float}
\sphinxAtStartPar
Minimum flux to consider

\end{description}

\end{fulllineitems}

\index{init\_optical\_flow\_listima() (pymusepipe.align\_pipe.AlignMuseDataset method)@\spxentry{init\_optical\_flow\_listima()}\spxextra{pymusepipe.align\_pipe.AlignMuseDataset method}}

\begin{fulllineitems}
\phantomsection\label{\detokenize{api/pymusepipe:pymusepipe.align_pipe.AlignMuseDataset.init_optical_flow_listima}}
\pysigstartsignatures
\pysiglinewithargsret{\sphinxbfcode{\sphinxupquote{init\_optical\_flow\_listima}}}{\sphinxparam{\DUrole{n,n}{list\_nima}\DUrole{o,o}{=}\DUrole{default_value}{None}}, \sphinxparam{\DUrole{o,o}{**}\DUrole{n,n}{kwargs}}}{}
\pysigstopsignatures
\sphinxAtStartPar
Initialise the optical flow on a list of images
given by a list of indices


\paragraph{Input}
\label{\detokenize{api/pymusepipe:id14}}\begin{description}
\sphinxlineitem{list\_nima: list of indices for images to process}
\sphinxAtStartPar
Should be a list. Default is None
and all images are processed

\end{description}

\end{fulllineitems}

\index{iterate\_on\_optical\_flow\_ima() (pymusepipe.align\_pipe.AlignMuseDataset method)@\spxentry{iterate\_on\_optical\_flow\_ima()}\spxextra{pymusepipe.align\_pipe.AlignMuseDataset method}}

\begin{fulllineitems}
\phantomsection\label{\detokenize{api/pymusepipe:pymusepipe.align_pipe.AlignMuseDataset.iterate_on_optical_flow_ima}}
\pysigstartsignatures
\pysiglinewithargsret{\sphinxbfcode{\sphinxupquote{iterate\_on\_optical\_flow\_ima}}}{\sphinxparam{\DUrole{n,n}{nima}\DUrole{o,o}{=}\DUrole{default_value}{0}}, \sphinxparam{\DUrole{n,n}{niter}\DUrole{o,o}{=}\DUrole{default_value}{5}}, \sphinxparam{\DUrole{n,n}{verbose}\DUrole{o,o}{=}\DUrole{default_value}{False}}, \sphinxparam{\DUrole{n,n}{use\_rotation}\DUrole{o,o}{=}\DUrole{default_value}{True}}, \sphinxparam{\DUrole{o,o}{**}\DUrole{n,n}{kwargs}}}{}
\pysigstopsignatures
\sphinxAtStartPar
Iterate solution using the optical flow guess


\paragraph{Input}
\label{\detokenize{api/pymusepipe:id15}}\begin{description}
\sphinxlineitem{nima: int}
\sphinxAtStartPar
Index of image to consider

\sphinxlineitem{niter: int}
\sphinxAtStartPar
Number of iterations

\end{description}

\end{fulllineitems}

\index{iterate\_on\_optical\_flow\_listima() (pymusepipe.align\_pipe.AlignMuseDataset method)@\spxentry{iterate\_on\_optical\_flow\_listima()}\spxextra{pymusepipe.align\_pipe.AlignMuseDataset method}}

\begin{fulllineitems}
\phantomsection\label{\detokenize{api/pymusepipe:pymusepipe.align_pipe.AlignMuseDataset.iterate_on_optical_flow_listima}}
\pysigstartsignatures
\pysiglinewithargsret{\sphinxbfcode{\sphinxupquote{iterate\_on\_optical\_flow\_listima}}}{\sphinxparam{\DUrole{n,n}{list\_nima}\DUrole{o,o}{=}\DUrole{default_value}{None}}, \sphinxparam{\DUrole{n,n}{use\_rotation}\DUrole{o,o}{=}\DUrole{default_value}{True}}, \sphinxparam{\DUrole{o,o}{**}\DUrole{n,n}{kwargs}}}{}
\pysigstopsignatures
\sphinxAtStartPar
Run the iteration for the optical flow on a list of images
given by a list of indices


\paragraph{Input}
\label{\detokenize{api/pymusepipe:id16}}\begin{description}
\sphinxlineitem{list\_nima: list of indices for images to process}
\sphinxAtStartPar
Should be a list. Default is None
and all images are processed

\sphinxlineitem{niter: int}
\sphinxAtStartPar
Number of iterations. Optional. If  not provided, will use the
default in self.iterate\_on\_optical\_flow\_ima

\end{description}

\end{fulllineitems}

\index{list\_states() (pymusepipe.align\_pipe.AlignMuseDataset method)@\spxentry{list\_states()}\spxextra{pymusepipe.align\_pipe.AlignMuseDataset method}}

\begin{fulllineitems}
\phantomsection\label{\detokenize{api/pymusepipe:pymusepipe.align_pipe.AlignMuseDataset.list_states}}
\pysigstartsignatures
\pysiglinewithargsret{\sphinxbfcode{\sphinxupquote{list\_states}}}{\sphinxparam{\DUrole{n,n}{nstate\_max}\DUrole{o,o}{=}\DUrole{default_value}{None}}}{}
\pysigstopsignatures

\paragraph{Input}
\label{\detokenize{api/pymusepipe:id17}}
\end{fulllineitems}

\index{offset\_and\_compare\_ima() (pymusepipe.align\_pipe.AlignMuseDataset method)@\spxentry{offset\_and\_compare\_ima()}\spxextra{pymusepipe.align\_pipe.AlignMuseDataset method}}

\begin{fulllineitems}
\phantomsection\label{\detokenize{api/pymusepipe:pymusepipe.align_pipe.AlignMuseDataset.offset_and_compare_ima}}
\pysigstartsignatures
\pysiglinewithargsret{\sphinxbfcode{\sphinxupquote{offset\_and\_compare\_ima}}}{\sphinxparam{\DUrole{n,n}{nima}\DUrole{o,o}{=}\DUrole{default_value}{0}}, \sphinxparam{\DUrole{n,n}{extra\_pixel}\DUrole{o,o}{=}\DUrole{default_value}{None}}, \sphinxparam{\DUrole{n,n}{extra\_arcsec}\DUrole{o,o}{=}\DUrole{default_value}{None}}, \sphinxparam{\DUrole{n,n}{extra\_rotation}\DUrole{o,o}{=}\DUrole{default_value}{None}}, \sphinxparam{\DUrole{o,o}{**}\DUrole{n,n}{kwargs}}}{}
\pysigstopsignatures
\sphinxAtStartPar
Run the offset and comparison for a given image number


\paragraph{Input}
\label{\detokenize{api/pymusepipe:id18}}\begin{description}
\sphinxlineitem{nima: int}
\sphinxAtStartPar
Index of the image to consider

\sphinxlineitem{extra\_pixel: list of 2 floats}
\sphinxAtStartPar
Offsets in X and Y in pixels to add to the existing
guessed offsets
IMPORTANT NOTE: extra\_pixel will be considered first
(before extra\_arcsec).

\sphinxlineitem{extra\_arcsec: list of 2 floats}
\sphinxAtStartPar
Offsets in X and Y in arcsec to add to the existing
guessed offsets. Ignored if extra\_pixel is given or None

\sphinxlineitem{extra\_rotation: rotation in degrees}
\sphinxAtStartPar
Angle to rotate the image (in degrees). Ignore if None

\end{description}


\paragraph{Additional arguments}
\label{\detokenize{api/pymusepipe:additional-arguments}}\begin{description}
\sphinxlineitem{threshold: float {[}0{]}}
\sphinxAtStartPar
Threshold to consider when plotting the comparison

\sphinxlineitem{plot (bool): if True, will plot the comparison}\begin{description}
\sphinxlineitem{If not used, will use the default self.plot}\begin{itemize}
\item {} 
\sphinxAtStartPar
flux comparison (1 to 1)

\item {} 
\sphinxAtStartPar
Map of the flux ratio

\item {} 
\sphinxAtStartPar
Contours of the two scaled maps

\item {} 
\sphinxAtStartPar
Cuts of the division between the 2 maps

\end{itemize}

\end{description}

\end{description}

\sphinxAtStartPar
See also all arguments from self.compare

\end{fulllineitems}

\index{open\_hdu() (pymusepipe.align\_pipe.AlignMuseDataset method)@\spxentry{open\_hdu()}\spxextra{pymusepipe.align\_pipe.AlignMuseDataset method}}

\begin{fulllineitems}
\phantomsection\label{\detokenize{api/pymusepipe:pymusepipe.align_pipe.AlignMuseDataset.open_hdu}}
\pysigstartsignatures
\pysiglinewithargsret{\sphinxbfcode{\sphinxupquote{open\_hdu}}}{}{}
\pysigstopsignatures
\sphinxAtStartPar
Open the HDU of the MUSE and reference images

\end{fulllineitems}

\index{open\_offset\_table() (pymusepipe.align\_pipe.AlignMuseDataset method)@\spxentry{open\_offset\_table()}\spxextra{pymusepipe.align\_pipe.AlignMuseDataset method}}

\begin{fulllineitems}
\phantomsection\label{\detokenize{api/pymusepipe:pymusepipe.align_pipe.AlignMuseDataset.open_offset_table}}
\pysigstartsignatures
\pysiglinewithargsret{\sphinxbfcode{\sphinxupquote{open\_offset\_table}}}{\sphinxparam{\DUrole{n,n}{name\_table}\DUrole{o,o}{=}\DUrole{default_value}{None}}}{}
\pysigstopsignatures
\sphinxAtStartPar
Read offset table from fits file


\paragraph{Input}
\label{\detokenize{api/pymusepipe:id19}}\begin{description}
\sphinxlineitem{name\_table: str}
\sphinxAtStartPar
Name of the input OFFSET table

\end{description}
\begin{quote}\begin{description}
\sphinxlineitem{returns}\begin{itemize}
\item {} 
\sphinxAtStartPar
\sphinxstylestrong{status} (\sphinxstyleemphasis{None if no table name is given, False if file does not}) \textendash{} exist, True if it does

\item {} 
\sphinxAtStartPar
\sphinxstylestrong{Table} (\sphinxstyleemphasis{the result of a astropy.Table.read of the fits table})

\end{itemize}

\end{description}\end{quote}

\end{fulllineitems}

\index{print\_images\_names() (pymusepipe.align\_pipe.AlignMuseDataset method)@\spxentry{print\_images\_names()}\spxextra{pymusepipe.align\_pipe.AlignMuseDataset method}}

\begin{fulllineitems}
\phantomsection\label{\detokenize{api/pymusepipe:pymusepipe.align_pipe.AlignMuseDataset.print_images_names}}
\pysigstartsignatures
\pysiglinewithargsret{\sphinxbfcode{\sphinxupquote{print\_images\_names}}}{}{}
\pysigstopsignatures
\sphinxAtStartPar
Print out the names of the images being considered for alignment

\end{fulllineitems}

\index{print\_offsets\_and\_norms() (pymusepipe.align\_pipe.AlignMuseDataset method)@\spxentry{print\_offsets\_and\_norms()}\spxextra{pymusepipe.align\_pipe.AlignMuseDataset method}}

\begin{fulllineitems}
\phantomsection\label{\detokenize{api/pymusepipe:pymusepipe.align_pipe.AlignMuseDataset.print_offsets_and_norms}}
\pysigstartsignatures
\pysiglinewithargsret{\sphinxbfcode{\sphinxupquote{print\_offsets\_and\_norms}}}{\sphinxparam{\DUrole{n,n}{filename}\DUrole{o,o}{=}\DUrole{default_value}{\textquotesingle{}\_temp.txt\textquotesingle{}}}, \sphinxparam{\DUrole{n,n}{folder\_output\_file}\DUrole{o,o}{=}\DUrole{default_value}{None}}, \sphinxparam{\DUrole{n,n}{overwrite}\DUrole{o,o}{=}\DUrole{default_value}{True}}}{}
\pysigstopsignatures
\sphinxAtStartPar
Save all offsets and norms into filename. By default, file will
be overwritten.


\paragraph{Input}
\label{\detokenize{api/pymusepipe:id20}}\begin{description}
\sphinxlineitem{filename: str}
\sphinxAtStartPar
Name of file where the output will be written

\sphinxlineitem{folder\_output\_file: str}
\sphinxAtStartPar
Name of output folder where the file will be written

\sphinxlineitem{overwrite: bool}
\sphinxAtStartPar
Default is True

\end{description}


\paragraph{Creates}
\label{\detokenize{api/pymusepipe:id21}}\begin{quote}

\sphinxAtStartPar
Ascii file named via the filename input argument
\end{quote}

\end{fulllineitems}

\index{retrieve\_state() (pymusepipe.align\_pipe.AlignMuseDataset method)@\spxentry{retrieve\_state()}\spxextra{pymusepipe.align\_pipe.AlignMuseDataset method}}

\begin{fulllineitems}
\phantomsection\label{\detokenize{api/pymusepipe:pymusepipe.align_pipe.AlignMuseDataset.retrieve_state}}
\pysigstartsignatures
\pysiglinewithargsret{\sphinxbfcode{\sphinxupquote{retrieve\_state}}}{\sphinxparam{\DUrole{n,n}{nstate}\DUrole{o,o}{=}\DUrole{default_value}{1}}}{}
\pysigstopsignatures
\sphinxAtStartPar
Retrieve the state with offset and background and norm factors


\paragraph{Input}
\label{\detokenize{api/pymusepipe:id22}}
\sphinxAtStartPar
nstate = int default=1

\end{fulllineitems}

\index{run\_optical\_flow() (pymusepipe.align\_pipe.AlignMuseDataset method)@\spxentry{run\_optical\_flow()}\spxextra{pymusepipe.align\_pipe.AlignMuseDataset method}}

\begin{fulllineitems}
\phantomsection\label{\detokenize{api/pymusepipe:pymusepipe.align_pipe.AlignMuseDataset.run_optical_flow}}
\pysigstartsignatures
\pysiglinewithargsret{\sphinxbfcode{\sphinxupquote{run\_optical\_flow}}}{\sphinxparam{\DUrole{n,n}{list\_nima}\DUrole{o,o}{=}\DUrole{default_value}{None}}, \sphinxparam{\DUrole{n,n}{save\_plot}\DUrole{o,o}{=}\DUrole{default_value}{True}}, \sphinxparam{\DUrole{n,n}{use\_rotation}\DUrole{o,o}{=}\DUrole{default_value}{True}}, \sphinxparam{\DUrole{n,n}{verbose}\DUrole{o,o}{=}\DUrole{default_value}{False}}, \sphinxparam{\DUrole{o,o}{**}\DUrole{n,n}{kwargs}}}{}
\pysigstopsignatures
\sphinxAtStartPar
Run Optical flow, first with a guess offset and then iterating. The solution
is saved as extra offset in the class, and a op\_plot instance is created.
If save\_plot is True, it will save a set of default plots


\paragraph{Input}
\label{\detokenize{api/pymusepipe:id23}}\begin{description}
\sphinxlineitem{list\_nima: list}
\sphinxAtStartPar
List of indices. If None, will use the default list of all images

\sphinxlineitem{save\_plot}{[}bool{]}
\sphinxAtStartPar
Whether to save the optical flow diagnostic plots or not.

\sphinxlineitem{use\_rotation: bool}
\sphinxAtStartPar
True if you wish to have rotation. False otherwise

\end{description}

\sphinxAtStartPar
verbose: bool

\end{fulllineitems}

\index{run\_optical\_flow\_ima() (pymusepipe.align\_pipe.AlignMuseDataset method)@\spxentry{run\_optical\_flow\_ima()}\spxextra{pymusepipe.align\_pipe.AlignMuseDataset method}}

\begin{fulllineitems}
\phantomsection\label{\detokenize{api/pymusepipe:pymusepipe.align_pipe.AlignMuseDataset.run_optical_flow_ima}}
\pysigstartsignatures
\pysiglinewithargsret{\sphinxbfcode{\sphinxupquote{run\_optical\_flow\_ima}}}{\sphinxparam{\DUrole{n,n}{nima}\DUrole{o,o}{=}\DUrole{default_value}{0}}, \sphinxparam{\DUrole{n,n}{save\_plot}\DUrole{o,o}{=}\DUrole{default_value}{True}}, \sphinxparam{\DUrole{n,n}{use\_rotation}\DUrole{o,o}{=}\DUrole{default_value}{True}}, \sphinxparam{\DUrole{n,n}{verbose}\DUrole{o,o}{=}\DUrole{default_value}{False}}, \sphinxparam{\DUrole{o,o}{**}\DUrole{n,n}{kwargs}}}{}
\pysigstopsignatures
\sphinxAtStartPar
Run Optical flow on image with index nima,
first with a guess offset and then iterating. The solution
is saved as extra offset in the class, and a op\_plot instance is created.
If save\_plot is True, it will save a set of default plots


\paragraph{Input}
\label{\detokenize{api/pymusepipe:id24}}\begin{description}
\sphinxlineitem{nima: int}
\sphinxAtStartPar
Image index.

\sphinxlineitem{save\_plot}{[}bool{]}
\sphinxAtStartPar
Whether to save the optical flow diagnostic plots or not.

\end{description}

\end{fulllineitems}

\index{save\_fits\_offset\_table() (pymusepipe.align\_pipe.AlignMuseDataset method)@\spxentry{save\_fits\_offset\_table()}\spxextra{pymusepipe.align\_pipe.AlignMuseDataset method}}

\begin{fulllineitems}
\phantomsection\label{\detokenize{api/pymusepipe:pymusepipe.align_pipe.AlignMuseDataset.save_fits_offset_table}}
\pysigstartsignatures
\pysiglinewithargsret{\sphinxbfcode{\sphinxupquote{save\_fits\_offset\_table}}}{\sphinxparam{\DUrole{n,n}{name\_output\_table}\DUrole{o,o}{=}\DUrole{default_value}{None}}, \sphinxparam{\DUrole{n,n}{folder\_output\_table}\DUrole{o,o}{=}\DUrole{default_value}{None}}, \sphinxparam{\DUrole{n,n}{overwrite}\DUrole{o,o}{=}\DUrole{default_value}{False}}, \sphinxparam{\DUrole{n,n}{suffix}\DUrole{o,o}{=}\DUrole{default_value}{\textquotesingle{}\textquotesingle{}}}, \sphinxparam{\DUrole{n,n}{save\_flux\_scale}\DUrole{o,o}{=}\DUrole{default_value}{True}}, \sphinxparam{\DUrole{n,n}{save\_other\_params}\DUrole{o,o}{=}\DUrole{default_value}{True}}}{}
\pysigstopsignatures
\sphinxAtStartPar
Save the Offsets into a fits Table


\paragraph{Input}
\label{\detokenize{api/pymusepipe:id25}}\begin{description}
\sphinxlineitem{folder\_output\_table: str {[}None{]}}
\sphinxAtStartPar
Folder of the output table. If None (default) the folder
for the input offset table will be used or alternatively
the folder of the MUSE images.

\sphinxlineitem{name\_output\_table: str {[}None{]}}
\sphinxAtStartPar
Name of the output fits table. If None (default) it will
use the one given in self.name\_output\_table

\sphinxlineitem{overwrite: bool {[}False{]}}
\sphinxAtStartPar
If True, overwrite if the file exists

\sphinxlineitem{suffix: str {[}“”{]}}
\sphinxAtStartPar
Suffix to be used to add to the input name. This is handy
to just modify the given fits name with a suffix
(e.g., version number).

\sphinxlineitem{save\_flux\_scale: bool {[}True{]}}
\sphinxAtStartPar
If True, saving the flux in FLUX\_SCALE
If False, do not save the flux conversion

\sphinxlineitem{save\_other\_params: bool {[}True{]}}
\sphinxAtStartPar
If True, saving the background + rotation
If False, do not save these 2 parameters.

\end{description}


\paragraph{Creates}
\label{\detokenize{api/pymusepipe:id26}}
\sphinxAtStartPar
A fits table with the given name (using the suffix if any)

\end{fulllineitems}

\index{save\_image() (pymusepipe.align\_pipe.AlignMuseDataset method)@\spxentry{save\_image()}\spxextra{pymusepipe.align\_pipe.AlignMuseDataset method}}

\begin{fulllineitems}
\phantomsection\label{\detokenize{api/pymusepipe:pymusepipe.align_pipe.AlignMuseDataset.save_image}}
\pysigstartsignatures
\pysiglinewithargsret{\sphinxbfcode{\sphinxupquote{save\_image}}}{\sphinxparam{\DUrole{n,n}{newfits\_name}\DUrole{o,o}{=}\DUrole{default_value}{None}}, \sphinxparam{\DUrole{n,n}{nima}\DUrole{o,o}{=}\DUrole{default_value}{0}}}{}
\pysigstopsignatures
\sphinxAtStartPar
Save the newly determined hdu


\paragraph{Input}
\label{\detokenize{api/pymusepipe:id27}}\begin{description}
\sphinxlineitem{newfits\_name: str}
\sphinxAtStartPar
Name of the fits file to be used

\sphinxlineitem{nima: int {[}0{]}}
\sphinxAtStartPar
Index of the image to save

\end{description}


\paragraph{Creates}
\label{\detokenize{api/pymusepipe:id28}}
\sphinxAtStartPar
A new fits file

\end{fulllineitems}

\index{save\_polypar\_ima() (pymusepipe.align\_pipe.AlignMuseDataset method)@\spxentry{save\_polypar\_ima()}\spxextra{pymusepipe.align\_pipe.AlignMuseDataset method}}

\begin{fulllineitems}
\phantomsection\label{\detokenize{api/pymusepipe:pymusepipe.align_pipe.AlignMuseDataset.save_polypar_ima}}
\pysigstartsignatures
\pysiglinewithargsret{\sphinxbfcode{\sphinxupquote{save\_polypar\_ima}}}{\sphinxparam{\DUrole{n,n}{nima}\DUrole{o,o}{=}\DUrole{default_value}{0}}, \sphinxparam{\DUrole{n,n}{beta}\DUrole{o,o}{=}\DUrole{default_value}{None}}}{}
\pysigstopsignatures
\sphinxAtStartPar
Saving the input values into the fixed arrays for the polynomial


\paragraph{Input}
\label{\detokenize{api/pymusepipe:id29}}
\sphinxAtStartPar
beta: list/array of 2 floats

\end{fulllineitems}

\index{save\_state() (pymusepipe.align\_pipe.AlignMuseDataset method)@\spxentry{save\_state()}\spxextra{pymusepipe.align\_pipe.AlignMuseDataset method}}

\begin{fulllineitems}
\phantomsection\label{\detokenize{api/pymusepipe:pymusepipe.align_pipe.AlignMuseDataset.save_state}}
\pysigstartsignatures
\pysiglinewithargsret{\sphinxbfcode{\sphinxupquote{save\_state}}}{\sphinxparam{\DUrole{n,n}{force\_nstate}\DUrole{o,o}{=}\DUrole{default_value}{None}}}{}
\pysigstopsignatures
\sphinxAtStartPar
Save the offset and background and normalisation to a given attribute
defined by self.\_nstate


\paragraph{Input}
\label{\detokenize{api/pymusepipe:id30}}
\sphinxAtStartPar
force\_nstate: int default=None

\end{fulllineitems}

\index{show\_linearfit\_values() (pymusepipe.align\_pipe.AlignMuseDataset method)@\spxentry{show\_linearfit\_values()}\spxextra{pymusepipe.align\_pipe.AlignMuseDataset method}}

\begin{fulllineitems}
\phantomsection\label{\detokenize{api/pymusepipe:pymusepipe.align_pipe.AlignMuseDataset.show_linearfit_values}}
\pysigstartsignatures
\pysiglinewithargsret{\sphinxbfcode{\sphinxupquote{show\_linearfit\_values}}}{}{}
\pysigstopsignatures
\sphinxAtStartPar
Print some information about the linearly fitted parameters
pertaining to the normalisation.

\end{fulllineitems}

\index{show\_norm\_factors() (pymusepipe.align\_pipe.AlignMuseDataset method)@\spxentry{show\_norm\_factors()}\spxextra{pymusepipe.align\_pipe.AlignMuseDataset method}}

\begin{fulllineitems}
\phantomsection\label{\detokenize{api/pymusepipe:pymusepipe.align_pipe.AlignMuseDataset.show_norm_factors}}
\pysigstartsignatures
\pysiglinewithargsret{\sphinxbfcode{\sphinxupquote{show\_norm\_factors}}}{}{}
\pysigstopsignatures
\sphinxAtStartPar
Print some information about the normalisation factors.

\end{fulllineitems}

\index{show\_offset\_fromfits() (pymusepipe.align\_pipe.AlignMuseDataset method)@\spxentry{show\_offset\_fromfits()}\spxextra{pymusepipe.align\_pipe.AlignMuseDataset method}}

\begin{fulllineitems}
\phantomsection\label{\detokenize{api/pymusepipe:pymusepipe.align_pipe.AlignMuseDataset.show_offset_fromfits}}
\pysigstartsignatures
\pysiglinewithargsret{\sphinxbfcode{\sphinxupquote{show\_offset\_fromfits}}}{\sphinxparam{\DUrole{n,n}{name\_table}\DUrole{o,o}{=}\DUrole{default_value}{None}}}{}
\pysigstopsignatures
\sphinxAtStartPar
Print offset table from fits file


\paragraph{Input}
\label{\detokenize{api/pymusepipe:id31}}\begin{description}
\sphinxlineitem{name\_table: str}
\sphinxAtStartPar
Name of the input OFFSET table

\end{description}

\end{fulllineitems}

\index{show\_offsets() (pymusepipe.align\_pipe.AlignMuseDataset method)@\spxentry{show\_offsets()}\spxextra{pymusepipe.align\_pipe.AlignMuseDataset method}}

\begin{fulllineitems}
\phantomsection\label{\detokenize{api/pymusepipe:pymusepipe.align_pipe.AlignMuseDataset.show_offsets}}
\pysigstartsignatures
\pysiglinewithargsret{\sphinxbfcode{\sphinxupquote{show\_offsets}}}{}{}
\pysigstopsignatures
\sphinxAtStartPar
Print out the offset from the Alignment class

\end{fulllineitems}

\index{transfer\_extra\_to\_guess() (pymusepipe.align\_pipe.AlignMuseDataset method)@\spxentry{transfer\_extra\_to\_guess()}\spxextra{pymusepipe.align\_pipe.AlignMuseDataset method}}

\begin{fulllineitems}
\phantomsection\label{\detokenize{api/pymusepipe:pymusepipe.align_pipe.AlignMuseDataset.transfer_extra_to_guess}}
\pysigstartsignatures
\pysiglinewithargsret{\sphinxbfcode{\sphinxupquote{transfer\_extra\_to\_guess}}}{\sphinxparam{\DUrole{n,n}{transfer\_rotation}\DUrole{o,o}{=}\DUrole{default_value}{True}}}{}
\pysigstopsignatures
\sphinxAtStartPar
Transfer the values of the extra offset as a guess

\end{fulllineitems}


\end{fulllineitems}

\index{OffsetState (class in pymusepipe.align\_pipe)@\spxentry{OffsetState}\spxextra{class in pymusepipe.align\_pipe}}

\begin{fulllineitems}
\phantomsection\label{\detokenize{api/pymusepipe:pymusepipe.align_pipe.OffsetState}}
\pysigstartsignatures
\pysiglinewithargsret{\sphinxbfcode{\sphinxupquote{class\DUrole{w,w}{  }}}\sphinxcode{\sphinxupquote{pymusepipe.align\_pipe.}}\sphinxbfcode{\sphinxupquote{OffsetState}}}{\sphinxparam{\DUrole{n,n}{nstate}\DUrole{o,o}{=}\DUrole{default_value}{1}}, \sphinxparam{\DUrole{n,n}{info}\DUrole{o,o}{=}\DUrole{default_value}{None}}}{}
\pysigstopsignatures
\sphinxAtStartPar
Bases: \sphinxhref{https://docs.python.org/3.10/library/functions.html\#object}{\sphinxcode{\sphinxupquote{object}}}

\sphinxAtStartPar
A very simple class used to store the offsets

\end{fulllineitems}

\index{align\_hdu() (in module pymusepipe.align\_pipe)@\spxentry{align\_hdu()}\spxextra{in module pymusepipe.align\_pipe}}

\begin{fulllineitems}
\phantomsection\label{\detokenize{api/pymusepipe:pymusepipe.align_pipe.align_hdu}}
\pysigstartsignatures
\pysiglinewithargsret{\sphinxcode{\sphinxupquote{pymusepipe.align\_pipe.}}\sphinxbfcode{\sphinxupquote{align\_hdu}}}{\sphinxparam{\DUrole{n,n}{hdu\_target}\DUrole{o,o}{=}\DUrole{default_value}{None}}, \sphinxparam{\DUrole{n,n}{hdu\_to\_align}\DUrole{o,o}{=}\DUrole{default_value}{None}}, \sphinxparam{\DUrole{n,n}{target\_rotation}\DUrole{o,o}{=}\DUrole{default_value}{0.0}}, \sphinxparam{\DUrole{n,n}{to\_align\_rotation}\DUrole{o,o}{=}\DUrole{default_value}{0.0}}, \sphinxparam{\DUrole{n,n}{conversion\_factor}\DUrole{o,o}{=}\DUrole{default_value}{1.0}}, \sphinxparam{\DUrole{n,n}{use\_mpdaf}\DUrole{o,o}{=}\DUrole{default_value}{False}}}{}
\pysigstopsignatures
\sphinxAtStartPar
Project the reference image onto the MUSE dataset
Hidden function, as only used internally


\paragraph{Input}
\label{\detokenize{api/pymusepipe:id32}}\begin{description}
\sphinxlineitem{hdu\_target: HDU {[}None{]}}
\sphinxAtStartPar
Target hdu (on to which to project)

\sphinxlineitem{hdu\_to\_align: HDU {[}None{]}}
\sphinxAtStartPar
Hdu to be aligned

\sphinxlineitem{target\_rotation: float {[}0{]}}
\sphinxAtStartPar
Rotation angle in degrees of the target hdu

\sphinxlineitem{to\_align\_rotation: float {[}0{]}}
\sphinxAtStartPar
Rotation angle in degrees of the to be aligned hdu

\sphinxlineitem{conversion\_factor: float}
\sphinxAtStartPar
Factor to be applied to the to\_align hdu

\sphinxlineitem{use\_mpdaf: bool}
\sphinxAtStartPar
If True, use mpdaf to project. This is not recommended.
If False, use reproject. This is the recommended option (default)

\end{description}
\begin{quote}\begin{description}
\sphinxlineitem{returns}
\sphinxAtStartPar
\sphinxstylestrong{hdu\_repr} \textendash{} Reprojected HDU. None if nothing is provided

\sphinxlineitem{rtype}
\sphinxAtStartPar
HDU

\end{description}\end{quote}

\end{fulllineitems}

\index{arcsec\_to\_pixel() (in module pymusepipe.align\_pipe)@\spxentry{arcsec\_to\_pixel()}\spxextra{in module pymusepipe.align\_pipe}}

\begin{fulllineitems}
\phantomsection\label{\detokenize{api/pymusepipe:pymusepipe.align_pipe.arcsec_to_pixel}}
\pysigstartsignatures
\pysiglinewithargsret{\sphinxcode{\sphinxupquote{pymusepipe.align\_pipe.}}\sphinxbfcode{\sphinxupquote{arcsec\_to\_pixel}}}{\sphinxparam{\DUrole{n,n}{hdu}}, \sphinxparam{\DUrole{n,n}{xy\_arcsec}\DUrole{o,o}{=}\DUrole{default_value}{(0.0, 0.0)}}}{}
\pysigstopsignatures
\sphinxAtStartPar
Transform from arcsec to pixel for the muse image
using the hdu to extract the WCS, hence the scaling.


\paragraph{Input}
\label{\detokenize{api/pymusepipe:id33}}\begin{description}
\sphinxlineitem{hdu: astropy hdu (fits)}
\sphinxAtStartPar
Input hdu which includes a WCS

\sphinxlineitem{xy\_arcsec: list of 2 floats ({[}0,0{]})}
\sphinxAtStartPar
Coordinates to transform from arcsec to pixel.

\end{description}
\begin{quote}\begin{description}
\sphinxlineitem{returns}\begin{itemize}
\item {} 
\sphinxAtStartPar
\sphinxstylestrong{xpix, ypix} (\sphinxstyleemphasis{tuple or list of 2 floats}) \textendash{} Pixel coordinates

\item {} 
\sphinxAtStartPar
\sphinxstylestrong{See also} (\sphinxstyleemphasis{pixel\_to\_arcsec (align\_pipe.py)})

\end{itemize}

\end{description}\end{quote}

\end{fulllineitems}

\index{get\_conversion\_factor() (in module pymusepipe.align\_pipe)@\spxentry{get\_conversion\_factor()}\spxextra{in module pymusepipe.align\_pipe}}

\begin{fulllineitems}
\phantomsection\label{\detokenize{api/pymusepipe:pymusepipe.align_pipe.get_conversion_factor}}
\pysigstartsignatures
\pysiglinewithargsret{\sphinxcode{\sphinxupquote{pymusepipe.align\_pipe.}}\sphinxbfcode{\sphinxupquote{get\_conversion\_factor}}}{\sphinxparam{input\_unit, output\_unit, filter\_name=\textquotesingle{}WFI\textquotesingle{}, dict\_equiv=\{\textquotesingle{}DUPONT\_R\textquotesingle{}: {[}(Unit("erg / (Angstrom cm2 s)"), Unit("erg / (cm2 Hz s)"), \textless{}function spectral\_density.\textless{}locals\textgreater{}.f\_la\_to\_f\_nu\textgreater{}, \textless{}function spectral\_density.\textless{}locals\textgreater{}.f\_la\_from\_f\_nu\textgreater{}), (Unit("erg / (cm2 Hz s)"), Unit("erg / (cm2 s)"), \textless{}function spectral\_density.\textless{}locals\textgreater{}.f\_nu\_to\_nu\_f\_nu\textgreater{}, \textless{}function spectral\_density.\textless{}locals\textgreater{}.f\_nu\_from\_nu\_f\_nu\textgreater{}), (Unit("erg / (Angstrom cm2 s)"), Unit("erg / (cm2 s)"), \textless{}function spectral\_density.\textless{}locals\textgreater{}.f\_la\_to\_la\_f\_la\textgreater{}, \textless{}function spectral\_density.\textless{}locals\textgreater{}.f\_la\_from\_la\_f\_la\textgreater{}), (Unit("ph / (Angstrom cm2 s)"), Unit("erg / (Angstrom cm2 s)"), \textless{}function spectral\_density.\textless{}locals\textgreater{}.phot\_f\_la\_to\_f\_la\textgreater{}, \textless{}function spectral\_density.\textless{}locals\textgreater{}.phot\_f\_la\_from\_f\_la\textgreater{}), (Unit("ph / (Angstrom cm2 s)"), Unit("erg / (cm2 Hz s)"), \textless{}function spectral\_density.\textless{}locals\textgreater{}.phot\_f\_la\_to\_f\_nu\textgreater{}, \textless{}function spectral\_density.\textless{}locals\textgreater{}.phot\_f\_la\_from\_f\_nu\textgreater{}), (Unit("ph / (Angstrom cm2 s)"), Unit("ph / (cm2 Hz s)"), \textless{}function spectral\_density.\textless{}locals\textgreater{}.phot\_f\_la\_to\_phot\_f\_nu\textgreater{}, \textless{}function spectral\_density.\textless{}locals\textgreater{}.phot\_f\_la\_from\_phot\_f\_nu\textgreater{}), (Unit("ph / (cm2 Hz s)"), Unit("erg / (cm2 Hz s)"), \textless{}function spectral\_density.\textless{}locals\textgreater{}.phot\_f\_la\_to\_f\_la\textgreater{}, \textless{}function spectral\_density.\textless{}locals\textgreater{}.phot\_f\_la\_from\_f\_la\textgreater{}), (Unit("ph / (cm2 Hz s)"), Unit("erg / (Angstrom cm2 s)"), \textless{}function spectral\_density.\textless{}locals\textgreater{}.phot\_f\_nu\_to\_f\_la\textgreater{}, \textless{}function spectral\_density.\textless{}locals\textgreater{}.phot\_f\_nu\_from\_f\_la\textgreater{}), (Unit("ph / (cm2 s)"), Unit("erg / (cm2 s)"), \textless{}function spectral\_density.\textless{}locals\textgreater{}.phot\_f\_la\_to\_f\_la\textgreater{}, \textless{}function spectral\_density.\textless{}locals\textgreater{}.phot\_f\_la\_from\_f\_la\textgreater{}), (Unit("erg / (Angstrom s)"), Unit("erg / (Hz s)"), \textless{}function spectral\_density.\textless{}locals\textgreater{}.f\_la\_to\_f\_nu\textgreater{}, \textless{}function spectral\_density.\textless{}locals\textgreater{}.f\_la\_from\_f\_nu\textgreater{}), (Unit("erg / (Hz s)"), Unit("erg / s"), \textless{}function spectral\_density.\textless{}locals\textgreater{}.f\_nu\_to\_nu\_f\_nu\textgreater{}, \textless{}function spectral\_density.\textless{}locals\textgreater{}.f\_nu\_from\_nu\_f\_nu\textgreater{}), (Unit("erg / (Angstrom s)"), Unit("erg / s"), \textless{}function spectral\_density.\textless{}locals\textgreater{}.f\_la\_to\_la\_f\_la\textgreater{}, \textless{}function spectral\_density.\textless{}locals\textgreater{}.f\_la\_from\_la\_f\_la\textgreater{}), (Unit("ph / (Angstrom s)"), Unit("erg / (Angstrom s)"), \textless{}function spectral\_density.\textless{}locals\textgreater{}.phot\_f\_la\_to\_f\_la\textgreater{}, \textless{}function spectral\_density.\textless{}locals\textgreater{}.phot\_f\_la\_from\_f\_la\textgreater{}), (Unit("ph / (Angstrom s)"), Unit("erg / (Hz s)"), \textless{}function spectral\_density.\textless{}locals\textgreater{}.phot\_f\_la\_to\_f\_nu\textgreater{}, \textless{}function spectral\_density.\textless{}locals\textgreater{}.phot\_f\_la\_from\_f\_nu\textgreater{}), (Unit("ph / (Angstrom s)"), Unit("ph"), \textless{}function spectral\_density.\textless{}locals\textgreater{}.phot\_f\_la\_to\_phot\_f\_nu\textgreater{}, \textless{}function spectral\_density.\textless{}locals\textgreater{}.phot\_f\_la\_from\_phot\_f\_nu\textgreater{}), (Unit("ph"), Unit("erg / (Hz s)"), \textless{}function spectral\_density.\textless{}locals\textgreater{}.phot\_f\_la\_to\_f\_la\textgreater{}, \textless{}function spectral\_density.\textless{}locals\textgreater{}.phot\_f\_la\_from\_f\_la\textgreater{}), (Unit("ph"), Unit("erg / (Angstrom s)"), \textless{}function spectral\_density.\textless{}locals\textgreater{}.phot\_f\_nu\_to\_f\_la\textgreater{}, \textless{}function spectral\_density.\textless{}locals\textgreater{}.phot\_f\_nu\_from\_f\_la\textgreater{}), (Unit("erg / (Angstrom cm2 s sr)"), Unit("erg / (cm2 Hz s sr)"), \textless{}function spectral\_density.\textless{}locals\textgreater{}.f\_la\_to\_f\_nu\textgreater{}, \textless{}function spectral\_density.\textless{}locals\textgreater{}.f\_la\_from\_f\_nu\textgreater{}), (Unit("erg / (cm2 Hz s sr)"), Unit("erg / (cm2 s sr)"), \textless{}function spectral\_density.\textless{}locals\textgreater{}.f\_nu\_to\_nu\_f\_nu\textgreater{}, \textless{}function spectral\_density.\textless{}locals\textgreater{}.f\_nu\_from\_nu\_f\_nu\textgreater{}), (Unit("erg / (Angstrom cm2 s sr)"), Unit("erg / (cm2 s sr)"), \textless{}function spectral\_density.\textless{}locals\textgreater{}.f\_la\_to\_la\_f\_la\textgreater{}, \textless{}function spectral\_density.\textless{}locals\textgreater{}.f\_la\_from\_la\_f\_la\textgreater{}), (Unit("ph / (Angstrom cm2 s sr)"), Unit("erg / (Angstrom cm2 s sr)"), \textless{}function spectral\_density.\textless{}locals\textgreater{}.phot\_f\_la\_to\_f\_la\textgreater{}, \textless{}function spectral\_density.\textless{}locals\textgreater{}.phot\_f\_la\_from\_f\_la\textgreater{}), (Unit("ph / (Angstrom cm2 s sr)"), Unit("erg / (cm2 Hz s sr)"), \textless{}function spectral\_density.\textless{}locals\textgreater{}.phot\_f\_la\_to\_f\_nu\textgreater{}, \textless{}function spectral\_density.\textless{}locals\textgreater{}.phot\_f\_la\_from\_f\_nu\textgreater{}), (Unit("ph / (Angstrom cm2 s sr)"), Unit("ph / (cm2 Hz s sr)"), \textless{}function spectral\_density.\textless{}locals\textgreater{}.phot\_f\_la\_to\_phot\_f\_nu\textgreater{}, \textless{}function spectral\_density.\textless{}locals\textgreater{}.phot\_f\_la\_from\_phot\_f\_nu\textgreater{}), (Unit("ph / (cm2 Hz s sr)"), Unit("erg / (cm2 Hz s sr)"), \textless{}function spectral\_density.\textless{}locals\textgreater{}.phot\_f\_la\_to\_f\_la\textgreater{}, \textless{}function spectral\_density.\textless{}locals\textgreater{}.phot\_f\_la\_from\_f\_la\textgreater{}), (Unit("ph / (cm2 Hz s sr)"), Unit("erg / (Angstrom cm2 s sr)"), \textless{}function spectral\_density.\textless{}locals\textgreater{}.phot\_f\_nu\_to\_f\_la\textgreater{}, \textless{}function spectral\_density.\textless{}locals\textgreater{}.phot\_f\_nu\_from\_f\_la\textgreater{}), (Unit("erg / (Angstrom s sr)"), Unit("erg / (Hz s sr)"), \textless{}function spectral\_density.\textless{}locals\textgreater{}.f\_la\_to\_f\_nu\textgreater{}, \textless{}function spectral\_density.\textless{}locals\textgreater{}.f\_la\_from\_f\_nu\textgreater{}), (Unit("erg / (Hz s sr)"), Unit("erg / (s sr)"), \textless{}function spectral\_density.\textless{}locals\textgreater{}.f\_nu\_to\_nu\_f\_nu\textgreater{}, \textless{}function spectral\_density.\textless{}locals\textgreater{}.f\_nu\_from\_nu\_f\_nu\textgreater{}), (Unit("erg / (Angstrom s sr)"), Unit("erg / (s sr)"), \textless{}function spectral\_density.\textless{}locals\textgreater{}.f\_la\_to\_la\_f\_la\textgreater{}, \textless{}function spectral\_density.\textless{}locals\textgreater{}.f\_la\_from\_la\_f\_la\textgreater{}), (Unit("ph / (Angstrom s sr)"), Unit("erg / (Angstrom s sr)"), \textless{}function spectral\_density.\textless{}locals\textgreater{}.phot\_f\_la\_to\_f\_la\textgreater{}, \textless{}function spectral\_density.\textless{}locals\textgreater{}.phot\_f\_la\_from\_f\_la\textgreater{}), (Unit("ph / (Angstrom s sr)"), Unit("erg / (Hz s sr)"), \textless{}function spectral\_density.\textless{}locals\textgreater{}.phot\_f\_la\_to\_f\_nu\textgreater{}, \textless{}function spectral\_density.\textless{}locals\textgreater{}.phot\_f\_la\_from\_f\_nu\textgreater{}), (Unit("ph / (Angstrom s sr)"), Unit("ph / sr"), \textless{}function spectral\_density.\textless{}locals\textgreater{}.phot\_f\_la\_to\_phot\_f\_nu\textgreater{}, \textless{}function spectral\_density.\textless{}locals\textgreater{}.phot\_f\_la\_from\_phot\_f\_nu\textgreater{}), (Unit("ph / sr"), Unit("erg / (Hz s sr)"), \textless{}function spectral\_density.\textless{}locals\textgreater{}.phot\_f\_la\_to\_f\_la\textgreater{}, \textless{}function spectral\_density.\textless{}locals\textgreater{}.phot\_f\_la\_from\_f\_la\textgreater{}), (Unit("ph / sr"), Unit("erg / (Angstrom s sr)"), \textless{}function spectral\_density.\textless{}locals\textgreater{}.phot\_f\_nu\_to\_f\_la\textgreater{}, \textless{}function spectral\_density.\textless{}locals\textgreater{}.phot\_f\_nu\_from\_f\_la\textgreater{}){]}, \textquotesingle{}LEGACY\_R\textquotesingle{}: {[}(Unit("erg / (Angstrom cm2 s)"), Unit("erg / (cm2 Hz s)"), \textless{}function spectral\_density.\textless{}locals\textgreater{}.f\_la\_to\_f\_nu\textgreater{}, \textless{}function spectral\_density.\textless{}locals\textgreater{}.f\_la\_from\_f\_nu\textgreater{}), (Unit("erg / (cm2 Hz s)"), Unit("erg / (cm2 s)"), \textless{}function spectral\_density.\textless{}locals\textgreater{}.f\_nu\_to\_nu\_f\_nu\textgreater{}, \textless{}function spectral\_density.\textless{}locals\textgreater{}.f\_nu\_from\_nu\_f\_nu\textgreater{}), (Unit("erg / (Angstrom cm2 s)"), Unit("erg / (cm2 s)"), \textless{}function spectral\_density.\textless{}locals\textgreater{}.f\_la\_to\_la\_f\_la\textgreater{}, \textless{}function spectral\_density.\textless{}locals\textgreater{}.f\_la\_from\_la\_f\_la\textgreater{}), (Unit("ph / (Angstrom cm2 s)"), Unit("erg / (Angstrom cm2 s)"), \textless{}function spectral\_density.\textless{}locals\textgreater{}.phot\_f\_la\_to\_f\_la\textgreater{}, \textless{}function spectral\_density.\textless{}locals\textgreater{}.phot\_f\_la\_from\_f\_la\textgreater{}), (Unit("ph / (Angstrom cm2 s)"), Unit("erg / (cm2 Hz s)"), \textless{}function spectral\_density.\textless{}locals\textgreater{}.phot\_f\_la\_to\_f\_nu\textgreater{}, \textless{}function spectral\_density.\textless{}locals\textgreater{}.phot\_f\_la\_from\_f\_nu\textgreater{}), (Unit("ph / (Angstrom cm2 s)"), Unit("ph / (cm2 Hz s)"), \textless{}function spectral\_density.\textless{}locals\textgreater{}.phot\_f\_la\_to\_phot\_f\_nu\textgreater{}, \textless{}function spectral\_density.\textless{}locals\textgreater{}.phot\_f\_la\_from\_phot\_f\_nu\textgreater{}), (Unit("ph / (cm2 Hz s)"), Unit("erg / (cm2 Hz s)"), \textless{}function spectral\_density.\textless{}locals\textgreater{}.phot\_f\_la\_to\_f\_la\textgreater{}, \textless{}function spectral\_density.\textless{}locals\textgreater{}.phot\_f\_la\_from\_f\_la\textgreater{}), (Unit("ph / (cm2 Hz s)"), Unit("erg / (Angstrom cm2 s)"), \textless{}function spectral\_density.\textless{}locals\textgreater{}.phot\_f\_nu\_to\_f\_la\textgreater{}, \textless{}function spectral\_density.\textless{}locals\textgreater{}.phot\_f\_nu\_from\_f\_la\textgreater{}), (Unit("ph / (cm2 s)"), Unit("erg / (cm2 s)"), \textless{}function spectral\_density.\textless{}locals\textgreater{}.phot\_f\_la\_to\_f\_la\textgreater{}, \textless{}function spectral\_density.\textless{}locals\textgreater{}.phot\_f\_la\_from\_f\_la\textgreater{}), (Unit("erg / (Angstrom s)"), Unit("erg / (Hz s)"), \textless{}function spectral\_density.\textless{}locals\textgreater{}.f\_la\_to\_f\_nu\textgreater{}, \textless{}function spectral\_density.\textless{}locals\textgreater{}.f\_la\_from\_f\_nu\textgreater{}), (Unit("erg / (Hz s)"), Unit("erg / s"), \textless{}function spectral\_density.\textless{}locals\textgreater{}.f\_nu\_to\_nu\_f\_nu\textgreater{}, \textless{}function spectral\_density.\textless{}locals\textgreater{}.f\_nu\_from\_nu\_f\_nu\textgreater{}), (Unit("erg / (Angstrom s)"), Unit("erg / s"), \textless{}function spectral\_density.\textless{}locals\textgreater{}.f\_la\_to\_la\_f\_la\textgreater{}, \textless{}function spectral\_density.\textless{}locals\textgreater{}.f\_la\_from\_la\_f\_la\textgreater{}), (Unit("ph / (Angstrom s)"), Unit("erg / (Angstrom s)"), \textless{}function spectral\_density.\textless{}locals\textgreater{}.phot\_f\_la\_to\_f\_la\textgreater{}, \textless{}function spectral\_density.\textless{}locals\textgreater{}.phot\_f\_la\_from\_f\_la\textgreater{}), (Unit("ph / (Angstrom s)"), Unit("erg / (Hz s)"), \textless{}function spectral\_density.\textless{}locals\textgreater{}.phot\_f\_la\_to\_f\_nu\textgreater{}, \textless{}function spectral\_density.\textless{}locals\textgreater{}.phot\_f\_la\_from\_f\_nu\textgreater{}), (Unit("ph / (Angstrom s)"), Unit("ph"), \textless{}function spectral\_density.\textless{}locals\textgreater{}.phot\_f\_la\_to\_phot\_f\_nu\textgreater{}, \textless{}function spectral\_density.\textless{}locals\textgreater{}.phot\_f\_la\_from\_phot\_f\_nu\textgreater{}), (Unit("ph"), Unit("erg / (Hz s)"), \textless{}function spectral\_density.\textless{}locals\textgreater{}.phot\_f\_la\_to\_f\_la\textgreater{}, \textless{}function spectral\_density.\textless{}locals\textgreater{}.phot\_f\_la\_from\_f\_la\textgreater{}), (Unit("ph"), Unit("erg / (Angstrom s)"), \textless{}function spectral\_density.\textless{}locals\textgreater{}.phot\_f\_nu\_to\_f\_la\textgreater{}, \textless{}function spectral\_density.\textless{}locals\textgreater{}.phot\_f\_nu\_from\_f\_la\textgreater{}), (Unit("erg / (Angstrom cm2 s sr)"), Unit("erg / (cm2 Hz s sr)"), \textless{}function spectral\_density.\textless{}locals\textgreater{}.f\_la\_to\_f\_nu\textgreater{}, \textless{}function spectral\_density.\textless{}locals\textgreater{}.f\_la\_from\_f\_nu\textgreater{}), (Unit("erg / (cm2 Hz s sr)"), Unit("erg / (cm2 s sr)"), \textless{}function spectral\_density.\textless{}locals\textgreater{}.f\_nu\_to\_nu\_f\_nu\textgreater{}, \textless{}function spectral\_density.\textless{}locals\textgreater{}.f\_nu\_from\_nu\_f\_nu\textgreater{}), (Unit("erg / (Angstrom cm2 s sr)"), Unit("erg / (cm2 s sr)"), \textless{}function spectral\_density.\textless{}locals\textgreater{}.f\_la\_to\_la\_f\_la\textgreater{}, \textless{}function spectral\_density.\textless{}locals\textgreater{}.f\_la\_from\_la\_f\_la\textgreater{}), (Unit("ph / (Angstrom cm2 s sr)"), Unit("erg / (Angstrom cm2 s sr)"), \textless{}function spectral\_density.\textless{}locals\textgreater{}.phot\_f\_la\_to\_f\_la\textgreater{}, \textless{}function spectral\_density.\textless{}locals\textgreater{}.phot\_f\_la\_from\_f\_la\textgreater{}), (Unit("ph / (Angstrom cm2 s sr)"), Unit("erg / (cm2 Hz s sr)"), \textless{}function spectral\_density.\textless{}locals\textgreater{}.phot\_f\_la\_to\_f\_nu\textgreater{}, \textless{}function spectral\_density.\textless{}locals\textgreater{}.phot\_f\_la\_from\_f\_nu\textgreater{}), (Unit("ph / (Angstrom cm2 s sr)"), Unit("ph / (cm2 Hz s sr)"), \textless{}function spectral\_density.\textless{}locals\textgreater{}.phot\_f\_la\_to\_phot\_f\_nu\textgreater{}, \textless{}function spectral\_density.\textless{}locals\textgreater{}.phot\_f\_la\_from\_phot\_f\_nu\textgreater{}), (Unit("ph / (cm2 Hz s sr)"), Unit("erg / (cm2 Hz s sr)"), \textless{}function spectral\_density.\textless{}locals\textgreater{}.phot\_f\_la\_to\_f\_la\textgreater{}, \textless{}function spectral\_density.\textless{}locals\textgreater{}.phot\_f\_la\_from\_f\_la\textgreater{}), (Unit("ph / (cm2 Hz s sr)"), Unit("erg / (Angstrom cm2 s sr)"), \textless{}function spectral\_density.\textless{}locals\textgreater{}.phot\_f\_nu\_to\_f\_la\textgreater{}, \textless{}function spectral\_density.\textless{}locals\textgreater{}.phot\_f\_nu\_from\_f\_la\textgreater{}), (Unit("erg / (Angstrom s sr)"), Unit("erg / (Hz s sr)"), \textless{}function spectral\_density.\textless{}locals\textgreater{}.f\_la\_to\_f\_nu\textgreater{}, \textless{}function spectral\_density.\textless{}locals\textgreater{}.f\_la\_from\_f\_nu\textgreater{}), (Unit("erg / (Hz s sr)"), Unit("erg / (s sr)"), \textless{}function spectral\_density.\textless{}locals\textgreater{}.f\_nu\_to\_nu\_f\_nu\textgreater{}, \textless{}function spectral\_density.\textless{}locals\textgreater{}.f\_nu\_from\_nu\_f\_nu\textgreater{}), (Unit("erg / (Angstrom s sr)"), Unit("erg / (s sr)"), \textless{}function spectral\_density.\textless{}locals\textgreater{}.f\_la\_to\_la\_f\_la\textgreater{}, \textless{}function spectral\_density.\textless{}locals\textgreater{}.f\_la\_from\_la\_f\_la\textgreater{}), (Unit("ph / (Angstrom s sr)"), Unit("erg / (Angstrom s sr)"), \textless{}function spectral\_density.\textless{}locals\textgreater{}.phot\_f\_la\_to\_f\_la\textgreater{}, \textless{}function spectral\_density.\textless{}locals\textgreater{}.phot\_f\_la\_from\_f\_la\textgreater{}), (Unit("ph / (Angstrom s sr)"), Unit("erg / (Hz s sr)"), \textless{}function spectral\_density.\textless{}locals\textgreater{}.phot\_f\_la\_to\_f\_nu\textgreater{}, \textless{}function spectral\_density.\textless{}locals\textgreater{}.phot\_f\_la\_from\_f\_nu\textgreater{}), (Unit("ph / (Angstrom s sr)"), Unit("ph / sr"), \textless{}function spectral\_density.\textless{}locals\textgreater{}.phot\_f\_la\_to\_phot\_f\_nu\textgreater{}, \textless{}function spectral\_density.\textless{}locals\textgreater{}.phot\_f\_la\_from\_phot\_f\_nu\textgreater{}), (Unit("ph / sr"), Unit("erg / (Hz s sr)"), \textless{}function spectral\_density.\textless{}locals\textgreater{}.phot\_f\_la\_to\_f\_la\textgreater{}, \textless{}function spectral\_density.\textless{}locals\textgreater{}.phot\_f\_la\_from\_f\_la\textgreater{}), (Unit("ph / sr"), Unit("erg / (Angstrom s sr)"), \textless{}function spectral\_density.\textless{}locals\textgreater{}.phot\_f\_nu\_to\_f\_la\textgreater{}, \textless{}function spectral\_density.\textless{}locals\textgreater{}.phot\_f\_nu\_from\_f\_la\textgreater{}){]}, \textquotesingle{}WFI\_BB\textquotesingle{}: {[}(Unit("erg / (Angstrom cm2 s)"), Unit("erg / (cm2 Hz s)"), \textless{}function spectral\_density.\textless{}locals\textgreater{}.f\_la\_to\_f\_nu\textgreater{}, \textless{}function spectral\_density.\textless{}locals\textgreater{}.f\_la\_from\_f\_nu\textgreater{}), (Unit("erg / (cm2 Hz s)"), Unit("erg / (cm2 s)"), \textless{}function spectral\_density.\textless{}locals\textgreater{}.f\_nu\_to\_nu\_f\_nu\textgreater{}, \textless{}function spectral\_density.\textless{}locals\textgreater{}.f\_nu\_from\_nu\_f\_nu\textgreater{}), (Unit("erg / (Angstrom cm2 s)"), Unit("erg / (cm2 s)"), \textless{}function spectral\_density.\textless{}locals\textgreater{}.f\_la\_to\_la\_f\_la\textgreater{}, \textless{}function spectral\_density.\textless{}locals\textgreater{}.f\_la\_from\_la\_f\_la\textgreater{}), (Unit("ph / (Angstrom cm2 s)"), Unit("erg / (Angstrom cm2 s)"), \textless{}function spectral\_density.\textless{}locals\textgreater{}.phot\_f\_la\_to\_f\_la\textgreater{}, \textless{}function spectral\_density.\textless{}locals\textgreater{}.phot\_f\_la\_from\_f\_la\textgreater{}), (Unit("ph / (Angstrom cm2 s)"), Unit("erg / (cm2 Hz s)"), \textless{}function spectral\_density.\textless{}locals\textgreater{}.phot\_f\_la\_to\_f\_nu\textgreater{}, \textless{}function spectral\_density.\textless{}locals\textgreater{}.phot\_f\_la\_from\_f\_nu\textgreater{}), (Unit("ph / (Angstrom cm2 s)"), Unit("ph / (cm2 Hz s)"), \textless{}function spectral\_density.\textless{}locals\textgreater{}.phot\_f\_la\_to\_phot\_f\_nu\textgreater{}, \textless{}function spectral\_density.\textless{}locals\textgreater{}.phot\_f\_la\_from\_phot\_f\_nu\textgreater{}), (Unit("ph / (cm2 Hz s)"), Unit("erg / (cm2 Hz s)"), \textless{}function spectral\_density.\textless{}locals\textgreater{}.phot\_f\_la\_to\_f\_la\textgreater{}, \textless{}function spectral\_density.\textless{}locals\textgreater{}.phot\_f\_la\_from\_f\_la\textgreater{}), (Unit("ph / (cm2 Hz s)"), Unit("erg / (Angstrom cm2 s)"), \textless{}function spectral\_density.\textless{}locals\textgreater{}.phot\_f\_nu\_to\_f\_la\textgreater{}, \textless{}function spectral\_density.\textless{}locals\textgreater{}.phot\_f\_nu\_from\_f\_la\textgreater{}), (Unit("ph / (cm2 s)"), Unit("erg / (cm2 s)"), \textless{}function spectral\_density.\textless{}locals\textgreater{}.phot\_f\_la\_to\_f\_la\textgreater{}, \textless{}function spectral\_density.\textless{}locals\textgreater{}.phot\_f\_la\_from\_f\_la\textgreater{}), (Unit("erg / (Angstrom s)"), Unit("erg / (Hz s)"), \textless{}function spectral\_density.\textless{}locals\textgreater{}.f\_la\_to\_f\_nu\textgreater{}, \textless{}function spectral\_density.\textless{}locals\textgreater{}.f\_la\_from\_f\_nu\textgreater{}), (Unit("erg / (Hz s)"), Unit("erg / s"), \textless{}function spectral\_density.\textless{}locals\textgreater{}.f\_nu\_to\_nu\_f\_nu\textgreater{}, \textless{}function spectral\_density.\textless{}locals\textgreater{}.f\_nu\_from\_nu\_f\_nu\textgreater{}), (Unit("erg / (Angstrom s)"), Unit("erg / s"), \textless{}function spectral\_density.\textless{}locals\textgreater{}.f\_la\_to\_la\_f\_la\textgreater{}, \textless{}function spectral\_density.\textless{}locals\textgreater{}.f\_la\_from\_la\_f\_la\textgreater{}), (Unit("ph / (Angstrom s)"), Unit("erg / (Angstrom s)"), \textless{}function spectral\_density.\textless{}locals\textgreater{}.phot\_f\_la\_to\_f\_la\textgreater{}, \textless{}function spectral\_density.\textless{}locals\textgreater{}.phot\_f\_la\_from\_f\_la\textgreater{}), (Unit("ph / (Angstrom s)"), Unit("erg / (Hz s)"), \textless{}function spectral\_density.\textless{}locals\textgreater{}.phot\_f\_la\_to\_f\_nu\textgreater{}, \textless{}function spectral\_density.\textless{}locals\textgreater{}.phot\_f\_la\_from\_f\_nu\textgreater{}), (Unit("ph / (Angstrom s)"), Unit("ph"), \textless{}function spectral\_density.\textless{}locals\textgreater{}.phot\_f\_la\_to\_phot\_f\_nu\textgreater{}, \textless{}function spectral\_density.\textless{}locals\textgreater{}.phot\_f\_la\_from\_phot\_f\_nu\textgreater{}), (Unit("ph"), Unit("erg / (Hz s)"), \textless{}function spectral\_density.\textless{}locals\textgreater{}.phot\_f\_la\_to\_f\_la\textgreater{}, \textless{}function spectral\_density.\textless{}locals\textgreater{}.phot\_f\_la\_from\_f\_la\textgreater{}), (Unit("ph"), Unit("erg / (Angstrom s)"), \textless{}function spectral\_density.\textless{}locals\textgreater{}.phot\_f\_nu\_to\_f\_la\textgreater{}, \textless{}function spectral\_density.\textless{}locals\textgreater{}.phot\_f\_nu\_from\_f\_la\textgreater{}), (Unit("erg / (Angstrom cm2 s sr)"), Unit("erg / (cm2 Hz s sr)"), \textless{}function spectral\_density.\textless{}locals\textgreater{}.f\_la\_to\_f\_nu\textgreater{}, \textless{}function spectral\_density.\textless{}locals\textgreater{}.f\_la\_from\_f\_nu\textgreater{}), (Unit("erg / (cm2 Hz s sr)"), Unit("erg / (cm2 s sr)"), \textless{}function spectral\_density.\textless{}locals\textgreater{}.f\_nu\_to\_nu\_f\_nu\textgreater{}, \textless{}function spectral\_density.\textless{}locals\textgreater{}.f\_nu\_from\_nu\_f\_nu\textgreater{}), (Unit("erg / (Angstrom cm2 s sr)"), Unit("erg / (cm2 s sr)"), \textless{}function spectral\_density.\textless{}locals\textgreater{}.f\_la\_to\_la\_f\_la\textgreater{}, \textless{}function spectral\_density.\textless{}locals\textgreater{}.f\_la\_from\_la\_f\_la\textgreater{}), (Unit("ph / (Angstrom cm2 s sr)"), Unit("erg / (Angstrom cm2 s sr)"), \textless{}function spectral\_density.\textless{}locals\textgreater{}.phot\_f\_la\_to\_f\_la\textgreater{}, \textless{}function spectral\_density.\textless{}locals\textgreater{}.phot\_f\_la\_from\_f\_la\textgreater{}), (Unit("ph / (Angstrom cm2 s sr)"), Unit("erg / (cm2 Hz s sr)"), \textless{}function spectral\_density.\textless{}locals\textgreater{}.phot\_f\_la\_to\_f\_nu\textgreater{}, \textless{}function spectral\_density.\textless{}locals\textgreater{}.phot\_f\_la\_from\_f\_nu\textgreater{}), (Unit("ph / (Angstrom cm2 s sr)"), Unit("ph / (cm2 Hz s sr)"), \textless{}function spectral\_density.\textless{}locals\textgreater{}.phot\_f\_la\_to\_phot\_f\_nu\textgreater{}, \textless{}function spectral\_density.\textless{}locals\textgreater{}.phot\_f\_la\_from\_phot\_f\_nu\textgreater{}), (Unit("ph / (cm2 Hz s sr)"), Unit("erg / (cm2 Hz s sr)"), \textless{}function spectral\_density.\textless{}locals\textgreater{}.phot\_f\_la\_to\_f\_la\textgreater{}, \textless{}function spectral\_density.\textless{}locals\textgreater{}.phot\_f\_la\_from\_f\_la\textgreater{}), (Unit("ph / (cm2 Hz s sr)"), Unit("erg / (Angstrom cm2 s sr)"), \textless{}function spectral\_density.\textless{}locals\textgreater{}.phot\_f\_nu\_to\_f\_la\textgreater{}, \textless{}function spectral\_density.\textless{}locals\textgreater{}.phot\_f\_nu\_from\_f\_la\textgreater{}), (Unit("erg / (Angstrom s sr)"), Unit("erg / (Hz s sr)"), \textless{}function spectral\_density.\textless{}locals\textgreater{}.f\_la\_to\_f\_nu\textgreater{}, \textless{}function spectral\_density.\textless{}locals\textgreater{}.f\_la\_from\_f\_nu\textgreater{}), (Unit("erg / (Hz s sr)"), Unit("erg / (s sr)"), \textless{}function spectral\_density.\textless{}locals\textgreater{}.f\_nu\_to\_nu\_f\_nu\textgreater{}, \textless{}function spectral\_density.\textless{}locals\textgreater{}.f\_nu\_from\_nu\_f\_nu\textgreater{}), (Unit("erg / (Angstrom s sr)"), Unit("erg / (s sr)"), \textless{}function spectral\_density.\textless{}locals\textgreater{}.f\_la\_to\_la\_f\_la\textgreater{}, \textless{}function spectral\_density.\textless{}locals\textgreater{}.f\_la\_from\_la\_f\_la\textgreater{}), (Unit("ph / (Angstrom s sr)"), Unit("erg / (Angstrom s sr)"), \textless{}function spectral\_density.\textless{}locals\textgreater{}.phot\_f\_la\_to\_f\_la\textgreater{}, \textless{}function spectral\_density.\textless{}locals\textgreater{}.phot\_f\_la\_from\_f\_la\textgreater{}), (Unit("ph / (Angstrom s sr)"), Unit("erg / (Hz s sr)"), \textless{}function spectral\_density.\textless{}locals\textgreater{}.phot\_f\_la\_to\_f\_nu\textgreater{}, \textless{}function spectral\_density.\textless{}locals\textgreater{}.phot\_f\_la\_from\_f\_nu\textgreater{}), (Unit("ph / (Angstrom s sr)"), Unit("ph / sr"), \textless{}function spectral\_density.\textless{}locals\textgreater{}.phot\_f\_la\_to\_phot\_f\_nu\textgreater{}, \textless{}function spectral\_density.\textless{}locals\textgreater{}.phot\_f\_la\_from\_phot\_f\_nu\textgreater{}), (Unit("ph / sr"), Unit("erg / (Hz s sr)"), \textless{}function spectral\_density.\textless{}locals\textgreater{}.phot\_f\_la\_to\_f\_la\textgreater{}, \textless{}function spectral\_density.\textless{}locals\textgreater{}.phot\_f\_la\_from\_f\_la\textgreater{}), (Unit("ph / sr"), Unit("erg / (Angstrom s sr)"), \textless{}function spectral\_density.\textless{}locals\textgreater{}.phot\_f\_nu\_to\_f\_la\textgreater{}, \textless{}function spectral\_density.\textless{}locals\textgreater{}.phot\_f\_nu\_from\_f\_la\textgreater{}){]}\}}}{}
\pysigstopsignatures
\sphinxAtStartPar
Conversion of units from an input one
to an output one


\paragraph{Input}
\label{\detokenize{api/pymusepipe:id34}}\begin{description}
\sphinxlineitem{input\_unit: astropy unit}
\sphinxAtStartPar
Input astropy unit to analyse

\sphinxlineitem{output\_unit: astropy unit}
\sphinxAtStartPar
Astropy unit to compare to input unit.

\sphinxlineitem{filter\_name: str, optional}
\sphinxAtStartPar
Name of the filter

\sphinxlineitem{dict\_equiv: dict, optional}
\sphinxAtStartPar
Dictionary listing the equivalencies for various filters.
Will be using it if the filter name is the key list
to help with the conversion.

\end{description}
\begin{quote}\begin{description}
\sphinxlineitem{returns}
\sphinxAtStartPar
\sphinxstylestrong{conversion}

\sphinxlineitem{rtype}
\sphinxAtStartPar
astropy unit conversion

\end{description}\end{quote}

\end{fulllineitems}

\index{init\_plot\_optical\_flow() (in module pymusepipe.align\_pipe)@\spxentry{init\_plot\_optical\_flow()}\spxextra{in module pymusepipe.align\_pipe}}

\begin{fulllineitems}
\phantomsection\label{\detokenize{api/pymusepipe:pymusepipe.align_pipe.init_plot_optical_flow}}
\pysigstartsignatures
\pysiglinewithargsret{\sphinxcode{\sphinxupquote{pymusepipe.align\_pipe.}}\sphinxbfcode{\sphinxupquote{init\_plot\_optical\_flow}}}{\sphinxparam{\DUrole{n,n}{opflow}}}{}
\pysigstopsignatures
\sphinxAtStartPar
Initialise the optical flow plot using the AlignmentPlotting


\paragraph{Input}
\label{\detokenize{api/pymusepipe:id35}}
\sphinxAtStartPar
opflow: optical flow instance (see spacepylot)
\begin{quote}\begin{description}
\sphinxlineitem{rtype}
\sphinxAtStartPar
An optical flow plot instance

\end{description}\end{quote}

\end{fulllineitems}

\index{is\_sequence() (in module pymusepipe.align\_pipe)@\spxentry{is\_sequence()}\spxextra{in module pymusepipe.align\_pipe}}

\begin{fulllineitems}
\phantomsection\label{\detokenize{api/pymusepipe:pymusepipe.align_pipe.is_sequence}}
\pysigstartsignatures
\pysiglinewithargsret{\sphinxcode{\sphinxupquote{pymusepipe.align\_pipe.}}\sphinxbfcode{\sphinxupquote{is\_sequence}}}{\sphinxparam{\DUrole{n,n}{arg}}}{}
\pysigstopsignatures
\sphinxAtStartPar
Test if sequence and return the boolean result
\begin{quote}\begin{description}
\sphinxlineitem{Parameters}
\sphinxAtStartPar
\sphinxstyleliteralstrong{\sphinxupquote{arg}} (\sphinxstyleliteralemphasis{\sphinxupquote{input argument}}) \textendash{} 

\sphinxlineitem{Returns}
\sphinxAtStartPar
\sphinxstylestrong{result}

\sphinxlineitem{Return type}
\sphinxAtStartPar
boolean

\end{description}\end{quote}

\end{fulllineitems}

\index{pixel\_to\_arcsec() (in module pymusepipe.align\_pipe)@\spxentry{pixel\_to\_arcsec()}\spxextra{in module pymusepipe.align\_pipe}}

\begin{fulllineitems}
\phantomsection\label{\detokenize{api/pymusepipe:pymusepipe.align_pipe.pixel_to_arcsec}}
\pysigstartsignatures
\pysiglinewithargsret{\sphinxcode{\sphinxupquote{pymusepipe.align\_pipe.}}\sphinxbfcode{\sphinxupquote{pixel\_to\_arcsec}}}{\sphinxparam{\DUrole{n,n}{hdu}}, \sphinxparam{\DUrole{n,n}{xy\_pixel}\DUrole{o,o}{=}\DUrole{default_value}{(0.0, 0.0)}}}{}
\pysigstopsignatures
\sphinxAtStartPar
Transform from arcsec to pixel for the muse image
using the hdu to extract the WCS, hence the scaling.


\paragraph{Input}
\label{\detokenize{api/pymusepipe:id36}}\begin{description}
\sphinxlineitem{hdu: astropy hdu (fits)}
\sphinxAtStartPar
Input hdu which includes a WCS

\sphinxlineitem{xy\_pixel: tuple or list of 2 floats ((0,0))}
\sphinxAtStartPar
Coordinates to transform from pixel to arcsec

\end{description}
\begin{quote}\begin{description}
\sphinxlineitem{returns}\begin{itemize}
\item {} 
\sphinxAtStartPar
\sphinxstylestrong{xarc, yarc} (\sphinxstyleemphasis{2 floats}) \textendash{} Arcseconds coordinates

\item {} 
\sphinxAtStartPar
\sphinxstylestrong{See also} (\sphinxstyleemphasis{arcsec\_to\_pixel (align\_pipe.py)})

\end{itemize}

\end{description}\end{quote}

\end{fulllineitems}

\index{rotate\_pixtable() (in module pymusepipe.align\_pipe)@\spxentry{rotate\_pixtable()}\spxextra{in module pymusepipe.align\_pipe}}

\begin{fulllineitems}
\phantomsection\label{\detokenize{api/pymusepipe:pymusepipe.align_pipe.rotate_pixtable}}
\pysigstartsignatures
\pysiglinewithargsret{\sphinxcode{\sphinxupquote{pymusepipe.align\_pipe.}}\sphinxbfcode{\sphinxupquote{rotate\_pixtable}}}{\sphinxparam{\DUrole{n,n}{folder}\DUrole{o,o}{=}\DUrole{default_value}{\textquotesingle{}\textquotesingle{}}}, \sphinxparam{\DUrole{n,n}{name\_suffix}\DUrole{o,o}{=}\DUrole{default_value}{\textquotesingle{}\textquotesingle{}}}, \sphinxparam{\DUrole{n,n}{nifu}\DUrole{o,o}{=}\DUrole{default_value}{1}}, \sphinxparam{\DUrole{n,n}{angle}\DUrole{o,o}{=}\DUrole{default_value}{0.0}}, \sphinxparam{\DUrole{o,o}{**}\DUrole{n,n}{kwargs}}}{}
\pysigstopsignatures
\sphinxAtStartPar
Rotate a single IFU PIXTABLE\_OBJECT
Will thus update the HIERARCH ESO INS DROT POSANG keyword.


\paragraph{Input}
\label{\detokenize{api/pymusepipe:id37}}\begin{description}
\sphinxlineitem{folder: str}
\sphinxAtStartPar
name of the folder where the PIXTABLE are

\sphinxlineitem{name\_suffix: str}
\sphinxAtStartPar
name of the suffix to be used on top of PIXTABLE\_OBJECT

\sphinxlineitem{nifu: int}
\sphinxAtStartPar
Pixtable number. Default is 1

\sphinxlineitem{angle: float}
\sphinxAtStartPar
Angle to rotate (in degrees)

\end{description}

\end{fulllineitems}

\index{rotate\_pixtables() (in module pymusepipe.align\_pipe)@\spxentry{rotate\_pixtables()}\spxextra{in module pymusepipe.align\_pipe}}

\begin{fulllineitems}
\phantomsection\label{\detokenize{api/pymusepipe:pymusepipe.align_pipe.rotate_pixtables}}
\pysigstartsignatures
\pysiglinewithargsret{\sphinxcode{\sphinxupquote{pymusepipe.align\_pipe.}}\sphinxbfcode{\sphinxupquote{rotate\_pixtables}}}{\sphinxparam{\DUrole{n,n}{folder}\DUrole{o,o}{=}\DUrole{default_value}{\textquotesingle{}\textquotesingle{}}}, \sphinxparam{\DUrole{n,n}{name\_suffix}\DUrole{o,o}{=}\DUrole{default_value}{\textquotesingle{}\textquotesingle{}}}, \sphinxparam{\DUrole{n,n}{list\_ifu}\DUrole{o,o}{=}\DUrole{default_value}{None}}, \sphinxparam{\DUrole{n,n}{angle}\DUrole{o,o}{=}\DUrole{default_value}{0.0}}, \sphinxparam{\DUrole{o,o}{**}\DUrole{n,n}{kwargs}}}{}
\pysigstopsignatures
\sphinxAtStartPar
Will update the derotator angle in each of the 24 pixtables
Using a loop on rotate\_pixtable

\sphinxAtStartPar
Will thus update the HIERARCH ESO INS DROT POSANG keyword.


\paragraph{Input}
\label{\detokenize{api/pymusepipe:id38}}\begin{description}
\sphinxlineitem{folder: str}
\sphinxAtStartPar
name of the folder where the PIXTABLE are

\sphinxlineitem{name\_suffix: str}
\sphinxAtStartPar
name of the suffix to be used on top of PIXTABLE\_OBJECT

\sphinxlineitem{list\_ifu: list{[}int{]}}
\sphinxAtStartPar
List of Pixtable numbers. If None, will do all 24

\sphinxlineitem{angle: float}
\sphinxAtStartPar
Angle to rotate (in degrees)

\end{description}

\end{fulllineitems}



\subsubsection{pymusepipe.check\_pipe module}
\label{\detokenize{api/pymusepipe:module-pymusepipe.check_pipe}}\label{\detokenize{api/pymusepipe:pymusepipe-check-pipe-module}}\index{module@\spxentry{module}!pymusepipe.check\_pipe@\spxentry{pymusepipe.check\_pipe}}\index{pymusepipe.check\_pipe@\spxentry{pymusepipe.check\_pipe}!module@\spxentry{module}}
\sphinxAtStartPar
MUSE\sphinxhyphen{}PHANGS check pipeline module
\index{CheckPipe (class in pymusepipe.check\_pipe)@\spxentry{CheckPipe}\spxextra{class in pymusepipe.check\_pipe}}

\begin{fulllineitems}
\phantomsection\label{\detokenize{api/pymusepipe:pymusepipe.check_pipe.CheckPipe}}
\pysigstartsignatures
\pysiglinewithargsret{\sphinxbfcode{\sphinxupquote{class\DUrole{w,w}{  }}}\sphinxcode{\sphinxupquote{pymusepipe.check\_pipe.}}\sphinxbfcode{\sphinxupquote{CheckPipe}}}{\sphinxparam{\DUrole{n,n}{mycube}\DUrole{o,o}{=}\DUrole{default_value}{\textquotesingle{}DATACUBE\_FINAL.fits\textquotesingle{}}}, \sphinxparam{\DUrole{n,n}{pdf\_name}\DUrole{o,o}{=}\DUrole{default_value}{\textquotesingle{}check\_pipe.pdf\textquotesingle{}}}, \sphinxparam{\DUrole{n,n}{pipe}\DUrole{o,o}{=}\DUrole{default_value}{None}}, \sphinxparam{\DUrole{n,n}{standard\_set}\DUrole{o,o}{=}\DUrole{default_value}{True}}, \sphinxparam{\DUrole{o,o}{**}\DUrole{n,n}{kwargs}}}{}
\pysigstopsignatures
\sphinxAtStartPar
Bases: {\hyperref[\detokenize{api/pymusepipe:pymusepipe.musepipe.MusePipe}]{\sphinxcrossref{\sphinxcode{\sphinxupquote{MusePipe}}}}}

\sphinxAtStartPar
Checking the outcome of the data reduction
\index{check\_given\_images() (pymusepipe.check\_pipe.CheckPipe method)@\spxentry{check\_given\_images()}\spxextra{pymusepipe.check\_pipe.CheckPipe method}}

\begin{fulllineitems}
\phantomsection\label{\detokenize{api/pymusepipe:pymusepipe.check_pipe.CheckPipe.check_given_images}}
\pysigstartsignatures
\pysiglinewithargsret{\sphinxbfcode{\sphinxupquote{check\_given\_images}}}{\sphinxparam{\DUrole{n,n}{suffix}\DUrole{o,o}{=}\DUrole{default_value}{None}}}{}
\pysigstopsignatures
\sphinxAtStartPar
Check all images with given suffix

\end{fulllineitems}

\index{check\_master\_bias\_flat() (pymusepipe.check\_pipe.CheckPipe method)@\spxentry{check\_master\_bias\_flat()}\spxextra{pymusepipe.check\_pipe.CheckPipe method}}

\begin{fulllineitems}
\phantomsection\label{\detokenize{api/pymusepipe:pymusepipe.check_pipe.CheckPipe.check_master_bias_flat}}
\pysigstartsignatures
\pysiglinewithargsret{\sphinxbfcode{\sphinxupquote{check\_master\_bias\_flat}}}{}{}
\pysigstopsignatures
\sphinxAtStartPar
Checking the Master bias and Master flat

\end{fulllineitems}

\index{check\_quadrants() (pymusepipe.check\_pipe.CheckPipe method)@\spxentry{check\_quadrants()}\spxextra{pymusepipe.check\_pipe.CheckPipe method}}

\begin{fulllineitems}
\phantomsection\label{\detokenize{api/pymusepipe:pymusepipe.check_pipe.CheckPipe.check_quadrants}}
\pysigstartsignatures
\pysiglinewithargsret{\sphinxbfcode{\sphinxupquote{check\_quadrants}}}{}{}
\pysigstopsignatures
\sphinxAtStartPar
Checking spectra from the 4 quadrants

\end{fulllineitems}

\index{check\_sky\_spectra() (pymusepipe.check\_pipe.CheckPipe method)@\spxentry{check\_sky\_spectra()}\spxextra{pymusepipe.check\_pipe.CheckPipe method}}

\begin{fulllineitems}
\phantomsection\label{\detokenize{api/pymusepipe:pymusepipe.check_pipe.CheckPipe.check_sky_spectra}}
\pysigstartsignatures
\pysiglinewithargsret{\sphinxbfcode{\sphinxupquote{check\_sky\_spectra}}}{\sphinxparam{\DUrole{n,n}{suffix}}}{}
\pysigstopsignatures
\sphinxAtStartPar
Check all sky spectra from the exposures

\end{fulllineitems}

\index{check\_white\_line\_images() (pymusepipe.check\_pipe.CheckPipe method)@\spxentry{check\_white\_line\_images()}\spxextra{pymusepipe.check\_pipe.CheckPipe method}}

\begin{fulllineitems}
\phantomsection\label{\detokenize{api/pymusepipe:pymusepipe.check_pipe.CheckPipe.check_white_line_images}}
\pysigstartsignatures
\pysiglinewithargsret{\sphinxbfcode{\sphinxupquote{check\_white\_line\_images}}}{\sphinxparam{\DUrole{n,n}{line}\DUrole{o,o}{=}\DUrole{default_value}{\textquotesingle{}Ha\textquotesingle{}}}, \sphinxparam{\DUrole{n,n}{velocity}\DUrole{o,o}{=}\DUrole{default_value}{0.0}}}{}
\pysigstopsignatures
\sphinxAtStartPar
Building the White and Ha images and
Adding them on the page

\end{fulllineitems}


\end{fulllineitems}

\index{print\_plot() (in module pymusepipe.check\_pipe)@\spxentry{print\_plot()}\spxextra{in module pymusepipe.check\_pipe}}

\begin{fulllineitems}
\phantomsection\label{\detokenize{api/pymusepipe:pymusepipe.check_pipe.print_plot}}
\pysigstartsignatures
\pysiglinewithargsret{\sphinxcode{\sphinxupquote{pymusepipe.check\_pipe.}}\sphinxbfcode{\sphinxupquote{print\_plot}}}{\sphinxparam{\DUrole{n,n}{text}}}{}
\pysigstopsignatures
\end{fulllineitems}



\subsubsection{pymusepipe.combine module}
\label{\detokenize{api/pymusepipe:module-pymusepipe.combine}}\label{\detokenize{api/pymusepipe:pymusepipe-combine-module}}\index{module@\spxentry{module}!pymusepipe.combine@\spxentry{pymusepipe.combine}}\index{pymusepipe.combine@\spxentry{pymusepipe.combine}!module@\spxentry{module}}
\sphinxAtStartPar
MUSE\sphinxhyphen{}PHANGS combine module
\index{MusePointings (class in pymusepipe.combine)@\spxentry{MusePointings}\spxextra{class in pymusepipe.combine}}

\begin{fulllineitems}
\phantomsection\label{\detokenize{api/pymusepipe:pymusepipe.combine.MusePointings}}
\pysigstartsignatures
\pysiglinewithargsret{\sphinxbfcode{\sphinxupquote{class\DUrole{w,w}{  }}}\sphinxcode{\sphinxupquote{pymusepipe.combine.}}\sphinxbfcode{\sphinxupquote{MusePointings}}}{\sphinxparam{\DUrole{n,n}{targetname}\DUrole{o,o}{=}\DUrole{default_value}{None}}, \sphinxparam{\DUrole{n,n}{list\_datasets}\DUrole{o,o}{=}\DUrole{default_value}{None}}, \sphinxparam{\DUrole{n,n}{list\_pointings}\DUrole{o,o}{=}\DUrole{default_value}{None}}, \sphinxparam{\DUrole{n,n}{pointing\_table}\DUrole{o,o}{=}\DUrole{default_value}{None}}, \sphinxparam{\DUrole{n,n}{pointing\_table\_format}\DUrole{o,o}{=}\DUrole{default_value}{\textquotesingle{}ascii\textquotesingle{}}}, \sphinxparam{\DUrole{n,n}{pointing\_table\_folder}\DUrole{o,o}{=}\DUrole{default_value}{\textquotesingle{}\textquotesingle{}}}, \sphinxparam{\DUrole{n,n}{folder\_config}\DUrole{o,o}{=}\DUrole{default_value}{\textquotesingle{}\textquotesingle{}}}, \sphinxparam{\DUrole{n,n}{rc\_filename}\DUrole{o,o}{=}\DUrole{default_value}{None}}, \sphinxparam{\DUrole{n,n}{cal\_filename}\DUrole{o,o}{=}\DUrole{default_value}{None}}, \sphinxparam{\DUrole{n,n}{suffix}\DUrole{o,o}{=}\DUrole{default_value}{\textquotesingle{}\textquotesingle{}}}, \sphinxparam{\DUrole{n,n}{name\_offset\_table}\DUrole{o,o}{=}\DUrole{default_value}{None}}, \sphinxparam{\DUrole{n,n}{folder\_offset\_table}\DUrole{o,o}{=}\DUrole{default_value}{None}}, \sphinxparam{\DUrole{n,n}{log\_filename}\DUrole{o,o}{=}\DUrole{default_value}{\textquotesingle{}MusePipeCombine.log\textquotesingle{}}}, \sphinxparam{\DUrole{n,n}{verbose}\DUrole{o,o}{=}\DUrole{default_value}{True}}, \sphinxparam{\DUrole{n,n}{debug}\DUrole{o,o}{=}\DUrole{default_value}{False}}, \sphinxparam{\DUrole{o,o}{**}\DUrole{n,n}{kwargs}}}{}
\pysigstopsignatures
\sphinxAtStartPar
Bases: {\hyperref[\detokenize{api/pymusepipe:pymusepipe.create_sof.SofPipe}]{\sphinxcrossref{\sphinxcode{\sphinxupquote{SofPipe}}}}}, {\hyperref[\detokenize{api/pymusepipe:pymusepipe.recipes_pipe.PipeRecipes}]{\sphinxcrossref{\sphinxcode{\sphinxupquote{PipeRecipes}}}}}

\sphinxAtStartPar
Class for a set of MUSE Pointings which can be covering several
datasets. This provides a set of rules and methods to access the data and
process them.
\index{assign\_pointing\_table() (pymusepipe.combine.MusePointings method)@\spxentry{assign\_pointing\_table()}\spxextra{pymusepipe.combine.MusePointings method}}

\begin{fulllineitems}
\phantomsection\label{\detokenize{api/pymusepipe:pymusepipe.combine.MusePointings.assign_pointing_table}}
\pysigstartsignatures
\pysiglinewithargsret{\sphinxbfcode{\sphinxupquote{assign\_pointing\_table}}}{\sphinxparam{\DUrole{n,n}{input\_table}\DUrole{o,o}{=}\DUrole{default_value}{None}}, \sphinxparam{\DUrole{n,n}{folder}\DUrole{o,o}{=}\DUrole{default_value}{\textquotesingle{}\textquotesingle{}}}, \sphinxparam{\DUrole{n,n}{table\_format}\DUrole{o,o}{=}\DUrole{default_value}{\textquotesingle{}ascii\textquotesingle{}}}}{}
\pysigstopsignatures
\sphinxAtStartPar
Assign the pointing table as provided. If not provided it will create one from the
pixtable list


\paragraph{Input}
\label{\detokenize{api/pymusepipe:id39}}
\sphinxAtStartPar
input\_table: str, QTable, Table or PointingTable

\sphinxAtStartPar
Create pointing\_table attribute.

\end{fulllineitems}

\index{create\_all\_pointings\_wcs() (pymusepipe.combine.MusePointings method)@\spxentry{create\_all\_pointings\_wcs()}\spxextra{pymusepipe.combine.MusePointings method}}

\begin{fulllineitems}
\phantomsection\label{\detokenize{api/pymusepipe:pymusepipe.combine.MusePointings.create_all_pointings_wcs}}
\pysigstartsignatures
\pysiglinewithargsret{\sphinxbfcode{\sphinxupquote{create\_all\_pointings\_wcs}}}{\sphinxparam{\DUrole{n,n}{filter\_list}\DUrole{o,o}{=}\DUrole{default_value}{\textquotesingle{}white\textquotesingle{}}}, \sphinxparam{\DUrole{n,n}{list\_pointings}\DUrole{o,o}{=}\DUrole{default_value}{None}}, \sphinxparam{\DUrole{o,o}{**}\DUrole{n,n}{kwargs}}}{}
\pysigstopsignatures
\sphinxAtStartPar
Create all pointing masks one by one
as well as the wcs for each individual pointings. Using the grid
from the global WCS of the mosaic but restricting it to the
range of non\sphinxhyphen{}NaN.
Hence this needs a global WCS mosaic as a reference to work.


\paragraph{Input}
\label{\detokenize{api/pymusepipe:id40}}\begin{description}
\sphinxlineitem{filter\_list = list of str}
\sphinxAtStartPar
List of filter names to be used.

\end{description}

\end{fulllineitems}

\index{create\_combined\_wcs() (pymusepipe.combine.MusePointings method)@\spxentry{create\_combined\_wcs()}\spxextra{pymusepipe.combine.MusePointings method}}

\begin{fulllineitems}
\phantomsection\label{\detokenize{api/pymusepipe:pymusepipe.combine.MusePointings.create_combined_wcs}}
\pysigstartsignatures
\pysiglinewithargsret{\sphinxbfcode{\sphinxupquote{create\_combined\_wcs}}}{\sphinxparam{\DUrole{n,n}{refcube\_name}\DUrole{o,o}{=}\DUrole{default_value}{None}}, \sphinxparam{\DUrole{n,n}{lambdaminmax\_wcs}\DUrole{o,o}{=}\DUrole{default_value}{{[}6800, 6805{]}}}, \sphinxparam{\DUrole{o,o}{**}\DUrole{n,n}{kwargs}}}{}
\pysigstopsignatures
\sphinxAtStartPar
Create the reference WCS from the full mosaic
with a given range of lambda.


\paragraph{Input}
\label{\detokenize{api/pymusepipe:id41}}\begin{description}
\sphinxlineitem{refcube\_name: str}
\sphinxAtStartPar
Name of the cube. Can be None, and then the final
datacube from the combine folder will be used.

\sphinxlineitem{wave1: float \sphinxhyphen{} optional}
\sphinxAtStartPar
Wavelength taken for the extraction. Should only
be present in all spaxels you wish to get.

\sphinxlineitem{prefix\_wcs: str \sphinxhyphen{} optional}
\sphinxAtStartPar
Prefix to be added to the name of the input cube.
By default, will use “refwcs”.

\sphinxlineitem{add\_targetname: bool {[}True{]}}
\sphinxAtStartPar
Add the name of the target to the name of the output
WCS reference cube. Default is True.

\end{description}

\end{fulllineitems}

\index{create\_pointing\_wcs() (pymusepipe.combine.MusePointings method)@\spxentry{create\_pointing\_wcs()}\spxextra{pymusepipe.combine.MusePointings method}}

\begin{fulllineitems}
\phantomsection\label{\detokenize{api/pymusepipe:pymusepipe.combine.MusePointings.create_pointing_wcs}}
\pysigstartsignatures
\pysiglinewithargsret{\sphinxbfcode{\sphinxupquote{create\_pointing\_wcs}}}{\sphinxparam{\DUrole{n,n}{pointing}}, \sphinxparam{\DUrole{n,n}{lambdaminmax\_mosaic}\DUrole{o,o}{=}\DUrole{default_value}{{[}4700, 9400{]}}}, \sphinxparam{\DUrole{n,n}{filter\_list}\DUrole{o,o}{=}\DUrole{default_value}{\textquotesingle{}white\textquotesingle{}}}, \sphinxparam{\DUrole{o,o}{**}\DUrole{n,n}{kwargs}}}{}
\pysigstopsignatures
\sphinxAtStartPar
Create the mask of a given pointing
And also a WCS file which can then be used to compute individual
pointings with a fixed WCS.


\paragraph{Input}
\label{\detokenize{api/pymusepipe:id42}}\begin{description}
\sphinxlineitem{pointing: int}
\sphinxAtStartPar
Number of the pointing

\sphinxlineitem{lambdaminmax\_mosaic: array of 2 floats}
\sphinxAtStartPar
Default is lambdaminmax\_for\_mosaic, the starting end ending wavelengths needed for a mosaic.

\sphinxlineitem{filter\_list = list of str}
\sphinxAtStartPar
List of filter names to be used.

\sphinxlineitem{Creates:}
\sphinxAtStartPar
pointing mask WCS cube

\end{description}
\begin{quote}\begin{description}
\sphinxlineitem{returns}
\sphinxAtStartPar
Name of the created WCS cube

\end{description}\end{quote}

\end{fulllineitems}

\index{create\_reference\_wcs() (pymusepipe.combine.MusePointings method)@\spxentry{create\_reference\_wcs()}\spxextra{pymusepipe.combine.MusePointings method}}

\begin{fulllineitems}
\phantomsection\label{\detokenize{api/pymusepipe:pymusepipe.combine.MusePointings.create_reference_wcs}}
\pysigstartsignatures
\pysiglinewithargsret{\sphinxbfcode{\sphinxupquote{create\_reference\_wcs}}}{\sphinxparam{\DUrole{n,n}{pointings\_wcs}\DUrole{o,o}{=}\DUrole{default_value}{True}}, \sphinxparam{\DUrole{n,n}{mosaic\_wcs}\DUrole{o,o}{=}\DUrole{default_value}{True}}, \sphinxparam{\DUrole{n,n}{wcs\_refcube\_name}\DUrole{o,o}{=}\DUrole{default_value}{None}}, \sphinxparam{\DUrole{n,n}{refcube\_name}\DUrole{o,o}{=}\DUrole{default_value}{None}}, \sphinxparam{\DUrole{n,n}{folder\_refcube}\DUrole{o,o}{=}\DUrole{default_value}{\textquotesingle{}\textquotesingle{}}}, \sphinxparam{\DUrole{n,n}{list\_pointings}\DUrole{o,o}{=}\DUrole{default_value}{None}}, \sphinxparam{\DUrole{o,o}{**}\DUrole{n,n}{kwargs}}}{}
\pysigstopsignatures
\sphinxAtStartPar
Create the WCS reference files, for all individual pointings and for
the mosaic.


\paragraph{Input}
\label{\detokenize{api/pymusepipe:id43}}\begin{description}
\sphinxlineitem{pointings\_wcs: bool {[}True{]}}
\sphinxAtStartPar
Will run the individual pointings WCS

\sphinxlineitem{mosaic\_wcs: bool {[}True{]}}
\sphinxAtStartPar
Will run the combined WCS

\sphinxlineitem{wcs\_refcube\_name: str default=None, optional}
\sphinxAtStartPar
Name of the input WCS to be used. If None (default), we will look at the refcube\_name keyword.
If set to ‘auto’, we will used the default naming conventions to find it on disk.
If set to a bona fide name, it will be used as reference WCS.

\sphinxlineitem{refcube\_name: str default=None, optional}
\sphinxAtStartPar
Name of the input cube to guide the WCS building (only used if wcs\_refcube\_name is None).
If None, a run\_combine will ensure we have a good reference cube that can be used for the building of a
reference WCS. If provided, it will be used as the reference cube to then build the reference WCS.

\sphinxlineitem{folder\_refcube: str, optional}
\sphinxAtStartPar
Folder name for the reference cube or wcs.

\sphinxlineitem{list\_pointings: list of int default=None, optional}
\sphinxAtStartPar
List of pointings to consider

\sphinxlineitem{{\color{red}\bfseries{}**}kwargs: additional keywords including}
\sphinxAtStartPar
lambdaminmax: {[}float, float{]}

\end{description}

\end{fulllineitems}

\index{dict\_pixtabs\_in\_datasets (pymusepipe.combine.MusePointings property)@\spxentry{dict\_pixtabs\_in\_datasets}\spxextra{pymusepipe.combine.MusePointings property}}

\begin{fulllineitems}
\phantomsection\label{\detokenize{api/pymusepipe:pymusepipe.combine.MusePointings.dict_pixtabs_in_datasets}}
\pysigstartsignatures
\pysigline{\sphinxbfcode{\sphinxupquote{property\DUrole{w,w}{  }}}\sphinxbfcode{\sphinxupquote{dict\_pixtabs\_in\_datasets}}}
\pysigstopsignatures
\end{fulllineitems}

\index{dict\_pixtabs\_in\_pointings (pymusepipe.combine.MusePointings property)@\spxentry{dict\_pixtabs\_in\_pointings}\spxextra{pymusepipe.combine.MusePointings property}}

\begin{fulllineitems}
\phantomsection\label{\detokenize{api/pymusepipe:pymusepipe.combine.MusePointings.dict_pixtabs_in_pointings}}
\pysigstartsignatures
\pysigline{\sphinxbfcode{\sphinxupquote{property\DUrole{w,w}{  }}}\sphinxbfcode{\sphinxupquote{dict\_pixtabs\_in\_pointings}}}
\pysigstopsignatures
\end{fulllineitems}

\index{dict\_tplexpo\_per\_dataset (pymusepipe.combine.MusePointings property)@\spxentry{dict\_tplexpo\_per\_dataset}\spxextra{pymusepipe.combine.MusePointings property}}

\begin{fulllineitems}
\phantomsection\label{\detokenize{api/pymusepipe:pymusepipe.combine.MusePointings.dict_tplexpo_per_dataset}}
\pysigstartsignatures
\pysigline{\sphinxbfcode{\sphinxupquote{property\DUrole{w,w}{  }}}\sphinxbfcode{\sphinxupquote{dict\_tplexpo\_per\_dataset}}}
\pysigstopsignatures
\end{fulllineitems}

\index{dict\_tplexpo\_per\_pointing (pymusepipe.combine.MusePointings property)@\spxentry{dict\_tplexpo\_per\_pointing}\spxextra{pymusepipe.combine.MusePointings property}}

\begin{fulllineitems}
\phantomsection\label{\detokenize{api/pymusepipe:pymusepipe.combine.MusePointings.dict_tplexpo_per_pointing}}
\pysigstartsignatures
\pysigline{\sphinxbfcode{\sphinxupquote{property\DUrole{w,w}{  }}}\sphinxbfcode{\sphinxupquote{dict\_tplexpo\_per\_pointing}}}
\pysigstopsignatures
\end{fulllineitems}

\index{extract\_combined\_narrow\_wcs() (pymusepipe.combine.MusePointings method)@\spxentry{extract\_combined\_narrow\_wcs()}\spxextra{pymusepipe.combine.MusePointings method}}

\begin{fulllineitems}
\phantomsection\label{\detokenize{api/pymusepipe:pymusepipe.combine.MusePointings.extract_combined_narrow_wcs}}
\pysigstartsignatures
\pysiglinewithargsret{\sphinxbfcode{\sphinxupquote{extract\_combined\_narrow\_wcs}}}{\sphinxparam{\DUrole{n,n}{name\_cube}\DUrole{o,o}{=}\DUrole{default_value}{None}}, \sphinxparam{\DUrole{o,o}{**}\DUrole{n,n}{kwargs}}}{}
\pysigstopsignatures
\sphinxAtStartPar
Create the reference WCS from the full mosaic with only 2 lambdas


\paragraph{Input}
\label{\detokenize{api/pymusepipe:id46}}\begin{description}
\sphinxlineitem{name\_cube: str}
\sphinxAtStartPar
Name of the cube. Can be None, and then the final datacube from the combine
folder will be used.

\sphinxlineitem{wave1: float \sphinxhyphen{} optional}
\sphinxAtStartPar
Wavelength taken for the extraction. Should only be present in all spaxels
you wish to get.

\sphinxlineitem{prefix\_wcs: str \sphinxhyphen{} optional}
\sphinxAtStartPar
Prefix to be added to the name of the input cube. By default, will use “refwcs”.

\sphinxlineitem{add\_targetname: bool default=True}
\sphinxAtStartPar
Add the name of the target to the name of the output WCS reference cube.
Default is True.

\sphinxlineitem{Creates:}
\sphinxAtStartPar
Combined narrow band WCS cube

\end{description}
\begin{quote}\begin{description}
\sphinxlineitem{returns}
\sphinxAtStartPar
name of the created cube

\end{description}\end{quote}

\end{fulllineitems}

\index{filter\_pixables\_with\_list() (pymusepipe.combine.MusePointings method)@\spxentry{filter\_pixables\_with\_list()}\spxextra{pymusepipe.combine.MusePointings method}}

\begin{fulllineitems}
\phantomsection\label{\detokenize{api/pymusepipe:pymusepipe.combine.MusePointings.filter_pixables_with_list}}
\pysigstartsignatures
\pysiglinewithargsret{\sphinxbfcode{\sphinxupquote{filter\_pixables\_with\_list}}}{\sphinxparam{\DUrole{n,n}{list\_datasets}\DUrole{o,o}{=}\DUrole{default_value}{None}}, \sphinxparam{\DUrole{n,n}{list\_pointings}\DUrole{o,o}{=}\DUrole{default_value}{None}}}{}
\pysigstopsignatures
\sphinxAtStartPar
Filter a list of pixtables
\begin{quote}\begin{description}
\sphinxlineitem{Parameters}\begin{itemize}
\item {} 
\sphinxAtStartPar
\sphinxstyleliteralstrong{\sphinxupquote{list\_datasets}} (\sphinxhref{https://docs.python.org/3.10/library/stdtypes.html\#list}{\sphinxstyleliteralemphasis{\sphinxupquote{list}}}\sphinxstyleliteralemphasis{\sphinxupquote{ of }}\sphinxhref{https://docs.python.org/3.10/library/functions.html\#int}{\sphinxstyleliteralemphasis{\sphinxupquote{int}}}\sphinxstyleliteralemphasis{\sphinxupquote{, }}\sphinxstyleliteralemphasis{\sphinxupquote{optional}}) \textendash{} 

\item {} 
\sphinxAtStartPar
\sphinxstyleliteralstrong{\sphinxupquote{list\_pointings}} (\sphinxhref{https://docs.python.org/3.10/library/stdtypes.html\#list}{\sphinxstyleliteralemphasis{\sphinxupquote{list}}}\sphinxstyleliteralemphasis{\sphinxupquote{ of }}\sphinxhref{https://docs.python.org/3.10/library/functions.html\#int}{\sphinxstyleliteralemphasis{\sphinxupquote{int}}}\sphinxstyleliteralemphasis{\sphinxupquote{, }}\sphinxstyleliteralemphasis{\sphinxupquote{optional}}) \textendash{} 

\item {} 
\sphinxAtStartPar
\sphinxstyleliteralstrong{\sphinxupquote{pointings}} (\sphinxstyleliteralemphasis{\sphinxupquote{Filter out the pointing table using those datasets and}}) \textendash{} 

\end{itemize}

\end{description}\end{quote}

\end{fulllineitems}

\index{full\_list\_datasets (pymusepipe.combine.MusePointings property)@\spxentry{full\_list\_datasets}\spxextra{pymusepipe.combine.MusePointings property}}

\begin{fulllineitems}
\phantomsection\label{\detokenize{api/pymusepipe:pymusepipe.combine.MusePointings.full_list_datasets}}
\pysigstartsignatures
\pysigline{\sphinxbfcode{\sphinxupquote{property\DUrole{w,w}{  }}}\sphinxbfcode{\sphinxupquote{full\_list\_datasets}}}
\pysigstopsignatures
\end{fulllineitems}

\index{get\_all\_pixtables() (pymusepipe.combine.MusePointings method)@\spxentry{get\_all\_pixtables()}\spxextra{pymusepipe.combine.MusePointings method}}

\begin{fulllineitems}
\phantomsection\label{\detokenize{api/pymusepipe:pymusepipe.combine.MusePointings.get_all_pixtables}}
\pysigstartsignatures
\pysiglinewithargsret{\sphinxbfcode{\sphinxupquote{get\_all\_pixtables}}}{}{}
\pysigstopsignatures
\sphinxAtStartPar
List all pixtables in the data folder
Fill in the dict\_allpixtabs\_in\_datasets dictionary and creates the pointing table

\end{fulllineitems}

\index{get\_pointing\_table() (pymusepipe.combine.MusePointings method)@\spxentry{get\_pointing\_table()}\spxextra{pymusepipe.combine.MusePointings method}}

\begin{fulllineitems}
\phantomsection\label{\detokenize{api/pymusepipe:pymusepipe.combine.MusePointings.get_pointing_table}}
\pysigstartsignatures
\pysiglinewithargsret{\sphinxbfcode{\sphinxupquote{get\_pointing\_table}}}{\sphinxparam{\DUrole{o,o}{**}\DUrole{n,n}{kwargs}}}{}
\pysigstopsignatures
\sphinxAtStartPar
Create the pointing table from the all pixtables list

\sphinxAtStartPar
Add the pointing\_table attribute and do the selection according to list\_datasets and
list\_pointings

\end{fulllineitems}

\index{get\_qtable() (pymusepipe.combine.MusePointings method)@\spxentry{get\_qtable()}\spxextra{pymusepipe.combine.MusePointings method}}

\begin{fulllineitems}
\phantomsection\label{\detokenize{api/pymusepipe:pymusepipe.combine.MusePointings.get_qtable}}
\pysigstartsignatures
\pysiglinewithargsret{\sphinxbfcode{\sphinxupquote{get\_qtable}}}{\sphinxparam{\DUrole{o,o}{**}\DUrole{n,n}{kwargs}}}{}
\pysigstopsignatures
\sphinxAtStartPar
Create the qtable from the all pixtables list

\end{fulllineitems}

\index{goto\_folder() (pymusepipe.combine.MusePointings method)@\spxentry{goto\_folder()}\spxextra{pymusepipe.combine.MusePointings method}}

\begin{fulllineitems}
\phantomsection\label{\detokenize{api/pymusepipe:pymusepipe.combine.MusePointings.goto_folder}}
\pysigstartsignatures
\pysiglinewithargsret{\sphinxbfcode{\sphinxupquote{goto\_folder}}}{\sphinxparam{\DUrole{n,n}{newpath}}, \sphinxparam{\DUrole{n,n}{addtolog}\DUrole{o,o}{=}\DUrole{default_value}{False}}}{}
\pysigstopsignatures
\sphinxAtStartPar
Changing directory and keeping memory of the old working one


\paragraph{Input}
\label{\detokenize{api/pymusepipe:id47}}\begin{description}
\sphinxlineitem{newpath: str}
\sphinxAtStartPar
Nanme of the folder where to go.

\sphinxlineitem{addtolog: bool, optional}
\sphinxAtStartPar
Add this change of folder to the log file.

\end{description}

\end{fulllineitems}

\index{goto\_origfolder() (pymusepipe.combine.MusePointings method)@\spxentry{goto\_origfolder()}\spxextra{pymusepipe.combine.MusePointings method}}

\begin{fulllineitems}
\phantomsection\label{\detokenize{api/pymusepipe:pymusepipe.combine.MusePointings.goto_origfolder}}
\pysigstartsignatures
\pysiglinewithargsret{\sphinxbfcode{\sphinxupquote{goto\_origfolder}}}{\sphinxparam{\DUrole{n,n}{addtolog}\DUrole{o,o}{=}\DUrole{default_value}{False}}}{}
\pysigstopsignatures
\sphinxAtStartPar
Go back to original folder


\paragraph{Input}
\label{\detokenize{api/pymusepipe:id48}}\begin{description}
\sphinxlineitem{addtolog: bool, optional}
\sphinxAtStartPar
Add this change of folder to the log file.

\end{description}

\end{fulllineitems}

\index{goto\_prevfolder() (pymusepipe.combine.MusePointings method)@\spxentry{goto\_prevfolder()}\spxextra{pymusepipe.combine.MusePointings method}}

\begin{fulllineitems}
\phantomsection\label{\detokenize{api/pymusepipe:pymusepipe.combine.MusePointings.goto_prevfolder}}
\pysigstartsignatures
\pysiglinewithargsret{\sphinxbfcode{\sphinxupquote{goto\_prevfolder}}}{\sphinxparam{\DUrole{n,n}{addtolog}\DUrole{o,o}{=}\DUrole{default_value}{False}}}{}
\pysigstopsignatures
\sphinxAtStartPar
Go back to previous folder


\paragraph{Input}
\label{\detokenize{api/pymusepipe:id49}}\begin{description}
\sphinxlineitem{addtolog: bool, optional}
\sphinxAtStartPar
Add this change of folder to the log file.

\end{description}

\end{fulllineitems}

\index{run\_combine() (pymusepipe.combine.MusePointings method)@\spxentry{run\_combine()}\spxextra{pymusepipe.combine.MusePointings method}}

\begin{fulllineitems}
\phantomsection\label{\detokenize{api/pymusepipe:pymusepipe.combine.MusePointings.run_combine}}
\pysigstartsignatures
\pysiglinewithargsret{\sphinxbfcode{\sphinxupquote{run\_combine}}}{\sphinxparam{\DUrole{n,n}{sof\_filename}\DUrole{o,o}{=}\DUrole{default_value}{\textquotesingle{}pointings\_combine\textquotesingle{}}}, \sphinxparam{\DUrole{n,n}{lambdaminmax}\DUrole{o,o}{=}\DUrole{default_value}{(4000.0, 10000.0)}}, \sphinxparam{\DUrole{n,n}{list\_pointings}\DUrole{o,o}{=}\DUrole{default_value}{None}}, \sphinxparam{\DUrole{n,n}{suffix}\DUrole{o,o}{=}\DUrole{default_value}{\textquotesingle{}\textquotesingle{}}}, \sphinxparam{\DUrole{o,o}{**}\DUrole{n,n}{kwargs}}}{}
\pysigstopsignatures
\sphinxAtStartPar
MUSE Exp\_combine treatment of the reduced pixtables
Will run the esorex muse\_exp\_combine routine
\begin{quote}\begin{description}
\sphinxlineitem{Parameters}\begin{itemize}
\item {} 
\sphinxAtStartPar
\sphinxstyleliteralstrong{\sphinxupquote{sof\_filename}} (\sphinxstyleliteralemphasis{\sphinxupquote{string}}\sphinxstyleliteralemphasis{\sphinxupquote{ (}}\sphinxstyleliteralemphasis{\sphinxupquote{without the file extension}}\sphinxstyleliteralemphasis{\sphinxupquote{)}}) \textendash{} Name of the SOF file which will contain the Bias frames

\item {} 
\sphinxAtStartPar
\sphinxstyleliteralstrong{\sphinxupquote{lambdaminmax}} (\sphinxhref{https://docs.python.org/3.10/library/stdtypes.html\#list}{\sphinxstyleliteralemphasis{\sphinxupquote{list}}}\sphinxstyleliteralemphasis{\sphinxupquote{ of }}\sphinxstyleliteralemphasis{\sphinxupquote{2 floats}}) \textendash{} Minimum and maximum lambda values to consider for the combine

\item {} 
\sphinxAtStartPar
\sphinxstyleliteralstrong{\sphinxupquote{suffix}} (\sphinxhref{https://docs.python.org/3.10/library/stdtypes.html\#str}{\sphinxstyleliteralemphasis{\sphinxupquote{str}}}) \textendash{} Suffix to be used for the output name

\end{itemize}

\end{description}\end{quote}

\end{fulllineitems}

\index{run\_combine\_all\_single\_pointings() (pymusepipe.combine.MusePointings method)@\spxentry{run\_combine\_all\_single\_pointings()}\spxextra{pymusepipe.combine.MusePointings method}}

\begin{fulllineitems}
\phantomsection\label{\detokenize{api/pymusepipe:pymusepipe.combine.MusePointings.run_combine_all_single_pointings}}
\pysigstartsignatures
\pysiglinewithargsret{\sphinxbfcode{\sphinxupquote{run\_combine\_all\_single\_pointings}}}{\sphinxparam{\DUrole{n,n}{add\_suffix}\DUrole{o,o}{=}\DUrole{default_value}{\textquotesingle{}\textquotesingle{}}}, \sphinxparam{\DUrole{n,n}{sof\_filename}\DUrole{o,o}{=}\DUrole{default_value}{\textquotesingle{}pointings\_combine\textquotesingle{}}}, \sphinxparam{\DUrole{n,n}{list\_pointings}\DUrole{o,o}{=}\DUrole{default_value}{None}}, \sphinxparam{\DUrole{o,o}{**}\DUrole{n,n}{kwargs}}}{}
\pysigstopsignatures
\sphinxAtStartPar
Run for all pointings individually, provided in the
list of pointings, by just looping over the pointings.


\paragraph{Input}
\label{\detokenize{api/pymusepipe:id50}}\begin{description}
\sphinxlineitem{list\_pointings: list of int}
\sphinxAtStartPar
By default to None (using the default self.list\_pointings).
Otherwise a list of pointings you wish to conduct
a combine but for each individual pointing.

\sphinxlineitem{add\_suffix: str}
\sphinxAtStartPar
Additional suffix. ‘PXX’ where XX is the pointing number
will be automatically added to that add\_suffix for
each individual pointing.

\sphinxlineitem{sof\_filename: str}
\sphinxAtStartPar
Name (suffix only) of the sof file for this combine.
By default, it is set to ‘pointings\_combine’.

\sphinxlineitem{lambdaminmax: list of 2 floats {[}in Angstroems{]}}
\sphinxAtStartPar
Minimum and maximum lambda values to consider for the combine.
Default is 4000 and 10000 for the lower and upper limits, resp.

\end{description}

\end{fulllineitems}

\index{run\_combine\_single\_pointing() (pymusepipe.combine.MusePointings method)@\spxentry{run\_combine\_single\_pointing()}\spxextra{pymusepipe.combine.MusePointings method}}

\begin{fulllineitems}
\phantomsection\label{\detokenize{api/pymusepipe:pymusepipe.combine.MusePointings.run_combine_single_pointing}}
\pysigstartsignatures
\pysiglinewithargsret{\sphinxbfcode{\sphinxupquote{run\_combine\_single\_pointing}}}{\sphinxparam{\DUrole{n,n}{pointing}}, \sphinxparam{\DUrole{n,n}{add\_suffix}\DUrole{o,o}{=}\DUrole{default_value}{\textquotesingle{}\textquotesingle{}}}, \sphinxparam{\DUrole{n,n}{sof\_filename}\DUrole{o,o}{=}\DUrole{default_value}{\textquotesingle{}pointing\_combine\textquotesingle{}}}, \sphinxparam{\DUrole{o,o}{**}\DUrole{n,n}{kwargs}}}{}
\pysigstopsignatures
\sphinxAtStartPar
Running the combine routine on just one single pointing


\paragraph{Input}
\label{\detokenize{api/pymusepipe:id51}}\begin{description}
\sphinxlineitem{pointing: int}
\sphinxAtStartPar
Pointing number. No default: must be provided.

\sphinxlineitem{add\_suffix: str}
\sphinxAtStartPar
Additional suffix. ‘PXX’ where XX is the pointing number
will be automatically added to that add\_suffix.

\sphinxlineitem{sof\_filename: str}
\sphinxAtStartPar
Name (suffix only) of the sof file for this combine.
By default, it is set to ‘pointings\_combine’.

\sphinxlineitem{lambdaminmax: list of 2 floats {[}in Angstroems{]}}
\sphinxAtStartPar
Minimum and maximum lambda values to consider for the combine.
Default is 4000 and 10000 for the lower and upper limits, resp.

\sphinxlineitem{wcs\_from\_pointing: bool}
\sphinxAtStartPar
True by default, meaning that the WCS of the pointings will be used.
If not there, will ignore it.

\end{description}

\end{fulllineitems}

\index{set\_fullpath\_names() (pymusepipe.combine.MusePointings method)@\spxentry{set\_fullpath\_names()}\spxextra{pymusepipe.combine.MusePointings method}}

\begin{fulllineitems}
\phantomsection\label{\detokenize{api/pymusepipe:pymusepipe.combine.MusePointings.set_fullpath_names}}
\pysigstartsignatures
\pysiglinewithargsret{\sphinxbfcode{\sphinxupquote{set\_fullpath\_names}}}{}{}
\pysigstopsignatures
\sphinxAtStartPar
Create full path names to be used
That includes: root, data, target, but also \_dict\_paths, paths

\end{fulllineitems}


\end{fulllineitems}



\subsubsection{pymusepipe.config\_pipe module}
\label{\detokenize{api/pymusepipe:module-pymusepipe.config_pipe}}\label{\detokenize{api/pymusepipe:pymusepipe-config-pipe-module}}\index{module@\spxentry{module}!pymusepipe.config\_pipe@\spxentry{pymusepipe.config\_pipe}}\index{pymusepipe.config\_pipe@\spxentry{pymusepipe.config\_pipe}!module@\spxentry{module}}
\sphinxAtStartPar
MUSE\sphinxhyphen{}PHANGS configuration module
\index{get\_suffix\_product() (in module pymusepipe.config\_pipe)@\spxentry{get\_suffix\_product()}\spxextra{in module pymusepipe.config\_pipe}}

\begin{fulllineitems}
\phantomsection\label{\detokenize{api/pymusepipe:pymusepipe.config_pipe.get_suffix_product}}
\pysigstartsignatures
\pysiglinewithargsret{\sphinxcode{\sphinxupquote{pymusepipe.config\_pipe.}}\sphinxbfcode{\sphinxupquote{get\_suffix\_product}}}{\sphinxparam{\DUrole{n,n}{expotype}}}{}
\pysigstopsignatures
\end{fulllineitems}



\subsubsection{pymusepipe.create\_sof module}
\label{\detokenize{api/pymusepipe:module-pymusepipe.create_sof}}\label{\detokenize{api/pymusepipe:pymusepipe-create-sof-module}}\index{module@\spxentry{module}!pymusepipe.create\_sof@\spxentry{pymusepipe.create\_sof}}\index{pymusepipe.create\_sof@\spxentry{pymusepipe.create\_sof}!module@\spxentry{module}}
\sphinxAtStartPar
MUSE\sphinxhyphen{}PHANGS creating sof file module
\index{SofDict (class in pymusepipe.create\_sof)@\spxentry{SofDict}\spxextra{class in pymusepipe.create\_sof}}

\begin{fulllineitems}
\phantomsection\label{\detokenize{api/pymusepipe:pymusepipe.create_sof.SofDict}}
\pysigstartsignatures
\pysigline{\sphinxbfcode{\sphinxupquote{class\DUrole{w,w}{  }}}\sphinxcode{\sphinxupquote{pymusepipe.create\_sof.}}\sphinxbfcode{\sphinxupquote{SofDict}}}
\pysigstopsignatures
\sphinxAtStartPar
Bases: \sphinxhref{https://docs.python.org/3.10/library/collections.html\#collections.OrderedDict}{\sphinxcode{\sphinxupquote{OrderedDict}}}

\sphinxAtStartPar
New Dictionary for the SOF writing
Inheriting from ordered Dictionary

\end{fulllineitems}

\index{SofPipe (class in pymusepipe.create\_sof)@\spxentry{SofPipe}\spxextra{class in pymusepipe.create\_sof}}

\begin{fulllineitems}
\phantomsection\label{\detokenize{api/pymusepipe:pymusepipe.create_sof.SofPipe}}
\pysigstartsignatures
\pysigline{\sphinxbfcode{\sphinxupquote{class\DUrole{w,w}{  }}}\sphinxcode{\sphinxupquote{pymusepipe.create\_sof.}}\sphinxbfcode{\sphinxupquote{SofPipe}}}
\pysigstopsignatures
\sphinxAtStartPar
Bases: \sphinxhref{https://docs.python.org/3.10/library/functions.html\#object}{\sphinxcode{\sphinxupquote{object}}}

\sphinxAtStartPar
SofPipe class containing all the SOF writing modules
\index{write\_sof() (pymusepipe.create\_sof.SofPipe method)@\spxentry{write\_sof()}\spxextra{pymusepipe.create\_sof.SofPipe method}}

\begin{fulllineitems}
\phantomsection\label{\detokenize{api/pymusepipe:pymusepipe.create_sof.SofPipe.write_sof}}
\pysigstartsignatures
\pysiglinewithargsret{\sphinxbfcode{\sphinxupquote{write\_sof}}}{\sphinxparam{\DUrole{n,n}{sof\_filename}}, \sphinxparam{\DUrole{n,n}{new}\DUrole{o,o}{=}\DUrole{default_value}{False}}, \sphinxparam{\DUrole{n,n}{verbose}\DUrole{o,o}{=}\DUrole{default_value}{None}}}{}
\pysigstopsignatures
\sphinxAtStartPar
Feeding an sof file with input filenames from a dictionary

\end{fulllineitems}


\end{fulllineitems}



\subsubsection{pymusepipe.cube\_convolve module}
\label{\detokenize{api/pymusepipe:module-pymusepipe.cube_convolve}}\label{\detokenize{api/pymusepipe:pymusepipe-cube-convolve-module}}\index{module@\spxentry{module}!pymusepipe.cube\_convolve@\spxentry{pymusepipe.cube\_convolve}}\index{pymusepipe.cube\_convolve@\spxentry{pymusepipe.cube\_convolve}!module@\spxentry{module}}
\sphinxAtStartPar
MUSE\sphinxhyphen{}PHANGS convolve module
\index{convolution\_kernel() (in module pymusepipe.cube\_convolve)@\spxentry{convolution\_kernel()}\spxextra{in module pymusepipe.cube\_convolve}}

\begin{fulllineitems}
\phantomsection\label{\detokenize{api/pymusepipe:pymusepipe.cube_convolve.convolution_kernel}}
\pysigstartsignatures
\pysiglinewithargsret{\sphinxcode{\sphinxupquote{pymusepipe.cube\_convolve.}}\sphinxbfcode{\sphinxupquote{convolution\_kernel}}}{\sphinxparam{\DUrole{n,n}{input\_psf}}, \sphinxparam{\DUrole{n,n}{target\_psf}}, \sphinxparam{\DUrole{n,n}{scale}\DUrole{o,o}{=}\DUrole{default_value}{0.2}}}{}
\pysigstopsignatures
\sphinxAtStartPar
Create the 3D convolution kernel starting from a 3D model of the original
PSF and a 2D model of the target PSF using pypher.
\begin{description}
\sphinxlineitem{Parameters}
\sphinxAtStartPar
input\_psf (np.ndarray): 3D array with the model of the original PSF
target\_psf (np.ndarray): 2D array with a model of the target PSF
scale (float): spatial scale of both PSF in arcsec/pix

\sphinxlineitem{Returns}\begin{description}
\sphinxlineitem{conv\_kernel (np.ndarray): 3D array with a convolution kernel}
\sphinxAtStartPar
that varies as a function of wavelength.

\end{description}

\end{description}

\end{fulllineitems}

\index{convolution\_kernel\_gaussian() (in module pymusepipe.cube\_convolve)@\spxentry{convolution\_kernel\_gaussian()}\spxextra{in module pymusepipe.cube\_convolve}}

\begin{fulllineitems}
\phantomsection\label{\detokenize{api/pymusepipe:pymusepipe.cube_convolve.convolution_kernel_gaussian}}
\pysigstartsignatures
\pysiglinewithargsret{\sphinxcode{\sphinxupquote{pymusepipe.cube\_convolve.}}\sphinxbfcode{\sphinxupquote{convolution\_kernel\_gaussian}}}{\sphinxparam{\DUrole{n,n}{fwhm\_wave}}, \sphinxparam{\DUrole{n,n}{target\_fwhm}}, \sphinxparam{\DUrole{n,n}{target\_psf}}, \sphinxparam{\DUrole{n,n}{scale}\DUrole{o,o}{=}\DUrole{default_value}{0.2}}}{}
\pysigstopsignatures
\sphinxAtStartPar
Create the 3D convolution kernel starting from a 3D model of the original
PSF and a 2D model of the target PSF using both gaussian functions.
\begin{quote}\begin{description}
\sphinxlineitem{Parameters}\begin{itemize}
\item {} 
\sphinxAtStartPar
\sphinxstyleliteralstrong{\sphinxupquote{fwhm\_wave}} (\sphinxstyleliteralemphasis{\sphinxupquote{array}}) \textendash{} FWHM of the original PSF as a function of
wavelength

\item {} 
\sphinxAtStartPar
\sphinxstyleliteralstrong{\sphinxupquote{target\_fwhm}} (\sphinxhref{https://docs.python.org/3.10/library/functions.html\#float}{\sphinxstyleliteralemphasis{\sphinxupquote{float}}}) \textendash{} fwhm of the target PSF

\item {} 
\sphinxAtStartPar
\sphinxstyleliteralstrong{\sphinxupquote{target\_psf}} (\sphinxstyleliteralemphasis{\sphinxupquote{np.ndarray}}) \textendash{} target psf2d

\item {} 
\sphinxAtStartPar
\sphinxstyleliteralstrong{\sphinxupquote{scale}} (\sphinxhref{https://docs.python.org/3.10/library/functions.html\#float}{\sphinxstyleliteralemphasis{\sphinxupquote{float}}}) \textendash{} spatial scale of both PSF in arcsec/pix

\end{itemize}

\sphinxlineitem{Returns}
\sphinxAtStartPar
\begin{description}
\sphinxlineitem{np.ndarray}
\sphinxAtStartPar
3D array with a convolution kernel that varies as a function of
wavelength.

\end{description}


\sphinxlineitem{Return type}
\sphinxAtStartPar
conv\_kernel

\end{description}\end{quote}

\end{fulllineitems}

\index{cube\_convolve() (in module pymusepipe.cube\_convolve)@\spxentry{cube\_convolve()}\spxextra{in module pymusepipe.cube\_convolve}}

\begin{fulllineitems}
\phantomsection\label{\detokenize{api/pymusepipe:pymusepipe.cube_convolve.cube_convolve}}
\pysigstartsignatures
\pysiglinewithargsret{\sphinxcode{\sphinxupquote{pymusepipe.cube\_convolve.}}\sphinxbfcode{\sphinxupquote{cube\_convolve}}}{\sphinxparam{\DUrole{n,n}{data}}, \sphinxparam{\DUrole{n,n}{kernel}}, \sphinxparam{\DUrole{n,n}{variance}\DUrole{o,o}{=}\DUrole{default_value}{None}}, \sphinxparam{\DUrole{n,n}{fft}\DUrole{o,o}{=}\DUrole{default_value}{True}}, \sphinxparam{\DUrole{n,n}{fill\_value}\DUrole{o,o}{=}\DUrole{default_value}{nan}}}{}
\pysigstopsignatures
\sphinxAtStartPar
Convolve a 3D datacube
\begin{quote}\begin{description}
\sphinxlineitem{Parameters}\begin{itemize}
\item {} 
\sphinxAtStartPar
\sphinxstyleliteralstrong{\sphinxupquote{datacube}} \textendash{} 

\item {} 
\sphinxAtStartPar
\sphinxstyleliteralstrong{\sphinxupquote{kernel}} \textendash{} 

\end{itemize}

\sphinxlineitem{Returns}
\sphinxAtStartPar
the convolved 3D data and its variance

\end{description}\end{quote}

\end{fulllineitems}

\index{cube\_kernel() (in module pymusepipe.cube\_convolve)@\spxentry{cube\_kernel()}\spxextra{in module pymusepipe.cube\_convolve}}

\begin{fulllineitems}
\phantomsection\label{\detokenize{api/pymusepipe:pymusepipe.cube_convolve.cube_kernel}}
\pysigstartsignatures
\pysiglinewithargsret{\sphinxcode{\sphinxupquote{pymusepipe.cube\_convolve.}}\sphinxbfcode{\sphinxupquote{cube\_kernel}}}{\sphinxparam{\DUrole{n,n}{shape}}, \sphinxparam{\DUrole{n,n}{wave}}, \sphinxparam{\DUrole{n,n}{input\_fwhm}}, \sphinxparam{\DUrole{n,n}{target\_fwhm}}, \sphinxparam{\DUrole{n,n}{input\_function}}, \sphinxparam{\DUrole{n,n}{target\_function}}, \sphinxparam{\DUrole{n,n}{lambda0}\DUrole{o,o}{=}\DUrole{default_value}{6483.58}}, \sphinxparam{\DUrole{n,n}{input\_nmoffat}\DUrole{o,o}{=}\DUrole{default_value}{None}}, \sphinxparam{\DUrole{n,n}{target\_nmoffat}\DUrole{o,o}{=}\DUrole{default_value}{None}}, \sphinxparam{\DUrole{n,n}{b}\DUrole{o,o}{=}\DUrole{default_value}{\sphinxhyphen{}3e\sphinxhyphen{}05}}, \sphinxparam{\DUrole{n,n}{scale}\DUrole{o,o}{=}\DUrole{default_value}{0.2}}, \sphinxparam{\DUrole{n,n}{compute\_kernel}\DUrole{o,o}{=}\DUrole{default_value}{\textquotesingle{}pypher\textquotesingle{}}}}{}
\pysigstopsignatures
\sphinxAtStartPar
Main function to create the convolution kernel for the datacube
\begin{quote}\begin{description}
\sphinxlineitem{Parameters}\begin{itemize}
\item {} 
\sphinxAtStartPar
\sphinxstyleliteralstrong{\sphinxupquote{shape}} (\sphinxstyleliteralemphasis{\sphinxupquote{array}}) \textendash{} the shape of the datacube that is going to be convolved.
It must be in the form (z, y, x).

\item {} 
\sphinxAtStartPar
\sphinxstyleliteralstrong{\sphinxupquote{wave}} (\sphinxstyleliteralemphasis{\sphinxupquote{float array}}) \textendash{} wavelengths for the datacube

\item {} 
\sphinxAtStartPar
\sphinxstyleliteralstrong{\sphinxupquote{target\_fwhm}} (\sphinxhref{https://docs.python.org/3.10/library/functions.html\#float}{\sphinxstyleliteralemphasis{\sphinxupquote{float}}}) \textendash{} fwhm of the target PSF.

\item {} 
\sphinxAtStartPar
\sphinxstyleliteralstrong{\sphinxupquote{input\_fwhm}} (\sphinxhref{https://docs.python.org/3.10/library/functions.html\#float}{\sphinxstyleliteralemphasis{\sphinxupquote{float}}}) \textendash{} fwhm of the original PSF at the reference
wavelength lambda0

\item {} 
\sphinxAtStartPar
\sphinxstyleliteralstrong{\sphinxupquote{input\_function}} (\sphinxhref{https://docs.python.org/3.10/library/stdtypes.html\#str}{\sphinxstyleliteralemphasis{\sphinxupquote{str}}}) \textendash{} function to be used to describe the input PSF

\item {} 
\sphinxAtStartPar
\sphinxstyleliteralstrong{\sphinxupquote{target\_function}} (\sphinxhref{https://docs.python.org/3.10/library/stdtypes.html\#str}{\sphinxstyleliteralemphasis{\sphinxupquote{str}}}) \textendash{} function to be used to describe the target PSF

\item {} 
\sphinxAtStartPar
\sphinxstyleliteralstrong{\sphinxupquote{lambda0}} \textendash{} float, optional
the wavelength at which the original\_fwhm has been measured.
Default: 6483.58 (central wavelenght of WFI\_BB filter)

\item {} 
\sphinxAtStartPar
\sphinxstyleliteralstrong{\sphinxupquote{input\_nmoffat}} (\sphinxhref{https://docs.python.org/3.10/library/functions.html\#float}{\sphinxstyleliteralemphasis{\sphinxupquote{float}}}) \textendash{} power index of the original PSF if Moffat {[}None{]}

\item {} 
\sphinxAtStartPar
\sphinxstyleliteralstrong{\sphinxupquote{target\_nmoffat}} (\sphinxhref{https://docs.python.org/3.10/library/functions.html\#float}{\sphinxstyleliteralemphasis{\sphinxupquote{float}}}) \textendash{} power index for the target PSF if Moffat {[}None{]}

\item {} 
\sphinxAtStartPar
\sphinxstyleliteralstrong{\sphinxupquote{b}} (\sphinxhref{https://docs.python.org/3.10/library/functions.html\#float}{\sphinxstyleliteralemphasis{\sphinxupquote{float}}}) \textendash{} steepness of the fwhm vs wavelength relation. Default: \sphinxhyphen{}3e\sphinxhyphen{}5

\item {} 
\sphinxAtStartPar
\sphinxstyleliteralstrong{\sphinxupquote{step}} (\sphinxhref{https://docs.python.org/3.10/library/functions.html\#float}{\sphinxstyleliteralemphasis{\sphinxupquote{float}}}) \textendash{} wavelength dispersion in AA/px

\item {} 
\sphinxAtStartPar
\sphinxstyleliteralstrong{\sphinxupquote{scale}} (\sphinxhref{https://docs.python.org/3.10/library/functions.html\#float}{\sphinxstyleliteralemphasis{\sphinxupquote{float}}}) \textendash{} spatial pixel scale of the PSFs in arcsec/pix

\item {} 
\sphinxAtStartPar
\sphinxstyleliteralstrong{\sphinxupquote{compute\_kernel}} (\sphinxhref{https://docs.python.org/3.10/library/stdtypes.html\#str}{\sphinxstyleliteralemphasis{\sphinxupquote{str}}}) \textendash{} method to compute the convolution kernel.
It can be ‘pypher’ or ‘gaussian’

\end{itemize}

\sphinxlineitem{Returns}
\sphinxAtStartPar
\begin{description}
\sphinxlineitem{np.ndarray}
\sphinxAtStartPar
3D array to be used in the convolution

\end{description}


\sphinxlineitem{Return type}
\sphinxAtStartPar
Kernel

\end{description}\end{quote}

\end{fulllineitems}

\index{gaussian\_kernel() (in module pymusepipe.cube\_convolve)@\spxentry{gaussian\_kernel()}\spxextra{in module pymusepipe.cube\_convolve}}

\begin{fulllineitems}
\phantomsection\label{\detokenize{api/pymusepipe:pymusepipe.cube_convolve.gaussian_kernel}}
\pysigstartsignatures
\pysiglinewithargsret{\sphinxcode{\sphinxupquote{pymusepipe.cube\_convolve.}}\sphinxbfcode{\sphinxupquote{gaussian\_kernel}}}{\sphinxparam{\DUrole{n,n}{fwhm}}, \sphinxparam{\DUrole{n,n}{size}}, \sphinxparam{\DUrole{n,n}{scale}\DUrole{o,o}{=}\DUrole{default_value}{0.2}}, \sphinxparam{\DUrole{o,o}{**}\DUrole{n,n}{kwargs}}}{}
\pysigstopsignatures
\sphinxAtStartPar
Gaussian kernel.
Input:
\begin{quote}

\sphinxAtStartPar
fwhm (float): fwhm of the Gaussian kernel, in arcsec.
size (int, ndarray): size of the requested kernel along each axis.
\begin{quote}

\sphinxAtStartPar
If {\color{red}\bfseries{}\textasciigrave{}\textasciigrave{}}size’’ is a scalar number the final kernel will be a square of
side {\color{red}\bfseries{}\textasciigrave{}\textasciigrave{}}size’’. If {\color{red}\bfseries{}\textasciigrave{}\textasciigrave{}}size’’ has two element they must be in
(y\_size, x\_size) order. In each case size must be an integer number
of pixels.
\end{quote}

\sphinxAtStartPar
scale (float): pixel scale of the image
{\color{red}\bfseries{}**}kwargs: is there to absorb any additional parameter which could be
\begin{quote}

\sphinxAtStartPar
provided (but won’t be used).
\end{quote}
\end{quote}

\end{fulllineitems}

\index{moffat\_kernel() (in module pymusepipe.cube\_convolve)@\spxentry{moffat\_kernel()}\spxextra{in module pymusepipe.cube\_convolve}}

\begin{fulllineitems}
\phantomsection\label{\detokenize{api/pymusepipe:pymusepipe.cube_convolve.moffat_kernel}}
\pysigstartsignatures
\pysiglinewithargsret{\sphinxcode{\sphinxupquote{pymusepipe.cube\_convolve.}}\sphinxbfcode{\sphinxupquote{moffat\_kernel}}}{\sphinxparam{\DUrole{n,n}{fwhm}}, \sphinxparam{\DUrole{n,n}{size}}, \sphinxparam{\DUrole{n,n}{n}\DUrole{o,o}{=}\DUrole{default_value}{1.0}}, \sphinxparam{\DUrole{n,n}{scale}\DUrole{o,o}{=}\DUrole{default_value}{0.2}}}{}
\pysigstopsignatures
\sphinxAtStartPar
Moffat kernel. Returns a Moffat function array according to given
input parameters. Using astropy Moffat2DKernel.
\begin{quote}\begin{description}
\sphinxlineitem{Parameters}\begin{itemize}
\item {} 
\sphinxAtStartPar
\sphinxstyleliteralstrong{\sphinxupquote{fwhm}} (\sphinxhref{https://docs.python.org/3.10/library/functions.html\#float}{\sphinxstyleliteralemphasis{\sphinxupquote{float}}}) \textendash{} fwhm of the Moffat kernel, in arcsec.

\item {} 
\sphinxAtStartPar
\sphinxstyleliteralstrong{\sphinxupquote{n}} (\sphinxhref{https://docs.python.org/3.10/library/functions.html\#float}{\sphinxstyleliteralemphasis{\sphinxupquote{float}}}) \textendash{} power index of the Moffat

\item {} 
\sphinxAtStartPar
\sphinxstyleliteralstrong{\sphinxupquote{size}} (\sphinxstyleliteralemphasis{\sphinxupquote{int numpy array}}) \textendash{} size of the requested kernel along each axis.
If {\color{red}\bfseries{}\textasciigrave{}\textasciigrave{}}size’’ is a scalar number the final kernel will be a square of
side {\color{red}\bfseries{}\textasciigrave{}\textasciigrave{}}size’’. If {\color{red}\bfseries{}\textasciigrave{}\textasciigrave{}}size’’ has two element they must be in
(y\_size, x\_size) order. In each case size must be an integer
number of pixels.

\item {} 
\sphinxAtStartPar
\sphinxstyleliteralstrong{\sphinxupquote{scale}} (\sphinxhref{https://docs.python.org/3.10/library/functions.html\#float}{\sphinxstyleliteralemphasis{\sphinxupquote{float}}}) \textendash{} pixel scale of the image {[}optional{]}

\end{itemize}

\end{description}\end{quote}

\end{fulllineitems}

\index{psf2d() (in module pymusepipe.cube\_convolve)@\spxentry{psf2d()}\spxextra{in module pymusepipe.cube\_convolve}}

\begin{fulllineitems}
\phantomsection\label{\detokenize{api/pymusepipe:pymusepipe.cube_convolve.psf2d}}
\pysigstartsignatures
\pysiglinewithargsret{\sphinxcode{\sphinxupquote{pymusepipe.cube\_convolve.}}\sphinxbfcode{\sphinxupquote{psf2d}}}{\sphinxparam{\DUrole{n,n}{size}}, \sphinxparam{\DUrole{n,n}{fwhm}}, \sphinxparam{\DUrole{n,n}{function}\DUrole{o,o}{=}\DUrole{default_value}{\textquotesingle{}gaussian\textquotesingle{}}}, \sphinxparam{\DUrole{n,n}{nmoffat}\DUrole{o,o}{=}\DUrole{default_value}{None}}, \sphinxparam{\DUrole{n,n}{scale}\DUrole{o,o}{=}\DUrole{default_value}{0.2}}}{}
\pysigstopsignatures
\sphinxAtStartPar
Create a model of the target PSF of the convolution. The target PSF does
not vary as a function of wavelenght, therefore the output is a 2D array.
\begin{description}
\sphinxlineitem{Parameters}\begin{description}
\sphinxlineitem{size: int, array\sphinxhyphen{}like}
\sphinxAtStartPar
the size of the final array. If {\color{red}\bfseries{}\textasciigrave{}\textasciigrave{}}size’’ is a scalar number the
kernel will be a square of side {\color{red}\bfseries{}\textasciigrave{}\textasciigrave{}}size’’. If {\color{red}\bfseries{}\textasciigrave{}\textasciigrave{}}size’’ has two
elements they must be in (y\_size, x\_size) order.

\sphinxlineitem{fwhm: float}
\sphinxAtStartPar
the FWHM of the psf

\sphinxlineitem{function: str, optional}
\sphinxAtStartPar
the function to model the target PSF. Only ‘gaussian’ or ‘moffat’
are accepted. Default: ‘gaussian’

\end{description}

\sphinxAtStartPar
nmoffat (float): Moffat power index. It must be defined if
function = ‘moffat’.
\begin{quote}

\sphinxAtStartPar
Default: None
\end{quote}
\begin{description}
\sphinxlineitem{scale: float, optional}
\sphinxAtStartPar
the spatial scale of the final kernel

\end{description}

\sphinxlineitem{Returns}\begin{description}
\sphinxlineitem{target: np.ndarray}
\sphinxAtStartPar
a 2D array with the model of the target PSF.

\end{description}

\end{description}

\end{fulllineitems}

\index{psf3d() (in module pymusepipe.cube\_convolve)@\spxentry{psf3d()}\spxextra{in module pymusepipe.cube\_convolve}}

\begin{fulllineitems}
\phantomsection\label{\detokenize{api/pymusepipe:pymusepipe.cube_convolve.psf3d}}
\pysigstartsignatures
\pysiglinewithargsret{\sphinxcode{\sphinxupquote{pymusepipe.cube\_convolve.}}\sphinxbfcode{\sphinxupquote{psf3d}}}{\sphinxparam{\DUrole{n,n}{wave}}, \sphinxparam{\DUrole{n,n}{size}}, \sphinxparam{\DUrole{n,n}{fwhm0}}, \sphinxparam{\DUrole{n,n}{lambda0}\DUrole{o,o}{=}\DUrole{default_value}{6483.58}}, \sphinxparam{\DUrole{n,n}{b}\DUrole{o,o}{=}\DUrole{default_value}{\sphinxhyphen{}3e\sphinxhyphen{}05}}, \sphinxparam{\DUrole{n,n}{scale}\DUrole{o,o}{=}\DUrole{default_value}{0.2}}, \sphinxparam{\DUrole{n,n}{nmoffat}\DUrole{o,o}{=}\DUrole{default_value}{None}}, \sphinxparam{\DUrole{n,n}{function}\DUrole{o,o}{=}\DUrole{default_value}{\textquotesingle{}moffat\textquotesingle{}}}}{}
\pysigstopsignatures
\sphinxAtStartPar
Function to create the cube with the  lambda dependent PSF, following
a given slope and nominal wavelength.
\begin{description}
\sphinxlineitem{Parameters}\begin{description}
\sphinxlineitem{wave: np.ndarray}
\sphinxAtStartPar
array with the wavelength axis of the datacube

\sphinxlineitem{size: int, array\sphinxhyphen{}like}
\sphinxAtStartPar
the size of the 2D PSF.  If {\color{red}\bfseries{}\textasciigrave{}\textasciigrave{}}size’’ is a scalar number the 2D PSF
kernel will be a square of side {\color{red}\bfseries{}\textasciigrave{}\textasciigrave{}}size’’. If {\color{red}\bfseries{}\textasciigrave{}\textasciigrave{}}size’’ has two element
they must be in (y\_size, x\_size) order.

\sphinxlineitem{fwhm0: float}
\sphinxAtStartPar
the fwhm at the reference wavelength in arcseconds.

\sphinxlineitem{n: float}
\sphinxAtStartPar
Power index of the Moffat profile. It is usually 2.8 for NOAO cubes
and 2.3 for AO cubes.

\sphinxlineitem{lambda0: float}
\sphinxAtStartPar
reference wavelength at which fwhm0 is measured. Default: 6483.58.
(It’s the average wavelength for the WFI\_BB filter)

\sphinxlineitem{b: float, optional}
\sphinxAtStartPar
the steepness of the relation between wavelength and FWHM.
Default: \sphinxhyphen{}3e\sphinxhyphen{}5 (arcsec/A) (From MUSE team)

\sphinxlineitem{scale: float, optional}
\sphinxAtStartPar
spatial scale of the new datacube in arcsec. Default: 0.2 (MUSE
spatial resolution).

\end{description}

\sphinxAtStartPar
function (str): ‘moffat’ or ‘gaussian’

\sphinxlineitem{Returns}\begin{description}
\sphinxlineitem{psf\_cube: np.array}
\sphinxAtStartPar
Datacube containing MUSE PSF as a function of wavelength.

\end{description}

\end{description}

\end{fulllineitems}

\index{pypher\_script() (in module pymusepipe.cube\_convolve)@\spxentry{pypher\_script()}\spxextra{in module pymusepipe.cube\_convolve}}

\begin{fulllineitems}
\phantomsection\label{\detokenize{api/pymusepipe:pymusepipe.cube_convolve.pypher_script}}
\pysigstartsignatures
\pysiglinewithargsret{\sphinxcode{\sphinxupquote{pymusepipe.cube\_convolve.}}\sphinxbfcode{\sphinxupquote{pypher\_script}}}{\sphinxparam{\DUrole{n,n}{psf\_source}}, \sphinxparam{\DUrole{n,n}{psf\_target}}, \sphinxparam{\DUrole{n,n}{pixscale\_source}\DUrole{o,o}{=}\DUrole{default_value}{0.2}}, \sphinxparam{\DUrole{n,n}{pixscale\_target}\DUrole{o,o}{=}\DUrole{default_value}{0.2}}, \sphinxparam{\DUrole{n,n}{angle\_source}\DUrole{o,o}{=}\DUrole{default_value}{0.0}}, \sphinxparam{\DUrole{n,n}{angle\_target}\DUrole{o,o}{=}\DUrole{default_value}{0.0}}, \sphinxparam{\DUrole{n,n}{reg\_fact}\DUrole{o,o}{=}\DUrole{default_value}{0.0001}}, \sphinxparam{\DUrole{n,n}{verbose}\DUrole{o,o}{=}\DUrole{default_value}{False}}}{}
\pysigstopsignatures
\sphinxAtStartPar
calculate the convolution kernel to move from one PSF to a target one.
This is an adaptation of the main pypher script that it is meant to be used
from the terminal.
\begin{quote}\begin{description}
\sphinxlineitem{Parameters}\begin{itemize}
\item {} 
\sphinxAtStartPar
\sphinxstyleliteralstrong{\sphinxupquote{psf\_source}} (\sphinxstyleliteralemphasis{\sphinxupquote{ndarray}}) \textendash{} 2D PSF of the source image.

\item {} 
\sphinxAtStartPar
\sphinxstyleliteralstrong{\sphinxupquote{psf\_target}} (\sphinxstyleliteralemphasis{\sphinxupquote{ndarray}}) \textendash{} target 2D PSF

\item {} 
\sphinxAtStartPar
\sphinxstyleliteralstrong{\sphinxupquote{pixscale\_source}} (\sphinxhref{https://docs.python.org/3.10/library/functions.html\#float}{\sphinxstyleliteralemphasis{\sphinxupquote{float}}}) \textendash{} pixel scale of the source PSF {[}0.2{]}

\item {} 
\sphinxAtStartPar
\sphinxstyleliteralstrong{\sphinxupquote{pixscale\_target}} (\sphinxhref{https://docs.python.org/3.10/library/functions.html\#float}{\sphinxstyleliteralemphasis{\sphinxupquote{float}}}) \textendash{} pixel scale of the target PSF {[}0.2{]}

\item {} 
\sphinxAtStartPar
\sphinxstyleliteralstrong{\sphinxupquote{angle\_source}} (\sphinxhref{https://docs.python.org/3.10/library/functions.html\#float}{\sphinxstyleliteralemphasis{\sphinxupquote{float}}}) \textendash{} position angle of the source PSF. {[}0{]}

\item {} 
\sphinxAtStartPar
\sphinxstyleliteralstrong{\sphinxupquote{angle\_target}} (\sphinxhref{https://docs.python.org/3.10/library/functions.html\#float}{\sphinxstyleliteralemphasis{\sphinxupquote{float}}}) \textendash{} position angle of the target PSF. {[}0{]}

\item {} 
\sphinxAtStartPar
\sphinxstyleliteralstrong{\sphinxupquote{reg\_fact}} (\sphinxhref{https://docs.python.org/3.10/library/functions.html\#float}{\sphinxstyleliteralemphasis{\sphinxupquote{float}}}) \textendash{} Regularisation parameter for the Wiener filter {[}1.e\sphinxhyphen{}4{]}

\item {} 
\sphinxAtStartPar
\sphinxstyleliteralstrong{\sphinxupquote{verbose}} (\sphinxhref{https://docs.python.org/3.10/library/functions.html\#bool}{\sphinxstyleliteralemphasis{\sphinxupquote{bool}}}) \textendash{} If True it prints more info on screen {[}False{]}

\end{itemize}

\sphinxlineitem{Returns}
\sphinxAtStartPar
\begin{description}
\sphinxlineitem{a 2D kernel that convolved with the source PSF}
\sphinxAtStartPar
returns the target PSF

\end{description}


\sphinxlineitem{Return type}
\sphinxAtStartPar
kernel

\end{description}\end{quote}

\end{fulllineitems}



\subsubsection{pymusepipe.emission\_lines module}
\label{\detokenize{api/pymusepipe:module-pymusepipe.emission_lines}}\label{\detokenize{api/pymusepipe:pymusepipe-emission-lines-module}}\index{module@\spxentry{module}!pymusepipe.emission\_lines@\spxentry{pymusepipe.emission\_lines}}\index{pymusepipe.emission\_lines@\spxentry{pymusepipe.emission\_lines}!module@\spxentry{module}}
\sphinxAtStartPar
Utility files and functions for wavelengths
\index{doppler\_shift() (in module pymusepipe.emission\_lines)@\spxentry{doppler\_shift()}\spxextra{in module pymusepipe.emission\_lines}}

\begin{fulllineitems}
\phantomsection\label{\detokenize{api/pymusepipe:pymusepipe.emission_lines.doppler_shift}}
\pysigstartsignatures
\pysiglinewithargsret{\sphinxcode{\sphinxupquote{pymusepipe.emission\_lines.}}\sphinxbfcode{\sphinxupquote{doppler\_shift}}}{\sphinxparam{\DUrole{n,n}{wavelength}}, \sphinxparam{\DUrole{n,n}{velocity}\DUrole{o,o}{=}\DUrole{default_value}{0.0}}}{}
\pysigstopsignatures
\sphinxAtStartPar
Return the redshifted wavelength

\end{fulllineitems}

\index{get\_emissionline\_band() (in module pymusepipe.emission\_lines)@\spxentry{get\_emissionline\_band()}\spxextra{in module pymusepipe.emission\_lines}}

\begin{fulllineitems}
\phantomsection\label{\detokenize{api/pymusepipe:pymusepipe.emission_lines.get_emissionline_band}}
\pysigstartsignatures
\pysiglinewithargsret{\sphinxcode{\sphinxupquote{pymusepipe.emission\_lines.}}\sphinxbfcode{\sphinxupquote{get\_emissionline\_band}}}{\sphinxparam{\DUrole{n,n}{line}\DUrole{o,o}{=}\DUrole{default_value}{\textquotesingle{}Ha\textquotesingle{}}}, \sphinxparam{\DUrole{n,n}{velocity}\DUrole{o,o}{=}\DUrole{default_value}{0.0}}, \sphinxparam{\DUrole{n,n}{redshift}\DUrole{o,o}{=}\DUrole{default_value}{None}}, \sphinxparam{\DUrole{n,n}{medium}\DUrole{o,o}{=}\DUrole{default_value}{\textquotesingle{}air\textquotesingle{}}}, \sphinxparam{\DUrole{n,n}{lambda\_window}\DUrole{o,o}{=}\DUrole{default_value}{10.0}}}{}
\pysigstopsignatures
\sphinxAtStartPar
Get the wavelengths of an emission line, including a correction
for the redshift (or velocity) and a lambda\_window around that line (in Angstroems)
\begin{quote}\begin{description}
\sphinxlineitem{Parameters}\begin{itemize}
\item {} 
\sphinxAtStartPar
\sphinxstyleliteralstrong{\sphinxupquote{line}} (\sphinxstyleliteralemphasis{\sphinxupquote{name}}\sphinxstyleliteralemphasis{\sphinxupquote{ of }}\sphinxstyleliteralemphasis{\sphinxupquote{the line}}\sphinxstyleliteralemphasis{\sphinxupquote{ (}}\sphinxstyleliteralemphasis{\sphinxupquote{string}}\sphinxstyleliteralemphasis{\sphinxupquote{)}}\sphinxstyleliteralemphasis{\sphinxupquote{. Default is \textquotesingle{}Ha\textquotesingle{}}}) \textendash{} 

\item {} 
\sphinxAtStartPar
\sphinxstyleliteralstrong{\sphinxupquote{velocity}} (\sphinxstyleliteralemphasis{\sphinxupquote{shift in velocity}}\sphinxstyleliteralemphasis{\sphinxupquote{ (}}\sphinxstyleliteralemphasis{\sphinxupquote{km/s}}\sphinxstyleliteralemphasis{\sphinxupquote{)}}) \textendash{} 

\item {} 
\sphinxAtStartPar
\sphinxstyleliteralstrong{\sphinxupquote{medium}} (\sphinxstyleliteralemphasis{\sphinxupquote{\textquotesingle{}air\textquotesingle{}}}\sphinxstyleliteralemphasis{\sphinxupquote{ or }}\sphinxstyleliteralemphasis{\sphinxupquote{\textquotesingle{}vacuum\textquotesingle{}}}) \textendash{} 

\item {} 
\sphinxAtStartPar
\sphinxstyleliteralstrong{\sphinxupquote{lambda\_window}} (\sphinxstyleliteralemphasis{\sphinxupquote{lambda\_window in Angstroem}}) \textendash{} 

\end{itemize}

\end{description}\end{quote}

\end{fulllineitems}

\index{get\_emissionline\_wavelength() (in module pymusepipe.emission\_lines)@\spxentry{get\_emissionline\_wavelength()}\spxextra{in module pymusepipe.emission\_lines}}

\begin{fulllineitems}
\phantomsection\label{\detokenize{api/pymusepipe:pymusepipe.emission_lines.get_emissionline_wavelength}}
\pysigstartsignatures
\pysiglinewithargsret{\sphinxcode{\sphinxupquote{pymusepipe.emission\_lines.}}\sphinxbfcode{\sphinxupquote{get\_emissionline\_wavelength}}}{\sphinxparam{\DUrole{n,n}{line}\DUrole{o,o}{=}\DUrole{default_value}{\textquotesingle{}Ha\textquotesingle{}}}, \sphinxparam{\DUrole{n,n}{velocity}\DUrole{o,o}{=}\DUrole{default_value}{0.0}}, \sphinxparam{\DUrole{n,n}{redshift}\DUrole{o,o}{=}\DUrole{default_value}{None}}, \sphinxparam{\DUrole{n,n}{medium}\DUrole{o,o}{=}\DUrole{default_value}{\textquotesingle{}air\textquotesingle{}}}}{}
\pysigstopsignatures
\sphinxAtStartPar
Get the wavelength of an emission line, including a correction
for the redshift (or velocity)

\end{fulllineitems}

\index{print\_emission\_lines() (in module pymusepipe.emission\_lines)@\spxentry{print\_emission\_lines()}\spxextra{in module pymusepipe.emission\_lines}}

\begin{fulllineitems}
\phantomsection\label{\detokenize{api/pymusepipe:pymusepipe.emission_lines.print_emission_lines}}
\pysigstartsignatures
\pysiglinewithargsret{\sphinxcode{\sphinxupquote{pymusepipe.emission\_lines.}}\sphinxbfcode{\sphinxupquote{print\_emission\_lines}}}{}{}
\pysigstopsignatures
\sphinxAtStartPar
Printing the names of the various emission lines

\end{fulllineitems}



\subsubsection{pymusepipe.graph\_pipe module}
\label{\detokenize{api/pymusepipe:module-pymusepipe.graph_pipe}}\label{\detokenize{api/pymusepipe:pymusepipe-graph-pipe-module}}\index{module@\spxentry{module}!pymusepipe.graph\_pipe@\spxentry{pymusepipe.graph\_pipe}}\index{pymusepipe.graph\_pipe@\spxentry{pymusepipe.graph\_pipe}!module@\spxentry{module}}
\sphinxAtStartPar
MUSE\sphinxhyphen{}PHANGS plotting routines
\index{GraphMuse (class in pymusepipe.graph\_pipe)@\spxentry{GraphMuse}\spxextra{class in pymusepipe.graph\_pipe}}

\begin{fulllineitems}
\phantomsection\label{\detokenize{api/pymusepipe:pymusepipe.graph_pipe.GraphMuse}}
\pysigstartsignatures
\pysiglinewithargsret{\sphinxbfcode{\sphinxupquote{class\DUrole{w,w}{  }}}\sphinxcode{\sphinxupquote{pymusepipe.graph\_pipe.}}\sphinxbfcode{\sphinxupquote{GraphMuse}}}{\sphinxparam{\DUrole{n,n}{pdf\_name}\DUrole{o,o}{=}\DUrole{default_value}{\textquotesingle{}drs\_check.pdf\textquotesingle{}}}, \sphinxparam{\DUrole{n,n}{figsize}\DUrole{o,o}{=}\DUrole{default_value}{(10, 14)}}, \sphinxparam{\DUrole{n,n}{rect\_layout}\DUrole{o,o}{=}\DUrole{default_value}{{[}0, 0.03, 1, 0.95{]}}}, \sphinxparam{\DUrole{n,n}{verbose}\DUrole{o,o}{=}\DUrole{default_value}{True}}}{}
\pysigstopsignatures
\sphinxAtStartPar
Bases: \sphinxhref{https://docs.python.org/3.10/library/functions.html\#object}{\sphinxcode{\sphinxupquote{object}}}

\sphinxAtStartPar
Graphic output to check MUSE data reduction products
\index{close() (pymusepipe.graph\_pipe.GraphMuse method)@\spxentry{close()}\spxextra{pymusepipe.graph\_pipe.GraphMuse method}}

\begin{fulllineitems}
\phantomsection\label{\detokenize{api/pymusepipe:pymusepipe.graph_pipe.GraphMuse.close}}
\pysigstartsignatures
\pysiglinewithargsret{\sphinxbfcode{\sphinxupquote{close}}}{}{}
\pysigstopsignatures
\end{fulllineitems}

\index{plot\_page() (pymusepipe.graph\_pipe.GraphMuse method)@\spxentry{plot\_page()}\spxextra{pymusepipe.graph\_pipe.GraphMuse method}}

\begin{fulllineitems}
\phantomsection\label{\detokenize{api/pymusepipe:pymusepipe.graph_pipe.GraphMuse.plot_page}}
\pysigstartsignatures
\pysiglinewithargsret{\sphinxbfcode{\sphinxupquote{plot\_page}}}{\sphinxparam{\DUrole{n,n}{list\_data}}}{}
\pysigstopsignatures
\sphinxAtStartPar
Plot a set of blocks, each made of a set of spectra or
images. This is for 1 page
It first counts the number of lines needed according to the
separation for images (default is 2 per line, each image taking 2 lines)
and spectra (1 spectrum per line over 2 columns)

\end{fulllineitems}

\index{plot\_set\_images() (pymusepipe.graph\_pipe.GraphMuse method)@\spxentry{plot\_set\_images()}\spxextra{pymusepipe.graph\_pipe.GraphMuse method}}

\begin{fulllineitems}
\phantomsection\label{\detokenize{api/pymusepipe:pymusepipe.graph_pipe.GraphMuse.plot_set_images}}
\pysigstartsignatures
\pysiglinewithargsret{\sphinxbfcode{\sphinxupquote{plot\_set\_images}}}{\sphinxparam{\DUrole{n,n}{set\_of\_images}\DUrole{o,o}{=}\DUrole{default_value}{None}}}{}
\pysigstopsignatures
\sphinxAtStartPar
Plotting a set of images
Skipping the ones that are ‘None’

\end{fulllineitems}

\index{plot\_set\_spectra() (pymusepipe.graph\_pipe.GraphMuse method)@\spxentry{plot\_set\_spectra()}\spxextra{pymusepipe.graph\_pipe.GraphMuse method}}

\begin{fulllineitems}
\phantomsection\label{\detokenize{api/pymusepipe:pymusepipe.graph_pipe.GraphMuse.plot_set_spectra}}
\pysigstartsignatures
\pysiglinewithargsret{\sphinxbfcode{\sphinxupquote{plot\_set\_spectra}}}{\sphinxparam{\DUrole{n,n}{set\_of\_spectra}\DUrole{o,o}{=}\DUrole{default_value}{None}}}{}
\pysigstopsignatures
\sphinxAtStartPar
Plotting a set of spectra
Skipping the ones that are ‘None’

\end{fulllineitems}

\index{savepage() (pymusepipe.graph\_pipe.GraphMuse method)@\spxentry{savepage()}\spxextra{pymusepipe.graph\_pipe.GraphMuse method}}

\begin{fulllineitems}
\phantomsection\label{\detokenize{api/pymusepipe:pymusepipe.graph_pipe.GraphMuse.savepage}}
\pysigstartsignatures
\pysiglinewithargsret{\sphinxbfcode{\sphinxupquote{savepage}}}{}{}
\pysigstopsignatures
\end{fulllineitems}

\index{start\_page() (pymusepipe.graph\_pipe.GraphMuse method)@\spxentry{start\_page()}\spxextra{pymusepipe.graph\_pipe.GraphMuse method}}

\begin{fulllineitems}
\phantomsection\label{\detokenize{api/pymusepipe:pymusepipe.graph_pipe.GraphMuse.start_page}}
\pysigstartsignatures
\pysiglinewithargsret{\sphinxbfcode{\sphinxupquote{start\_page}}}{}{}
\pysigstopsignatures
\sphinxAtStartPar
Start the page

\end{fulllineitems}


\end{fulllineitems}

\index{open\_new\_wcs\_figure() (in module pymusepipe.graph\_pipe)@\spxentry{open\_new\_wcs\_figure()}\spxextra{in module pymusepipe.graph\_pipe}}

\begin{fulllineitems}
\phantomsection\label{\detokenize{api/pymusepipe:pymusepipe.graph_pipe.open_new_wcs_figure}}
\pysigstartsignatures
\pysiglinewithargsret{\sphinxcode{\sphinxupquote{pymusepipe.graph\_pipe.}}\sphinxbfcode{\sphinxupquote{open\_new\_wcs\_figure}}}{\sphinxparam{\DUrole{n,n}{nfig}}, \sphinxparam{\DUrole{n,n}{mywcs}\DUrole{o,o}{=}\DUrole{default_value}{None}}}{}
\pysigstopsignatures
\sphinxAtStartPar
Open a new figure (with number nfig) with given wcs.
If not WCS is provided, just opens a subplot in that figure.


\paragraph{Input}
\label{\detokenize{api/pymusepipe:id78}}\begin{description}
\sphinxlineitem{nfig}{[}int{]}
\sphinxAtStartPar
Figure number to consider

\sphinxlineitem{mywcs}{[}astropy.wcs.WCS{]}
\sphinxAtStartPar
Input WCS to open a new figure (Default value = None)

\end{description}
\begin{quote}\begin{description}
\sphinxlineitem{returns}
\sphinxAtStartPar
Figure itself with the subplots using the wcs projection

\sphinxlineitem{rtype}
\sphinxAtStartPar
fig, subplot

\end{description}\end{quote}

\end{fulllineitems}

\index{plot\_compare\_contours() (in module pymusepipe.graph\_pipe)@\spxentry{plot\_compare\_contours()}\spxextra{in module pymusepipe.graph\_pipe}}

\begin{fulllineitems}
\phantomsection\label{\detokenize{api/pymusepipe:pymusepipe.graph_pipe.plot_compare_contours}}
\pysigstartsignatures
\pysiglinewithargsret{\sphinxcode{\sphinxupquote{pymusepipe.graph\_pipe.}}\sphinxbfcode{\sphinxupquote{plot\_compare\_contours}}}{\sphinxparam{\DUrole{n,n}{data1}}, \sphinxparam{\DUrole{n,n}{data2}}, \sphinxparam{\DUrole{n,n}{plotwcs}\DUrole{o,o}{=}\DUrole{default_value}{None}}, \sphinxparam{\DUrole{n,n}{labels}\DUrole{o,o}{=}\DUrole{default_value}{(\textquotesingle{}Data1\textquotesingle{}, \textquotesingle{}Data2\textquotesingle{})}}, \sphinxparam{\DUrole{n,n}{levels}\DUrole{o,o}{=}\DUrole{default_value}{None}}, \sphinxparam{\DUrole{n,n}{nlevels}\DUrole{o,o}{=}\DUrole{default_value}{10}}, \sphinxparam{\DUrole{n,n}{fignum}\DUrole{o,o}{=}\DUrole{default_value}{1}}, \sphinxparam{\DUrole{n,n}{namefig}\DUrole{o,o}{=}\DUrole{default_value}{\textquotesingle{}dummy\_contours.png\textquotesingle{}}}, \sphinxparam{\DUrole{n,n}{figfolder}\DUrole{o,o}{=}\DUrole{default_value}{\textquotesingle{}\textquotesingle{}}}, \sphinxparam{\DUrole{n,n}{savefig}\DUrole{o,o}{=}\DUrole{default_value}{False}}, \sphinxparam{\DUrole{o,o}{**}\DUrole{n,n}{kwargs}}}{}
\pysigstopsignatures
\sphinxAtStartPar
Creates a plot with the contours of two input datasets for comparison


\paragraph{Input}
\label{\detokenize{api/pymusepipe:id79}}
\sphinxAtStartPar
data1
data2: 2d np.arrays
\begin{quote}

\sphinxAtStartPar
Input arrays to compare
\end{quote}
\begin{description}
\sphinxlineitem{plotwcs: WCS}
\sphinxAtStartPar
WCS used to set the plot if provided

\sphinxlineitem{labels: tuple/list of 2 str}
\sphinxAtStartPar
Labels for the plot

\sphinxlineitem{levels: list of floats}
\sphinxAtStartPar
Levels to be used for the contours. Calculated if None.

\sphinxlineitem{fignum: int}
\sphinxAtStartPar
Number for the figure

\sphinxlineitem{namefig: str}
\sphinxAtStartPar
Name of the figure to be saved (if savefig is True)

\sphinxlineitem{figfolder: str}
\sphinxAtStartPar
Name of the folder for the figure

\sphinxlineitem{savefig: bool}
\sphinxAtStartPar
If True, will save the figure as namefig

\end{description}


\paragraph{Creates}
\label{\detokenize{api/pymusepipe:id80}}
\sphinxAtStartPar
Plot with contours of the two input dataset

\end{fulllineitems}

\index{plot\_compare\_cuts() (in module pymusepipe.graph\_pipe)@\spxentry{plot\_compare\_cuts()}\spxextra{in module pymusepipe.graph\_pipe}}

\begin{fulllineitems}
\phantomsection\label{\detokenize{api/pymusepipe:pymusepipe.graph_pipe.plot_compare_cuts}}
\pysigstartsignatures
\pysiglinewithargsret{\sphinxcode{\sphinxupquote{pymusepipe.graph\_pipe.}}\sphinxbfcode{\sphinxupquote{plot\_compare\_cuts}}}{\sphinxparam{\DUrole{n,n}{data1}}, \sphinxparam{\DUrole{n,n}{data2}}, \sphinxparam{\DUrole{n,n}{labels}\DUrole{o,o}{=}\DUrole{default_value}{(\textquotesingle{}X\textquotesingle{}, \textquotesingle{}Y\textquotesingle{})}}, \sphinxparam{\DUrole{n,n}{figfolder}\DUrole{o,o}{=}\DUrole{default_value}{\textquotesingle{}\textquotesingle{}}}, \sphinxparam{\DUrole{n,n}{fignum}\DUrole{o,o}{=}\DUrole{default_value}{1}}, \sphinxparam{\DUrole{n,n}{namefig}\DUrole{o,o}{=}\DUrole{default_value}{\textquotesingle{}dummy\_polypar.png\textquotesingle{}}}, \sphinxparam{\DUrole{n,n}{ncuts}\DUrole{o,o}{=}\DUrole{default_value}{11}}, \sphinxparam{\DUrole{n,n}{savefig}\DUrole{o,o}{=}\DUrole{default_value}{False}}, \sphinxparam{\DUrole{o,o}{**}\DUrole{n,n}{kwargs}}}{}
\pysigstopsignatures

\paragraph{Input}
\label{\detokenize{api/pymusepipe:id81}}
\sphinxAtStartPar
data1
data2
label1
label2
figfolder
fignum
namefig
savefig
kwargs


\paragraph{Creates}
\label{\detokenize{api/pymusepipe:id82}}
\sphinxAtStartPar
Plot with a comparison of the two data arrays using regular X and Y cuts

\end{fulllineitems}

\index{plot\_compare\_diff() (in module pymusepipe.graph\_pipe)@\spxentry{plot\_compare\_diff()}\spxextra{in module pymusepipe.graph\_pipe}}

\begin{fulllineitems}
\phantomsection\label{\detokenize{api/pymusepipe:pymusepipe.graph_pipe.plot_compare_diff}}
\pysigstartsignatures
\pysiglinewithargsret{\sphinxcode{\sphinxupquote{pymusepipe.graph\_pipe.}}\sphinxbfcode{\sphinxupquote{plot\_compare\_diff}}}{\sphinxparam{\DUrole{n,n}{data1}}, \sphinxparam{\DUrole{n,n}{data2}}, \sphinxparam{\DUrole{n,n}{plotwcs}\DUrole{o,o}{=}\DUrole{default_value}{None}}, \sphinxparam{\DUrole{n,n}{figfolder}\DUrole{o,o}{=}\DUrole{default_value}{\textquotesingle{}\textquotesingle{}}}, \sphinxparam{\DUrole{n,n}{percentage}\DUrole{o,o}{=}\DUrole{default_value}{5}}, \sphinxparam{\DUrole{n,n}{fignum}\DUrole{o,o}{=}\DUrole{default_value}{1}}, \sphinxparam{\DUrole{n,n}{namefig}\DUrole{o,o}{=}\DUrole{default_value}{\textquotesingle{}dummy\_diff.ong\textquotesingle{}}}, \sphinxparam{\DUrole{n,n}{savefig}\DUrole{o,o}{=}\DUrole{default_value}{False}}, \sphinxparam{\DUrole{o,o}{**}\DUrole{n,n}{kwargs}}}{}
\pysigstopsignatures\begin{quote}\begin{description}
\sphinxlineitem{Parameters}\begin{itemize}
\item {} 
\sphinxAtStartPar
\sphinxstyleliteralstrong{\sphinxupquote{data1}} \textendash{} 

\item {} 
\sphinxAtStartPar
\sphinxstyleliteralstrong{\sphinxupquote{data2}} \textendash{} 

\item {} 
\sphinxAtStartPar
\sphinxstyleliteralstrong{\sphinxupquote{figfolder}} \textendash{} 

\item {} 
\sphinxAtStartPar
\sphinxstyleliteralstrong{\sphinxupquote{fignum}} \textendash{} 

\item {} 
\sphinxAtStartPar
\sphinxstyleliteralstrong{\sphinxupquote{namefig}} \textendash{} 

\item {} 
\sphinxAtStartPar
\sphinxstyleliteralstrong{\sphinxupquote{savefig}} \textendash{} 

\item {} 
\sphinxAtStartPar
\sphinxstyleliteralstrong{\sphinxupquote{kwargs}} \textendash{} 

\end{itemize}

\end{description}\end{quote}

\end{fulllineitems}

\index{plot\_polypar() (in module pymusepipe.graph\_pipe)@\spxentry{plot\_polypar()}\spxextra{in module pymusepipe.graph\_pipe}}

\begin{fulllineitems}
\phantomsection\label{\detokenize{api/pymusepipe:pymusepipe.graph_pipe.plot_polypar}}
\pysigstartsignatures
\pysiglinewithargsret{\sphinxcode{\sphinxupquote{pymusepipe.graph\_pipe.}}\sphinxbfcode{\sphinxupquote{plot\_polypar}}}{\sphinxparam{\DUrole{n,n}{polypar}}, \sphinxparam{\DUrole{n,n}{labels}\DUrole{o,o}{=}\DUrole{default_value}{(\textquotesingle{}Data 1\textquotesingle{}, \textquotesingle{}Data 2\textquotesingle{})}}, \sphinxparam{\DUrole{n,n}{figfolder}\DUrole{o,o}{=}\DUrole{default_value}{\textquotesingle{}\textquotesingle{}}}, \sphinxparam{\DUrole{n,n}{fignum}\DUrole{o,o}{=}\DUrole{default_value}{1}}, \sphinxparam{\DUrole{n,n}{namefig}\DUrole{o,o}{=}\DUrole{default_value}{\textquotesingle{}dummy\_polypar.png\textquotesingle{}}}, \sphinxparam{\DUrole{n,n}{savefig}\DUrole{o,o}{=}\DUrole{default_value}{False}}, \sphinxparam{\DUrole{o,o}{**}\DUrole{n,n}{kwargs}}}{}
\pysigstopsignatures
\sphinxAtStartPar
Creating a plot showing the normalisation arising from a polypar object
\begin{quote}\begin{description}
\sphinxlineitem{Parameters}\begin{itemize}
\item {} 
\sphinxAtStartPar
\sphinxstyleliteralstrong{\sphinxupquote{polypar}} \textendash{} 

\item {} 
\sphinxAtStartPar
\sphinxstyleliteralstrong{\sphinxupquote{label1}} \textendash{} 

\item {} 
\sphinxAtStartPar
\sphinxstyleliteralstrong{\sphinxupquote{label2}} \textendash{} 

\item {} 
\sphinxAtStartPar
\sphinxstyleliteralstrong{\sphinxupquote{foldfig}} \textendash{} 

\item {} 
\sphinxAtStartPar
\sphinxstyleliteralstrong{\sphinxupquote{namefig}} \textendash{} 

\end{itemize}

\end{description}\end{quote}

\end{fulllineitems}

\index{print\_fig() (in module pymusepipe.graph\_pipe)@\spxentry{print\_fig()}\spxextra{in module pymusepipe.graph\_pipe}}

\begin{fulllineitems}
\phantomsection\label{\detokenize{api/pymusepipe:pymusepipe.graph_pipe.print_fig}}
\pysigstartsignatures
\pysiglinewithargsret{\sphinxcode{\sphinxupquote{pymusepipe.graph\_pipe.}}\sphinxbfcode{\sphinxupquote{print\_fig}}}{\sphinxparam{\DUrole{n,n}{text}}}{}
\pysigstopsignatures
\end{fulllineitems}



\subsubsection{pymusepipe.init\_musepipe module}
\label{\detokenize{api/pymusepipe:module-pymusepipe.init_musepipe}}\label{\detokenize{api/pymusepipe:pymusepipe-init-musepipe-module}}\index{module@\spxentry{module}!pymusepipe.init\_musepipe@\spxentry{pymusepipe.init\_musepipe}}\index{pymusepipe.init\_musepipe@\spxentry{pymusepipe.init\_musepipe}!module@\spxentry{module}}
\sphinxAtStartPar
MUSE\sphinxhyphen{}PHANGS pipeline wrapper
initialisation of folders
\index{InitMuseParameters (class in pymusepipe.init\_musepipe)@\spxentry{InitMuseParameters}\spxextra{class in pymusepipe.init\_musepipe}}

\begin{fulllineitems}
\phantomsection\label{\detokenize{api/pymusepipe:pymusepipe.init_musepipe.InitMuseParameters}}
\pysigstartsignatures
\pysiglinewithargsret{\sphinxbfcode{\sphinxupquote{class\DUrole{w,w}{  }}}\sphinxcode{\sphinxupquote{pymusepipe.init\_musepipe.}}\sphinxbfcode{\sphinxupquote{InitMuseParameters}}}{\sphinxparam{\DUrole{n,n}{folder\_config}\DUrole{o,o}{=}\DUrole{default_value}{\textquotesingle{}Config/\textquotesingle{}}}, \sphinxparam{\DUrole{n,n}{rc\_filename}\DUrole{o,o}{=}\DUrole{default_value}{None}}, \sphinxparam{\DUrole{n,n}{cal\_filename}\DUrole{o,o}{=}\DUrole{default_value}{None}}, \sphinxparam{\DUrole{n,n}{verbose}\DUrole{o,o}{=}\DUrole{default_value}{True}}}{}
\pysigstopsignatures
\sphinxAtStartPar
Bases: \sphinxhref{https://docs.python.org/3.10/library/functions.html\#object}{\sphinxcode{\sphinxupquote{object}}}
\index{init\_default\_param() (pymusepipe.init\_musepipe.InitMuseParameters method)@\spxentry{init\_default\_param()}\spxextra{pymusepipe.init\_musepipe.InitMuseParameters method}}

\begin{fulllineitems}
\phantomsection\label{\detokenize{api/pymusepipe:pymusepipe.init_musepipe.InitMuseParameters.init_default_param}}
\pysigstartsignatures
\pysiglinewithargsret{\sphinxbfcode{\sphinxupquote{init\_default\_param}}}{\sphinxparam{\DUrole{n,n}{dict\_param}}, \sphinxparam{\DUrole{n,n}{subattr}\DUrole{o,o}{=}\DUrole{default_value}{None}}}{}
\pysigstopsignatures
\sphinxAtStartPar
Initialise the parameters as defined in the input dictionary
Hardcoded in config\_pipe.py


\paragraph{Input}
\label{\detokenize{api/pymusepipe:id83}}\begin{description}
\sphinxlineitem{dict\_param: dict}
\sphinxAtStartPar
Input dictionary defining the attributes

\sphinxlineitem{subattr: str}
\sphinxAtStartPar
Use subattr to add attributes under self.subattr

\end{description}

\end{fulllineitems}

\index{read\_param\_file() (pymusepipe.init\_musepipe.InitMuseParameters method)@\spxentry{read\_param\_file()}\spxextra{pymusepipe.init\_musepipe.InitMuseParameters method}}

\begin{fulllineitems}
\phantomsection\label{\detokenize{api/pymusepipe:pymusepipe.init_musepipe.InitMuseParameters.read_param_file}}
\pysigstartsignatures
\pysiglinewithargsret{\sphinxbfcode{\sphinxupquote{read\_param\_file}}}{\sphinxparam{\DUrole{n,n}{filename}}, \sphinxparam{\DUrole{n,n}{dict\_param}}}{}
\pysigstopsignatures
\sphinxAtStartPar
Reading an input parameter initialisation file

\end{fulllineitems}


\end{fulllineitems}

\index{PipeObject (class in pymusepipe.init\_musepipe)@\spxentry{PipeObject}\spxextra{class in pymusepipe.init\_musepipe}}

\begin{fulllineitems}
\phantomsection\label{\detokenize{api/pymusepipe:pymusepipe.init_musepipe.PipeObject}}
\pysigstartsignatures
\pysiglinewithargsret{\sphinxbfcode{\sphinxupquote{class\DUrole{w,w}{  }}}\sphinxcode{\sphinxupquote{pymusepipe.init\_musepipe.}}\sphinxbfcode{\sphinxupquote{PipeObject}}}{\sphinxparam{\DUrole{n,n}{info}\DUrole{o,o}{=}\DUrole{default_value}{None}}}{}
\pysigstopsignatures
\sphinxAtStartPar
Bases: \sphinxhref{https://docs.python.org/3.10/library/functions.html\#object}{\sphinxcode{\sphinxupquote{object}}}

\sphinxAtStartPar
A very simple class used to store astropy tables.

\end{fulllineitems}

\index{add\_suffix\_tokeys() (in module pymusepipe.init\_musepipe)@\spxentry{add\_suffix\_tokeys()}\spxextra{in module pymusepipe.init\_musepipe}}

\begin{fulllineitems}
\phantomsection\label{\detokenize{api/pymusepipe:pymusepipe.init_musepipe.add_suffix_tokeys}}
\pysigstartsignatures
\pysiglinewithargsret{\sphinxcode{\sphinxupquote{pymusepipe.init\_musepipe.}}\sphinxbfcode{\sphinxupquote{add\_suffix\_tokeys}}}{\sphinxparam{\DUrole{n,n}{dic}}, \sphinxparam{\DUrole{n,n}{suffix}\DUrole{o,o}{=}\DUrole{default_value}{\textquotesingle{}\_folder\textquotesingle{}}}}{}
\pysigstopsignatures
\end{fulllineitems}



\subsubsection{pymusepipe.mpdaf\_pipe module}
\label{\detokenize{api/pymusepipe:module-pymusepipe.mpdaf_pipe}}\label{\detokenize{api/pymusepipe:pymusepipe-mpdaf-pipe-module}}\index{module@\spxentry{module}!pymusepipe.mpdaf\_pipe@\spxentry{pymusepipe.mpdaf\_pipe}}\index{pymusepipe.mpdaf\_pipe@\spxentry{pymusepipe.mpdaf\_pipe}!module@\spxentry{module}}
\sphinxAtStartPar
MUSE\sphinxhyphen{}PHANGS mpdaf\sphinxhyphen{}functions module
\index{BasicFile (class in pymusepipe.mpdaf\_pipe)@\spxentry{BasicFile}\spxextra{class in pymusepipe.mpdaf\_pipe}}

\begin{fulllineitems}
\phantomsection\label{\detokenize{api/pymusepipe:pymusepipe.mpdaf_pipe.BasicFile}}
\pysigstartsignatures
\pysiglinewithargsret{\sphinxbfcode{\sphinxupquote{class\DUrole{w,w}{  }}}\sphinxcode{\sphinxupquote{pymusepipe.mpdaf\_pipe.}}\sphinxbfcode{\sphinxupquote{BasicFile}}}{\sphinxparam{\DUrole{n,n}{filename}}, \sphinxparam{\DUrole{o,o}{**}\DUrole{n,n}{kwargs}}}{}
\pysigstopsignatures
\sphinxAtStartPar
Bases: \sphinxhref{https://docs.python.org/3.10/library/functions.html\#object}{\sphinxcode{\sphinxupquote{object}}}

\sphinxAtStartPar
Basic file with just the name and some properties
to attach to that Cube

\end{fulllineitems}

\index{BasicPSF (class in pymusepipe.mpdaf\_pipe)@\spxentry{BasicPSF}\spxextra{class in pymusepipe.mpdaf\_pipe}}

\begin{fulllineitems}
\phantomsection\label{\detokenize{api/pymusepipe:pymusepipe.mpdaf_pipe.BasicPSF}}
\pysigstartsignatures
\pysiglinewithargsret{\sphinxbfcode{\sphinxupquote{class\DUrole{w,w}{  }}}\sphinxcode{\sphinxupquote{pymusepipe.mpdaf\_pipe.}}\sphinxbfcode{\sphinxupquote{BasicPSF}}}{\sphinxparam{\DUrole{n,n}{function}\DUrole{o,o}{=}\DUrole{default_value}{\textquotesingle{}gaussian\textquotesingle{}}}, \sphinxparam{\DUrole{n,n}{fwhm0}\DUrole{o,o}{=}\DUrole{default_value}{0.0}}, \sphinxparam{\DUrole{n,n}{nmoffat}\DUrole{o,o}{=}\DUrole{default_value}{2.8}}, \sphinxparam{\DUrole{n,n}{b}\DUrole{o,o}{=}\DUrole{default_value}{0.0}}, \sphinxparam{\DUrole{n,n}{l0}\DUrole{o,o}{=}\DUrole{default_value}{6483.58}}, \sphinxparam{\DUrole{n,n}{psf\_array}\DUrole{o,o}{=}\DUrole{default_value}{None}}}{}
\pysigstopsignatures
\sphinxAtStartPar
Bases: \sphinxhref{https://docs.python.org/3.10/library/functions.html\#object}{\sphinxcode{\sphinxupquote{object}}}

\sphinxAtStartPar
Basic PSF function and parameters
\index{psf\_array (pymusepipe.mpdaf\_pipe.BasicPSF property)@\spxentry{psf\_array}\spxextra{pymusepipe.mpdaf\_pipe.BasicPSF property}}

\begin{fulllineitems}
\phantomsection\label{\detokenize{api/pymusepipe:pymusepipe.mpdaf_pipe.BasicPSF.psf_array}}
\pysigstartsignatures
\pysigline{\sphinxbfcode{\sphinxupquote{property\DUrole{w,w}{  }}}\sphinxbfcode{\sphinxupquote{psf\_array}}}
\pysigstopsignatures
\end{fulllineitems}


\end{fulllineitems}

\index{MuseCube (class in pymusepipe.mpdaf\_pipe)@\spxentry{MuseCube}\spxextra{class in pymusepipe.mpdaf\_pipe}}

\begin{fulllineitems}
\phantomsection\label{\detokenize{api/pymusepipe:pymusepipe.mpdaf_pipe.MuseCube}}
\pysigstartsignatures
\pysiglinewithargsret{\sphinxbfcode{\sphinxupquote{class\DUrole{w,w}{  }}}\sphinxcode{\sphinxupquote{pymusepipe.mpdaf\_pipe.}}\sphinxbfcode{\sphinxupquote{MuseCube}}}{\sphinxparam{\DUrole{n,n}{source}\DUrole{o,o}{=}\DUrole{default_value}{None}}, \sphinxparam{\DUrole{n,n}{verbose}\DUrole{o,o}{=}\DUrole{default_value}{False}}, \sphinxparam{\DUrole{o,o}{**}\DUrole{n,n}{kwargs}}}{}
\pysigstopsignatures
\sphinxAtStartPar
Bases: \sphinxcode{\sphinxupquote{Cube}}

\sphinxAtStartPar
Wrapper around the mpdaf Cube functionalities
\index{astropy\_convolve() (pymusepipe.mpdaf\_pipe.MuseCube method)@\spxentry{astropy\_convolve()}\spxextra{pymusepipe.mpdaf\_pipe.MuseCube method}}

\begin{fulllineitems}
\phantomsection\label{\detokenize{api/pymusepipe:pymusepipe.mpdaf_pipe.MuseCube.astropy_convolve}}
\pysigstartsignatures
\pysiglinewithargsret{\sphinxbfcode{\sphinxupquote{astropy\_convolve}}}{\sphinxparam{\DUrole{n,n}{other}}, \sphinxparam{\DUrole{n,n}{fft}\DUrole{o,o}{=}\DUrole{default_value}{True}}, \sphinxparam{\DUrole{n,n}{inplace}\DUrole{o,o}{=}\DUrole{default_value}{False}}}{}
\pysigstopsignatures
\sphinxAtStartPar
Convolve a DataArray with an array of the same number of dimensions
using a specified convolution function.

\sphinxAtStartPar
Copy of \_convolve for a cube, but doing it per slice or not

\sphinxAtStartPar
Masked values in self.data and self.var are replaced with
zeros before the convolution is performed. However masked
pixels in the input data remain masked in the output.

\sphinxAtStartPar
Any variances in self.var are propagated correctly.

\sphinxAtStartPar
If self.var exists, the variances are propagated using the equation:

\begin{sphinxVerbatim}[commandchars=\\\{\}]
\PYG{n}{result}\PYG{o}{.}\PYG{n}{var} \PYG{o}{=} \PYG{n+nb+bp}{self}\PYG{o}{.}\PYG{n}{var} \PYG{p}{(}\PYG{o}{*}\PYG{p}{)} \PYG{n}{other}\PYG{o}{*}\PYG{o}{*}\PYG{l+m+mi}{2}
\end{sphinxVerbatim}

\sphinxAtStartPar
where (*) indicates convolution. This equation can be derived
by applying the usual rules of error\sphinxhyphen{}propagation to the
discrete convolution equation.

\sphinxAtStartPar
Uses {\color{red}\bfseries{}\textasciigrave{}}astropy.convolution.convolve\_fft’ or ‘astropy.convolution.convolve’
\begin{quote}\begin{description}
\sphinxlineitem{Parameters}\begin{itemize}
\item {} 
\sphinxAtStartPar
\sphinxstyleliteralstrong{\sphinxupquote{fft}} (\sphinxstyleliteralemphasis{\sphinxupquote{boolean}}) \textendash{} 
\sphinxAtStartPar
The convolution function to use, chosen from:
\begin{itemize}
\item {} 
\sphinxAtStartPar
{\color{red}\bfseries{}\textasciigrave{}}astropy.convolution.convolve\_fft’

\item {} 
\sphinxAtStartPar
{\color{red}\bfseries{}\textasciigrave{}}astropy.convolution.convolve’

\end{itemize}

\sphinxAtStartPar
In general convolve\_fft() is faster than convolve() except when
other.data only contains a few pixels. However convolve\_fft uses
a lot more memory than convolve(), so convolve() is sometimes the
only reasonable choice. In particular, convolve\_fft allocates two
arrays whose dimensions are the sum of self.shape and other.shape,
rounded up to a power of two. These arrays can be impractically
large for some input data\sphinxhyphen{}sets.


\item {} 
\sphinxAtStartPar
\sphinxstyleliteralstrong{\sphinxupquote{other}} (\sphinxstyleliteralemphasis{\sphinxupquote{DataArray}}\sphinxstyleliteralemphasis{\sphinxupquote{ or }}\sphinxhref{https://numpy.org/doc/stable/reference/generated/numpy.ndarray.html\#numpy.ndarray}{\sphinxstyleliteralemphasis{\sphinxupquote{numpy.ndarray}}}) \textendash{} 
\sphinxAtStartPar
The array with which to convolve the contents of self.  This must
have the same number of dimensions as self, but it can have fewer
elements. When this array contains a symmetric filtering function,
the center of the function should be placed at the center of pixel,
\sphinxcode{\sphinxupquote{(other.shape \sphinxhyphen{} 1)//2}}.

\sphinxAtStartPar
Note that passing a DataArray object is equivalent to just
passing its DataArray.data member. If it has any variances,
these are ignored.


\item {} 
\sphinxAtStartPar
\sphinxstyleliteralstrong{\sphinxupquote{inplace}} (\sphinxhref{https://docs.python.org/3.10/library/functions.html\#bool}{\sphinxstyleliteralemphasis{\sphinxupquote{bool}}}) \textendash{} If False (the default), return a new object containing the
convolved array.
If True, record the convolved array in self and return self.

\end{itemize}

\sphinxlineitem{Return type}
\sphinxAtStartPar
\sphinxtitleref{\textasciitilde{}mpdaf.obj.DataArray}

\end{description}\end{quote}

\end{fulllineitems}

\index{build\_filterlist\_images() (pymusepipe.mpdaf\_pipe.MuseCube method)@\spxentry{build\_filterlist\_images()}\spxextra{pymusepipe.mpdaf\_pipe.MuseCube method}}

\begin{fulllineitems}
\phantomsection\label{\detokenize{api/pymusepipe:pymusepipe.mpdaf_pipe.MuseCube.build_filterlist_images}}
\pysigstartsignatures
\pysiglinewithargsret{\sphinxbfcode{\sphinxupquote{build\_filterlist\_images}}}{\sphinxparam{\DUrole{n,n}{filter\_list}}, \sphinxparam{\DUrole{n,n}{prefix}\DUrole{o,o}{=}\DUrole{default_value}{\textquotesingle{}IMAGE\_FOV\textquotesingle{}}}, \sphinxparam{\DUrole{n,n}{suffix}\DUrole{o,o}{=}\DUrole{default_value}{\textquotesingle{}\textquotesingle{}}}, \sphinxparam{\DUrole{n,n}{folder}\DUrole{o,o}{=}\DUrole{default_value}{None}}, \sphinxparam{\DUrole{o,o}{**}\DUrole{n,n}{kwargs}}}{}
\pysigstopsignatures\begin{quote}\begin{description}
\sphinxlineitem{Parameters}\begin{itemize}
\item {} 
\sphinxAtStartPar
\sphinxstyleliteralstrong{\sphinxupquote{filter\_list}} \textendash{} 

\item {} 
\sphinxAtStartPar
\sphinxstyleliteralstrong{\sphinxupquote{prefix}} \textendash{} 

\item {} 
\sphinxAtStartPar
\sphinxstyleliteralstrong{\sphinxupquote{suffix}} \textendash{} 

\item {} 
\sphinxAtStartPar
\sphinxstyleliteralstrong{\sphinxupquote{folder}} \textendash{} 

\item {} 
\sphinxAtStartPar
\sphinxstyleliteralstrong{\sphinxupquote{**kwargs}} \textendash{} 

\end{itemize}

\end{description}\end{quote}

\sphinxAtStartPar
Returns:

\end{fulllineitems}

\index{convolve\_cube\_to\_psf() (pymusepipe.mpdaf\_pipe.MuseCube method)@\spxentry{convolve\_cube\_to\_psf()}\spxextra{pymusepipe.mpdaf\_pipe.MuseCube method}}

\begin{fulllineitems}
\phantomsection\label{\detokenize{api/pymusepipe:pymusepipe.mpdaf_pipe.MuseCube.convolve_cube_to_psf}}
\pysigstartsignatures
\pysiglinewithargsret{\sphinxbfcode{\sphinxupquote{convolve\_cube\_to\_psf}}}{\sphinxparam{\DUrole{n,n}{target\_fwhm}}, \sphinxparam{\DUrole{n,n}{target\_nmoffat}\DUrole{o,o}{=}\DUrole{default_value}{None}}, \sphinxparam{\DUrole{n,n}{target\_function}\DUrole{o,o}{=}\DUrole{default_value}{\textquotesingle{}gaussian\textquotesingle{}}}, \sphinxparam{\DUrole{n,n}{outcube\_folder}\DUrole{o,o}{=}\DUrole{default_value}{None}}, \sphinxparam{\DUrole{n,n}{outcube\_name}\DUrole{o,o}{=}\DUrole{default_value}{None}}, \sphinxparam{\DUrole{n,n}{factor\_fwhm}\DUrole{o,o}{=}\DUrole{default_value}{3}}, \sphinxparam{\DUrole{n,n}{fft}\DUrole{o,o}{=}\DUrole{default_value}{True}}, \sphinxparam{\DUrole{n,n}{erode\_edges}\DUrole{o,o}{=}\DUrole{default_value}{True}}, \sphinxparam{\DUrole{n,n}{npixels\_erosion}\DUrole{o,o}{=}\DUrole{default_value}{2}}}{}
\pysigstopsignatures
\sphinxAtStartPar
Convolve the cube for a target function ‘gaussian’ or ‘moffat’
\begin{quote}\begin{description}
\sphinxlineitem{Parameters}\begin{itemize}
\item {} 
\sphinxAtStartPar
\sphinxstyleliteralstrong{\sphinxupquote{target\_fwhm}} (\sphinxhref{https://docs.python.org/3.10/library/functions.html\#float}{\sphinxstyleliteralemphasis{\sphinxupquote{float}}}) \textendash{} target FWHM in arcsecond

\item {} 
\sphinxAtStartPar
\sphinxstyleliteralstrong{\sphinxupquote{target\_nmoffat}} \textendash{} target n if Moffat function

\item {} 
\sphinxAtStartPar
\sphinxstyleliteralstrong{\sphinxupquote{target\_function}} (\sphinxhref{https://docs.python.org/3.10/library/stdtypes.html\#str}{\sphinxstyleliteralemphasis{\sphinxupquote{str}}}) \textendash{} ‘gaussian’ or ‘moffat’ {[}‘gaussian’{]}

\item {} 
\sphinxAtStartPar
\sphinxstyleliteralstrong{\sphinxupquote{factor\_fwhm}} (\sphinxhref{https://docs.python.org/3.10/library/functions.html\#float}{\sphinxstyleliteralemphasis{\sphinxupquote{float}}}) \textendash{} number of FWHM for size of Kernel

\item {} 
\sphinxAtStartPar
\sphinxstyleliteralstrong{\sphinxupquote{fft}} (\sphinxhref{https://docs.python.org/3.10/library/functions.html\#bool}{\sphinxstyleliteralemphasis{\sphinxupquote{bool}}}) \textendash{} use FFT to convolve or not {[}False{]}

\item {} 
\sphinxAtStartPar
\sphinxstyleliteralstrong{\sphinxupquote{perslice}} (\sphinxhref{https://docs.python.org/3.10/library/functions.html\#bool}{\sphinxstyleliteralemphasis{\sphinxupquote{bool}}}) \textendash{} doing it per slice, or not {[}True{]}
If doing it per slice, using a direct astropy fft. If
doing it with the cube, it uses much more memory but is
more efficient as the convolution is done via mpdaf directly.

\end{itemize}

\end{description}\end{quote}
\begin{description}
\sphinxlineitem{Creates:}
\sphinxAtStartPar
Folder and convolved cube names

\end{description}

\end{fulllineitems}

\index{create\_reference\_cube() (pymusepipe.mpdaf\_pipe.MuseCube method)@\spxentry{create\_reference\_cube()}\spxextra{pymusepipe.mpdaf\_pipe.MuseCube method}}

\begin{fulllineitems}
\phantomsection\label{\detokenize{api/pymusepipe:pymusepipe.mpdaf_pipe.MuseCube.create_reference_cube}}
\pysigstartsignatures
\pysiglinewithargsret{\sphinxbfcode{\sphinxupquote{create\_reference\_cube}}}{\sphinxparam{\DUrole{n,n}{lambdamin}\DUrole{o,o}{=}\DUrole{default_value}{4700}}, \sphinxparam{\DUrole{n,n}{lambdamax}\DUrole{o,o}{=}\DUrole{default_value}{9400}}, \sphinxparam{\DUrole{n,n}{step}\DUrole{o,o}{=}\DUrole{default_value}{1.25}}, \sphinxparam{\DUrole{n,n}{outcube\_name}\DUrole{o,o}{=}\DUrole{default_value}{None}}, \sphinxparam{\DUrole{n,n}{filter\_for\_nan}\DUrole{o,o}{=}\DUrole{default_value}{False}}, \sphinxparam{\DUrole{o,o}{**}\DUrole{n,n}{kwargs}}}{}
\pysigstopsignatures
\sphinxAtStartPar
Create a reference cube using an input one, and overriding
the lambda part, to get a new WCS
\begin{quote}\begin{description}
\sphinxlineitem{Parameters}\begin{itemize}
\item {} 
\sphinxAtStartPar
\sphinxstyleliteralstrong{\sphinxupquote{lambdamin}} \textendash{} 

\item {} 
\sphinxAtStartPar
\sphinxstyleliteralstrong{\sphinxupquote{lambdamax}} \textendash{} 

\item {} 
\sphinxAtStartPar
\sphinxstyleliteralstrong{\sphinxupquote{step}} \textendash{} 

\item {} 
\sphinxAtStartPar
\sphinxstyleliteralstrong{\sphinxupquote{outcube\_name}} \textendash{} 

\item {} 
\sphinxAtStartPar
\sphinxstyleliteralstrong{\sphinxupquote{filter\_for\_nan}} \textendash{} 

\item {} 
\sphinxAtStartPar
\sphinxstyleliteralstrong{\sphinxupquote{**kwargs}} \textendash{} 

\end{itemize}

\sphinxlineitem{Returns}
\sphinxAtStartPar
\begin{description}
\sphinxlineitem{the name of the folder where}
\sphinxAtStartPar
the output cube is, and its name

\end{description}


\sphinxlineitem{Return type}
\sphinxAtStartPar
cube\_folder, outcube\_name (\sphinxhref{https://docs.python.org/3.10/library/stdtypes.html\#str}{str}, \sphinxhref{https://docs.python.org/3.10/library/stdtypes.html\#str}{str})

\end{description}\end{quote}

\end{fulllineitems}

\index{extract\_onespectral\_cube() (pymusepipe.mpdaf\_pipe.MuseCube method)@\spxentry{extract\_onespectral\_cube()}\spxextra{pymusepipe.mpdaf\_pipe.MuseCube method}}

\begin{fulllineitems}
\phantomsection\label{\detokenize{api/pymusepipe:pymusepipe.mpdaf_pipe.MuseCube.extract_onespectral_cube}}
\pysigstartsignatures
\pysiglinewithargsret{\sphinxbfcode{\sphinxupquote{extract\_onespectral\_cube}}}{\sphinxparam{\DUrole{n,n}{wave1}\DUrole{o,o}{=}\DUrole{default_value}{6500.0}}, \sphinxparam{\DUrole{n,n}{outcube\_name}\DUrole{o,o}{=}\DUrole{default_value}{None}}, \sphinxparam{\DUrole{o,o}{**}\DUrole{n,n}{kwargs}}}{}
\pysigstopsignatures
\sphinxAtStartPar
Create a single pixel cube extracted from this one.
\begin{quote}\begin{description}
\sphinxlineitem{Parameters}\begin{itemize}
\item {} 
\sphinxAtStartPar
\sphinxstyleliteralstrong{\sphinxupquote{wave1}} (\sphinxhref{https://docs.python.org/3.10/library/functions.html\#float}{\sphinxstyleliteralemphasis{\sphinxupquote{float}}}) \textendash{} Value of the wavelength to extract. In Angstroems.

\item {} 
\sphinxAtStartPar
\sphinxstyleliteralstrong{\sphinxupquote{outcube\_name}} (\sphinxhref{https://docs.python.org/3.10/library/stdtypes.html\#str}{\sphinxstyleliteralemphasis{\sphinxupquote{str}}}) \textendash{} Name of the output cube

\item {} 
\sphinxAtStartPar
\sphinxstyleliteralstrong{\sphinxupquote{prefix}} (\sphinxhref{https://docs.python.org/3.10/library/stdtypes.html\#str}{\sphinxstyleliteralemphasis{\sphinxupquote{str}}}) \textendash{} If outcube\_name is None (default), use that prefix to append
in front of the input cube name (same folder)

\end{itemize}

\sphinxlineitem{Returns}
\sphinxAtStartPar
\begin{itemize}
\item {} 
\sphinxAtStartPar
\sphinxstyleemphasis{A new cube with only 2 lambda. To be used as a WCS reference for}

\item {} 
\sphinxAtStartPar
\sphinxstyleemphasis{masks.}

\end{itemize}


\end{description}\end{quote}

\end{fulllineitems}

\index{get\_emissionline\_image() (pymusepipe.mpdaf\_pipe.MuseCube method)@\spxentry{get\_emissionline\_image()}\spxextra{pymusepipe.mpdaf\_pipe.MuseCube method}}

\begin{fulllineitems}
\phantomsection\label{\detokenize{api/pymusepipe:pymusepipe.mpdaf_pipe.MuseCube.get_emissionline_image}}
\pysigstartsignatures
\pysiglinewithargsret{\sphinxbfcode{\sphinxupquote{get\_emissionline\_image}}}{\sphinxparam{\DUrole{n,n}{line}\DUrole{o,o}{=}\DUrole{default_value}{None}}, \sphinxparam{\DUrole{n,n}{velocity}\DUrole{o,o}{=}\DUrole{default_value}{0.0}}, \sphinxparam{\DUrole{n,n}{redshift}\DUrole{o,o}{=}\DUrole{default_value}{None}}, \sphinxparam{\DUrole{n,n}{lambda\_window}\DUrole{o,o}{=}\DUrole{default_value}{10.0}}, \sphinxparam{\DUrole{n,n}{medium}\DUrole{o,o}{=}\DUrole{default_value}{\textquotesingle{}vacuum\textquotesingle{}}}}{}
\pysigstopsignatures
\sphinxAtStartPar
Get a narrow band image around Ha


\paragraph{Input}
\label{\detokenize{api/pymusepipe:id90}}
\sphinxAtStartPar
lambda\_window: in Angstroems (10 by default). Width of the window of integration
medium: vacuum or air (string, ‘vacuum’ by default)
velocity: default is 0. (km/s)
redshift: default is None. Overwrite velocity if provided.
line: name of the emission line (see emission\_lines dictionary)

\end{fulllineitems}

\index{get\_filter\_image() (pymusepipe.mpdaf\_pipe.MuseCube method)@\spxentry{get\_filter\_image()}\spxextra{pymusepipe.mpdaf\_pipe.MuseCube method}}

\begin{fulllineitems}
\phantomsection\label{\detokenize{api/pymusepipe:pymusepipe.mpdaf_pipe.MuseCube.get_filter_image}}
\pysigstartsignatures
\pysiglinewithargsret{\sphinxbfcode{\sphinxupquote{get\_filter\_image}}}{\sphinxparam{\DUrole{n,n}{filter\_name}\DUrole{o,o}{=}\DUrole{default_value}{None}}, \sphinxparam{\DUrole{n,n}{own\_filter\_file}\DUrole{o,o}{=}\DUrole{default_value}{None}}, \sphinxparam{\DUrole{n,n}{filter\_folder}\DUrole{o,o}{=}\DUrole{default_value}{\textquotesingle{}\textquotesingle{}}}, \sphinxparam{\DUrole{n,n}{dict\_filters}\DUrole{o,o}{=}\DUrole{default_value}{None}}}{}
\pysigstopsignatures
\sphinxAtStartPar
Get an image given by a filter. If the filter belongs to
the filter list, then use that, otherwise use the given file

\end{fulllineitems}

\index{get\_image\_from\_cube() (pymusepipe.mpdaf\_pipe.MuseCube method)@\spxentry{get\_image\_from\_cube()}\spxextra{pymusepipe.mpdaf\_pipe.MuseCube method}}

\begin{fulllineitems}
\phantomsection\label{\detokenize{api/pymusepipe:pymusepipe.mpdaf_pipe.MuseCube.get_image_from_cube}}
\pysigstartsignatures
\pysiglinewithargsret{\sphinxbfcode{\sphinxupquote{get\_image\_from\_cube}}}{\sphinxparam{\DUrole{n,n}{central\_lambda}\DUrole{o,o}{=}\DUrole{default_value}{None}}, \sphinxparam{\DUrole{n,n}{lambda\_window}\DUrole{o,o}{=}\DUrole{default_value}{0}}}{}
\pysigstopsignatures
\sphinxAtStartPar
Get image from integrated cube, with spectral pixel
centred at central\_lambda and with a lambda\_window of lambda\_window

\end{fulllineitems}

\index{get\_quadrant\_spectra\_from\_cube() (pymusepipe.mpdaf\_pipe.MuseCube method)@\spxentry{get\_quadrant\_spectra\_from\_cube()}\spxextra{pymusepipe.mpdaf\_pipe.MuseCube method}}

\begin{fulllineitems}
\phantomsection\label{\detokenize{api/pymusepipe:pymusepipe.mpdaf_pipe.MuseCube.get_quadrant_spectra_from_cube}}
\pysigstartsignatures
\pysiglinewithargsret{\sphinxbfcode{\sphinxupquote{get\_quadrant\_spectra\_from\_cube}}}{\sphinxparam{\DUrole{n,n}{pixel\_window}\DUrole{o,o}{=}\DUrole{default_value}{0}}}{}
\pysigstopsignatures
\sphinxAtStartPar
Get quadrant spectra from the Cube


\paragraph{Input}
\label{\detokenize{api/pymusepipe:id91}}
\sphinxAtStartPar
pixel\_window : pixel\_window of integration

\end{fulllineitems}

\index{get\_set\_spectra() (pymusepipe.mpdaf\_pipe.MuseCube method)@\spxentry{get\_set\_spectra()}\spxextra{pymusepipe.mpdaf\_pipe.MuseCube method}}

\begin{fulllineitems}
\phantomsection\label{\detokenize{api/pymusepipe:pymusepipe.mpdaf_pipe.MuseCube.get_set_spectra}}
\pysigstartsignatures
\pysiglinewithargsret{\sphinxbfcode{\sphinxupquote{get\_set\_spectra}}}{}{}
\pysigstopsignatures
\sphinxAtStartPar
Get a set of standard spectra from the Cube

\end{fulllineitems}

\index{get\_spectrum\_from\_cube() (pymusepipe.mpdaf\_pipe.MuseCube method)@\spxentry{get\_spectrum\_from\_cube()}\spxextra{pymusepipe.mpdaf\_pipe.MuseCube method}}

\begin{fulllineitems}
\phantomsection\label{\detokenize{api/pymusepipe:pymusepipe.mpdaf_pipe.MuseCube.get_spectrum_from_cube}}
\pysigstartsignatures
\pysiglinewithargsret{\sphinxbfcode{\sphinxupquote{get\_spectrum\_from\_cube}}}{\sphinxparam{\DUrole{n,n}{nx}\DUrole{o,o}{=}\DUrole{default_value}{None}}, \sphinxparam{\DUrole{n,n}{ny}\DUrole{o,o}{=}\DUrole{default_value}{None}}, \sphinxparam{\DUrole{n,n}{pixel\_window}\DUrole{o,o}{=}\DUrole{default_value}{0}}, \sphinxparam{\DUrole{n,n}{title}\DUrole{o,o}{=}\DUrole{default_value}{\textquotesingle{}Spectrum\textquotesingle{}}}}{}
\pysigstopsignatures
\sphinxAtStartPar
Get a spectrum from the cube with centre defined in pixels
with nx, ny and a window of ‘pixel\_window’

\end{fulllineitems}

\index{get\_whiteimage\_from\_cube() (pymusepipe.mpdaf\_pipe.MuseCube method)@\spxentry{get\_whiteimage\_from\_cube()}\spxextra{pymusepipe.mpdaf\_pipe.MuseCube method}}

\begin{fulllineitems}
\phantomsection\label{\detokenize{api/pymusepipe:pymusepipe.mpdaf_pipe.MuseCube.get_whiteimage_from_cube}}
\pysigstartsignatures
\pysiglinewithargsret{\sphinxbfcode{\sphinxupquote{get\_whiteimage\_from\_cube}}}{}{}
\pysigstopsignatures
\end{fulllineitems}

\index{mask\_trail() (pymusepipe.mpdaf\_pipe.MuseCube method)@\spxentry{mask\_trail()}\spxextra{pymusepipe.mpdaf\_pipe.MuseCube method}}

\begin{fulllineitems}
\phantomsection\label{\detokenize{api/pymusepipe:pymusepipe.mpdaf_pipe.MuseCube.mask_trail}}
\pysigstartsignatures
\pysiglinewithargsret{\sphinxbfcode{\sphinxupquote{mask\_trail}}}{\sphinxparam{\DUrole{n,n}{pq1}\DUrole{o,o}{=}\DUrole{default_value}{{[}0, 0{]}}}, \sphinxparam{\DUrole{n,n}{pq2}\DUrole{o,o}{=}\DUrole{default_value}{{[}10, 10{]}}}, \sphinxparam{\DUrole{n,n}{width}\DUrole{o,o}{=}\DUrole{default_value}{1.0}}, \sphinxparam{\DUrole{n,n}{margins}\DUrole{o,o}{=}\DUrole{default_value}{0.0}}, \sphinxparam{\DUrole{n,n}{reset}\DUrole{o,o}{=}\DUrole{default_value}{False}}, \sphinxparam{\DUrole{n,n}{save}\DUrole{o,o}{=}\DUrole{default_value}{True}}, \sphinxparam{\DUrole{o,o}{**}\DUrole{n,n}{kwargs}}}{}
\pysigstopsignatures
\sphinxAtStartPar
Build a cube mask from 2 points measured from a trail on an image


\paragraph{Input}
\label{\detokenize{api/pymusepipe:id92}}\begin{description}
\sphinxlineitem{pq1: array or tuple (float)}
\sphinxAtStartPar
p and q coordinates of point 1 along the trail

\sphinxlineitem{pq2: array or tuple (float)}
\sphinxAtStartPar
p and q coordinates of point 2 along the trail

\sphinxlineitem{width: float}
\sphinxAtStartPar
Value (in pixel) of the full slit width to exclude

\sphinxlineitem{margins: float}
\sphinxAtStartPar
Value (in pixel) to extend the slit beyond the 2 extrema
If 0, this means limiting it to the extrema themselves.
Default is None, which mean infinitely long slit

\end{description}

\sphinxAtStartPar
reset (bool): if True, reset the mask before masking the slit
save (bool): if True, save the masked cube

\end{fulllineitems}

\index{rebin\_spatial() (pymusepipe.mpdaf\_pipe.MuseCube method)@\spxentry{rebin\_spatial()}\spxextra{pymusepipe.mpdaf\_pipe.MuseCube method}}

\begin{fulllineitems}
\phantomsection\label{\detokenize{api/pymusepipe:pymusepipe.mpdaf_pipe.MuseCube.rebin_spatial}}
\pysigstartsignatures
\pysiglinewithargsret{\sphinxbfcode{\sphinxupquote{rebin\_spatial}}}{\sphinxparam{\DUrole{n,n}{factor}}, \sphinxparam{\DUrole{n,n}{mean}\DUrole{o,o}{=}\DUrole{default_value}{False}}, \sphinxparam{\DUrole{n,n}{inplace}\DUrole{o,o}{=}\DUrole{default_value}{False}}, \sphinxparam{\DUrole{n,n}{full\_covariance}\DUrole{o,o}{=}\DUrole{default_value}{False}}, \sphinxparam{\DUrole{o,o}{**}\DUrole{n,n}{kwargs}}}{}
\pysigstopsignatures
\sphinxAtStartPar
Combine neighboring pixels to reduce the size of a cube by integer factors along each axis.

\sphinxAtStartPar
Each output pixel is the mean of n pixels, where n is the product of the
reduction factors in the factor argument.
Uses mpdaf rebin function, but add a normalisation factor if mean=False (sum).
It also updates the unit by just copying the old one.


\paragraph{Input}
\label{\detokenize{api/pymusepipe:id93}}\begin{description}
\sphinxlineitem{factor}{[}(int or (int,int)){]}
\sphinxAtStartPar
Factor by which the spatial dimensions are reduced

\sphinxlineitem{mean}{[}bool{]}
\sphinxAtStartPar
If True, taking the mean, if False (default) summing

\sphinxlineitem{inplace}{[}bool{]}
\sphinxAtStartPar
If False (default) making a copy. Otherwise using the present cube.

\sphinxlineitem{full\_covariance: bool}
\sphinxAtStartPar
If True, will assume that spaxels are fully covariant. This means that
the variance will be normalised by sqrt(N) where N is the number of
summed spaxels. Default is False

\end{description}
\begin{quote}\begin{description}
\sphinxlineitem{returns}
\sphinxAtStartPar
\sphinxstylestrong{Cube}

\sphinxlineitem{rtype}
\sphinxAtStartPar
rebinned cube

\end{description}\end{quote}

\end{fulllineitems}

\index{save\_mask() (pymusepipe.mpdaf\_pipe.MuseCube method)@\spxentry{save\_mask()}\spxextra{pymusepipe.mpdaf\_pipe.MuseCube method}}

\begin{fulllineitems}
\phantomsection\label{\detokenize{api/pymusepipe:pymusepipe.mpdaf_pipe.MuseCube.save_mask}}
\pysigstartsignatures
\pysiglinewithargsret{\sphinxbfcode{\sphinxupquote{save\_mask}}}{\sphinxparam{\DUrole{n,n}{mask\_name}\DUrole{o,o}{=}\DUrole{default_value}{\textquotesingle{}dummy\_mask.fits\textquotesingle{}}}}{}
\pysigstopsignatures
\sphinxAtStartPar
Save the mask into a 0\sphinxhyphen{}1 image

\end{fulllineitems}


\end{fulllineitems}

\index{MuseCubeMosaic (class in pymusepipe.mpdaf\_pipe)@\spxentry{MuseCubeMosaic}\spxextra{class in pymusepipe.mpdaf\_pipe}}

\begin{fulllineitems}
\phantomsection\label{\detokenize{api/pymusepipe:pymusepipe.mpdaf_pipe.MuseCubeMosaic}}
\pysigstartsignatures
\pysiglinewithargsret{\sphinxbfcode{\sphinxupquote{class\DUrole{w,w}{  }}}\sphinxcode{\sphinxupquote{pymusepipe.mpdaf\_pipe.}}\sphinxbfcode{\sphinxupquote{MuseCubeMosaic}}}{\sphinxparam{\DUrole{n,n}{ref\_wcs}}, \sphinxparam{\DUrole{n,n}{folder\_refcube}\DUrole{o,o}{=}\DUrole{default_value}{\textquotesingle{}\textquotesingle{}}}, \sphinxparam{\DUrole{n,n}{folder\_cubes}\DUrole{o,o}{=}\DUrole{default_value}{\textquotesingle{}\textquotesingle{}}}, \sphinxparam{\DUrole{n,n}{prefix\_cubes}\DUrole{o,o}{=}\DUrole{default_value}{\textquotesingle{}DATACUBE\_FINAL\_WCS\textquotesingle{}}}, \sphinxparam{\DUrole{n,n}{list\_suffix}\DUrole{o,o}{=}\DUrole{default_value}{{[}{]}}}, \sphinxparam{\DUrole{n,n}{use\_fixed\_cubes}\DUrole{o,o}{=}\DUrole{default_value}{True}}, \sphinxparam{\DUrole{n,n}{excluded\_suffix}\DUrole{o,o}{=}\DUrole{default_value}{{[}{]}}}, \sphinxparam{\DUrole{n,n}{included\_suffix}\DUrole{o,o}{=}\DUrole{default_value}{{[}{]}}}, \sphinxparam{\DUrole{n,n}{prefix\_fixed\_cubes}\DUrole{o,o}{=}\DUrole{default_value}{\textquotesingle{}tmask\textquotesingle{}}}, \sphinxparam{\DUrole{n,n}{verbose}\DUrole{o,o}{=}\DUrole{default_value}{False}}, \sphinxparam{\DUrole{n,n}{pointing\_table}\DUrole{o,o}{=}\DUrole{default_value}{None}}, \sphinxparam{\DUrole{n,n}{list\_pointings}\DUrole{o,o}{=}\DUrole{default_value}{None}}, \sphinxparam{\DUrole{n,n}{dict\_psf}\DUrole{o,o}{=}\DUrole{default_value}{\{\}}}, \sphinxparam{\DUrole{n,n}{list\_cubes}\DUrole{o,o}{=}\DUrole{default_value}{None}}, \sphinxparam{\DUrole{o,o}{**}\DUrole{n,n}{kwargs}}}{}
\pysigstopsignatures
\sphinxAtStartPar
Bases: \sphinxcode{\sphinxupquote{CubeMosaic}}
\index{build\_list() (pymusepipe.mpdaf\_pipe.MuseCubeMosaic method)@\spxentry{build\_list()}\spxextra{pymusepipe.mpdaf\_pipe.MuseCubeMosaic method}}

\begin{fulllineitems}
\phantomsection\label{\detokenize{api/pymusepipe:pymusepipe.mpdaf_pipe.MuseCubeMosaic.build_list}}
\pysigstartsignatures
\pysiglinewithargsret{\sphinxbfcode{\sphinxupquote{build\_list}}}{\sphinxparam{\DUrole{n,n}{folder\_cubes}\DUrole{o,o}{=}\DUrole{default_value}{None}}, \sphinxparam{\DUrole{n,n}{prefix\_cubes}\DUrole{o,o}{=}\DUrole{default_value}{None}}, \sphinxparam{\DUrole{n,n}{list\_cubes}\DUrole{o,o}{=}\DUrole{default_value}{None}}, \sphinxparam{\DUrole{o,o}{**}\DUrole{n,n}{kwargs}}}{}
\pysigstopsignatures
\sphinxAtStartPar
Building the list of cubes to process
\begin{quote}\begin{description}
\sphinxlineitem{Parameters}\begin{itemize}
\item {} 
\sphinxAtStartPar
\sphinxstyleliteralstrong{\sphinxupquote{folder\_cubes}} (\sphinxhref{https://docs.python.org/3.10/library/stdtypes.html\#str}{\sphinxstyleliteralemphasis{\sphinxupquote{str}}}) \textendash{} folder for the cubes

\item {} 
\sphinxAtStartPar
\sphinxstyleliteralstrong{\sphinxupquote{prefix\_cubes}} (\sphinxhref{https://docs.python.org/3.10/library/stdtypes.html\#str}{\sphinxstyleliteralemphasis{\sphinxupquote{str}}}) \textendash{} prefix to be used

\end{itemize}

\end{description}\end{quote}

\end{fulllineitems}

\index{convolve\_cubes() (pymusepipe.mpdaf\_pipe.MuseCubeMosaic method)@\spxentry{convolve\_cubes()}\spxextra{pymusepipe.mpdaf\_pipe.MuseCubeMosaic method}}

\begin{fulllineitems}
\phantomsection\label{\detokenize{api/pymusepipe:pymusepipe.mpdaf_pipe.MuseCubeMosaic.convolve_cubes}}
\pysigstartsignatures
\pysiglinewithargsret{\sphinxbfcode{\sphinxupquote{convolve\_cubes}}}{\sphinxparam{\DUrole{n,n}{target\_fwhm}}, \sphinxparam{\DUrole{n,n}{target\_nmoffat}\DUrole{o,o}{=}\DUrole{default_value}{None}}, \sphinxparam{\DUrole{n,n}{target\_function}\DUrole{o,o}{=}\DUrole{default_value}{\textquotesingle{}gaussian\textquotesingle{}}}, \sphinxparam{\DUrole{n,n}{suffix}\DUrole{o,o}{=}\DUrole{default_value}{\textquotesingle{}conv\textquotesingle{}}}, \sphinxparam{\DUrole{o,o}{**}\DUrole{n,n}{kwargs}}}{}
\pysigstopsignatures\begin{quote}\begin{description}
\sphinxlineitem{Parameters}\begin{itemize}
\item {} 
\sphinxAtStartPar
\sphinxstyleliteralstrong{\sphinxupquote{target\_fwhm}} \textendash{} 

\item {} 
\sphinxAtStartPar
\sphinxstyleliteralstrong{\sphinxupquote{target\_nmoffat}} \textendash{} 

\item {} 
\sphinxAtStartPar
\sphinxstyleliteralstrong{\sphinxupquote{input\_function}} \textendash{} 

\item {} 
\sphinxAtStartPar
\sphinxstyleliteralstrong{\sphinxupquote{target\_function}} \textendash{} 

\item {} 
\sphinxAtStartPar
\sphinxstyleliteralstrong{\sphinxupquote{suffix}} \textendash{} 

\item {} 
\sphinxAtStartPar
\sphinxstyleliteralstrong{\sphinxupquote{**kwargs}} \textendash{} 

\end{itemize}

\end{description}\end{quote}

\sphinxAtStartPar
Returns:

\end{fulllineitems}

\index{cube\_names (pymusepipe.mpdaf\_pipe.MuseCubeMosaic property)@\spxentry{cube\_names}\spxextra{pymusepipe.mpdaf\_pipe.MuseCubeMosaic property}}

\begin{fulllineitems}
\phantomsection\label{\detokenize{api/pymusepipe:pymusepipe.mpdaf_pipe.MuseCubeMosaic.cube_names}}
\pysigstartsignatures
\pysigline{\sphinxbfcode{\sphinxupquote{property\DUrole{w,w}{  }}}\sphinxbfcode{\sphinxupquote{cube\_names}}}
\pysigstopsignatures
\end{fulllineitems}

\index{list\_cubes (pymusepipe.mpdaf\_pipe.MuseCubeMosaic property)@\spxentry{list\_cubes}\spxextra{pymusepipe.mpdaf\_pipe.MuseCubeMosaic property}}

\begin{fulllineitems}
\phantomsection\label{\detokenize{api/pymusepipe:pymusepipe.mpdaf_pipe.MuseCubeMosaic.list_cubes}}
\pysigstartsignatures
\pysigline{\sphinxbfcode{\sphinxupquote{property\DUrole{w,w}{  }}}\sphinxbfcode{\sphinxupquote{list\_cubes}}}
\pysigstopsignatures
\end{fulllineitems}

\index{madcombine() (pymusepipe.mpdaf\_pipe.MuseCubeMosaic method)@\spxentry{madcombine()}\spxextra{pymusepipe.mpdaf\_pipe.MuseCubeMosaic method}}

\begin{fulllineitems}
\phantomsection\label{\detokenize{api/pymusepipe:pymusepipe.mpdaf_pipe.MuseCubeMosaic.madcombine}}
\pysigstartsignatures
\pysiglinewithargsret{\sphinxbfcode{\sphinxupquote{madcombine}}}{\sphinxparam{\DUrole{n,n}{folder\_cubes}\DUrole{o,o}{=}\DUrole{default_value}{None}}, \sphinxparam{\DUrole{n,n}{outcube\_name}\DUrole{o,o}{=}\DUrole{default_value}{\textquotesingle{}dummy.fits\textquotesingle{}}}, \sphinxparam{\DUrole{n,n}{fakemode}\DUrole{o,o}{=}\DUrole{default_value}{False}}, \sphinxparam{\DUrole{n,n}{mad}\DUrole{o,o}{=}\DUrole{default_value}{True}}}{}
\pysigstopsignatures
\sphinxAtStartPar
Combine the CubeMosaic and write it out.
\begin{quote}\begin{description}
\sphinxlineitem{Parameters}\begin{itemize}
\item {} 
\sphinxAtStartPar
\sphinxstyleliteralstrong{\sphinxupquote{folder\_cubes}} (\sphinxhref{https://docs.python.org/3.10/library/stdtypes.html\#str}{\sphinxstyleliteralemphasis{\sphinxupquote{str}}}) \textendash{} name of the folder for the cube {[}None{]}

\item {} 
\sphinxAtStartPar
\sphinxstyleliteralstrong{\sphinxupquote{outcube\_name}} (\sphinxhref{https://docs.python.org/3.10/library/stdtypes.html\#str}{\sphinxstyleliteralemphasis{\sphinxupquote{str}}}) \textendash{} name of the outcube

\item {} 
\sphinxAtStartPar
\sphinxstyleliteralstrong{\sphinxupquote{mad}} (\sphinxhref{https://docs.python.org/3.10/library/functions.html\#bool}{\sphinxstyleliteralemphasis{\sphinxupquote{bool}}}) \textendash{} using mad or not {[}True{]}

\end{itemize}

\end{description}\end{quote}
\begin{description}
\sphinxlineitem{Creates:}
\sphinxAtStartPar
A new cube, combination of all input cubes listes in CubeMosaic

\end{description}

\end{fulllineitems}

\index{ncubes (pymusepipe.mpdaf\_pipe.MuseCubeMosaic property)@\spxentry{ncubes}\spxextra{pymusepipe.mpdaf\_pipe.MuseCubeMosaic property}}

\begin{fulllineitems}
\phantomsection\label{\detokenize{api/pymusepipe:pymusepipe.mpdaf_pipe.MuseCubeMosaic.ncubes}}
\pysigstartsignatures
\pysigline{\sphinxbfcode{\sphinxupquote{property\DUrole{w,w}{  }}}\sphinxbfcode{\sphinxupquote{ncubes}}}
\pysigstopsignatures
\end{fulllineitems}

\index{print\_cube\_names() (pymusepipe.mpdaf\_pipe.MuseCubeMosaic method)@\spxentry{print\_cube\_names()}\spxextra{pymusepipe.mpdaf\_pipe.MuseCubeMosaic method}}

\begin{fulllineitems}
\phantomsection\label{\detokenize{api/pymusepipe:pymusepipe.mpdaf_pipe.MuseCubeMosaic.print_cube_names}}
\pysigstartsignatures
\pysiglinewithargsret{\sphinxbfcode{\sphinxupquote{print\_cube\_names}}}{}{}
\pysigstopsignatures
\end{fulllineitems}


\end{fulllineitems}

\index{MuseFilter (class in pymusepipe.mpdaf\_pipe)@\spxentry{MuseFilter}\spxextra{class in pymusepipe.mpdaf\_pipe}}

\begin{fulllineitems}
\phantomsection\label{\detokenize{api/pymusepipe:pymusepipe.mpdaf_pipe.MuseFilter}}
\pysigstartsignatures
\pysiglinewithargsret{\sphinxbfcode{\sphinxupquote{class\DUrole{w,w}{  }}}\sphinxcode{\sphinxupquote{pymusepipe.mpdaf\_pipe.}}\sphinxbfcode{\sphinxupquote{MuseFilter}}}{\sphinxparam{\DUrole{n,n}{filter\_name}\DUrole{o,o}{=}\DUrole{default_value}{\textquotesingle{}Cousins\_R\textquotesingle{}}}, \sphinxparam{\DUrole{n,n}{filter\_fits\_file}\DUrole{o,o}{=}\DUrole{default_value}{\textquotesingle{}filter\_list.fits\textquotesingle{}}}, \sphinxparam{\DUrole{n,n}{filter\_ascii\_file}\DUrole{o,o}{=}\DUrole{default_value}{None}}}{}
\pysigstopsignatures
\sphinxAtStartPar
Bases: \sphinxhref{https://docs.python.org/3.10/library/functions.html\#object}{\sphinxcode{\sphinxupquote{object}}}
\index{read() (pymusepipe.mpdaf\_pipe.MuseFilter method)@\spxentry{read()}\spxextra{pymusepipe.mpdaf\_pipe.MuseFilter method}}

\begin{fulllineitems}
\phantomsection\label{\detokenize{api/pymusepipe:pymusepipe.mpdaf_pipe.MuseFilter.read}}
\pysigstartsignatures
\pysiglinewithargsret{\sphinxbfcode{\sphinxupquote{read}}}{}{}
\pysigstopsignatures
\sphinxAtStartPar
Reading the data in the file

\end{fulllineitems}


\end{fulllineitems}

\index{MuseImage (class in pymusepipe.mpdaf\_pipe)@\spxentry{MuseImage}\spxextra{class in pymusepipe.mpdaf\_pipe}}

\begin{fulllineitems}
\phantomsection\label{\detokenize{api/pymusepipe:pymusepipe.mpdaf_pipe.MuseImage}}
\pysigstartsignatures
\pysiglinewithargsret{\sphinxbfcode{\sphinxupquote{class\DUrole{w,w}{  }}}\sphinxcode{\sphinxupquote{pymusepipe.mpdaf\_pipe.}}\sphinxbfcode{\sphinxupquote{MuseImage}}}{\sphinxparam{\DUrole{n,n}{source}\DUrole{o,o}{=}\DUrole{default_value}{None}}, \sphinxparam{\DUrole{o,o}{**}\DUrole{n,n}{kwargs}}}{}
\pysigstopsignatures
\sphinxAtStartPar
Bases: \sphinxcode{\sphinxupquote{Image}}

\sphinxAtStartPar
Wrapper around the mpdaf Image functionalities
\index{get\_fwhm\_startend() (pymusepipe.mpdaf\_pipe.MuseImage method)@\spxentry{get\_fwhm\_startend()}\spxextra{pymusepipe.mpdaf\_pipe.MuseImage method}}

\begin{fulllineitems}
\phantomsection\label{\detokenize{api/pymusepipe:pymusepipe.mpdaf_pipe.MuseImage.get_fwhm_startend}}
\pysigstartsignatures
\pysiglinewithargsret{\sphinxbfcode{\sphinxupquote{get\_fwhm\_startend}}}{}{}
\pysigstopsignatures
\sphinxAtStartPar
Get range of FWHM

\end{fulllineitems}

\index{mask\_trail() (pymusepipe.mpdaf\_pipe.MuseImage method)@\spxentry{mask\_trail()}\spxextra{pymusepipe.mpdaf\_pipe.MuseImage method}}

\begin{fulllineitems}
\phantomsection\label{\detokenize{api/pymusepipe:pymusepipe.mpdaf_pipe.MuseImage.mask_trail}}
\pysigstartsignatures
\pysiglinewithargsret{\sphinxbfcode{\sphinxupquote{mask\_trail}}}{\sphinxparam{\DUrole{n,n}{pq1}\DUrole{o,o}{=}\DUrole{default_value}{{[}0, 0{]}}}, \sphinxparam{\DUrole{n,n}{pq2}\DUrole{o,o}{=}\DUrole{default_value}{{[}10, 10{]}}}, \sphinxparam{\DUrole{n,n}{width}\DUrole{o,o}{=}\DUrole{default_value}{0.0}}, \sphinxparam{\DUrole{n,n}{reset}\DUrole{o,o}{=}\DUrole{default_value}{False}}, \sphinxparam{\DUrole{n,n}{extent}\DUrole{o,o}{=}\DUrole{default_value}{None}}}{}
\pysigstopsignatures
\sphinxAtStartPar
Build an image mask from 2 points measured from a trail


\paragraph{Input}
\label{\detokenize{api/pymusepipe:id94}}\begin{description}
\sphinxlineitem{pq1: array or tuple (float)}
\sphinxAtStartPar
p and q coordinates of point 1 along the trail

\sphinxlineitem{pq2: array or tuple (float)}
\sphinxAtStartPar
p and q coordinates of point 2 along the trail

\sphinxlineitem{width: float}
\sphinxAtStartPar
Value (in pixel) of the full slit width to exclude

\sphinxlineitem{extent: float}
\sphinxAtStartPar
Value (in pixel) to extend the slit beyond the 2 extrema
If 0, this means limiting it to the extrema themselves.
Default is None, which mean infinitely long slit

\end{description}

\end{fulllineitems}

\index{reset\_mask() (pymusepipe.mpdaf\_pipe.MuseImage method)@\spxentry{reset\_mask()}\spxextra{pymusepipe.mpdaf\_pipe.MuseImage method}}

\begin{fulllineitems}
\phantomsection\label{\detokenize{api/pymusepipe:pymusepipe.mpdaf_pipe.MuseImage.reset_mask}}
\pysigstartsignatures
\pysiglinewithargsret{\sphinxbfcode{\sphinxupquote{reset\_mask}}}{}{}
\pysigstopsignatures
\sphinxAtStartPar
Resetting the Image mask

\end{fulllineitems}

\index{save\_mask() (pymusepipe.mpdaf\_pipe.MuseImage method)@\spxentry{save\_mask()}\spxextra{pymusepipe.mpdaf\_pipe.MuseImage method}}

\begin{fulllineitems}
\phantomsection\label{\detokenize{api/pymusepipe:pymusepipe.mpdaf_pipe.MuseImage.save_mask}}
\pysigstartsignatures
\pysiglinewithargsret{\sphinxbfcode{\sphinxupquote{save\_mask}}}{\sphinxparam{\DUrole{n,n}{mask\_name}\DUrole{o,o}{=}\DUrole{default_value}{\textquotesingle{}dummy\_mask.fits\textquotesingle{}}}}{}
\pysigstopsignatures
\sphinxAtStartPar
Save the mask into a 0\sphinxhyphen{}1 image

\end{fulllineitems}


\end{fulllineitems}

\index{MuseSetImages (class in pymusepipe.mpdaf\_pipe)@\spxentry{MuseSetImages}\spxextra{class in pymusepipe.mpdaf\_pipe}}

\begin{fulllineitems}
\phantomsection\label{\detokenize{api/pymusepipe:pymusepipe.mpdaf_pipe.MuseSetImages}}
\pysigstartsignatures
\pysiglinewithargsret{\sphinxbfcode{\sphinxupquote{class\DUrole{w,w}{  }}}\sphinxcode{\sphinxupquote{pymusepipe.mpdaf\_pipe.}}\sphinxbfcode{\sphinxupquote{MuseSetImages}}}{\sphinxparam{\DUrole{o,o}{*}\DUrole{n,n}{args}}, \sphinxparam{\DUrole{o,o}{**}\DUrole{n,n}{kwargs}}}{}
\pysigstopsignatures
\sphinxAtStartPar
Bases: \sphinxhref{https://docs.python.org/3.10/library/stdtypes.html\#list}{\sphinxcode{\sphinxupquote{list}}}

\sphinxAtStartPar
Set of images
\index{update() (pymusepipe.mpdaf\_pipe.MuseSetImages method)@\spxentry{update()}\spxextra{pymusepipe.mpdaf\_pipe.MuseSetImages method}}

\begin{fulllineitems}
\phantomsection\label{\detokenize{api/pymusepipe:pymusepipe.mpdaf_pipe.MuseSetImages.update}}
\pysigstartsignatures
\pysiglinewithargsret{\sphinxbfcode{\sphinxupquote{update}}}{\sphinxparam{\DUrole{o,o}{**}\DUrole{n,n}{kwargs}}}{}
\pysigstopsignatures
\end{fulllineitems}


\end{fulllineitems}

\index{MuseSetSpectra (class in pymusepipe.mpdaf\_pipe)@\spxentry{MuseSetSpectra}\spxextra{class in pymusepipe.mpdaf\_pipe}}

\begin{fulllineitems}
\phantomsection\label{\detokenize{api/pymusepipe:pymusepipe.mpdaf_pipe.MuseSetSpectra}}
\pysigstartsignatures
\pysiglinewithargsret{\sphinxbfcode{\sphinxupquote{class\DUrole{w,w}{  }}}\sphinxcode{\sphinxupquote{pymusepipe.mpdaf\_pipe.}}\sphinxbfcode{\sphinxupquote{MuseSetSpectra}}}{\sphinxparam{\DUrole{o,o}{*}\DUrole{n,n}{args}}, \sphinxparam{\DUrole{o,o}{**}\DUrole{n,n}{kwargs}}}{}
\pysigstopsignatures
\sphinxAtStartPar
Bases: \sphinxhref{https://docs.python.org/3.10/library/stdtypes.html\#list}{\sphinxcode{\sphinxupquote{list}}}

\sphinxAtStartPar
Set of spectra
\index{update() (pymusepipe.mpdaf\_pipe.MuseSetSpectra method)@\spxentry{update()}\spxextra{pymusepipe.mpdaf\_pipe.MuseSetSpectra method}}

\begin{fulllineitems}
\phantomsection\label{\detokenize{api/pymusepipe:pymusepipe.mpdaf_pipe.MuseSetSpectra.update}}
\pysigstartsignatures
\pysiglinewithargsret{\sphinxbfcode{\sphinxupquote{update}}}{\sphinxparam{\DUrole{o,o}{**}\DUrole{n,n}{kwargs}}}{}
\pysigstopsignatures
\end{fulllineitems}


\end{fulllineitems}

\index{MuseSkyContinuum (class in pymusepipe.mpdaf\_pipe)@\spxentry{MuseSkyContinuum}\spxextra{class in pymusepipe.mpdaf\_pipe}}

\begin{fulllineitems}
\phantomsection\label{\detokenize{api/pymusepipe:pymusepipe.mpdaf_pipe.MuseSkyContinuum}}
\pysigstartsignatures
\pysiglinewithargsret{\sphinxbfcode{\sphinxupquote{class\DUrole{w,w}{  }}}\sphinxcode{\sphinxupquote{pymusepipe.mpdaf\_pipe.}}\sphinxbfcode{\sphinxupquote{MuseSkyContinuum}}}{\sphinxparam{\DUrole{n,n}{filename}}}{}
\pysigstopsignatures
\sphinxAtStartPar
Bases: \sphinxhref{https://docs.python.org/3.10/library/functions.html\#object}{\sphinxcode{\sphinxupquote{object}}}
\index{integrate() (pymusepipe.mpdaf\_pipe.MuseSkyContinuum method)@\spxentry{integrate()}\spxextra{pymusepipe.mpdaf\_pipe.MuseSkyContinuum method}}

\begin{fulllineitems}
\phantomsection\label{\detokenize{api/pymusepipe:pymusepipe.mpdaf_pipe.MuseSkyContinuum.integrate}}
\pysigstartsignatures
\pysiglinewithargsret{\sphinxbfcode{\sphinxupquote{integrate}}}{\sphinxparam{\DUrole{n,n}{muse\_filter}}, \sphinxparam{\DUrole{n,n}{ao\_mask}\DUrole{o,o}{=}\DUrole{default_value}{False}}}{}
\pysigstopsignatures
\sphinxAtStartPar
Integrate a sky continuum spectrum using a certain filter file.
If the file is a fits file, use it as the MUSE filter list.
Otherwise use it as an ascii file


\paragraph{Input}
\label{\detokenize{api/pymusepipe:id95}}
\sphinxAtStartPar
muse\_filter: MuseFilter

\end{fulllineitems}

\index{read() (pymusepipe.mpdaf\_pipe.MuseSkyContinuum method)@\spxentry{read()}\spxextra{pymusepipe.mpdaf\_pipe.MuseSkyContinuum method}}

\begin{fulllineitems}
\phantomsection\label{\detokenize{api/pymusepipe:pymusepipe.mpdaf_pipe.MuseSkyContinuum.read}}
\pysigstartsignatures
\pysiglinewithargsret{\sphinxbfcode{\sphinxupquote{read}}}{}{}
\pysigstopsignatures
\sphinxAtStartPar
Read sky continuum spectrum from MUSE data reduction

\end{fulllineitems}

\index{save\_normalised() (pymusepipe.mpdaf\_pipe.MuseSkyContinuum method)@\spxentry{save\_normalised()}\spxextra{pymusepipe.mpdaf\_pipe.MuseSkyContinuum method}}

\begin{fulllineitems}
\phantomsection\label{\detokenize{api/pymusepipe:pymusepipe.mpdaf_pipe.MuseSkyContinuum.save_normalised}}
\pysigstartsignatures
\pysiglinewithargsret{\sphinxbfcode{\sphinxupquote{save\_normalised}}}{\sphinxparam{\DUrole{n,n}{norm\_factor}\DUrole{o,o}{=}\DUrole{default_value}{1.0}}, \sphinxparam{\DUrole{n,n}{prefix}\DUrole{o,o}{=}\DUrole{default_value}{\textquotesingle{}norm\textquotesingle{}}}, \sphinxparam{\DUrole{n,n}{overwrite}\DUrole{o,o}{=}\DUrole{default_value}{False}}}{}
\pysigstopsignatures
\sphinxAtStartPar
Normalises a sky continuum spectrum and save it
within a new fits file


\paragraph{Input}
\label{\detokenize{api/pymusepipe:id96}}\begin{description}
\sphinxlineitem{norm\_factor: float}
\sphinxAtStartPar
Scale factor to multiply the input continuum

\sphinxlineitem{prefix: str}
\sphinxAtStartPar
Prefix for the new continuum fits name. Default
is ‘norm’, so that the new file is ‘norm\_oldname.fits’

\sphinxlineitem{overwrite: bool}
\sphinxAtStartPar
If True, existing file will be overwritten.
Default is False.

\end{description}

\end{fulllineitems}

\index{set\_normfactor() (pymusepipe.mpdaf\_pipe.MuseSkyContinuum method)@\spxentry{set\_normfactor()}\spxextra{pymusepipe.mpdaf\_pipe.MuseSkyContinuum method}}

\begin{fulllineitems}
\phantomsection\label{\detokenize{api/pymusepipe:pymusepipe.mpdaf_pipe.MuseSkyContinuum.set_normfactor}}
\pysigstartsignatures
\pysiglinewithargsret{\sphinxbfcode{\sphinxupquote{set\_normfactor}}}{\sphinxparam{\DUrole{n,n}{background}}, \sphinxparam{\DUrole{n,n}{filter\_name}\DUrole{o,o}{=}\DUrole{default_value}{\textquotesingle{}Cousins\_R\textquotesingle{}}}}{}
\pysigstopsignatures
\sphinxAtStartPar
Get the normalisation factor given a background value
Takes the background value and the sky continuuum spectrum
and convert this to the scaling Ks needed for this sky continuum
The principle relies on having the background measured as:
MUSE\_calib = ((MUSE \sphinxhyphen{} Sky\_cont) + Background) * Norm

\sphinxAtStartPar
as measured from the alignment procedure.

\sphinxAtStartPar
Since we want:
MUSE\_calib = ((MUSE \sphinxhyphen{} Ks * Sky\_cont) + 0) * Norm

\sphinxAtStartPar
This means that: Ks * Sky\_cont = Sky\_cont \sphinxhyphen{} Background
==\textgreater{} Ks = 1 \sphinxhyphen{} Background / Sky\_cont

\sphinxAtStartPar
So we integrate the Sky\_cont to get the corresponding S value
and then provide Ks as 1\sphinxhyphen{}B/S


\paragraph{Input}
\label{\detokenize{api/pymusepipe:id97}}\begin{description}
\sphinxlineitem{background: float}
\sphinxAtStartPar
Value of the background to consider

\sphinxlineitem{filter\_name: str}
\sphinxAtStartPar
Name of the filter to consider

\end{description}

\end{fulllineitems}


\end{fulllineitems}

\index{MuseSpectrum (class in pymusepipe.mpdaf\_pipe)@\spxentry{MuseSpectrum}\spxextra{class in pymusepipe.mpdaf\_pipe}}

\begin{fulllineitems}
\phantomsection\label{\detokenize{api/pymusepipe:pymusepipe.mpdaf_pipe.MuseSpectrum}}
\pysigstartsignatures
\pysiglinewithargsret{\sphinxbfcode{\sphinxupquote{class\DUrole{w,w}{  }}}\sphinxcode{\sphinxupquote{pymusepipe.mpdaf\_pipe.}}\sphinxbfcode{\sphinxupquote{MuseSpectrum}}}{\sphinxparam{\DUrole{n,n}{source}\DUrole{o,o}{=}\DUrole{default_value}{None}}, \sphinxparam{\DUrole{o,o}{**}\DUrole{n,n}{kwargs}}}{}
\pysigstopsignatures
\sphinxAtStartPar
Bases: \sphinxcode{\sphinxupquote{Spectrum}}

\sphinxAtStartPar
Wrapper around the mpdaf Spectrum functionalities

\end{fulllineitems}

\index{PixTableToMask (class in pymusepipe.mpdaf\_pipe)@\spxentry{PixTableToMask}\spxextra{class in pymusepipe.mpdaf\_pipe}}

\begin{fulllineitems}
\phantomsection\label{\detokenize{api/pymusepipe:pymusepipe.mpdaf_pipe.PixTableToMask}}
\pysigstartsignatures
\pysiglinewithargsret{\sphinxbfcode{\sphinxupquote{class\DUrole{w,w}{  }}}\sphinxcode{\sphinxupquote{pymusepipe.mpdaf\_pipe.}}\sphinxbfcode{\sphinxupquote{PixTableToMask}}}{\sphinxparam{\DUrole{n,n}{pixtable\_name}}, \sphinxparam{\DUrole{n,n}{image\_name}}, \sphinxparam{\DUrole{n,n}{suffix\_out}\DUrole{o,o}{=}\DUrole{default_value}{\textquotesingle{}tmask\textquotesingle{}}}}{}
\pysigstopsignatures
\sphinxAtStartPar
Bases: \sphinxhref{https://docs.python.org/3.10/library/functions.html\#object}{\sphinxcode{\sphinxupquote{object}}}

\sphinxAtStartPar
This class is meant to just be a simple tool to
mask out some regions from the PixTable using Image masks
\index{create\_mask() (pymusepipe.mpdaf\_pipe.PixTableToMask method)@\spxentry{create\_mask()}\spxextra{pymusepipe.mpdaf\_pipe.PixTableToMask method}}

\begin{fulllineitems}
\phantomsection\label{\detokenize{api/pymusepipe:pymusepipe.mpdaf_pipe.PixTableToMask.create_mask}}
\pysigstartsignatures
\pysiglinewithargsret{\sphinxbfcode{\sphinxupquote{create\_mask}}}{\sphinxparam{\DUrole{n,n}{pq1}\DUrole{o,o}{=}\DUrole{default_value}{{[}0, 0{]}}}, \sphinxparam{\DUrole{n,n}{pq2}\DUrole{o,o}{=}\DUrole{default_value}{{[}10, 10{]}}}, \sphinxparam{\DUrole{n,n}{width}\DUrole{o,o}{=}\DUrole{default_value}{0.0}}, \sphinxparam{\DUrole{n,n}{reset}\DUrole{o,o}{=}\DUrole{default_value}{False}}, \sphinxparam{\DUrole{n,n}{mask\_name}\DUrole{o,o}{=}\DUrole{default_value}{\textquotesingle{}dummy\_mask.fits\textquotesingle{}}}, \sphinxparam{\DUrole{n,n}{extent}\DUrole{o,o}{=}\DUrole{default_value}{None}}, \sphinxparam{\DUrole{o,o}{**}\DUrole{n,n}{kwargs}}}{}
\pysigstopsignatures
\sphinxAtStartPar
Create the mask and save it in one go


\paragraph{Input}
\label{\detokenize{api/pymusepipe:id98}}\begin{description}
\sphinxlineitem{pq1: array or tuple (float)}
\sphinxAtStartPar
p and q coordinates of point 1 along the trail

\sphinxlineitem{pq2: array or tuple (float)}
\sphinxAtStartPar
p and q coordinates of point 2 along the trail

\sphinxlineitem{width: float}
\sphinxAtStartPar
Value (in pixel) of the full slit width to exclude

\sphinxlineitem{reset: bool}
\sphinxAtStartPar
By default False, so the mask goes on top of the existing one
If True, will reset the mask before building it.

\sphinxlineitem{extent: float}
\sphinxAtStartPar
Value (in pixel) to extend the slit beyond the 2 extrema
If 0, this means limiting it to the extrema themselves.
Default is None, which mean infinitely long slit

\end{description}

\end{fulllineitems}

\index{imshow() (pymusepipe.mpdaf\_pipe.PixTableToMask method)@\spxentry{imshow()}\spxextra{pymusepipe.mpdaf\_pipe.PixTableToMask method}}

\begin{fulllineitems}
\phantomsection\label{\detokenize{api/pymusepipe:pymusepipe.mpdaf_pipe.PixTableToMask.imshow}}
\pysigstartsignatures
\pysiglinewithargsret{\sphinxbfcode{\sphinxupquote{imshow}}}{\sphinxparam{\DUrole{o,o}{**}\DUrole{n,n}{kwargs}}}{}
\pysigstopsignatures
\sphinxAtStartPar
Just showing the image

\end{fulllineitems}

\index{mask\_pixtable() (pymusepipe.mpdaf\_pipe.PixTableToMask method)@\spxentry{mask\_pixtable()}\spxextra{pymusepipe.mpdaf\_pipe.PixTableToMask method}}

\begin{fulllineitems}
\phantomsection\label{\detokenize{api/pymusepipe:pymusepipe.mpdaf_pipe.PixTableToMask.mask_pixtable}}
\pysigstartsignatures
\pysiglinewithargsret{\sphinxbfcode{\sphinxupquote{mask\_pixtable}}}{\sphinxparam{\DUrole{n,n}{mask\_name}\DUrole{o,o}{=}\DUrole{default_value}{None}}, \sphinxparam{\DUrole{o,o}{**}\DUrole{n,n}{kwargs}}}{}
\pysigstopsignatures
\sphinxAtStartPar
Use the Image Mask and create a new Pixtable


\paragraph{Input}
\label{\detokenize{api/pymusepipe:id99}}\begin{description}
\sphinxlineitem{mask\_name: str}
\sphinxAtStartPar
Name of the mask to be used (FITS file)

\sphinxlineitem{use\_folder: bool}
\sphinxAtStartPar
If True, use the same folder as the Pixtable
Otherwise just write where you stand

\sphinxlineitem{suffix\_out: str}
\sphinxAtStartPar
Suffix for the name of the output Pixtable
If provided, will overwrite the one in self.suffix\_out

\end{description}

\end{fulllineitems}

\index{save\_mask() (pymusepipe.mpdaf\_pipe.PixTableToMask method)@\spxentry{save\_mask()}\spxextra{pymusepipe.mpdaf\_pipe.PixTableToMask method}}

\begin{fulllineitems}
\phantomsection\label{\detokenize{api/pymusepipe:pymusepipe.mpdaf_pipe.PixTableToMask.save_mask}}
\pysigstartsignatures
\pysiglinewithargsret{\sphinxbfcode{\sphinxupquote{save\_mask}}}{\sphinxparam{\DUrole{n,n}{mask\_name}\DUrole{o,o}{=}\DUrole{default_value}{\textquotesingle{}dummy\_mask.fits\textquotesingle{}}}, \sphinxparam{\DUrole{n,n}{use\_folder}\DUrole{o,o}{=}\DUrole{default_value}{True}}}{}
\pysigstopsignatures
\sphinxAtStartPar
Saving the mask from the Image into a fits file


\paragraph{Input}
\label{\detokenize{api/pymusepipe:id100}}\begin{description}
\sphinxlineitem{mask\_name: str}
\sphinxAtStartPar
Name of the fits file for the mask

\sphinxlineitem{use\_folder: bool}
\sphinxAtStartPar
If True (default) will look for the mask in the image\_folder.
If False, will just look for it where the command is run.

\end{description}


\paragraph{Creates}
\label{\detokenize{api/pymusepipe:id101}}
\sphinxAtStartPar
A fits file with the mask as 0 and 1

\end{fulllineitems}


\end{fulllineitems}

\index{get\_sky\_spectrum() (in module pymusepipe.mpdaf\_pipe)@\spxentry{get\_sky\_spectrum()}\spxextra{in module pymusepipe.mpdaf\_pipe}}

\begin{fulllineitems}
\phantomsection\label{\detokenize{api/pymusepipe:pymusepipe.mpdaf_pipe.get_sky_spectrum}}
\pysigstartsignatures
\pysiglinewithargsret{\sphinxcode{\sphinxupquote{pymusepipe.mpdaf\_pipe.}}\sphinxbfcode{\sphinxupquote{get\_sky\_spectrum}}}{\sphinxparam{\DUrole{n,n}{specname}}}{}
\pysigstopsignatures
\sphinxAtStartPar
Read sky spectrum from MUSE data reduction

\end{fulllineitems}

\index{integrate\_spectrum() (in module pymusepipe.mpdaf\_pipe)@\spxentry{integrate\_spectrum()}\spxextra{in module pymusepipe.mpdaf\_pipe}}

\begin{fulllineitems}
\phantomsection\label{\detokenize{api/pymusepipe:pymusepipe.mpdaf_pipe.integrate_spectrum}}
\pysigstartsignatures
\pysiglinewithargsret{\sphinxcode{\sphinxupquote{pymusepipe.mpdaf\_pipe.}}\sphinxbfcode{\sphinxupquote{integrate\_spectrum}}}{\sphinxparam{\DUrole{n,n}{spectrum}}, \sphinxparam{\DUrole{n,n}{wave\_filter}}, \sphinxparam{\DUrole{n,n}{throughput\_filter}}, \sphinxparam{\DUrole{n,n}{ao\_mask}\DUrole{o,o}{=}\DUrole{default_value}{False}}}{}
\pysigstopsignatures
\sphinxAtStartPar
Integrate a spectrum using a certain Muse Filter file.


\paragraph{Input}
\label{\detokenize{api/pymusepipe:id102}}\begin{description}
\sphinxlineitem{spectrum: Spectrum}
\sphinxAtStartPar
Input spectrum given as an mpdaf Spectrum

\sphinxlineitem{wave\_filter: float array}
\sphinxAtStartPar
Array of wavelength for the filter

\sphinxlineitem{throughput\_filter: float array}
\sphinxAtStartPar
Array of throughput (between 0 and 1) for the filter. Should be the
same dimension (1D, N floats) as wave\_filter

\end{description}

\end{fulllineitems}

\index{rotate\_cube\_wcs() (in module pymusepipe.mpdaf\_pipe)@\spxentry{rotate\_cube\_wcs()}\spxextra{in module pymusepipe.mpdaf\_pipe}}

\begin{fulllineitems}
\phantomsection\label{\detokenize{api/pymusepipe:pymusepipe.mpdaf_pipe.rotate_cube_wcs}}
\pysigstartsignatures
\pysiglinewithargsret{\sphinxcode{\sphinxupquote{pymusepipe.mpdaf\_pipe.}}\sphinxbfcode{\sphinxupquote{rotate\_cube\_wcs}}}{\sphinxparam{\DUrole{n,n}{cube\_name}}, \sphinxparam{\DUrole{n,n}{cube\_folder}\DUrole{o,o}{=}\DUrole{default_value}{\textquotesingle{}\textquotesingle{}}}, \sphinxparam{\DUrole{n,n}{outwcs\_folder}\DUrole{o,o}{=}\DUrole{default_value}{None}}, \sphinxparam{\DUrole{n,n}{rotangle}\DUrole{o,o}{=}\DUrole{default_value}{0.0}}, \sphinxparam{\DUrole{o,o}{**}\DUrole{n,n}{kwargs}}}{}
\pysigstopsignatures
\sphinxAtStartPar
Routine to remove potential Nan around an image and reconstruct
an optimal WCS reference image. The rotation angle is provided as a way
to optimise the extent of the output image, removing Nan along X and Y
at that angle.
\begin{quote}\begin{description}
\sphinxlineitem{Parameters}\begin{itemize}
\item {} 
\sphinxAtStartPar
\sphinxstyleliteralstrong{\sphinxupquote{cube\_name}} (\sphinxhref{https://docs.python.org/3.10/library/stdtypes.html\#str}{\sphinxstyleliteralemphasis{\sphinxupquote{str}}}) \textendash{} input image name. No default.

\item {} 
\sphinxAtStartPar
\sphinxstyleliteralstrong{\sphinxupquote{cube\_folder}} (\sphinxhref{https://docs.python.org/3.10/library/stdtypes.html\#str}{\sphinxstyleliteralemphasis{\sphinxupquote{str}}}) \textendash{} input image folder {[}‘’{]}

\item {} 
\sphinxAtStartPar
\sphinxstyleliteralstrong{\sphinxupquote{outwcs\_folder}} (\sphinxhref{https://docs.python.org/3.10/library/stdtypes.html\#str}{\sphinxstyleliteralemphasis{\sphinxupquote{str}}}) \textendash{} folder where to write the output frame. Default is
None which means that it will use the folder of the input image.

\item {} 
\sphinxAtStartPar
\sphinxstyleliteralstrong{\sphinxupquote{rotangle}} (\sphinxhref{https://docs.python.org/3.10/library/functions.html\#float}{\sphinxstyleliteralemphasis{\sphinxupquote{float}}}) \textendash{} rotation angle in degrees {[}0{]}

\item {} 
\sphinxAtStartPar
\sphinxstyleliteralstrong{\sphinxupquote{**kwargs}} \textendash{} in\_suffix (str): in suffix to remove from name {[}‘prealign’{]}
out\_suffix (str): out suffix to add to name {[}‘rotwcs’{]}
margin\_factor (float): factor to extend the image {[}1.1{]}

\end{itemize}

\end{description}\end{quote}

\sphinxAtStartPar
Returns:

\end{fulllineitems}

\index{rotate\_image\_wcs() (in module pymusepipe.mpdaf\_pipe)@\spxentry{rotate\_image\_wcs()}\spxextra{in module pymusepipe.mpdaf\_pipe}}

\begin{fulllineitems}
\phantomsection\label{\detokenize{api/pymusepipe:pymusepipe.mpdaf_pipe.rotate_image_wcs}}
\pysigstartsignatures
\pysiglinewithargsret{\sphinxcode{\sphinxupquote{pymusepipe.mpdaf\_pipe.}}\sphinxbfcode{\sphinxupquote{rotate\_image\_wcs}}}{\sphinxparam{\DUrole{n,n}{ima\_name}}, \sphinxparam{\DUrole{n,n}{ima\_folder}\DUrole{o,o}{=}\DUrole{default_value}{\textquotesingle{}\textquotesingle{}}}, \sphinxparam{\DUrole{n,n}{outwcs\_folder}\DUrole{o,o}{=}\DUrole{default_value}{None}}, \sphinxparam{\DUrole{n,n}{rotangle}\DUrole{o,o}{=}\DUrole{default_value}{0.0}}, \sphinxparam{\DUrole{o,o}{**}\DUrole{n,n}{kwargs}}}{}
\pysigstopsignatures
\sphinxAtStartPar
Routine to remove potential Nan around an image and reconstruct
an optimal WCS reference image. The rotation angle is provided as a way
to optimise the extent of the output image, removing Nan along X and Y
at that angle.


\paragraph{Input}
\label{\detokenize{api/pymusepipe:id103}}\begin{description}
\sphinxlineitem{ima\_name: str}
\sphinxAtStartPar
input image name. No default.

\sphinxlineitem{ima\_folder: str default=’’, optional}
\sphinxAtStartPar
input image folder

\sphinxlineitem{outwcs\_folder: str, optional}
\sphinxAtStartPar
folder where to write the output frame. Default is
None which means that it will use the folder of the input image.

\sphinxlineitem{rotangle: float default=0, optional}
\sphinxAtStartPar
rotation angle in degrees

\sphinxlineitem{in\_suffix: str default=’prealign’}
\sphinxAtStartPar
in suffix to remove from name

\sphinxlineitem{out\_suffix: str default=’rotwcs’}
\sphinxAtStartPar
out suffix to add to name

\sphinxlineitem{margin\_factor: float}
\sphinxAtStartPar
factor to extend the image {[}1.1{]}

\end{description}

\end{fulllineitems}



\subsubsection{pymusepipe.musepipe module}
\label{\detokenize{api/pymusepipe:module-pymusepipe.musepipe}}\label{\detokenize{api/pymusepipe:pymusepipe-musepipe-module}}\index{module@\spxentry{module}!pymusepipe.musepipe@\spxentry{pymusepipe.musepipe}}\index{pymusepipe.musepipe@\spxentry{pymusepipe.musepipe}!module@\spxentry{module}}
\sphinxAtStartPar
MUSE\sphinxhyphen{}PHANGS core module.
This defines the main class (MusePipe) which can be used throughout this package.

\sphinxAtStartPar
This module is a complete rewrite of a pipeline wrapper for the MUSE dataset.
All classes and objects were refactored.

\sphinxAtStartPar
However, the starting point of this package has been initially
inspired by several pieces of python codes developed by various individiduals,
including Kyriakos and Martina from the GTO MUSE MAD team and further
rewritten by Mark van den Brok.
Hence: a big Thanks to all three for this!

\sphinxAtStartPar
Note that several python packages exist which would provide similar
(or better) functionalities.

\sphinxAtStartPar
Eric Emsellem adapted a version from early 2017, provided by Mark and adapted
it for the needs of the PHANGS project (PI Schinnerer). It was further
refactored starting from scratch but keeping a few initial ideas.
\index{MusePipe (class in pymusepipe.musepipe)@\spxentry{MusePipe}\spxextra{class in pymusepipe.musepipe}}

\begin{fulllineitems}
\phantomsection\label{\detokenize{api/pymusepipe:pymusepipe.musepipe.MusePipe}}
\pysigstartsignatures
\pysiglinewithargsret{\sphinxbfcode{\sphinxupquote{class\DUrole{w,w}{  }}}\sphinxcode{\sphinxupquote{pymusepipe.musepipe.}}\sphinxbfcode{\sphinxupquote{MusePipe}}}{\sphinxparam{\DUrole{n,n}{targetname}\DUrole{o,o}{=}\DUrole{default_value}{None}}, \sphinxparam{\DUrole{n,n}{dataset}\DUrole{o,o}{=}\DUrole{default_value}{1}}, \sphinxparam{\DUrole{n,n}{folder\_config}\DUrole{o,o}{=}\DUrole{default_value}{\textquotesingle{}Config/\textquotesingle{}}}, \sphinxparam{\DUrole{n,n}{rc\_filename}\DUrole{o,o}{=}\DUrole{default_value}{None}}, \sphinxparam{\DUrole{n,n}{cal\_filename}\DUrole{o,o}{=}\DUrole{default_value}{None}}, \sphinxparam{\DUrole{n,n}{log\_filename}\DUrole{o,o}{=}\DUrole{default_value}{\textquotesingle{}MusePipe.log\textquotesingle{}}}, \sphinxparam{\DUrole{n,n}{verbose}\DUrole{o,o}{=}\DUrole{default_value}{True}}, \sphinxparam{\DUrole{n,n}{musemode}\DUrole{o,o}{=}\DUrole{default_value}{\textquotesingle{}WFM\sphinxhyphen{}NOAO\sphinxhyphen{}N\textquotesingle{}}}, \sphinxparam{\DUrole{n,n}{checkmode}\DUrole{o,o}{=}\DUrole{default_value}{True}}, \sphinxparam{\DUrole{n,n}{strong\_checkmode}\DUrole{o,o}{=}\DUrole{default_value}{False}}, \sphinxparam{\DUrole{o,o}{**}\DUrole{n,n}{kwargs}}}{}
\pysigstopsignatures
\sphinxAtStartPar
Bases: {\hyperref[\detokenize{api/pymusepipe:pymusepipe.prep_recipes_pipe.PipePrep}]{\sphinxcrossref{\sphinxcode{\sphinxupquote{PipePrep}}}}}, {\hyperref[\detokenize{api/pymusepipe:pymusepipe.recipes_pipe.PipeRecipes}]{\sphinxcrossref{\sphinxcode{\sphinxupquote{PipeRecipes}}}}}

\sphinxAtStartPar
Main Class to define and run the MUSE pipeline, given a certain galaxy
name. This is the main class used throughout the running of the pipeline
which contains functions and attributes all associated with the reduction
of MUSE exposures.

\sphinxAtStartPar
It inherits from the PipePrep class, which prepares the recipes for the
running of the MUSE pipeline, and Piperecipes which has the recipes
described.
\index{goto\_folder() (pymusepipe.musepipe.MusePipe method)@\spxentry{goto\_folder()}\spxextra{pymusepipe.musepipe.MusePipe method}}

\begin{fulllineitems}
\phantomsection\label{\detokenize{api/pymusepipe:pymusepipe.musepipe.MusePipe.goto_folder}}
\pysigstartsignatures
\pysiglinewithargsret{\sphinxbfcode{\sphinxupquote{goto\_folder}}}{\sphinxparam{\DUrole{n,n}{newpath}}, \sphinxparam{\DUrole{n,n}{addtolog}\DUrole{o,o}{=}\DUrole{default_value}{False}}}{}
\pysigstopsignatures
\sphinxAtStartPar
Changing directory and keeping memory of the old working one
\begin{quote}\begin{description}
\sphinxlineitem{Parameters}\begin{itemize}
\item {} 
\sphinxAtStartPar
\sphinxstyleliteralstrong{\sphinxupquote{newpath}} (\sphinxhref{https://docs.python.org/3.10/library/stdtypes.html\#str}{\sphinxstyleliteralemphasis{\sphinxupquote{str}}}) \textendash{} Path where to go to

\item {} 
\sphinxAtStartPar
\sphinxstyleliteralstrong{\sphinxupquote{addtolog}} (\sphinxhref{https://docs.python.org/3.10/library/functions.html\#bool}{\sphinxstyleliteralemphasis{\sphinxupquote{bool}}}\sphinxstyleliteralemphasis{\sphinxupquote{ {[}}}\sphinxstyleliteralemphasis{\sphinxupquote{False}}\sphinxstyleliteralemphasis{\sphinxupquote{{]}}}) \textendash{} Adding the folder move to the log file

\end{itemize}

\end{description}\end{quote}

\end{fulllineitems}

\index{goto\_origfolder() (pymusepipe.musepipe.MusePipe method)@\spxentry{goto\_origfolder()}\spxextra{pymusepipe.musepipe.MusePipe method}}

\begin{fulllineitems}
\phantomsection\label{\detokenize{api/pymusepipe:pymusepipe.musepipe.MusePipe.goto_origfolder}}
\pysigstartsignatures
\pysiglinewithargsret{\sphinxbfcode{\sphinxupquote{goto\_origfolder}}}{\sphinxparam{\DUrole{n,n}{addtolog}\DUrole{o,o}{=}\DUrole{default_value}{False}}}{}
\pysigstopsignatures
\sphinxAtStartPar
Go back to original folder

\end{fulllineitems}

\index{goto\_prevfolder() (pymusepipe.musepipe.MusePipe method)@\spxentry{goto\_prevfolder()}\spxextra{pymusepipe.musepipe.MusePipe method}}

\begin{fulllineitems}
\phantomsection\label{\detokenize{api/pymusepipe:pymusepipe.musepipe.MusePipe.goto_prevfolder}}
\pysigstartsignatures
\pysiglinewithargsret{\sphinxbfcode{\sphinxupquote{goto\_prevfolder}}}{\sphinxparam{\DUrole{n,n}{addtolog}\DUrole{o,o}{=}\DUrole{default_value}{False}}}{}
\pysigstopsignatures
\sphinxAtStartPar
Go back to previous folder
\begin{quote}\begin{description}
\sphinxlineitem{Parameters}
\sphinxAtStartPar
\sphinxstyleliteralstrong{\sphinxupquote{addtolog}} (\sphinxhref{https://docs.python.org/3.10/library/functions.html\#bool}{\sphinxstyleliteralemphasis{\sphinxupquote{bool}}}\sphinxstyleliteralemphasis{\sphinxupquote{ {[}}}\sphinxstyleliteralemphasis{\sphinxupquote{False}}\sphinxstyleliteralemphasis{\sphinxupquote{{]}}}) \textendash{} Adding the folder move to the log file

\end{description}\end{quote}

\end{fulllineitems}

\index{init\_raw\_table() (pymusepipe.musepipe.MusePipe method)@\spxentry{init\_raw\_table()}\spxextra{pymusepipe.musepipe.MusePipe method}}

\begin{fulllineitems}
\phantomsection\label{\detokenize{api/pymusepipe:pymusepipe.musepipe.MusePipe.init_raw_table}}
\pysigstartsignatures
\pysiglinewithargsret{\sphinxbfcode{\sphinxupquote{init\_raw\_table}}}{\sphinxparam{\DUrole{n,n}{reset}\DUrole{o,o}{=}\DUrole{default_value}{False}}, \sphinxparam{\DUrole{o,o}{**}\DUrole{n,n}{kwargs}}}{}
\pysigstopsignatures
\sphinxAtStartPar
Create a fits table with all the information from
the Raw files. Also create an astropy table with the same info
\begin{quote}\begin{description}
\sphinxlineitem{Parameters}
\sphinxAtStartPar
\sphinxstyleliteralstrong{\sphinxupquote{reset}} (\sphinxhref{https://docs.python.org/3.10/library/functions.html\#bool}{\sphinxstyleliteralemphasis{\sphinxupquote{bool}}}\sphinxstyleliteralemphasis{\sphinxupquote{ {[}}}\sphinxstyleliteralemphasis{\sphinxupquote{False}}\sphinxstyleliteralemphasis{\sphinxupquote{{]}}}) \textendash{} Resetting the raw astropy table if True

\end{description}\end{quote}

\end{fulllineitems}

\index{musemode (pymusepipe.musepipe.MusePipe property)@\spxentry{musemode}\spxextra{pymusepipe.musepipe.MusePipe property}}

\begin{fulllineitems}
\phantomsection\label{\detokenize{api/pymusepipe:pymusepipe.musepipe.MusePipe.musemode}}
\pysigstartsignatures
\pysigline{\sphinxbfcode{\sphinxupquote{property\DUrole{w,w}{  }}}\sphinxbfcode{\sphinxupquote{musemode}}}
\pysigstopsignatures
\sphinxAtStartPar
Mode for MUSE

\end{fulllineitems}

\index{print\_musemodes() (pymusepipe.musepipe.MusePipe method)@\spxentry{print\_musemodes()}\spxextra{pymusepipe.musepipe.MusePipe method}}

\begin{fulllineitems}
\phantomsection\label{\detokenize{api/pymusepipe:pymusepipe.musepipe.MusePipe.print_musemodes}}
\pysigstartsignatures
\pysiglinewithargsret{\sphinxbfcode{\sphinxupquote{print\_musemodes}}}{}{}
\pysigstopsignatures
\sphinxAtStartPar
Print out the list of allowed muse modes

\end{fulllineitems}

\index{read\_all\_astro\_tables() (pymusepipe.musepipe.MusePipe method)@\spxentry{read\_all\_astro\_tables()}\spxextra{pymusepipe.musepipe.MusePipe method}}

\begin{fulllineitems}
\phantomsection\label{\detokenize{api/pymusepipe:pymusepipe.musepipe.MusePipe.read_all_astro_tables}}
\pysigstartsignatures
\pysiglinewithargsret{\sphinxbfcode{\sphinxupquote{read\_all\_astro\_tables}}}{\sphinxparam{\DUrole{n,n}{reset}\DUrole{o,o}{=}\DUrole{default_value}{False}}}{}
\pysigstopsignatures
\sphinxAtStartPar
Initialise all existing Astropy Tables

\end{fulllineitems}

\index{read\_astropy\_table() (pymusepipe.musepipe.MusePipe method)@\spxentry{read\_astropy\_table()}\spxextra{pymusepipe.musepipe.MusePipe method}}

\begin{fulllineitems}
\phantomsection\label{\detokenize{api/pymusepipe:pymusepipe.musepipe.MusePipe.read_astropy_table}}
\pysigstartsignatures
\pysiglinewithargsret{\sphinxbfcode{\sphinxupquote{read\_astropy\_table}}}{\sphinxparam{\DUrole{n,n}{expotype}\DUrole{o,o}{=}\DUrole{default_value}{None}}, \sphinxparam{\DUrole{n,n}{stage}\DUrole{o,o}{=}\DUrole{default_value}{\textquotesingle{}master\textquotesingle{}}}}{}
\pysigstopsignatures
\sphinxAtStartPar
Read an existing Masterfile data table to start the pipeline

\end{fulllineitems}

\index{retrieve\_geoastro\_name() (pymusepipe.musepipe.MusePipe method)@\spxentry{retrieve\_geoastro\_name()}\spxextra{pymusepipe.musepipe.MusePipe method}}

\begin{fulllineitems}
\phantomsection\label{\detokenize{api/pymusepipe:pymusepipe.musepipe.MusePipe.retrieve_geoastro_name}}
\pysigstartsignatures
\pysiglinewithargsret{\sphinxbfcode{\sphinxupquote{retrieve\_geoastro\_name}}}{\sphinxparam{\DUrole{n,n}{date\_str}}, \sphinxparam{\DUrole{n,n}{filetype}\DUrole{o,o}{=}\DUrole{default_value}{\textquotesingle{}geo\textquotesingle{}}}, \sphinxparam{\DUrole{n,n}{fieldmode}\DUrole{o,o}{=}\DUrole{default_value}{\textquotesingle{}wfm\textquotesingle{}}}}{}
\pysigstopsignatures
\sphinxAtStartPar
Retrieving the astrometry or geometry fits file name
\begin{quote}\begin{description}
\sphinxlineitem{Parameters}\begin{itemize}
\item {} 
\sphinxAtStartPar
\sphinxstyleliteralstrong{\sphinxupquote{date\_str}} (\sphinxhref{https://docs.python.org/3.10/library/stdtypes.html\#str}{\sphinxstyleliteralemphasis{\sphinxupquote{str}}}) \textendash{} Date as a string (DD/MM/YYYY)

\item {} 
\sphinxAtStartPar
\sphinxstyleliteralstrong{\sphinxupquote{filetype}} (\sphinxhref{https://docs.python.org/3.10/library/stdtypes.html\#str}{\sphinxstyleliteralemphasis{\sphinxupquote{str}}}) \textendash{} ‘geo’ or ‘astro’, type of the needed file

\item {} 
\sphinxAtStartPar
\sphinxstyleliteralstrong{\sphinxupquote{fieldmode}} (\sphinxhref{https://docs.python.org/3.10/library/stdtypes.html\#str}{\sphinxstyleliteralemphasis{\sphinxupquote{str}}}) \textendash{} ‘wfm’ or ‘nfm’ \sphinxhyphen{} MUSE mode

\end{itemize}

\end{description}\end{quote}

\end{fulllineitems}

\index{save\_expo\_table() (pymusepipe.musepipe.MusePipe method)@\spxentry{save\_expo\_table()}\spxextra{pymusepipe.musepipe.MusePipe method}}

\begin{fulllineitems}
\phantomsection\label{\detokenize{api/pymusepipe:pymusepipe.musepipe.MusePipe.save_expo_table}}
\pysigstartsignatures
\pysiglinewithargsret{\sphinxbfcode{\sphinxupquote{save\_expo\_table}}}{\sphinxparam{\DUrole{n,n}{expotype}}, \sphinxparam{\DUrole{n,n}{tpl\_gtable}}, \sphinxparam{\DUrole{n,n}{stage}\DUrole{o,o}{=}\DUrole{default_value}{\textquotesingle{}master\textquotesingle{}}}, \sphinxparam{\DUrole{n,n}{fits\_tablename}\DUrole{o,o}{=}\DUrole{default_value}{None}}, \sphinxparam{\DUrole{n,n}{aggregate}\DUrole{o,o}{=}\DUrole{default_value}{True}}, \sphinxparam{\DUrole{n,n}{suffix}\DUrole{o,o}{=}\DUrole{default_value}{\textquotesingle{}\textquotesingle{}}}, \sphinxparam{\DUrole{n,n}{overwrite}\DUrole{o,o}{=}\DUrole{default_value}{None}}, \sphinxparam{\DUrole{n,n}{update}\DUrole{o,o}{=}\DUrole{default_value}{None}}}{}
\pysigstopsignatures
\sphinxAtStartPar
Save the Expo (Master or not) Table corresponding to the expotype

\end{fulllineitems}

\index{set\_fullpath\_names() (pymusepipe.musepipe.MusePipe method)@\spxentry{set\_fullpath\_names()}\spxextra{pymusepipe.musepipe.MusePipe method}}

\begin{fulllineitems}
\phantomsection\label{\detokenize{api/pymusepipe:pymusepipe.musepipe.MusePipe.set_fullpath_names}}
\pysigstartsignatures
\pysiglinewithargsret{\sphinxbfcode{\sphinxupquote{set\_fullpath\_names}}}{}{}
\pysigstopsignatures
\sphinxAtStartPar
Create full path names to be used

\end{fulllineitems}

\index{sort\_raw\_tables() (pymusepipe.musepipe.MusePipe method)@\spxentry{sort\_raw\_tables()}\spxextra{pymusepipe.musepipe.MusePipe method}}

\begin{fulllineitems}
\phantomsection\label{\detokenize{api/pymusepipe:pymusepipe.musepipe.MusePipe.sort_raw_tables}}
\pysigstartsignatures
\pysiglinewithargsret{\sphinxbfcode{\sphinxupquote{sort\_raw\_tables}}}{\sphinxparam{\DUrole{n,n}{checkmode}\DUrole{o,o}{=}\DUrole{default_value}{None}}, \sphinxparam{\DUrole{n,n}{strong\_checkmode}\DUrole{o,o}{=}\DUrole{default_value}{None}}}{}
\pysigstopsignatures
\sphinxAtStartPar
Provide lists of exposures with types defined in the dictionary
after excluding those with the wrong MUSE mode if checkmode is set up.


\paragraph{Input}
\label{\detokenize{api/pymusepipe:id104}}\begin{description}
\sphinxlineitem{checkmode: boolean}
\sphinxAtStartPar
Checking the MUSE mode or not. Default to None, namely it won’t use
the value set here but the value predefined in self.checkmode.

\sphinxlineitem{strong\_checkmode: boolean}
\sphinxAtStartPar
Strong check, namely in case you still wish to force the MUSE mode
even for files which are not mode specific (e.g., BIAS).
Default to None, namely it uses the self.strong\_checkmode which was
already set up at start.

\end{description}

\end{fulllineitems}


\end{fulllineitems}



\subsubsection{pymusepipe.prep\_recipes\_pipe module}
\label{\detokenize{api/pymusepipe:module-pymusepipe.prep_recipes_pipe}}\label{\detokenize{api/pymusepipe:pymusepipe-prep-recipes-pipe-module}}\index{module@\spxentry{module}!pymusepipe.prep\_recipes\_pipe@\spxentry{pymusepipe.prep\_recipes\_pipe}}\index{pymusepipe.prep\_recipes\_pipe@\spxentry{pymusepipe.prep\_recipes\_pipe}!module@\spxentry{module}}
\sphinxAtStartPar
MUSE\sphinxhyphen{}PHANGS preparation recipe module
\index{PipePrep (class in pymusepipe.prep\_recipes\_pipe)@\spxentry{PipePrep}\spxextra{class in pymusepipe.prep\_recipes\_pipe}}

\begin{fulllineitems}
\phantomsection\label{\detokenize{api/pymusepipe:pymusepipe.prep_recipes_pipe.PipePrep}}
\pysigstartsignatures
\pysiglinewithargsret{\sphinxbfcode{\sphinxupquote{class\DUrole{w,w}{  }}}\sphinxcode{\sphinxupquote{pymusepipe.prep\_recipes\_pipe.}}\sphinxbfcode{\sphinxupquote{PipePrep}}}{\sphinxparam{\DUrole{n,n}{first\_recipe}\DUrole{o,o}{=}\DUrole{default_value}{1}}, \sphinxparam{\DUrole{n,n}{last\_recipe}\DUrole{o,o}{=}\DUrole{default_value}{None}}}{}
\pysigstopsignatures
\sphinxAtStartPar
Bases: {\hyperref[\detokenize{api/pymusepipe:pymusepipe.create_sof.SofPipe}]{\sphinxcrossref{\sphinxcode{\sphinxupquote{SofPipe}}}}}

\sphinxAtStartPar
PipePrep class prepare the SOF files and launch the recipes
\index{get\_align\_group() (pymusepipe.prep\_recipes\_pipe.PipePrep method)@\spxentry{get\_align\_group()}\spxextra{pymusepipe.prep\_recipes\_pipe.PipePrep method}}

\begin{fulllineitems}
\phantomsection\label{\detokenize{api/pymusepipe:pymusepipe.prep_recipes_pipe.PipePrep.get_align_group}}
\pysigstartsignatures
\pysiglinewithargsret{\sphinxbfcode{\sphinxupquote{get\_align\_group}}}{\sphinxparam{\DUrole{n,n}{name\_ima\_reference}\DUrole{o,o}{=}\DUrole{default_value}{None}}, \sphinxparam{\DUrole{n,n}{list\_expo}\DUrole{o,o}{=}\DUrole{default_value}{{[}{]}}}, \sphinxparam{\DUrole{n,n}{line}\DUrole{o,o}{=}\DUrole{default_value}{None}}, \sphinxparam{\DUrole{n,n}{suffix}\DUrole{o,o}{=}\DUrole{default_value}{\textquotesingle{}\textquotesingle{}}}, \sphinxparam{\DUrole{n,n}{bygroup}\DUrole{o,o}{=}\DUrole{default_value}{False}}, \sphinxparam{\DUrole{o,o}{**}\DUrole{n,n}{kwargs}}}{}
\pysigstopsignatures
\sphinxAtStartPar
Extract the needed information for a set of exposures to be aligned

\end{fulllineitems}

\index{print\_recipes() (pymusepipe.prep\_recipes\_pipe.PipePrep static method)@\spxentry{print\_recipes()}\spxextra{pymusepipe.prep\_recipes\_pipe.PipePrep static method}}

\begin{fulllineitems}
\phantomsection\label{\detokenize{api/pymusepipe:pymusepipe.prep_recipes_pipe.PipePrep.print_recipes}}
\pysigstartsignatures
\pysiglinewithargsret{\sphinxbfcode{\sphinxupquote{static\DUrole{w,w}{  }}}\sphinxbfcode{\sphinxupquote{print\_recipes}}}{}{}
\pysigstopsignatures
\sphinxAtStartPar
Printing the list of recipes

\end{fulllineitems}

\index{run\_align\_bydataset() (pymusepipe.prep\_recipes\_pipe.PipePrep method)@\spxentry{run\_align\_bydataset()}\spxextra{pymusepipe.prep\_recipes\_pipe.PipePrep method}}

\begin{fulllineitems}
\phantomsection\label{\detokenize{api/pymusepipe:pymusepipe.prep_recipes_pipe.PipePrep.run_align_bydataset}}
\pysigstartsignatures
\pysiglinewithargsret{\sphinxbfcode{\sphinxupquote{run\_align\_bydataset}}}{\sphinxparam{\DUrole{n,n}{sof\_filename}\DUrole{o,o}{=}\DUrole{default_value}{\textquotesingle{}exp\_align\_bydataset\textquotesingle{}}}, \sphinxparam{\DUrole{n,n}{expotype}\DUrole{o,o}{=}\DUrole{default_value}{\textquotesingle{}OBJECT\textquotesingle{}}}, \sphinxparam{\DUrole{n,n}{list\_expo}\DUrole{o,o}{=}\DUrole{default_value}{{[}{]}}}, \sphinxparam{\DUrole{n,n}{stage}\DUrole{o,o}{=}\DUrole{default_value}{\textquotesingle{}processed\textquotesingle{}}}, \sphinxparam{\DUrole{n,n}{line}\DUrole{o,o}{=}\DUrole{default_value}{None}}, \sphinxparam{\DUrole{n,n}{suffix}\DUrole{o,o}{=}\DUrole{default_value}{\textquotesingle{}\textquotesingle{}}}, \sphinxparam{\DUrole{n,n}{tpl}\DUrole{o,o}{=}\DUrole{default_value}{\textquotesingle{}ALL\textquotesingle{}}}, \sphinxparam{\DUrole{o,o}{**}\DUrole{n,n}{kwargs}}}{}
\pysigstopsignatures
\sphinxAtStartPar
Aligning the individual exposures from a dataset
using the emission line region
With the muse exp\_align routine
\begin{quote}\begin{description}
\sphinxlineitem{Parameters}\begin{itemize}
\item {} 
\sphinxAtStartPar
\sphinxstyleliteralstrong{\sphinxupquote{sof\_filename}} (\sphinxstyleliteralemphasis{\sphinxupquote{string}}\sphinxstyleliteralemphasis{\sphinxupquote{ (}}\sphinxstyleliteralemphasis{\sphinxupquote{without the file extension}}\sphinxstyleliteralemphasis{\sphinxupquote{)}}) \textendash{} Name of the SOF file which will contain the Bias frames

\item {} 
\sphinxAtStartPar
\sphinxstyleliteralstrong{\sphinxupquote{tpl}} (\sphinxstyleliteralemphasis{\sphinxupquote{ALL by default}}\sphinxstyleliteralemphasis{\sphinxupquote{ or }}\sphinxstyleliteralemphasis{\sphinxupquote{a special tpl time}}) \textendash{} 

\end{itemize}

\end{description}\end{quote}

\end{fulllineitems}

\index{run\_align\_bygroup() (pymusepipe.prep\_recipes\_pipe.PipePrep method)@\spxentry{run\_align\_bygroup()}\spxextra{pymusepipe.prep\_recipes\_pipe.PipePrep method}}

\begin{fulllineitems}
\phantomsection\label{\detokenize{api/pymusepipe:pymusepipe.prep_recipes_pipe.PipePrep.run_align_bygroup}}
\pysigstartsignatures
\pysiglinewithargsret{\sphinxbfcode{\sphinxupquote{run\_align\_bygroup}}}{\sphinxparam{\DUrole{n,n}{sof\_filename}\DUrole{o,o}{=}\DUrole{default_value}{\textquotesingle{}exp\_align\_bygroup\textquotesingle{}}}, \sphinxparam{\DUrole{n,n}{expotype}\DUrole{o,o}{=}\DUrole{default_value}{\textquotesingle{}OBJECT\textquotesingle{}}}, \sphinxparam{\DUrole{n,n}{list\_expo}\DUrole{o,o}{=}\DUrole{default_value}{{[}{]}}}, \sphinxparam{\DUrole{n,n}{stage}\DUrole{o,o}{=}\DUrole{default_value}{\textquotesingle{}processed\textquotesingle{}}}, \sphinxparam{\DUrole{n,n}{line}\DUrole{o,o}{=}\DUrole{default_value}{None}}, \sphinxparam{\DUrole{n,n}{suffix}\DUrole{o,o}{=}\DUrole{default_value}{\textquotesingle{}\textquotesingle{}}}, \sphinxparam{\DUrole{n,n}{tpl}\DUrole{o,o}{=}\DUrole{default_value}{\textquotesingle{}ALL\textquotesingle{}}}, \sphinxparam{\DUrole{o,o}{**}\DUrole{n,n}{kwargs}}}{}
\pysigstopsignatures
\sphinxAtStartPar
Aligning the individual exposures from a dataset
using the emission line region
With the muse exp\_align routine
\begin{quote}\begin{description}
\sphinxlineitem{Parameters}\begin{itemize}
\item {} 
\sphinxAtStartPar
\sphinxstyleliteralstrong{\sphinxupquote{sof\_filename}} (\sphinxstyleliteralemphasis{\sphinxupquote{string}}\sphinxstyleliteralemphasis{\sphinxupquote{ (}}\sphinxstyleliteralemphasis{\sphinxupquote{without the file extension}}\sphinxstyleliteralemphasis{\sphinxupquote{)}}) \textendash{} Name of the SOF file which will contain the Bias frames

\item {} 
\sphinxAtStartPar
\sphinxstyleliteralstrong{\sphinxupquote{tpl}} (\sphinxstyleliteralemphasis{\sphinxupquote{ALL by default}}\sphinxstyleliteralemphasis{\sphinxupquote{ or }}\sphinxstyleliteralemphasis{\sphinxupquote{a special tpl time}}) \textendash{} 

\end{itemize}

\end{description}\end{quote}

\end{fulllineitems}

\index{run\_autocal\_sky() (pymusepipe.prep\_recipes\_pipe.PipePrep method)@\spxentry{run\_autocal\_sky()}\spxextra{pymusepipe.prep\_recipes\_pipe.PipePrep method}}

\begin{fulllineitems}
\phantomsection\label{\detokenize{api/pymusepipe:pymusepipe.prep_recipes_pipe.PipePrep.run_autocal_sky}}
\pysigstartsignatures
\pysiglinewithargsret{\sphinxbfcode{\sphinxupquote{run\_autocal\_sky}}}{\sphinxparam{\DUrole{n,n}{sof\_filename}\DUrole{o,o}{=}\DUrole{default_value}{\textquotesingle{}scipost\textquotesingle{}}}, \sphinxparam{\DUrole{n,n}{expotype}\DUrole{o,o}{=}\DUrole{default_value}{\textquotesingle{}SKY\textquotesingle{}}}, \sphinxparam{\DUrole{n,n}{AC\_suffix}\DUrole{o,o}{=}\DUrole{default_value}{\textquotesingle{}\_AC\textquotesingle{}}}, \sphinxparam{\DUrole{n,n}{tpl}\DUrole{o,o}{=}\DUrole{default_value}{\textquotesingle{}ALL\textquotesingle{}}}, \sphinxparam{\DUrole{o,o}{**}\DUrole{n,n}{extra\_kwargs}}}{}
\pysigstopsignatures
\sphinxAtStartPar
Launch the scipost command to get individual exposures in a narrow
band filter

\end{fulllineitems}

\index{run\_bias() (pymusepipe.prep\_recipes\_pipe.PipePrep method)@\spxentry{run\_bias()}\spxextra{pymusepipe.prep\_recipes\_pipe.PipePrep method}}

\begin{fulllineitems}
\phantomsection\label{\detokenize{api/pymusepipe:pymusepipe.prep_recipes_pipe.PipePrep.run_bias}}
\pysigstartsignatures
\pysiglinewithargsret{\sphinxbfcode{\sphinxupquote{run\_bias}}}{\sphinxparam{\DUrole{n,n}{sof\_filename}\DUrole{o,o}{=}\DUrole{default_value}{\textquotesingle{}bias\textquotesingle{}}}, \sphinxparam{\DUrole{n,n}{tpl}\DUrole{o,o}{=}\DUrole{default_value}{\textquotesingle{}ALL\textquotesingle{}}}, \sphinxparam{\DUrole{n,n}{update}\DUrole{o,o}{=}\DUrole{default_value}{None}}}{}
\pysigstopsignatures
\sphinxAtStartPar
Reducing the Bias files and creating a Master Bias
Will run the esorex muse\_bias command on all Biases
\begin{quote}\begin{description}
\sphinxlineitem{Parameters}\begin{itemize}
\item {} 
\sphinxAtStartPar
\sphinxstyleliteralstrong{\sphinxupquote{sof\_filename}} (\sphinxstyleliteralemphasis{\sphinxupquote{string}}\sphinxstyleliteralemphasis{\sphinxupquote{ (}}\sphinxstyleliteralemphasis{\sphinxupquote{without the file extension}}\sphinxstyleliteralemphasis{\sphinxupquote{)}}) \textendash{} Name of the SOF file which will contain the Bias frames

\item {} 
\sphinxAtStartPar
\sphinxstyleliteralstrong{\sphinxupquote{tpl}} (\sphinxstyleliteralemphasis{\sphinxupquote{ALL by default}}\sphinxstyleliteralemphasis{\sphinxupquote{ or }}\sphinxstyleliteralemphasis{\sphinxupquote{a special tpl time}}) \textendash{} 

\end{itemize}

\end{description}\end{quote}

\end{fulllineitems}

\index{run\_check\_align() (pymusepipe.prep\_recipes\_pipe.PipePrep method)@\spxentry{run\_check\_align()}\spxextra{pymusepipe.prep\_recipes\_pipe.PipePrep method}}

\begin{fulllineitems}
\phantomsection\label{\detokenize{api/pymusepipe:pymusepipe.prep_recipes_pipe.PipePrep.run_check_align}}
\pysigstartsignatures
\pysiglinewithargsret{\sphinxbfcode{\sphinxupquote{run\_check\_align}}}{\sphinxparam{\DUrole{n,n}{name\_offset\_table}}, \sphinxparam{\DUrole{n,n}{sof\_filename}\DUrole{o,o}{=}\DUrole{default_value}{\textquotesingle{}scipost\textquotesingle{}}}, \sphinxparam{\DUrole{n,n}{expotype}\DUrole{o,o}{=}\DUrole{default_value}{\textquotesingle{}OBJECT\textquotesingle{}}}, \sphinxparam{\DUrole{n,n}{tpl}\DUrole{o,o}{=}\DUrole{default_value}{\textquotesingle{}ALL\textquotesingle{}}}, \sphinxparam{\DUrole{n,n}{line}\DUrole{o,o}{=}\DUrole{default_value}{None}}, \sphinxparam{\DUrole{n,n}{suffix}\DUrole{o,o}{=}\DUrole{default_value}{\textquotesingle{}\textquotesingle{}}}, \sphinxparam{\DUrole{n,n}{folder\_offset\_table}\DUrole{o,o}{=}\DUrole{default_value}{None}}, \sphinxparam{\DUrole{o,o}{**}\DUrole{n,n}{extra\_kwargs}}}{}
\pysigstopsignatures
\sphinxAtStartPar
Launch the scipost command to get individual exposures in a narrow
band filter to check if the alignments are ok (after rotation
and using a given offset\_table)

\end{fulllineitems}

\index{run\_combine\_dataset() (pymusepipe.prep\_recipes\_pipe.PipePrep method)@\spxentry{run\_combine\_dataset()}\spxextra{pymusepipe.prep\_recipes\_pipe.PipePrep method}}

\begin{fulllineitems}
\phantomsection\label{\detokenize{api/pymusepipe:pymusepipe.prep_recipes_pipe.PipePrep.run_combine_dataset}}
\pysigstartsignatures
\pysiglinewithargsret{\sphinxbfcode{\sphinxupquote{run\_combine\_dataset}}}{\sphinxparam{\DUrole{n,n}{sof\_filename}\DUrole{o,o}{=}\DUrole{default_value}{\textquotesingle{}exp\_combine\textquotesingle{}}}, \sphinxparam{\DUrole{n,n}{expotype}\DUrole{o,o}{=}\DUrole{default_value}{\textquotesingle{}OBJECT\textquotesingle{}}}, \sphinxparam{\DUrole{n,n}{list\_expo}\DUrole{o,o}{=}\DUrole{default_value}{{[}{]}}}, \sphinxparam{\DUrole{n,n}{stage}\DUrole{o,o}{=}\DUrole{default_value}{\textquotesingle{}processed\textquotesingle{}}}, \sphinxparam{\DUrole{n,n}{tpl}\DUrole{o,o}{=}\DUrole{default_value}{\textquotesingle{}ALL\textquotesingle{}}}, \sphinxparam{\DUrole{n,n}{lambdaminmax}\DUrole{o,o}{=}\DUrole{default_value}{{[}4000.0, 10000.0{]}}}, \sphinxparam{\DUrole{n,n}{suffix}\DUrole{o,o}{=}\DUrole{default_value}{\textquotesingle{}\textquotesingle{}}}, \sphinxparam{\DUrole{o,o}{**}\DUrole{n,n}{kwargs}}}{}
\pysigstopsignatures
\sphinxAtStartPar
Produce a cube from all frames in the dataset
list\_expo or tpl specific arguments can still reduce the selection if needed

\end{fulllineitems}

\index{run\_fine\_alignment() (pymusepipe.prep\_recipes\_pipe.PipePrep method)@\spxentry{run\_fine\_alignment()}\spxextra{pymusepipe.prep\_recipes\_pipe.PipePrep method}}

\begin{fulllineitems}
\phantomsection\label{\detokenize{api/pymusepipe:pymusepipe.prep_recipes_pipe.PipePrep.run_fine_alignment}}
\pysigstartsignatures
\pysiglinewithargsret{\sphinxbfcode{\sphinxupquote{run\_fine\_alignment}}}{\sphinxparam{\DUrole{n,n}{name\_ima\_reference}\DUrole{o,o}{=}\DUrole{default_value}{None}}, \sphinxparam{\DUrole{n,n}{nexpo}\DUrole{o,o}{=}\DUrole{default_value}{1}}, \sphinxparam{\DUrole{n,n}{list\_expo}\DUrole{o,o}{=}\DUrole{default_value}{{[}{]}}}, \sphinxparam{\DUrole{n,n}{line}\DUrole{o,o}{=}\DUrole{default_value}{None}}, \sphinxparam{\DUrole{n,n}{bygroup}\DUrole{o,o}{=}\DUrole{default_value}{False}}, \sphinxparam{\DUrole{n,n}{reset}\DUrole{o,o}{=}\DUrole{default_value}{False}}, \sphinxparam{\DUrole{o,o}{**}\DUrole{n,n}{kwargs}}}{}
\pysigstopsignatures
\sphinxAtStartPar
Run the alignment on this dataset using or not a reference image

\end{fulllineitems}

\index{run\_flat() (pymusepipe.prep\_recipes\_pipe.PipePrep method)@\spxentry{run\_flat()}\spxextra{pymusepipe.prep\_recipes\_pipe.PipePrep method}}

\begin{fulllineitems}
\phantomsection\label{\detokenize{api/pymusepipe:pymusepipe.prep_recipes_pipe.PipePrep.run_flat}}
\pysigstartsignatures
\pysiglinewithargsret{\sphinxbfcode{\sphinxupquote{run\_flat}}}{\sphinxparam{\DUrole{n,n}{sof\_filename}\DUrole{o,o}{=}\DUrole{default_value}{\textquotesingle{}flat\textquotesingle{}}}, \sphinxparam{\DUrole{n,n}{tpl}\DUrole{o,o}{=}\DUrole{default_value}{\textquotesingle{}ALL\textquotesingle{}}}, \sphinxparam{\DUrole{n,n}{update}\DUrole{o,o}{=}\DUrole{default_value}{None}}}{}
\pysigstopsignatures
\sphinxAtStartPar
Reducing the Flat files and creating a Master Flat
Will run the esorex muse\_flat command on all Flats
\begin{quote}\begin{description}
\sphinxlineitem{Parameters}\begin{itemize}
\item {} 
\sphinxAtStartPar
\sphinxstyleliteralstrong{\sphinxupquote{sof\_filename}} (\sphinxstyleliteralemphasis{\sphinxupquote{string}}\sphinxstyleliteralemphasis{\sphinxupquote{ (}}\sphinxstyleliteralemphasis{\sphinxupquote{without the file extension}}\sphinxstyleliteralemphasis{\sphinxupquote{)}}) \textendash{} Name of the SOF file which will contain the Bias frames

\item {} 
\sphinxAtStartPar
\sphinxstyleliteralstrong{\sphinxupquote{tpl}} (\sphinxstyleliteralemphasis{\sphinxupquote{ALL by default}}\sphinxstyleliteralemphasis{\sphinxupquote{ or }}\sphinxstyleliteralemphasis{\sphinxupquote{a special tpl time}}) \textendash{} 

\end{itemize}

\end{description}\end{quote}

\end{fulllineitems}

\index{run\_lsf() (pymusepipe.prep\_recipes\_pipe.PipePrep method)@\spxentry{run\_lsf()}\spxextra{pymusepipe.prep\_recipes\_pipe.PipePrep method}}

\begin{fulllineitems}
\phantomsection\label{\detokenize{api/pymusepipe:pymusepipe.prep_recipes_pipe.PipePrep.run_lsf}}
\pysigstartsignatures
\pysiglinewithargsret{\sphinxbfcode{\sphinxupquote{run\_lsf}}}{\sphinxparam{\DUrole{n,n}{sof\_filename}\DUrole{o,o}{=}\DUrole{default_value}{\textquotesingle{}lsf\textquotesingle{}}}, \sphinxparam{\DUrole{n,n}{tpl}\DUrole{o,o}{=}\DUrole{default_value}{\textquotesingle{}ALL\textquotesingle{}}}, \sphinxparam{\DUrole{n,n}{update}\DUrole{o,o}{=}\DUrole{default_value}{None}}}{}
\pysigstopsignatures
\sphinxAtStartPar
Reducing the LSF files and creating the LSF PROFILE
Will run the esorex muse\_lsf command on all Flats
\begin{quote}\begin{description}
\sphinxlineitem{Parameters}\begin{itemize}
\item {} 
\sphinxAtStartPar
\sphinxstyleliteralstrong{\sphinxupquote{sof\_filename}} (\sphinxstyleliteralemphasis{\sphinxupquote{string}}\sphinxstyleliteralemphasis{\sphinxupquote{ (}}\sphinxstyleliteralemphasis{\sphinxupquote{without the file extension}}\sphinxstyleliteralemphasis{\sphinxupquote{)}}) \textendash{} Name of the SOF file which will contain the Bias frames

\item {} 
\sphinxAtStartPar
\sphinxstyleliteralstrong{\sphinxupquote{tpl}} (\sphinxstyleliteralemphasis{\sphinxupquote{ALL by default}}\sphinxstyleliteralemphasis{\sphinxupquote{ or }}\sphinxstyleliteralemphasis{\sphinxupquote{a special tpl time}}) \textendash{} 

\end{itemize}

\end{description}\end{quote}

\end{fulllineitems}

\index{run\_phangs\_recipes() (pymusepipe.prep\_recipes\_pipe.PipePrep method)@\spxentry{run\_phangs\_recipes()}\spxextra{pymusepipe.prep\_recipes\_pipe.PipePrep method}}

\begin{fulllineitems}
\phantomsection\label{\detokenize{api/pymusepipe:pymusepipe.prep_recipes_pipe.PipePrep.run_phangs_recipes}}
\pysigstartsignatures
\pysiglinewithargsret{\sphinxbfcode{\sphinxupquote{run\_phangs\_recipes}}}{\sphinxparam{\DUrole{n,n}{fraction}\DUrole{o,o}{=}\DUrole{default_value}{0.8}}, \sphinxparam{\DUrole{n,n}{illum}\DUrole{o,o}{=}\DUrole{default_value}{True}}, \sphinxparam{\DUrole{n,n}{skymethod}\DUrole{o,o}{=}\DUrole{default_value}{\textquotesingle{}model\textquotesingle{}}}, \sphinxparam{\DUrole{o,o}{**}\DUrole{n,n}{kwargs}}}{}
\pysigstopsignatures
\sphinxAtStartPar
Running all PHANGS recipes in one shot
Using the basic set up for the general list of recipes


\paragraph{Input}
\label{\detokenize{api/pymusepipe:id105}}\begin{description}
\sphinxlineitem{fraction: float}
\sphinxAtStartPar
Fraction of sky to consider in sky frames for the sky spectrum
Default is 0.8.

\sphinxlineitem{illum: bool}
\sphinxAtStartPar
Default is True (use illumination during twilight calibration)

\sphinxlineitem{skymethod: str}
\sphinxAtStartPar
Default is “model”.

\end{description}

\end{fulllineitems}

\index{run\_prep\_align() (pymusepipe.prep\_recipes\_pipe.PipePrep method)@\spxentry{run\_prep\_align()}\spxextra{pymusepipe.prep\_recipes\_pipe.PipePrep method}}

\begin{fulllineitems}
\phantomsection\label{\detokenize{api/pymusepipe:pymusepipe.prep_recipes_pipe.PipePrep.run_prep_align}}
\pysigstartsignatures
\pysiglinewithargsret{\sphinxbfcode{\sphinxupquote{run\_prep\_align}}}{\sphinxparam{\DUrole{n,n}{sof\_filename}\DUrole{o,o}{=}\DUrole{default_value}{\textquotesingle{}scipost\textquotesingle{}}}, \sphinxparam{\DUrole{n,n}{expotype}\DUrole{o,o}{=}\DUrole{default_value}{\textquotesingle{}OBJECT\textquotesingle{}}}, \sphinxparam{\DUrole{n,n}{tpl}\DUrole{o,o}{=}\DUrole{default_value}{\textquotesingle{}ALL\textquotesingle{}}}, \sphinxparam{\DUrole{n,n}{line}\DUrole{o,o}{=}\DUrole{default_value}{None}}, \sphinxparam{\DUrole{n,n}{suffix}\DUrole{o,o}{=}\DUrole{default_value}{\textquotesingle{}\textquotesingle{}}}, \sphinxparam{\DUrole{o,o}{**}\DUrole{n,n}{extra\_kwargs}}}{}
\pysigstopsignatures
\sphinxAtStartPar
Launch the scipost command to get individual exposures in a narrow
band filter

\end{fulllineitems}

\index{run\_recipes() (pymusepipe.prep\_recipes\_pipe.PipePrep method)@\spxentry{run\_recipes()}\spxextra{pymusepipe.prep\_recipes\_pipe.PipePrep method}}

\begin{fulllineitems}
\phantomsection\label{\detokenize{api/pymusepipe:pymusepipe.prep_recipes_pipe.PipePrep.run_recipes}}
\pysigstartsignatures
\pysiglinewithargsret{\sphinxbfcode{\sphinxupquote{run\_recipes}}}{\sphinxparam{\DUrole{o,o}{**}\DUrole{n,n}{kwargs}}}{}
\pysigstopsignatures
\sphinxAtStartPar
Running all recipes in one shot


\paragraph{Input}
\label{\detokenize{api/pymusepipe:id106}}\begin{description}
\sphinxlineitem{fraction: float}
\sphinxAtStartPar
Fraction of sky to consider in sky frames for the sky spectrum
Default is 0.8.

\sphinxlineitem{illum: bool}
\sphinxAtStartPar
Default is True (use illumination during twilight calibration)

\sphinxlineitem{skymethod: str}
\sphinxAtStartPar
Default is “model”.

\sphinxlineitem{filter\_for\_alignment: str}
\sphinxAtStartPar
Default is defined in config\_pipe

\sphinxlineitem{line: str}
\sphinxAtStartPar
Default is None as defined in config\_pipe

\sphinxlineitem{lambda\_window: float}
\sphinxAtStartPar
Default is 10.0 as defined in config\_pipe

\end{description}

\end{fulllineitems}

\index{run\_scibasic() (pymusepipe.prep\_recipes\_pipe.PipePrep method)@\spxentry{run\_scibasic()}\spxextra{pymusepipe.prep\_recipes\_pipe.PipePrep method}}

\begin{fulllineitems}
\phantomsection\label{\detokenize{api/pymusepipe:pymusepipe.prep_recipes_pipe.PipePrep.run_scibasic}}
\pysigstartsignatures
\pysiglinewithargsret{\sphinxbfcode{\sphinxupquote{run\_scibasic}}}{\sphinxparam{\DUrole{n,n}{sof\_filename}\DUrole{o,o}{=}\DUrole{default_value}{\textquotesingle{}scibasic\textquotesingle{}}}, \sphinxparam{\DUrole{n,n}{expotype}\DUrole{o,o}{=}\DUrole{default_value}{\textquotesingle{}OBJECT\textquotesingle{}}}, \sphinxparam{\DUrole{n,n}{tpl}\DUrole{o,o}{=}\DUrole{default_value}{\textquotesingle{}ALL\textquotesingle{}}}, \sphinxparam{\DUrole{n,n}{illum}\DUrole{o,o}{=}\DUrole{default_value}{True}}, \sphinxparam{\DUrole{n,n}{update}\DUrole{o,o}{=}\DUrole{default_value}{True}}, \sphinxparam{\DUrole{n,n}{overwrite}\DUrole{o,o}{=}\DUrole{default_value}{True}}}{}
\pysigstopsignatures
\sphinxAtStartPar
Reducing the files of a certain category and creating the PIXTABLES
Will run the esorex muse\_scibasic
\begin{quote}\begin{description}
\sphinxlineitem{Parameters}\begin{itemize}
\item {} 
\sphinxAtStartPar
\sphinxstyleliteralstrong{\sphinxupquote{sof\_filename}} (\sphinxstyleliteralemphasis{\sphinxupquote{string}}\sphinxstyleliteralemphasis{\sphinxupquote{ (}}\sphinxstyleliteralemphasis{\sphinxupquote{without the file extension}}\sphinxstyleliteralemphasis{\sphinxupquote{)}}) \textendash{} Name of the SOF file which will contain the Bias frames

\item {} 
\sphinxAtStartPar
\sphinxstyleliteralstrong{\sphinxupquote{tpl}} (\sphinxstyleliteralemphasis{\sphinxupquote{ALL by default}}\sphinxstyleliteralemphasis{\sphinxupquote{ or }}\sphinxstyleliteralemphasis{\sphinxupquote{a special tpl time}}) \textendash{} 

\end{itemize}

\end{description}\end{quote}

\end{fulllineitems}

\index{run\_scibasic\_all() (pymusepipe.prep\_recipes\_pipe.PipePrep method)@\spxentry{run\_scibasic\_all()}\spxextra{pymusepipe.prep\_recipes\_pipe.PipePrep method}}

\begin{fulllineitems}
\phantomsection\label{\detokenize{api/pymusepipe:pymusepipe.prep_recipes_pipe.PipePrep.run_scibasic_all}}
\pysigstartsignatures
\pysiglinewithargsret{\sphinxbfcode{\sphinxupquote{run\_scibasic\_all}}}{\sphinxparam{\DUrole{n,n}{list\_object}\DUrole{o,o}{=}\DUrole{default_value}{{[}\textquotesingle{}OBJECT\textquotesingle{}, \textquotesingle{}SKY\textquotesingle{}, \textquotesingle{}STD\textquotesingle{}{]}}}, \sphinxparam{\DUrole{n,n}{tpl}\DUrole{o,o}{=}\DUrole{default_value}{\textquotesingle{}ALL\textquotesingle{}}}, \sphinxparam{\DUrole{n,n}{illum}\DUrole{o,o}{=}\DUrole{default_value}{True}}, \sphinxparam{\DUrole{o,o}{**}\DUrole{n,n}{kwargs}}}{}
\pysigstopsignatures
\sphinxAtStartPar
Running scibasic for all objects in list\_object
Making different sof for each category

\end{fulllineitems}

\index{run\_scipost() (pymusepipe.prep\_recipes\_pipe.PipePrep method)@\spxentry{run\_scipost()}\spxextra{pymusepipe.prep\_recipes\_pipe.PipePrep method}}

\begin{fulllineitems}
\phantomsection\label{\detokenize{api/pymusepipe:pymusepipe.prep_recipes_pipe.PipePrep.run_scipost}}
\pysigstartsignatures
\pysiglinewithargsret{\sphinxbfcode{\sphinxupquote{run\_scipost}}}{\sphinxparam{\DUrole{n,n}{sof\_filename}\DUrole{o,o}{=}\DUrole{default_value}{\textquotesingle{}scipost\textquotesingle{}}}, \sphinxparam{\DUrole{n,n}{expotype}\DUrole{o,o}{=}\DUrole{default_value}{\textquotesingle{}OBJECT\textquotesingle{}}}, \sphinxparam{\DUrole{n,n}{tpl}\DUrole{o,o}{=}\DUrole{default_value}{\textquotesingle{}ALL\textquotesingle{}}}, \sphinxparam{\DUrole{n,n}{stage}\DUrole{o,o}{=}\DUrole{default_value}{\textquotesingle{}processed\textquotesingle{}}}, \sphinxparam{\DUrole{n,n}{list\_expo}\DUrole{o,o}{=}\DUrole{default_value}{{[}{]}}}, \sphinxparam{\DUrole{n,n}{lambdaminmax}\DUrole{o,o}{=}\DUrole{default_value}{{[}4000.0, 10000.0{]}}}, \sphinxparam{\DUrole{n,n}{suffix}\DUrole{o,o}{=}\DUrole{default_value}{\textquotesingle{}\textquotesingle{}}}, \sphinxparam{\DUrole{o,o}{**}\DUrole{n,n}{kwargs}}}{}
\pysigstopsignatures
\sphinxAtStartPar
Scipost treatment of the objects
Will run the esorex muse\_scipost routine


\paragraph{Input}
\label{\detokenize{api/pymusepipe:id107}}\begin{description}
\sphinxlineitem{sof\_filename: string (without the file extension)}
\sphinxAtStartPar
Name of the SOF file which will contain the Bias frames

\end{description}

\sphinxAtStartPar
tpl: ALL by default or a special tpl time
list\_expo: list of integers
\begin{quote}

\sphinxAtStartPar
Exposure numbers. By default, an empty list which means that all
exposures will be used.
\end{quote}
\begin{description}
\sphinxlineitem{lambdaminmax: tuple of 2 floats}
\sphinxAtStartPar
Minimum and Maximum wavelength to pass to the muse\_scipost recipe

\sphinxlineitem{suffix: str}
\sphinxAtStartPar
Suffix to add to the input pixtables.

\sphinxlineitem{norm\_skycontinuum: bool}
\sphinxAtStartPar
Normalise the skycontinuum or not. Default is False.
If normalisation is to be done, it will use the offset\_table
and the tabulated background value to renormalise the sky
continuum.

\sphinxlineitem{skymethod: str}
\sphinxAtStartPar
Type of skymethod. See MUSE manual.

\sphinxlineitem{offset\_list: bool}
\sphinxAtStartPar
If True, using an OFFSET list. Default is True.

\sphinxlineitem{name\_offset\_table: str}
\sphinxAtStartPar
Name of the offset table table. If not provided, will use the
default name produced during the pipeline run.

\sphinxlineitem{filter\_for\_alignment: str}
\sphinxAtStartPar
Name of the filter used for alignment.
Default is self.filter\_for\_alignment

\sphinxlineitem{filter\_list: str}
\sphinxAtStartPar
List of filters to be considered for reconstructed images.
By Default will use the list in self.filter\_list.

\end{description}

\end{fulllineitems}

\index{run\_scipost\_perexpo() (pymusepipe.prep\_recipes\_pipe.PipePrep method)@\spxentry{run\_scipost\_perexpo()}\spxextra{pymusepipe.prep\_recipes\_pipe.PipePrep method}}

\begin{fulllineitems}
\phantomsection\label{\detokenize{api/pymusepipe:pymusepipe.prep_recipes_pipe.PipePrep.run_scipost_perexpo}}
\pysigstartsignatures
\pysiglinewithargsret{\sphinxbfcode{\sphinxupquote{run\_scipost\_perexpo}}}{\sphinxparam{\DUrole{n,n}{sof\_filename}\DUrole{o,o}{=}\DUrole{default_value}{\textquotesingle{}scipost\textquotesingle{}}}, \sphinxparam{\DUrole{n,n}{expotype}\DUrole{o,o}{=}\DUrole{default_value}{\textquotesingle{}OBJECT\textquotesingle{}}}, \sphinxparam{\DUrole{n,n}{list\_tplexpo}\DUrole{o,o}{=}\DUrole{default_value}{\textquotesingle{}ALL\textquotesingle{}}}, \sphinxparam{\DUrole{n,n}{stage}\DUrole{o,o}{=}\DUrole{default_value}{\textquotesingle{}processed\textquotesingle{}}}, \sphinxparam{\DUrole{n,n}{suffix}\DUrole{o,o}{=}\DUrole{default_value}{\textquotesingle{}\textquotesingle{}}}, \sphinxparam{\DUrole{n,n}{offset\_list}\DUrole{o,o}{=}\DUrole{default_value}{False}}, \sphinxparam{\DUrole{o,o}{**}\DUrole{n,n}{kwargs}}}{}
\pysigstopsignatures
\sphinxAtStartPar
Launch the scipost command exposure per exposure


\paragraph{Input}
\label{\detokenize{api/pymusepipe:id108}}
\sphinxAtStartPar
See run\_scipost parameters

\end{fulllineitems}

\index{run\_scipost\_sky() (pymusepipe.prep\_recipes\_pipe.PipePrep method)@\spxentry{run\_scipost\_sky()}\spxextra{pymusepipe.prep\_recipes\_pipe.PipePrep method}}

\begin{fulllineitems}
\phantomsection\label{\detokenize{api/pymusepipe:pymusepipe.prep_recipes_pipe.PipePrep.run_scipost_sky}}
\pysigstartsignatures
\pysiglinewithargsret{\sphinxbfcode{\sphinxupquote{run\_scipost\_sky}}}{}{}
\pysigstopsignatures
\sphinxAtStartPar
Run scipost for the SKY with no offset list and no skymethod

\end{fulllineitems}

\index{run\_sky() (pymusepipe.prep\_recipes\_pipe.PipePrep method)@\spxentry{run\_sky()}\spxextra{pymusepipe.prep\_recipes\_pipe.PipePrep method}}

\begin{fulllineitems}
\phantomsection\label{\detokenize{api/pymusepipe:pymusepipe.prep_recipes_pipe.PipePrep.run_sky}}
\pysigstartsignatures
\pysiglinewithargsret{\sphinxbfcode{\sphinxupquote{run\_sky}}}{\sphinxparam{\DUrole{n,n}{sof\_filename}\DUrole{o,o}{=}\DUrole{default_value}{\textquotesingle{}sky\textquotesingle{}}}, \sphinxparam{\DUrole{n,n}{tpl}\DUrole{o,o}{=}\DUrole{default_value}{\textquotesingle{}ALL\textquotesingle{}}}, \sphinxparam{\DUrole{n,n}{fraction}\DUrole{o,o}{=}\DUrole{default_value}{0.8}}, \sphinxparam{\DUrole{n,n}{update}\DUrole{o,o}{=}\DUrole{default_value}{None}}, \sphinxparam{\DUrole{n,n}{overwrite}\DUrole{o,o}{=}\DUrole{default_value}{True}}}{}
\pysigstopsignatures
\sphinxAtStartPar
Reducing the SKY after they have been scibasic reduced
Will run the esorex muse\_create\_sky routine
\begin{quote}\begin{description}
\sphinxlineitem{Parameters}\begin{itemize}
\item {} 
\sphinxAtStartPar
\sphinxstyleliteralstrong{\sphinxupquote{sof\_filename}} (\sphinxstyleliteralemphasis{\sphinxupquote{string}}\sphinxstyleliteralemphasis{\sphinxupquote{ (}}\sphinxstyleliteralemphasis{\sphinxupquote{without the file extension}}\sphinxstyleliteralemphasis{\sphinxupquote{)}}) \textendash{} Name of the SOF file which will contain the Bias frames

\item {} 
\sphinxAtStartPar
\sphinxstyleliteralstrong{\sphinxupquote{tpl}} (\sphinxstyleliteralemphasis{\sphinxupquote{ALL by default}}\sphinxstyleliteralemphasis{\sphinxupquote{ or }}\sphinxstyleliteralemphasis{\sphinxupquote{a special tpl time}}) \textendash{} 

\end{itemize}

\end{description}\end{quote}

\end{fulllineitems}

\index{run\_standard() (pymusepipe.prep\_recipes\_pipe.PipePrep method)@\spxentry{run\_standard()}\spxextra{pymusepipe.prep\_recipes\_pipe.PipePrep method}}

\begin{fulllineitems}
\phantomsection\label{\detokenize{api/pymusepipe:pymusepipe.prep_recipes_pipe.PipePrep.run_standard}}
\pysigstartsignatures
\pysiglinewithargsret{\sphinxbfcode{\sphinxupquote{run\_standard}}}{\sphinxparam{\DUrole{n,n}{sof\_filename}\DUrole{o,o}{=}\DUrole{default_value}{\textquotesingle{}standard\textquotesingle{}}}, \sphinxparam{\DUrole{n,n}{tpl}\DUrole{o,o}{=}\DUrole{default_value}{\textquotesingle{}ALL\textquotesingle{}}}, \sphinxparam{\DUrole{n,n}{update}\DUrole{o,o}{=}\DUrole{default_value}{None}}, \sphinxparam{\DUrole{n,n}{overwrite}\DUrole{o,o}{=}\DUrole{default_value}{True}}}{}
\pysigstopsignatures
\sphinxAtStartPar
Reducing the STD files after they have been obtained
Running the muse\_standard routine
\begin{quote}\begin{description}
\sphinxlineitem{Parameters}\begin{itemize}
\item {} 
\sphinxAtStartPar
\sphinxstyleliteralstrong{\sphinxupquote{sof\_filename}} (\sphinxstyleliteralemphasis{\sphinxupquote{string}}\sphinxstyleliteralemphasis{\sphinxupquote{ (}}\sphinxstyleliteralemphasis{\sphinxupquote{without the file extension}}\sphinxstyleliteralemphasis{\sphinxupquote{)}}) \textendash{} Name of the SOF file which will contain the Bias frames

\item {} 
\sphinxAtStartPar
\sphinxstyleliteralstrong{\sphinxupquote{tpl}} (\sphinxstyleliteralemphasis{\sphinxupquote{ALL by default}}\sphinxstyleliteralemphasis{\sphinxupquote{ or }}\sphinxstyleliteralemphasis{\sphinxupquote{a special tpl time}}) \textendash{} 

\end{itemize}

\end{description}\end{quote}

\end{fulllineitems}

\index{run\_twilight() (pymusepipe.prep\_recipes\_pipe.PipePrep method)@\spxentry{run\_twilight()}\spxextra{pymusepipe.prep\_recipes\_pipe.PipePrep method}}

\begin{fulllineitems}
\phantomsection\label{\detokenize{api/pymusepipe:pymusepipe.prep_recipes_pipe.PipePrep.run_twilight}}
\pysigstartsignatures
\pysiglinewithargsret{\sphinxbfcode{\sphinxupquote{run\_twilight}}}{\sphinxparam{\DUrole{n,n}{sof\_filename}\DUrole{o,o}{=}\DUrole{default_value}{\textquotesingle{}twilight\textquotesingle{}}}, \sphinxparam{\DUrole{n,n}{tpl}\DUrole{o,o}{=}\DUrole{default_value}{\textquotesingle{}ALL\textquotesingle{}}}, \sphinxparam{\DUrole{n,n}{update}\DUrole{o,o}{=}\DUrole{default_value}{None}}, \sphinxparam{\DUrole{n,n}{illum}\DUrole{o,o}{=}\DUrole{default_value}{True}}}{}
\pysigstopsignatures
\sphinxAtStartPar
Reducing the  files and creating the TWILIGHT CUBE.
Will run the esorex muse\_twilight command on all TWILIGHT
\begin{quote}\begin{description}
\sphinxlineitem{Parameters}\begin{itemize}
\item {} 
\sphinxAtStartPar
\sphinxstyleliteralstrong{\sphinxupquote{sof\_filename}} (\sphinxstyleliteralemphasis{\sphinxupquote{string}}\sphinxstyleliteralemphasis{\sphinxupquote{ (}}\sphinxstyleliteralemphasis{\sphinxupquote{without the file extension}}\sphinxstyleliteralemphasis{\sphinxupquote{)}}) \textendash{} Name of the SOF file which will contain the Bias frames

\item {} 
\sphinxAtStartPar
\sphinxstyleliteralstrong{\sphinxupquote{tpl}} (\sphinxstyleliteralemphasis{\sphinxupquote{ALL by default}}\sphinxstyleliteralemphasis{\sphinxupquote{ or }}\sphinxstyleliteralemphasis{\sphinxupquote{a special tpl time}}) \textendash{} 

\end{itemize}

\end{description}\end{quote}

\end{fulllineitems}

\index{run\_wave() (pymusepipe.prep\_recipes\_pipe.PipePrep method)@\spxentry{run\_wave()}\spxextra{pymusepipe.prep\_recipes\_pipe.PipePrep method}}

\begin{fulllineitems}
\phantomsection\label{\detokenize{api/pymusepipe:pymusepipe.prep_recipes_pipe.PipePrep.run_wave}}
\pysigstartsignatures
\pysiglinewithargsret{\sphinxbfcode{\sphinxupquote{run\_wave}}}{\sphinxparam{\DUrole{n,n}{sof\_filename}\DUrole{o,o}{=}\DUrole{default_value}{\textquotesingle{}wave\textquotesingle{}}}, \sphinxparam{\DUrole{n,n}{tpl}\DUrole{o,o}{=}\DUrole{default_value}{\textquotesingle{}ALL\textquotesingle{}}}, \sphinxparam{\DUrole{n,n}{update}\DUrole{o,o}{=}\DUrole{default_value}{None}}}{}
\pysigstopsignatures
\sphinxAtStartPar
Reducing the WAVE\sphinxhyphen{}CAL files and creating the Master Wave
Will run the esorex muse\_wave command on all Flats
\begin{quote}\begin{description}
\sphinxlineitem{Parameters}\begin{itemize}
\item {} 
\sphinxAtStartPar
\sphinxstyleliteralstrong{\sphinxupquote{sof\_filename}} (\sphinxstyleliteralemphasis{\sphinxupquote{string}}\sphinxstyleliteralemphasis{\sphinxupquote{ (}}\sphinxstyleliteralemphasis{\sphinxupquote{without the file extension}}\sphinxstyleliteralemphasis{\sphinxupquote{)}}) \textendash{} Name of the SOF file which will contain the Bias frames

\item {} 
\sphinxAtStartPar
\sphinxstyleliteralstrong{\sphinxupquote{tpl}} (\sphinxstyleliteralemphasis{\sphinxupquote{ALL by default}}\sphinxstyleliteralemphasis{\sphinxupquote{ or }}\sphinxstyleliteralemphasis{\sphinxupquote{a special tpl time}}) \textendash{} 

\end{itemize}

\end{description}\end{quote}

\end{fulllineitems}

\index{save\_fine\_alignment() (pymusepipe.prep\_recipes\_pipe.PipePrep method)@\spxentry{save\_fine\_alignment()}\spxextra{pymusepipe.prep\_recipes\_pipe.PipePrep method}}

\begin{fulllineitems}
\phantomsection\label{\detokenize{api/pymusepipe:pymusepipe.prep_recipes_pipe.PipePrep.save_fine_alignment}}
\pysigstartsignatures
\pysiglinewithargsret{\sphinxbfcode{\sphinxupquote{save\_fine\_alignment}}}{\sphinxparam{\DUrole{n,n}{name\_offset\_table}\DUrole{o,o}{=}\DUrole{default_value}{None}}}{}
\pysigstopsignatures
\sphinxAtStartPar
Save the fine dataset alignment

\end{fulllineitems}

\index{select\_tpl\_files() (pymusepipe.prep\_recipes\_pipe.PipePrep method)@\spxentry{select\_tpl\_files()}\spxextra{pymusepipe.prep\_recipes\_pipe.PipePrep method}}

\begin{fulllineitems}
\phantomsection\label{\detokenize{api/pymusepipe:pymusepipe.prep_recipes_pipe.PipePrep.select_tpl_files}}
\pysigstartsignatures
\pysiglinewithargsret{\sphinxbfcode{\sphinxupquote{select\_tpl\_files}}}{\sphinxparam{\DUrole{n,n}{expotype}\DUrole{o,o}{=}\DUrole{default_value}{None}}, \sphinxparam{\DUrole{n,n}{tpl}\DUrole{o,o}{=}\DUrole{default_value}{\textquotesingle{}ALL\textquotesingle{}}}, \sphinxparam{\DUrole{n,n}{stage}\DUrole{o,o}{=}\DUrole{default_value}{\textquotesingle{}raw\textquotesingle{}}}}{}
\pysigstopsignatures
\sphinxAtStartPar
Selecting a subset of files from a certain type

\end{fulllineitems}


\end{fulllineitems}

\index{add\_listpath() (in module pymusepipe.prep\_recipes\_pipe)@\spxentry{add\_listpath()}\spxextra{in module pymusepipe.prep\_recipes\_pipe}}

\begin{fulllineitems}
\phantomsection\label{\detokenize{api/pymusepipe:pymusepipe.prep_recipes_pipe.add_listpath}}
\pysigstartsignatures
\pysiglinewithargsret{\sphinxcode{\sphinxupquote{pymusepipe.prep\_recipes\_pipe.}}\sphinxbfcode{\sphinxupquote{add\_listpath}}}{\sphinxparam{\DUrole{n,n}{suffix}}, \sphinxparam{\DUrole{n,n}{paths}}}{}
\pysigstopsignatures
\sphinxAtStartPar
Add a suffix to a list of path
and normalise them

\end{fulllineitems}

\index{norm\_listpath() (in module pymusepipe.prep\_recipes\_pipe)@\spxentry{norm\_listpath()}\spxextra{in module pymusepipe.prep\_recipes\_pipe}}

\begin{fulllineitems}
\phantomsection\label{\detokenize{api/pymusepipe:pymusepipe.prep_recipes_pipe.norm_listpath}}
\pysigstartsignatures
\pysiglinewithargsret{\sphinxcode{\sphinxupquote{pymusepipe.prep\_recipes\_pipe.}}\sphinxbfcode{\sphinxupquote{norm\_listpath}}}{\sphinxparam{\DUrole{n,n}{paths}}}{}
\pysigstopsignatures
\sphinxAtStartPar
Normalise the path for a list of paths

\end{fulllineitems}

\index{print\_my\_function\_name() (in module pymusepipe.prep\_recipes\_pipe)@\spxentry{print\_my\_function\_name()}\spxextra{in module pymusepipe.prep\_recipes\_pipe}}

\begin{fulllineitems}
\phantomsection\label{\detokenize{api/pymusepipe:pymusepipe.prep_recipes_pipe.print_my_function_name}}
\pysigstartsignatures
\pysiglinewithargsret{\sphinxcode{\sphinxupquote{pymusepipe.prep\_recipes\_pipe.}}\sphinxbfcode{\sphinxupquote{print\_my\_function\_name}}}{\sphinxparam{\DUrole{n,n}{f}}}{}
\pysigstopsignatures
\sphinxAtStartPar
Function to provide a print of the name of the function
Can be used as a decorator

\end{fulllineitems}



\subsubsection{pymusepipe.recipes\_pipe module}
\label{\detokenize{api/pymusepipe:module-pymusepipe.recipes_pipe}}\label{\detokenize{api/pymusepipe:pymusepipe-recipes-pipe-module}}\index{module@\spxentry{module}!pymusepipe.recipes\_pipe@\spxentry{pymusepipe.recipes\_pipe}}\index{pymusepipe.recipes\_pipe@\spxentry{pymusepipe.recipes\_pipe}!module@\spxentry{module}}
\sphinxAtStartPar
MUSE\sphinxhyphen{}PHANGS recipe module
\index{PipeRecipes (class in pymusepipe.recipes\_pipe)@\spxentry{PipeRecipes}\spxextra{class in pymusepipe.recipes\_pipe}}

\begin{fulllineitems}
\phantomsection\label{\detokenize{api/pymusepipe:pymusepipe.recipes_pipe.PipeRecipes}}
\pysigstartsignatures
\pysiglinewithargsret{\sphinxbfcode{\sphinxupquote{class\DUrole{w,w}{  }}}\sphinxcode{\sphinxupquote{pymusepipe.recipes\_pipe.}}\sphinxbfcode{\sphinxupquote{PipeRecipes}}}{\sphinxparam{\DUrole{n,n}{nifu}\DUrole{o,o}{=}\DUrole{default_value}{\sphinxhyphen{}1}}, \sphinxparam{\DUrole{n,n}{first\_cpu}\DUrole{o,o}{=}\DUrole{default_value}{0}}, \sphinxparam{\DUrole{n,n}{ncpu}\DUrole{o,o}{=}\DUrole{default_value}{24}}, \sphinxparam{\DUrole{n,n}{list\_cpu}\DUrole{o,o}{=}\DUrole{default_value}{{[}{]}}}, \sphinxparam{\DUrole{n,n}{likwid}\DUrole{o,o}{=}\DUrole{default_value}{\textquotesingle{}likwid\sphinxhyphen{}pin \sphinxhyphen{}c N:\textquotesingle{}}}, \sphinxparam{\DUrole{n,n}{fakemode}\DUrole{o,o}{=}\DUrole{default_value}{False}}, \sphinxparam{\DUrole{n,n}{domerge}\DUrole{o,o}{=}\DUrole{default_value}{True}}, \sphinxparam{\DUrole{n,n}{nocache}\DUrole{o,o}{=}\DUrole{default_value}{False}}, \sphinxparam{\DUrole{n,n}{nochecksum}\DUrole{o,o}{=}\DUrole{default_value}{True}}}{}
\pysigstopsignatures
\sphinxAtStartPar
Bases: \sphinxhref{https://docs.python.org/3.10/library/functions.html\#object}{\sphinxcode{\sphinxupquote{object}}}

\sphinxAtStartPar
PipeRecipes class containing all the esorex recipes for MUSE data reduction
\index{checksum (pymusepipe.recipes\_pipe.PipeRecipes property)@\spxentry{checksum}\spxextra{pymusepipe.recipes\_pipe.PipeRecipes property}}

\begin{fulllineitems}
\phantomsection\label{\detokenize{api/pymusepipe:pymusepipe.recipes_pipe.PipeRecipes.checksum}}
\pysigstartsignatures
\pysigline{\sphinxbfcode{\sphinxupquote{property\DUrole{w,w}{  }}}\sphinxbfcode{\sphinxupquote{checksum}}}
\pysigstopsignatures
\end{fulllineitems}

\index{esorex (pymusepipe.recipes\_pipe.PipeRecipes property)@\spxentry{esorex}\spxextra{pymusepipe.recipes\_pipe.PipeRecipes property}}

\begin{fulllineitems}
\phantomsection\label{\detokenize{api/pymusepipe:pymusepipe.recipes_pipe.PipeRecipes.esorex}}
\pysigstartsignatures
\pysigline{\sphinxbfcode{\sphinxupquote{property\DUrole{w,w}{  }}}\sphinxbfcode{\sphinxupquote{esorex}}}
\pysigstopsignatures
\end{fulllineitems}

\index{joinprod() (pymusepipe.recipes\_pipe.PipeRecipes method)@\spxentry{joinprod()}\spxextra{pymusepipe.recipes\_pipe.PipeRecipes method}}

\begin{fulllineitems}
\phantomsection\label{\detokenize{api/pymusepipe:pymusepipe.recipes_pipe.PipeRecipes.joinprod}}
\pysigstartsignatures
\pysiglinewithargsret{\sphinxbfcode{\sphinxupquote{joinprod}}}{\sphinxparam{\DUrole{n,n}{name}}}{}
\pysigstopsignatures
\end{fulllineitems}

\index{merge (pymusepipe.recipes\_pipe.PipeRecipes property)@\spxentry{merge}\spxextra{pymusepipe.recipes\_pipe.PipeRecipes property}}

\begin{fulllineitems}
\phantomsection\label{\detokenize{api/pymusepipe:pymusepipe.recipes_pipe.PipeRecipes.merge}}
\pysigstartsignatures
\pysigline{\sphinxbfcode{\sphinxupquote{property\DUrole{w,w}{  }}}\sphinxbfcode{\sphinxupquote{merge}}}
\pysigstopsignatures
\end{fulllineitems}

\index{recipe\_align() (pymusepipe.recipes\_pipe.PipeRecipes method)@\spxentry{recipe\_align()}\spxextra{pymusepipe.recipes\_pipe.PipeRecipes method}}

\begin{fulllineitems}
\phantomsection\label{\detokenize{api/pymusepipe:pymusepipe.recipes_pipe.PipeRecipes.recipe_align}}
\pysigstartsignatures
\pysiglinewithargsret{\sphinxbfcode{\sphinxupquote{recipe\_align}}}{\sphinxparam{\DUrole{n,n}{sof}}, \sphinxparam{\DUrole{n,n}{dir\_products}}, \sphinxparam{\DUrole{n,n}{namein\_products}}, \sphinxparam{\DUrole{n,n}{nameout\_products}}, \sphinxparam{\DUrole{n,n}{tpl}}, \sphinxparam{\DUrole{n,n}{group}}, \sphinxparam{\DUrole{n,n}{threshold}\DUrole{o,o}{=}\DUrole{default_value}{10.0}}, \sphinxparam{\DUrole{n,n}{srcmin}\DUrole{o,o}{=}\DUrole{default_value}{3}}, \sphinxparam{\DUrole{n,n}{srcmax}\DUrole{o,o}{=}\DUrole{default_value}{80}}, \sphinxparam{\DUrole{n,n}{fwhm}\DUrole{o,o}{=}\DUrole{default_value}{5.0}}}{}
\pysigstopsignatures
\sphinxAtStartPar
Running the muse\_exp\_align recipe

\end{fulllineitems}

\index{recipe\_bias() (pymusepipe.recipes\_pipe.PipeRecipes method)@\spxentry{recipe\_bias()}\spxextra{pymusepipe.recipes\_pipe.PipeRecipes method}}

\begin{fulllineitems}
\phantomsection\label{\detokenize{api/pymusepipe:pymusepipe.recipes_pipe.PipeRecipes.recipe_bias}}
\pysigstartsignatures
\pysiglinewithargsret{\sphinxbfcode{\sphinxupquote{recipe\_bias}}}{\sphinxparam{\DUrole{n,n}{sof}}, \sphinxparam{\DUrole{n,n}{dir\_bias}}, \sphinxparam{\DUrole{n,n}{name\_bias}}, \sphinxparam{\DUrole{n,n}{tpl}}}{}
\pysigstopsignatures
\sphinxAtStartPar
Running the esorex muse\_bias recipe

\end{fulllineitems}

\index{recipe\_combine() (pymusepipe.recipes\_pipe.PipeRecipes method)@\spxentry{recipe\_combine()}\spxextra{pymusepipe.recipes\_pipe.PipeRecipes method}}

\begin{fulllineitems}
\phantomsection\label{\detokenize{api/pymusepipe:pymusepipe.recipes_pipe.PipeRecipes.recipe_combine}}
\pysigstartsignatures
\pysiglinewithargsret{\sphinxbfcode{\sphinxupquote{recipe\_combine}}}{\sphinxparam{\DUrole{n,n}{sof}}, \sphinxparam{\DUrole{n,n}{dir\_products}}, \sphinxparam{\DUrole{n,n}{name\_products}}, \sphinxparam{\DUrole{n,n}{tpl}}, \sphinxparam{\DUrole{n,n}{expotype}}, \sphinxparam{\DUrole{n,n}{suffix\_products}\DUrole{o,o}{=}\DUrole{default_value}{\textquotesingle{}\textquotesingle{}}}, \sphinxparam{\DUrole{n,n}{suffix\_prefinalnames}\DUrole{o,o}{=}\DUrole{default_value}{\textquotesingle{}\textquotesingle{}}}, \sphinxparam{\DUrole{n,n}{prefix\_products}\DUrole{o,o}{=}\DUrole{default_value}{\textquotesingle{}\textquotesingle{}}}, \sphinxparam{\DUrole{n,n}{save}\DUrole{o,o}{=}\DUrole{default_value}{\textquotesingle{}cube\textquotesingle{}}}, \sphinxparam{\DUrole{n,n}{pixfrac}\DUrole{o,o}{=}\DUrole{default_value}{0.6}}, \sphinxparam{\DUrole{n,n}{suffix}\DUrole{o,o}{=}\DUrole{default_value}{\textquotesingle{}\textquotesingle{}}}, \sphinxparam{\DUrole{n,n}{format\_out}\DUrole{o,o}{=}\DUrole{default_value}{\textquotesingle{}Cube\textquotesingle{}}}, \sphinxparam{\DUrole{n,n}{filter\_list}\DUrole{o,o}{=}\DUrole{default_value}{\textquotesingle{}white\textquotesingle{}}}, \sphinxparam{\DUrole{n,n}{lambdamin}\DUrole{o,o}{=}\DUrole{default_value}{4000.0}}, \sphinxparam{\DUrole{n,n}{lambdamax}\DUrole{o,o}{=}\DUrole{default_value}{10000.0}}}{}
\pysigstopsignatures
\sphinxAtStartPar
Running the muse\_exp\_combine recipe for one single dataset

\end{fulllineitems}

\index{recipe\_combine\_pointings() (pymusepipe.recipes\_pipe.PipeRecipes method)@\spxentry{recipe\_combine\_pointings()}\spxextra{pymusepipe.recipes\_pipe.PipeRecipes method}}

\begin{fulllineitems}
\phantomsection\label{\detokenize{api/pymusepipe:pymusepipe.recipes_pipe.PipeRecipes.recipe_combine_pointings}}
\pysigstartsignatures
\pysiglinewithargsret{\sphinxbfcode{\sphinxupquote{recipe\_combine\_pointings}}}{\sphinxparam{\DUrole{n,n}{sof}}, \sphinxparam{\DUrole{n,n}{dir\_products}}, \sphinxparam{\DUrole{n,n}{name\_products}}, \sphinxparam{\DUrole{n,n}{suffix\_products}\DUrole{o,o}{=}\DUrole{default_value}{\textquotesingle{}\textquotesingle{}}}, \sphinxparam{\DUrole{n,n}{suffix\_prefinalnames}\DUrole{o,o}{=}\DUrole{default_value}{\textquotesingle{}\textquotesingle{}}}, \sphinxparam{\DUrole{n,n}{prefix\_products}\DUrole{o,o}{=}\DUrole{default_value}{\textquotesingle{}\textquotesingle{}}}, \sphinxparam{\DUrole{n,n}{save}\DUrole{o,o}{=}\DUrole{default_value}{\textquotesingle{}cube\textquotesingle{}}}, \sphinxparam{\DUrole{n,n}{pixfrac}\DUrole{o,o}{=}\DUrole{default_value}{0.6}}, \sphinxparam{\DUrole{n,n}{suffix}\DUrole{o,o}{=}\DUrole{default_value}{\textquotesingle{}\textquotesingle{}}}, \sphinxparam{\DUrole{n,n}{format\_out}\DUrole{o,o}{=}\DUrole{default_value}{\textquotesingle{}Cube\textquotesingle{}}}, \sphinxparam{\DUrole{n,n}{filter\_list}\DUrole{o,o}{=}\DUrole{default_value}{\textquotesingle{}white\textquotesingle{}}}, \sphinxparam{\DUrole{n,n}{lambdamin}\DUrole{o,o}{=}\DUrole{default_value}{4000.0}}, \sphinxparam{\DUrole{n,n}{lambdamax}\DUrole{o,o}{=}\DUrole{default_value}{10000.0}}}{}
\pysigstopsignatures
\sphinxAtStartPar
Running the muse\_exp\_combine recipe for pointings

\end{fulllineitems}

\index{recipe\_flat() (pymusepipe.recipes\_pipe.PipeRecipes method)@\spxentry{recipe\_flat()}\spxextra{pymusepipe.recipes\_pipe.PipeRecipes method}}

\begin{fulllineitems}
\phantomsection\label{\detokenize{api/pymusepipe:pymusepipe.recipes_pipe.PipeRecipes.recipe_flat}}
\pysigstartsignatures
\pysiglinewithargsret{\sphinxbfcode{\sphinxupquote{recipe\_flat}}}{\sphinxparam{\DUrole{n,n}{sof}}, \sphinxparam{\DUrole{n,n}{dir\_flat}}, \sphinxparam{\DUrole{n,n}{name\_flat}}, \sphinxparam{\DUrole{n,n}{dir\_trace}}, \sphinxparam{\DUrole{n,n}{name\_trace}}, \sphinxparam{\DUrole{n,n}{tpl}}}{}
\pysigstopsignatures
\sphinxAtStartPar
Running the esorex muse\_flat recipe

\end{fulllineitems}

\index{recipe\_lsf() (pymusepipe.recipes\_pipe.PipeRecipes method)@\spxentry{recipe\_lsf()}\spxextra{pymusepipe.recipes\_pipe.PipeRecipes method}}

\begin{fulllineitems}
\phantomsection\label{\detokenize{api/pymusepipe:pymusepipe.recipes_pipe.PipeRecipes.recipe_lsf}}
\pysigstartsignatures
\pysiglinewithargsret{\sphinxbfcode{\sphinxupquote{recipe\_lsf}}}{\sphinxparam{\DUrole{n,n}{sof}}, \sphinxparam{\DUrole{n,n}{dir\_lsf}}, \sphinxparam{\DUrole{n,n}{name\_lsf}}, \sphinxparam{\DUrole{n,n}{tpl}}}{}
\pysigstopsignatures
\sphinxAtStartPar
Running the esorex muse\_lsf recipe

\end{fulllineitems}

\index{recipe\_scibasic() (pymusepipe.recipes\_pipe.PipeRecipes method)@\spxentry{recipe\_scibasic()}\spxextra{pymusepipe.recipes\_pipe.PipeRecipes method}}

\begin{fulllineitems}
\phantomsection\label{\detokenize{api/pymusepipe:pymusepipe.recipes_pipe.PipeRecipes.recipe_scibasic}}
\pysigstartsignatures
\pysiglinewithargsret{\sphinxbfcode{\sphinxupquote{recipe\_scibasic}}}{\sphinxparam{\DUrole{n,n}{sof}}, \sphinxparam{\DUrole{n,n}{tpl}}, \sphinxparam{\DUrole{n,n}{expotype}}, \sphinxparam{\DUrole{n,n}{dir\_products}\DUrole{o,o}{=}\DUrole{default_value}{None}}, \sphinxparam{\DUrole{n,n}{name\_products}\DUrole{o,o}{=}\DUrole{default_value}{{[}{]}}}, \sphinxparam{\DUrole{n,n}{suffix}\DUrole{o,o}{=}\DUrole{default_value}{\textquotesingle{}\textquotesingle{}}}}{}
\pysigstopsignatures
\sphinxAtStartPar
Running the esorex muse\_scibasic recipe

\end{fulllineitems}

\index{recipe\_scipost() (pymusepipe.recipes\_pipe.PipeRecipes method)@\spxentry{recipe\_scipost()}\spxextra{pymusepipe.recipes\_pipe.PipeRecipes method}}

\begin{fulllineitems}
\phantomsection\label{\detokenize{api/pymusepipe:pymusepipe.recipes_pipe.PipeRecipes.recipe_scipost}}
\pysigstartsignatures
\pysiglinewithargsret{\sphinxbfcode{\sphinxupquote{recipe\_scipost}}}{\sphinxparam{\DUrole{n,n}{sof}}, \sphinxparam{\DUrole{n,n}{tpl}}, \sphinxparam{\DUrole{n,n}{expotype}}, \sphinxparam{\DUrole{n,n}{dir\_products}\DUrole{o,o}{=}\DUrole{default_value}{\textquotesingle{}\textquotesingle{}}}, \sphinxparam{\DUrole{n,n}{name\_products}\DUrole{o,o}{=}\DUrole{default_value}{{[}\textquotesingle{}\textquotesingle{}{]}}}, \sphinxparam{\DUrole{n,n}{suffix\_products}\DUrole{o,o}{=}\DUrole{default_value}{{[}\textquotesingle{}\textquotesingle{}{]}}}, \sphinxparam{\DUrole{n,n}{suffix\_prefinalnames}\DUrole{o,o}{=}\DUrole{default_value}{{[}\textquotesingle{}\textquotesingle{}{]}}}, \sphinxparam{\DUrole{n,n}{suffix\_postfinalnames}\DUrole{o,o}{=}\DUrole{default_value}{{[}\textquotesingle{}\textquotesingle{}{]}}}, \sphinxparam{\DUrole{n,n}{list\_expo}\DUrole{o,o}{=}\DUrole{default_value}{{[}{]}}}, \sphinxparam{\DUrole{n,n}{save}\DUrole{o,o}{=}\DUrole{default_value}{\textquotesingle{}cube,skymodel\textquotesingle{}}}, \sphinxparam{\DUrole{n,n}{filter\_list}\DUrole{o,o}{=}\DUrole{default_value}{\textquotesingle{}white\textquotesingle{}}}, \sphinxparam{\DUrole{n,n}{skymethod}\DUrole{o,o}{=}\DUrole{default_value}{\textquotesingle{}model\textquotesingle{}}}, \sphinxparam{\DUrole{n,n}{pixfrac}\DUrole{o,o}{=}\DUrole{default_value}{0.8}}, \sphinxparam{\DUrole{n,n}{darcheck}\DUrole{o,o}{=}\DUrole{default_value}{\textquotesingle{}none\textquotesingle{}}}, \sphinxparam{\DUrole{n,n}{skymodel\_frac}\DUrole{o,o}{=}\DUrole{default_value}{0.05}}, \sphinxparam{\DUrole{n,n}{astrometry}\DUrole{o,o}{=}\DUrole{default_value}{\textquotesingle{}TRUE\textquotesingle{}}}, \sphinxparam{\DUrole{n,n}{lambdamin}\DUrole{o,o}{=}\DUrole{default_value}{4000.0}}, \sphinxparam{\DUrole{n,n}{lambdamax}\DUrole{o,o}{=}\DUrole{default_value}{10000.0}}, \sphinxparam{\DUrole{n,n}{suffix}\DUrole{o,o}{=}\DUrole{default_value}{\textquotesingle{}\textquotesingle{}}}, \sphinxparam{\DUrole{n,n}{autocalib}\DUrole{o,o}{=}\DUrole{default_value}{\textquotesingle{}none\textquotesingle{}}}, \sphinxparam{\DUrole{n,n}{rvcorr}\DUrole{o,o}{=}\DUrole{default_value}{\textquotesingle{}bary\textquotesingle{}}}, \sphinxparam{\DUrole{o,o}{**}\DUrole{n,n}{kwargs}}}{}
\pysigstopsignatures
\sphinxAtStartPar
Running the esorex muse\_scipost recipe

\end{fulllineitems}

\index{recipe\_sky() (pymusepipe.recipes\_pipe.PipeRecipes method)@\spxentry{recipe\_sky()}\spxextra{pymusepipe.recipes\_pipe.PipeRecipes method}}

\begin{fulllineitems}
\phantomsection\label{\detokenize{api/pymusepipe:pymusepipe.recipes_pipe.PipeRecipes.recipe_sky}}
\pysigstartsignatures
\pysiglinewithargsret{\sphinxbfcode{\sphinxupquote{recipe\_sky}}}{\sphinxparam{\DUrole{n,n}{sof}}, \sphinxparam{\DUrole{n,n}{dir\_sky}}, \sphinxparam{\DUrole{n,n}{name\_sky}}, \sphinxparam{\DUrole{n,n}{tpl}}, \sphinxparam{\DUrole{n,n}{iexpo}\DUrole{o,o}{=}\DUrole{default_value}{1}}, \sphinxparam{\DUrole{n,n}{fraction}\DUrole{o,o}{=}\DUrole{default_value}{0.8}}}{}
\pysigstopsignatures
\sphinxAtStartPar
Running the esorex muse\_stc recipe

\end{fulllineitems}

\index{recipe\_std() (pymusepipe.recipes\_pipe.PipeRecipes method)@\spxentry{recipe\_std()}\spxextra{pymusepipe.recipes\_pipe.PipeRecipes method}}

\begin{fulllineitems}
\phantomsection\label{\detokenize{api/pymusepipe:pymusepipe.recipes_pipe.PipeRecipes.recipe_std}}
\pysigstartsignatures
\pysiglinewithargsret{\sphinxbfcode{\sphinxupquote{recipe\_std}}}{\sphinxparam{\DUrole{n,n}{sof}}, \sphinxparam{\DUrole{n,n}{dir\_std}}, \sphinxparam{\DUrole{n,n}{name\_std}}, \sphinxparam{\DUrole{n,n}{tpl}}}{}
\pysigstopsignatures
\sphinxAtStartPar
Running the esorex muse\_stc recipe

\end{fulllineitems}

\index{recipe\_twilight() (pymusepipe.recipes\_pipe.PipeRecipes method)@\spxentry{recipe\_twilight()}\spxextra{pymusepipe.recipes\_pipe.PipeRecipes method}}

\begin{fulllineitems}
\phantomsection\label{\detokenize{api/pymusepipe:pymusepipe.recipes_pipe.PipeRecipes.recipe_twilight}}
\pysigstartsignatures
\pysiglinewithargsret{\sphinxbfcode{\sphinxupquote{recipe\_twilight}}}{\sphinxparam{\DUrole{n,n}{sof}}, \sphinxparam{\DUrole{n,n}{dir\_twilight}}, \sphinxparam{\DUrole{n,n}{name\_twilight}}, \sphinxparam{\DUrole{n,n}{tpl}}}{}
\pysigstopsignatures
\sphinxAtStartPar
Running the esorex muse\_twilight recipe

\end{fulllineitems}

\index{recipe\_wave() (pymusepipe.recipes\_pipe.PipeRecipes method)@\spxentry{recipe\_wave()}\spxextra{pymusepipe.recipes\_pipe.PipeRecipes method}}

\begin{fulllineitems}
\phantomsection\label{\detokenize{api/pymusepipe:pymusepipe.recipes_pipe.PipeRecipes.recipe_wave}}
\pysigstartsignatures
\pysiglinewithargsret{\sphinxbfcode{\sphinxupquote{recipe\_wave}}}{\sphinxparam{\DUrole{n,n}{sof}}, \sphinxparam{\DUrole{n,n}{dir\_wave}}, \sphinxparam{\DUrole{n,n}{name\_wave}}, \sphinxparam{\DUrole{n,n}{tpl}}}{}
\pysigstopsignatures
\sphinxAtStartPar
Running the esorex muse\_wavecal recipe

\end{fulllineitems}

\index{run\_oscommand() (pymusepipe.recipes\_pipe.PipeRecipes method)@\spxentry{run\_oscommand()}\spxextra{pymusepipe.recipes\_pipe.PipeRecipes method}}

\begin{fulllineitems}
\phantomsection\label{\detokenize{api/pymusepipe:pymusepipe.recipes_pipe.PipeRecipes.run_oscommand}}
\pysigstartsignatures
\pysiglinewithargsret{\sphinxbfcode{\sphinxupquote{run\_oscommand}}}{\sphinxparam{\DUrole{n,n}{command}}, \sphinxparam{\DUrole{n,n}{log}\DUrole{o,o}{=}\DUrole{default_value}{True}}}{}
\pysigstopsignatures
\sphinxAtStartPar
Running an os.system shell command
Fake mode will just spit out the command but not actually do it.

\end{fulllineitems}

\index{write\_errlogfile() (pymusepipe.recipes\_pipe.PipeRecipes method)@\spxentry{write\_errlogfile()}\spxextra{pymusepipe.recipes\_pipe.PipeRecipes method}}

\begin{fulllineitems}
\phantomsection\label{\detokenize{api/pymusepipe:pymusepipe.recipes_pipe.PipeRecipes.write_errlogfile}}
\pysigstartsignatures
\pysiglinewithargsret{\sphinxbfcode{\sphinxupquote{write\_errlogfile}}}{\sphinxparam{\DUrole{n,n}{text}}}{}
\pysigstopsignatures
\sphinxAtStartPar
Writing in log file

\end{fulllineitems}

\index{write\_logfile() (pymusepipe.recipes\_pipe.PipeRecipes method)@\spxentry{write\_logfile()}\spxextra{pymusepipe.recipes\_pipe.PipeRecipes method}}

\begin{fulllineitems}
\phantomsection\label{\detokenize{api/pymusepipe:pymusepipe.recipes_pipe.PipeRecipes.write_logfile}}
\pysigstartsignatures
\pysiglinewithargsret{\sphinxbfcode{\sphinxupquote{write\_logfile}}}{\sphinxparam{\DUrole{n,n}{text}}, \sphinxparam{\DUrole{n,n}{addext}\DUrole{o,o}{=}\DUrole{default_value}{\textquotesingle{}\textquotesingle{}}}}{}
\pysigstopsignatures
\sphinxAtStartPar
Writing in log file

\end{fulllineitems}

\index{write\_outlogfile() (pymusepipe.recipes\_pipe.PipeRecipes method)@\spxentry{write\_outlogfile()}\spxextra{pymusepipe.recipes\_pipe.PipeRecipes method}}

\begin{fulllineitems}
\phantomsection\label{\detokenize{api/pymusepipe:pymusepipe.recipes_pipe.PipeRecipes.write_outlogfile}}
\pysigstartsignatures
\pysiglinewithargsret{\sphinxbfcode{\sphinxupquote{write\_outlogfile}}}{\sphinxparam{\DUrole{n,n}{text}}}{}
\pysigstopsignatures
\sphinxAtStartPar
Writing in log file

\end{fulllineitems}


\end{fulllineitems}



\subsubsection{pymusepipe.target\_sample module}
\label{\detokenize{api/pymusepipe:module-pymusepipe.target_sample}}\label{\detokenize{api/pymusepipe:pymusepipe-target-sample-module}}\index{module@\spxentry{module}!pymusepipe.target\_sample@\spxentry{pymusepipe.target\_sample}}\index{pymusepipe.target\_sample@\spxentry{pymusepipe.target\_sample}!module@\spxentry{module}}
\sphinxAtStartPar
MUSE\sphinxhyphen{}PHANGS target sample module
\index{MusePipeSample (class in pymusepipe.target\_sample)@\spxentry{MusePipeSample}\spxextra{class in pymusepipe.target\_sample}}

\begin{fulllineitems}
\phantomsection\label{\detokenize{api/pymusepipe:pymusepipe.target_sample.MusePipeSample}}
\pysigstartsignatures
\pysiglinewithargsret{\sphinxbfcode{\sphinxupquote{class\DUrole{w,w}{  }}}\sphinxcode{\sphinxupquote{pymusepipe.target\_sample.}}\sphinxbfcode{\sphinxupquote{MusePipeSample}}}{\sphinxparam{\DUrole{n,n}{TargetDic}}, \sphinxparam{\DUrole{n,n}{rc\_filename}\DUrole{o,o}{=}\DUrole{default_value}{None}}, \sphinxparam{\DUrole{n,n}{cal\_filename}\DUrole{o,o}{=}\DUrole{default_value}{None}}, \sphinxparam{\DUrole{n,n}{folder\_config}\DUrole{o,o}{=}\DUrole{default_value}{\textquotesingle{}\textquotesingle{}}}, \sphinxparam{\DUrole{n,n}{first\_recipe}\DUrole{o,o}{=}\DUrole{default_value}{1}}, \sphinxparam{\DUrole{o,o}{**}\DUrole{n,n}{kwargs}}}{}
\pysigstopsignatures
\sphinxAtStartPar
Bases: \sphinxhref{https://docs.python.org/3.10/library/functions.html\#object}{\sphinxcode{\sphinxupquote{object}}}
\index{combine\_target() (pymusepipe.target\_sample.MusePipeSample method)@\spxentry{combine\_target()}\spxextra{pymusepipe.target\_sample.MusePipeSample method}}

\begin{fulllineitems}
\phantomsection\label{\detokenize{api/pymusepipe:pymusepipe.target_sample.MusePipeSample.combine_target}}
\pysigstartsignatures
\pysiglinewithargsret{\sphinxbfcode{\sphinxupquote{combine\_target}}}{\sphinxparam{\DUrole{n,n}{targetname}\DUrole{o,o}{=}\DUrole{default_value}{None}}, \sphinxparam{\DUrole{o,o}{**}\DUrole{n,n}{kwargs}}}{}
\pysigstopsignatures
\sphinxAtStartPar
Run the combine recipe. Shortcut for combine{[}targetname{]}.run\_combine()

\end{fulllineitems}

\index{combine\_target\_per\_pointing() (pymusepipe.target\_sample.MusePipeSample method)@\spxentry{combine\_target\_per\_pointing()}\spxextra{pymusepipe.target\_sample.MusePipeSample method}}

\begin{fulllineitems}
\phantomsection\label{\detokenize{api/pymusepipe:pymusepipe.target_sample.MusePipeSample.combine_target_per_pointing}}
\pysigstartsignatures
\pysiglinewithargsret{\sphinxbfcode{\sphinxupquote{combine\_target\_per\_pointing}}}{\sphinxparam{\DUrole{n,n}{targetname}\DUrole{o,o}{=}\DUrole{default_value}{None}}, \sphinxparam{\DUrole{n,n}{wcs\_from\_pointing}\DUrole{o,o}{=}\DUrole{default_value}{True}}, \sphinxparam{\DUrole{o,o}{**}\DUrole{n,n}{kwargs}}}{}
\pysigstopsignatures
\sphinxAtStartPar
Run the combine recipe. Shortcut for combine{[}targetname{]}.run\_combine()

\end{fulllineitems}

\index{convolve\_mosaic\_per\_pointing() (pymusepipe.target\_sample.MusePipeSample method)@\spxentry{convolve\_mosaic\_per\_pointing()}\spxextra{pymusepipe.target\_sample.MusePipeSample method}}

\begin{fulllineitems}
\phantomsection\label{\detokenize{api/pymusepipe:pymusepipe.target_sample.MusePipeSample.convolve_mosaic_per_pointing}}
\pysigstartsignatures
\pysiglinewithargsret{\sphinxbfcode{\sphinxupquote{convolve\_mosaic\_per\_pointing}}}{\sphinxparam{\DUrole{n,n}{targetname}\DUrole{o,o}{=}\DUrole{default_value}{None}}, \sphinxparam{\DUrole{n,n}{list\_pointings}\DUrole{o,o}{=}\DUrole{default_value}{None}}, \sphinxparam{\DUrole{n,n}{dict\_psf}\DUrole{o,o}{=}\DUrole{default_value}{\{\}}}, \sphinxparam{\DUrole{n,n}{target\_fwhm}\DUrole{o,o}{=}\DUrole{default_value}{0.0}}, \sphinxparam{\DUrole{n,n}{target\_nmoffat}\DUrole{o,o}{=}\DUrole{default_value}{None}}, \sphinxparam{\DUrole{n,n}{target\_function}\DUrole{o,o}{=}\DUrole{default_value}{\textquotesingle{}gaussian\textquotesingle{}}}, \sphinxparam{\DUrole{n,n}{suffix}\DUrole{o,o}{=}\DUrole{default_value}{None}}, \sphinxparam{\DUrole{n,n}{best\_psf}\DUrole{o,o}{=}\DUrole{default_value}{True}}, \sphinxparam{\DUrole{n,n}{min\_dfwhm}\DUrole{o,o}{=}\DUrole{default_value}{0.2}}, \sphinxparam{\DUrole{n,n}{fakemode}\DUrole{o,o}{=}\DUrole{default_value}{False}}, \sphinxparam{\DUrole{o,o}{**}\DUrole{n,n}{kwargs}}}{}
\pysigstopsignatures
\sphinxAtStartPar
Convolve the datacubes listed in a mosaic with some target function
and FWHM. It will try to homogeneise all individual cubes to that
target PSF.
\begin{quote}\begin{description}
\sphinxlineitem{Parameters}\begin{itemize}
\item {} 
\sphinxAtStartPar
\sphinxstyleliteralstrong{\sphinxupquote{targetname}} (\sphinxhref{https://docs.python.org/3.10/library/stdtypes.html\#str}{\sphinxstyleliteralemphasis{\sphinxupquote{str}}}) \textendash{} name of the target

\item {} 
\sphinxAtStartPar
\sphinxstyleliteralstrong{\sphinxupquote{list\_pointings}} (\sphinxhref{https://docs.python.org/3.10/library/stdtypes.html\#list}{\sphinxstyleliteralemphasis{\sphinxupquote{list}}}) \textendash{} list of pointing numbers for the list of pointings
to consider

\item {} 
\sphinxAtStartPar
\sphinxstyleliteralstrong{\sphinxupquote{dict\_psf}} (\sphinxhref{https://docs.python.org/3.10/library/stdtypes.html\#dict}{\sphinxstyleliteralemphasis{\sphinxupquote{dict}}}) \textendash{} dictionary providing individual PSFs per pointing

\item {} 
\sphinxAtStartPar
\sphinxstyleliteralstrong{\sphinxupquote{target\_fwhm}} (\sphinxhref{https://docs.python.org/3.10/library/functions.html\#float}{\sphinxstyleliteralemphasis{\sphinxupquote{float}}}) \textendash{} target FWHM for the convolution {[}arcsec{]}

\item {} 
\sphinxAtStartPar
\sphinxstyleliteralstrong{\sphinxupquote{target\_nmoffat}} (\sphinxhref{https://docs.python.org/3.10/library/functions.html\#float}{\sphinxstyleliteralemphasis{\sphinxupquote{float}}}) \textendash{} tail factor for the moffat function.

\item {} 
\sphinxAtStartPar
\sphinxstyleliteralstrong{\sphinxupquote{target\_function}} (\sphinxhref{https://docs.python.org/3.10/library/stdtypes.html\#str}{\sphinxstyleliteralemphasis{\sphinxupquote{str}}}) \textendash{} ‘moffat’ or ‘gaussian’ {[}‘gaussian’{]}

\item {} 
\sphinxAtStartPar
\sphinxstyleliteralstrong{\sphinxupquote{suffix}} (\sphinxhref{https://docs.python.org/3.10/library/stdtypes.html\#str}{\sphinxstyleliteralemphasis{\sphinxupquote{str}}}) \textendash{} input string to be added

\item {} 
\sphinxAtStartPar
\sphinxstyleliteralstrong{\sphinxupquote{best\_psf}} (\sphinxhref{https://docs.python.org/3.10/library/functions.html\#bool}{\sphinxstyleliteralemphasis{\sphinxupquote{bool}}}) \textendash{} if True use the minimum overall possible value. If
True it will overwrite all the target parameters.

\item {} 
\sphinxAtStartPar
\sphinxstyleliteralstrong{\sphinxupquote{min\_dfwhm}} (\sphinxhref{https://docs.python.org/3.10/library/functions.html\#float}{\sphinxstyleliteralemphasis{\sphinxupquote{float}}}) \textendash{} minimum difference to be added in quadrature
{[}in arcsec{]}

\item {} 
\sphinxAtStartPar
\sphinxstyleliteralstrong{\sphinxupquote{filter\_list}} (\sphinxhref{https://docs.python.org/3.10/library/stdtypes.html\#list}{\sphinxstyleliteralemphasis{\sphinxupquote{list}}}) \textendash{} list of filters to be used for reconstructing
images

\item {} 
\sphinxAtStartPar
\sphinxstyleliteralstrong{\sphinxupquote{fakemode}} (\sphinxhref{https://docs.python.org/3.10/library/functions.html\#bool}{\sphinxstyleliteralemphasis{\sphinxupquote{bool}}}) \textendash{} if True, will only initialise parameters but not
proceed with the convolution.

\item {} 
\sphinxAtStartPar
\sphinxstyleliteralstrong{\sphinxupquote{**kwargs}} \textendash{} 

\end{itemize}

\end{description}\end{quote}

\sphinxAtStartPar
Returns:

\end{fulllineitems}

\index{create\_reference\_wcs() (pymusepipe.target\_sample.MusePipeSample method)@\spxentry{create\_reference\_wcs()}\spxextra{pymusepipe.target\_sample.MusePipeSample method}}

\begin{fulllineitems}
\phantomsection\label{\detokenize{api/pymusepipe:pymusepipe.target_sample.MusePipeSample.create_reference_wcs}}
\pysigstartsignatures
\pysiglinewithargsret{\sphinxbfcode{\sphinxupquote{create\_reference\_wcs}}}{\sphinxparam{\DUrole{n,n}{targetname}\DUrole{o,o}{=}\DUrole{default_value}{None}}, \sphinxparam{\DUrole{n,n}{pointings\_wcs}\DUrole{o,o}{=}\DUrole{default_value}{True}}, \sphinxparam{\DUrole{n,n}{mosaic\_wcs}\DUrole{o,o}{=}\DUrole{default_value}{True}}, \sphinxparam{\DUrole{n,n}{wcs\_refcube\_name}\DUrole{o,o}{=}\DUrole{default_value}{None}}, \sphinxparam{\DUrole{n,n}{refcube\_name}\DUrole{o,o}{=}\DUrole{default_value}{None}}, \sphinxparam{\DUrole{o,o}{**}\DUrole{n,n}{kwargs}}}{}
\pysigstopsignatures
\sphinxAtStartPar
Run the combine for individual exposures first building up
a mask.

\end{fulllineitems}

\index{finalise\_reduction() (pymusepipe.target\_sample.MusePipeSample method)@\spxentry{finalise\_reduction()}\spxextra{pymusepipe.target\_sample.MusePipeSample method}}

\begin{fulllineitems}
\phantomsection\label{\detokenize{api/pymusepipe:pymusepipe.target_sample.MusePipeSample.finalise_reduction}}
\pysigstartsignatures
\pysiglinewithargsret{\sphinxbfcode{\sphinxupquote{finalise\_reduction}}}{\sphinxparam{\DUrole{n,n}{targetname}\DUrole{o,o}{=}\DUrole{default_value}{None}}, \sphinxparam{\DUrole{n,n}{rot\_pixtab}\DUrole{o,o}{=}\DUrole{default_value}{False}}, \sphinxparam{\DUrole{n,n}{create\_wcs}\DUrole{o,o}{=}\DUrole{default_value}{True}}, \sphinxparam{\DUrole{n,n}{create\_expocubes}\DUrole{o,o}{=}\DUrole{default_value}{True}}, \sphinxparam{\DUrole{n,n}{create\_pixtables}\DUrole{o,o}{=}\DUrole{default_value}{True}}, \sphinxparam{\DUrole{n,n}{create\_pointingcubes}\DUrole{o,o}{=}\DUrole{default_value}{True}}, \sphinxparam{\DUrole{n,n}{name\_offset\_table}\DUrole{o,o}{=}\DUrole{default_value}{None}}, \sphinxparam{\DUrole{n,n}{folder\_offset\_table}\DUrole{o,o}{=}\DUrole{default_value}{None}}, \sphinxparam{\DUrole{n,n}{list\_datasets}\DUrole{o,o}{=}\DUrole{default_value}{None}}, \sphinxparam{\DUrole{o,o}{**}\DUrole{n,n}{kwargs}}}{}
\pysigstopsignatures
\sphinxAtStartPar
Finalise the reduction steps by using the offset table, rotating the
pixeltables, then reconstructing the PIXTABLE\_REDUCED, produce reference
WCS for each pointing, and then run the reconstruction of the final
individual cubes
\begin{quote}\begin{description}
\sphinxlineitem{Parameters}\begin{itemize}
\item {} 
\sphinxAtStartPar
\sphinxstyleliteralstrong{\sphinxupquote{targetname}} (\sphinxhref{https://docs.python.org/3.10/library/stdtypes.html\#str}{\sphinxstyleliteralemphasis{\sphinxupquote{str}}}) \textendash{} 

\item {} 
\sphinxAtStartPar
\sphinxstyleliteralstrong{\sphinxupquote{rot\_pixtab}} (\sphinxhref{https://docs.python.org/3.10/library/functions.html\#bool}{\sphinxstyleliteralemphasis{\sphinxupquote{bool}}}) \textendash{} 

\item {} 
\sphinxAtStartPar
\sphinxstyleliteralstrong{\sphinxupquote{create\_wcs}} (\sphinxhref{https://docs.python.org/3.10/library/functions.html\#bool}{\sphinxstyleliteralemphasis{\sphinxupquote{bool}}}) \textendash{} 

\item {} 
\sphinxAtStartPar
\sphinxstyleliteralstrong{\sphinxupquote{create\_expocubes}} (\sphinxhref{https://docs.python.org/3.10/library/functions.html\#bool}{\sphinxstyleliteralemphasis{\sphinxupquote{bool}}}) \textendash{} 

\item {} 
\sphinxAtStartPar
\sphinxstyleliteralstrong{\sphinxupquote{name\_offset\_table}} (\sphinxhref{https://docs.python.org/3.10/library/stdtypes.html\#str}{\sphinxstyleliteralemphasis{\sphinxupquote{str}}}) \textendash{} 

\item {} 
\sphinxAtStartPar
\sphinxstyleliteralstrong{\sphinxupquote{folder\_offset\_table}} (\sphinxhref{https://docs.python.org/3.10/library/stdtypes.html\#str}{\sphinxstyleliteralemphasis{\sphinxupquote{str}}}) \textendash{} 

\item {} 
\sphinxAtStartPar
\sphinxstyleliteralstrong{\sphinxupquote{**kwargs}} \textendash{} 
\sphinxAtStartPar
include
wcs\_refcube\_name: str
\begin{quote}

\sphinxAtStartPar
Reference WCS (cube) to be used to project all cubes
\end{quote}
\begin{description}
\sphinxlineitem{refcube\_name: str}
\sphinxAtStartPar
Reference cube which will be use to build a reference WCS

\sphinxlineitem{folder\_refcube: str}
\sphinxAtStartPar
Name of the folder where to find the reference WCS or cube

\end{description}


\end{itemize}

\end{description}\end{quote}

\sphinxAtStartPar
Returns:

\end{fulllineitems}

\index{init\_combine() (pymusepipe.target\_sample.MusePipeSample method)@\spxentry{init\_combine()}\spxextra{pymusepipe.target\_sample.MusePipeSample method}}

\begin{fulllineitems}
\phantomsection\label{\detokenize{api/pymusepipe:pymusepipe.target_sample.MusePipeSample.init_combine}}
\pysigstartsignatures
\pysiglinewithargsret{\sphinxbfcode{\sphinxupquote{init\_combine}}}{\sphinxparam{\DUrole{n,n}{targetname}\DUrole{o,o}{=}\DUrole{default_value}{None}}, \sphinxparam{\DUrole{n,n}{list\_pointings}\DUrole{o,o}{=}\DUrole{default_value}{None}}, \sphinxparam{\DUrole{n,n}{list\_datasets}\DUrole{o,o}{=}\DUrole{default_value}{None}}, \sphinxparam{\DUrole{n,n}{folder\_offset\_table}\DUrole{o,o}{=}\DUrole{default_value}{None}}, \sphinxparam{\DUrole{n,n}{name\_offset\_table}\DUrole{o,o}{=}\DUrole{default_value}{None}}, \sphinxparam{\DUrole{o,o}{**}\DUrole{n,n}{kwargs}}}{}
\pysigstopsignatures
\sphinxAtStartPar
Prepare the combination of targets. The use can provide a pointing table providing a
given selection.


\paragraph{Input}
\label{\detokenize{api/pymusepipe:id109}}\begin{description}
\sphinxlineitem{targetname: str {[}None{]}}
\sphinxAtStartPar
Name of target

\sphinxlineitem{list\_pointings: list {[}or None=default= all pointings{]}}
\sphinxAtStartPar
List of pointings (e.g., {[}1,2,3{]})

\sphinxlineitem{name\_offset\_table: str}
\sphinxAtStartPar
Name of Offset table

\sphinxlineitem{{\color{red}\bfseries{}**}kwargs: additional keywords including}
\sphinxAtStartPar
pointing\_table, pointing\_table\_folder, pointing\_table\_format

\end{description}

\end{fulllineitems}

\index{init\_mosaic() (pymusepipe.target\_sample.MusePipeSample method)@\spxentry{init\_mosaic()}\spxextra{pymusepipe.target\_sample.MusePipeSample method}}

\begin{fulllineitems}
\phantomsection\label{\detokenize{api/pymusepipe:pymusepipe.target_sample.MusePipeSample.init_mosaic}}
\pysigstartsignatures
\pysiglinewithargsret{\sphinxbfcode{\sphinxupquote{init\_mosaic}}}{\sphinxparam{\DUrole{n,n}{targetname}\DUrole{o,o}{=}\DUrole{default_value}{None}}, \sphinxparam{\DUrole{n,n}{list\_datasets}\DUrole{o,o}{=}\DUrole{default_value}{None}}, \sphinxparam{\DUrole{n,n}{list\_pointings}\DUrole{o,o}{=}\DUrole{default_value}{None}}, \sphinxparam{\DUrole{n,n}{pointing\_table}\DUrole{o,o}{=}\DUrole{default_value}{None}}, \sphinxparam{\DUrole{o,o}{**}\DUrole{n,n}{kwargs}}}{}
\pysigstopsignatures
\sphinxAtStartPar
Prepare the combination of targets


\paragraph{Input}
\label{\detokenize{api/pymusepipe:id112}}\begin{description}
\sphinxlineitem{targetname: str {[}None{]}}
\sphinxAtStartPar
Name of target

\sphinxlineitem{list\_datasets: list {[}or None=default meaning all datasets{]}}
\sphinxAtStartPar
List of datasets (e.g., {[}1,2,3{]})

\sphinxlineitem{pointing\_table: PointingTable}
\sphinxAtStartPar
Pointing Table to select a given set of exposures

\end{description}

\end{fulllineitems}

\index{mosaic() (pymusepipe.target\_sample.MusePipeSample method)@\spxentry{mosaic()}\spxextra{pymusepipe.target\_sample.MusePipeSample method}}

\begin{fulllineitems}
\phantomsection\label{\detokenize{api/pymusepipe:pymusepipe.target_sample.MusePipeSample.mosaic}}
\pysigstartsignatures
\pysiglinewithargsret{\sphinxbfcode{\sphinxupquote{mosaic}}}{\sphinxparam{\DUrole{n,n}{targetname}\DUrole{o,o}{=}\DUrole{default_value}{None}}, \sphinxparam{\DUrole{n,n}{list\_pointings}\DUrole{o,o}{=}\DUrole{default_value}{None}}, \sphinxparam{\DUrole{n,n}{init\_mosaic}\DUrole{o,o}{=}\DUrole{default_value}{True}}, \sphinxparam{\DUrole{n,n}{build\_cube}\DUrole{o,o}{=}\DUrole{default_value}{True}}, \sphinxparam{\DUrole{n,n}{build\_images}\DUrole{o,o}{=}\DUrole{default_value}{True}}, \sphinxparam{\DUrole{o,o}{**}\DUrole{n,n}{kwargs}}}{}
\pysigstopsignatures\begin{quote}\begin{description}
\sphinxlineitem{Parameters}\begin{itemize}
\item {} 
\sphinxAtStartPar
\sphinxstyleliteralstrong{\sphinxupquote{targetname}} \textendash{} 

\item {} 
\sphinxAtStartPar
\sphinxstyleliteralstrong{\sphinxupquote{list\_pointings}} \textendash{} 

\item {} 
\sphinxAtStartPar
\sphinxstyleliteralstrong{\sphinxupquote{**kwargs}} \textendash{} 

\end{itemize}

\end{description}\end{quote}

\sphinxAtStartPar
Returns:

\end{fulllineitems}

\index{reduce\_all\_targets() (pymusepipe.target\_sample.MusePipeSample method)@\spxentry{reduce\_all\_targets()}\spxextra{pymusepipe.target\_sample.MusePipeSample method}}

\begin{fulllineitems}
\phantomsection\label{\detokenize{api/pymusepipe:pymusepipe.target_sample.MusePipeSample.reduce_all_targets}}
\pysigstartsignatures
\pysiglinewithargsret{\sphinxbfcode{\sphinxupquote{reduce\_all\_targets}}}{\sphinxparam{\DUrole{o,o}{**}\DUrole{n,n}{kwargs}}}{}
\pysigstopsignatures
\sphinxAtStartPar
Reduce all targets already initialised


\paragraph{Input}
\label{\detokenize{api/pymusepipe:id113}}\begin{description}
\sphinxlineitem{first\_recipe: int or str}
\sphinxAtStartPar
One of the recipe to start with

\sphinxlineitem{last\_recipe: int or str}
\sphinxAtStartPar
One of the recipe to end with

\end{description}

\end{fulllineitems}

\index{reduce\_target() (pymusepipe.target\_sample.MusePipeSample method)@\spxentry{reduce\_target()}\spxextra{pymusepipe.target\_sample.MusePipeSample method}}

\begin{fulllineitems}
\phantomsection\label{\detokenize{api/pymusepipe:pymusepipe.target_sample.MusePipeSample.reduce_target}}
\pysigstartsignatures
\pysiglinewithargsret{\sphinxbfcode{\sphinxupquote{reduce\_target}}}{\sphinxparam{\DUrole{n,n}{targetname}\DUrole{o,o}{=}\DUrole{default_value}{None}}, \sphinxparam{\DUrole{n,n}{list\_datasets}\DUrole{o,o}{=}\DUrole{default_value}{None}}, \sphinxparam{\DUrole{o,o}{**}\DUrole{n,n}{kwargs}}}{}
\pysigstopsignatures
\sphinxAtStartPar
Reduce one target for a list of datasets


\paragraph{Input}
\label{\detokenize{api/pymusepipe:id114}}\begin{description}
\sphinxlineitem{targetname: str}
\sphinxAtStartPar
Name of the target

\sphinxlineitem{list\_datasets: list}
\sphinxAtStartPar
Dataset numbers. Default is None (meaning all datasets
indicated in the dictonary will be reduced)

\end{description}

\sphinxAtStartPar
first\_recipe: str or int {[}1{]}
last\_recipe: str or int {[}max of all recipes{]}
\begin{quote}

\sphinxAtStartPar
Name or number of the first and last recipes to process
\end{quote}

\end{fulllineitems}

\index{reduce\_target\_postalign() (pymusepipe.target\_sample.MusePipeSample method)@\spxentry{reduce\_target\_postalign()}\spxextra{pymusepipe.target\_sample.MusePipeSample method}}

\begin{fulllineitems}
\phantomsection\label{\detokenize{api/pymusepipe:pymusepipe.target_sample.MusePipeSample.reduce_target_postalign}}
\pysigstartsignatures
\pysiglinewithargsret{\sphinxbfcode{\sphinxupquote{reduce\_target\_postalign}}}{\sphinxparam{\DUrole{n,n}{targetname}\DUrole{o,o}{=}\DUrole{default_value}{None}}, \sphinxparam{\DUrole{n,n}{list\_datasets}\DUrole{o,o}{=}\DUrole{default_value}{None}}, \sphinxparam{\DUrole{o,o}{**}\DUrole{n,n}{kwargs}}}{}
\pysigstopsignatures
\sphinxAtStartPar
Reduce target for all steps after pre\sphinxhyphen{}alignment


\paragraph{Input}
\label{\detokenize{api/pymusepipe:id115}}\begin{description}
\sphinxlineitem{targetname: str}
\sphinxAtStartPar
Name of the target

\sphinxlineitem{list\_datasets: list}
\sphinxAtStartPar
Dataset numbers. Default is None (meaning all datasets
indicated in the dictonary will be reduced)

\end{description}

\end{fulllineitems}

\index{reduce\_target\_prealign() (pymusepipe.target\_sample.MusePipeSample method)@\spxentry{reduce\_target\_prealign()}\spxextra{pymusepipe.target\_sample.MusePipeSample method}}

\begin{fulllineitems}
\phantomsection\label{\detokenize{api/pymusepipe:pymusepipe.target_sample.MusePipeSample.reduce_target_prealign}}
\pysigstartsignatures
\pysiglinewithargsret{\sphinxbfcode{\sphinxupquote{reduce\_target\_prealign}}}{\sphinxparam{\DUrole{n,n}{targetname}\DUrole{o,o}{=}\DUrole{default_value}{None}}, \sphinxparam{\DUrole{n,n}{list\_datasets}\DUrole{o,o}{=}\DUrole{default_value}{None}}, \sphinxparam{\DUrole{o,o}{**}\DUrole{n,n}{kwargs}}}{}
\pysigstopsignatures
\sphinxAtStartPar
Reduce target for all steps before pre\sphinxhyphen{}alignment (included)


\paragraph{Input}
\label{\detokenize{api/pymusepipe:id116}}\begin{description}
\sphinxlineitem{targetname: str}
\sphinxAtStartPar
Name of the target

\sphinxlineitem{list\_datasets: list}
\sphinxAtStartPar
Dataset numbers. Default is None (meaning all datasets
indicated in the dictionary will be reduced)

\end{description}

\end{fulllineitems}

\index{rotate\_pixtables\_target() (pymusepipe.target\_sample.MusePipeSample method)@\spxentry{rotate\_pixtables\_target()}\spxextra{pymusepipe.target\_sample.MusePipeSample method}}

\begin{fulllineitems}
\phantomsection\label{\detokenize{api/pymusepipe:pymusepipe.target_sample.MusePipeSample.rotate_pixtables_target}}
\pysigstartsignatures
\pysiglinewithargsret{\sphinxbfcode{\sphinxupquote{rotate\_pixtables\_target}}}{\sphinxparam{\DUrole{n,n}{targetname}\DUrole{o,o}{=}\DUrole{default_value}{None}}, \sphinxparam{\DUrole{n,n}{list\_datasets}\DUrole{o,o}{=}\DUrole{default_value}{None}}, \sphinxparam{\DUrole{n,n}{folder\_offset\_table}\DUrole{o,o}{=}\DUrole{default_value}{None}}, \sphinxparam{\DUrole{n,n}{name\_offset\_table}\DUrole{o,o}{=}\DUrole{default_value}{None}}, \sphinxparam{\DUrole{n,n}{fakemode}\DUrole{o,o}{=}\DUrole{default_value}{False}}, \sphinxparam{\DUrole{o,o}{**}\DUrole{n,n}{kwargs}}}{}
\pysigstopsignatures
\sphinxAtStartPar
Rotate all pixel table of a certain targetname and datasets

\end{fulllineitems}

\index{run\_target\_recipe() (pymusepipe.target\_sample.MusePipeSample method)@\spxentry{run\_target\_recipe()}\spxextra{pymusepipe.target\_sample.MusePipeSample method}}

\begin{fulllineitems}
\phantomsection\label{\detokenize{api/pymusepipe:pymusepipe.target_sample.MusePipeSample.run_target_recipe}}
\pysigstartsignatures
\pysiglinewithargsret{\sphinxbfcode{\sphinxupquote{run\_target\_recipe}}}{\sphinxparam{\DUrole{n,n}{recipe\_name}}, \sphinxparam{\DUrole{n,n}{targetname}\DUrole{o,o}{=}\DUrole{default_value}{None}}, \sphinxparam{\DUrole{n,n}{list\_datasets}\DUrole{o,o}{=}\DUrole{default_value}{None}}, \sphinxparam{\DUrole{o,o}{**}\DUrole{n,n}{kwargs}}}{}
\pysigstopsignatures
\sphinxAtStartPar
Run just one recipe on target


\paragraph{Input}
\label{\detokenize{api/pymusepipe:id117}}
\sphinxAtStartPar
recipe\_name: str
targetname: str
\begin{quote}

\sphinxAtStartPar
Name of the target
\end{quote}
\begin{description}
\sphinxlineitem{list\_datasets: list}
\sphinxAtStartPar
Pointing numbers. Default is None (meaning all datasets
indicated in the dictonary will be reduced)

\end{description}

\end{fulllineitems}

\index{run\_target\_scipost\_perexpo() (pymusepipe.target\_sample.MusePipeSample method)@\spxentry{run\_target\_scipost\_perexpo()}\spxextra{pymusepipe.target\_sample.MusePipeSample method}}

\begin{fulllineitems}
\phantomsection\label{\detokenize{api/pymusepipe:pymusepipe.target_sample.MusePipeSample.run_target_scipost_perexpo}}
\pysigstartsignatures
\pysiglinewithargsret{\sphinxbfcode{\sphinxupquote{run\_target\_scipost\_perexpo}}}{\sphinxparam{\DUrole{n,n}{targetname}\DUrole{o,o}{=}\DUrole{default_value}{None}}, \sphinxparam{\DUrole{n,n}{list\_datasets}\DUrole{o,o}{=}\DUrole{default_value}{None}}, \sphinxparam{\DUrole{n,n}{list\_pointings}\DUrole{o,o}{=}\DUrole{default_value}{None}}, \sphinxparam{\DUrole{n,n}{folder\_offset\_table}\DUrole{o,o}{=}\DUrole{default_value}{None}}, \sphinxparam{\DUrole{n,n}{name\_offset\_table}\DUrole{o,o}{=}\DUrole{default_value}{None}}, \sphinxparam{\DUrole{o,o}{**}\DUrole{n,n}{kwargs}}}{}
\pysigstopsignatures
\sphinxAtStartPar
Build the cube per exposure using a given WCS
\begin{quote}\begin{description}
\sphinxlineitem{Parameters}\begin{itemize}
\item {} 
\sphinxAtStartPar
\sphinxstyleliteralstrong{\sphinxupquote{targetname}} \textendash{} 

\item {} 
\sphinxAtStartPar
\sphinxstyleliteralstrong{\sphinxupquote{list\_datasets}} \textendash{} 

\item {} 
\sphinxAtStartPar
\sphinxstyleliteralstrong{\sphinxupquote{**kwargs}} \textendash{} 

\end{itemize}

\end{description}\end{quote}

\sphinxAtStartPar
Returns:

\end{fulllineitems}

\index{set\_pipe\_target() (pymusepipe.target\_sample.MusePipeSample method)@\spxentry{set\_pipe\_target()}\spxextra{pymusepipe.target\_sample.MusePipeSample method}}

\begin{fulllineitems}
\phantomsection\label{\detokenize{api/pymusepipe:pymusepipe.target_sample.MusePipeSample.set_pipe_target}}
\pysigstartsignatures
\pysiglinewithargsret{\sphinxbfcode{\sphinxupquote{set\_pipe\_target}}}{\sphinxparam{\DUrole{n,n}{targetname}\DUrole{o,o}{=}\DUrole{default_value}{None}}, \sphinxparam{\DUrole{n,n}{list\_datasets}\DUrole{o,o}{=}\DUrole{default_value}{None}}, \sphinxparam{\DUrole{o,o}{**}\DUrole{n,n}{kwargs}}}{}
\pysigstopsignatures
\sphinxAtStartPar
Create the musepipe instance for that target and list of datasets


\paragraph{Input}
\label{\detokenize{api/pymusepipe:id118}}\begin{description}
\sphinxlineitem{targetname: str}
\sphinxAtStartPar
Name of the target

\sphinxlineitem{list\_datasets: list}
\sphinxAtStartPar
Dataset numbers. Default is None (meaning all datasets
indicated in the dictonary will be reduced)

\sphinxlineitem{config\_args: dic}
\sphinxAtStartPar
Dictionary including extra configuration parameters to pass
to MusePipe. This allows to define a global configuration.
If self.\_\_phangs is set to True, this is overwritten with the default
PHANGS configuration parameters as provided in config\_pipe.py.

\end{description}

\end{fulllineitems}


\end{fulllineitems}

\index{MusePipeTarget (class in pymusepipe.target\_sample)@\spxentry{MusePipeTarget}\spxextra{class in pymusepipe.target\_sample}}

\begin{fulllineitems}
\phantomsection\label{\detokenize{api/pymusepipe:pymusepipe.target_sample.MusePipeTarget}}
\pysigstartsignatures
\pysiglinewithargsret{\sphinxbfcode{\sphinxupquote{class\DUrole{w,w}{  }}}\sphinxcode{\sphinxupquote{pymusepipe.target\_sample.}}\sphinxbfcode{\sphinxupquote{MusePipeTarget}}}{\sphinxparam{\DUrole{n,n}{targetname}\DUrole{o,o}{=}\DUrole{default_value}{\textquotesingle{}\textquotesingle{}}}, \sphinxparam{\DUrole{n,n}{subfolder}\DUrole{o,o}{=}\DUrole{default_value}{\textquotesingle{}P100\textquotesingle{}}}, \sphinxparam{\DUrole{n,n}{list\_datasets}\DUrole{o,o}{=}\DUrole{default_value}{None}}}{}
\pysigstopsignatures
\sphinxAtStartPar
Bases: \sphinxhref{https://docs.python.org/3.10/library/functions.html\#object}{\sphinxcode{\sphinxupquote{object}}}

\end{fulllineitems}

\index{PipeDict (class in pymusepipe.target\_sample)@\spxentry{PipeDict}\spxextra{class in pymusepipe.target\_sample}}

\begin{fulllineitems}
\phantomsection\label{\detokenize{api/pymusepipe:pymusepipe.target_sample.PipeDict}}
\pysigstartsignatures
\pysiglinewithargsret{\sphinxbfcode{\sphinxupquote{class\DUrole{w,w}{  }}}\sphinxcode{\sphinxupquote{pymusepipe.target\_sample.}}\sphinxbfcode{\sphinxupquote{PipeDict}}}{\sphinxparam{\DUrole{o,o}{*}\DUrole{n,n}{args}}, \sphinxparam{\DUrole{o,o}{**}\DUrole{n,n}{kwargs}}}{}
\pysigstopsignatures
\sphinxAtStartPar
Bases: \sphinxhref{https://docs.python.org/3.10/library/stdtypes.html\#dict}{\sphinxcode{\sphinxupquote{dict}}}

\sphinxAtStartPar
Dictionary with extra attributes
\index{run\_on\_all\_keys() (pymusepipe.target\_sample.PipeDict method)@\spxentry{run\_on\_all\_keys()}\spxextra{pymusepipe.target\_sample.PipeDict method}}

\begin{fulllineitems}
\phantomsection\label{\detokenize{api/pymusepipe:pymusepipe.target_sample.PipeDict.run_on_all_keys}}
\pysigstartsignatures
\pysiglinewithargsret{\sphinxbfcode{\sphinxupquote{run\_on\_all\_keys}}}{\sphinxparam{\DUrole{n,n}{funcname}}}{}
\pysigstopsignatures
\sphinxAtStartPar
Runs the given function on all the keys

\end{fulllineitems}

\index{setdefault() (pymusepipe.target\_sample.PipeDict method)@\spxentry{setdefault()}\spxextra{pymusepipe.target\_sample.PipeDict method}}

\begin{fulllineitems}
\phantomsection\label{\detokenize{api/pymusepipe:pymusepipe.target_sample.PipeDict.setdefault}}
\pysigstartsignatures
\pysiglinewithargsret{\sphinxbfcode{\sphinxupquote{setdefault}}}{\sphinxparam{\DUrole{n,n}{key}}, \sphinxparam{\DUrole{n,n}{value}\DUrole{o,o}{=}\DUrole{default_value}{None}}}{}
\pysigstopsignatures
\sphinxAtStartPar
Insert key with a value of default if key is not in the dictionary.

\sphinxAtStartPar
Return the value for key if key is in the dictionary, else default.

\end{fulllineitems}

\index{update() (pymusepipe.target\_sample.PipeDict method)@\spxentry{update()}\spxextra{pymusepipe.target\_sample.PipeDict method}}

\begin{fulllineitems}
\phantomsection\label{\detokenize{api/pymusepipe:pymusepipe.target_sample.PipeDict.update}}
\pysigstartsignatures
\pysiglinewithargsret{\sphinxbfcode{\sphinxupquote{update}}}{\sphinxoptional{\sphinxparam{\DUrole{n,n}{E}}}, \sphinxparam{\DUrole{n,n}{**F}}}{{ $\rightarrow$ None.  Update D from dict/iterable E and F.}}
\pysigstopsignatures
\sphinxAtStartPar
If E is present and has a .keys() method, then does:  for k in E: D{[}k{]} = E{[}k{]}
If E is present and lacks a .keys() method, then does:  for k, v in E: D{[}k{]} = v
In either case, this is followed by: for k in F:  D{[}k{]} = F{[}k{]}

\end{fulllineitems}


\end{fulllineitems}

\index{insert\_suffix() (in module pymusepipe.target\_sample)@\spxentry{insert\_suffix()}\spxextra{in module pymusepipe.target\_sample}}

\begin{fulllineitems}
\phantomsection\label{\detokenize{api/pymusepipe:pymusepipe.target_sample.insert_suffix}}
\pysigstartsignatures
\pysiglinewithargsret{\sphinxcode{\sphinxupquote{pymusepipe.target\_sample.}}\sphinxbfcode{\sphinxupquote{insert\_suffix}}}{\sphinxparam{\DUrole{n,n}{filename}}, \sphinxparam{\DUrole{n,n}{suffix}\DUrole{o,o}{=}\DUrole{default_value}{\textquotesingle{}\textquotesingle{}}}}{}
\pysigstopsignatures
\sphinxAtStartPar
Create a new filename including the
suffix in the name


\paragraph{Input}
\label{\detokenize{api/pymusepipe:id119}}
\sphinxAtStartPar
filename: str
suffix: str

\end{fulllineitems}

\index{update\_calib\_file() (in module pymusepipe.target\_sample)@\spxentry{update\_calib\_file()}\spxextra{in module pymusepipe.target\_sample}}

\begin{fulllineitems}
\phantomsection\label{\detokenize{api/pymusepipe:pymusepipe.target_sample.update_calib_file}}
\pysigstartsignatures
\pysiglinewithargsret{\sphinxcode{\sphinxupquote{pymusepipe.target\_sample.}}\sphinxbfcode{\sphinxupquote{update\_calib\_file}}}{\sphinxparam{\DUrole{n,n}{filename}}, \sphinxparam{\DUrole{n,n}{subfolder}\DUrole{o,o}{=}\DUrole{default_value}{\textquotesingle{}\textquotesingle{}}}, \sphinxparam{\DUrole{n,n}{folder\_config}\DUrole{o,o}{=}\DUrole{default_value}{\textquotesingle{}\textquotesingle{}}}}{}
\pysigstopsignatures
\sphinxAtStartPar
Update the rcfile with a new root


\paragraph{Input}
\label{\detokenize{api/pymusepipe:id120}}\begin{description}
\sphinxlineitem{filename: str}
\sphinxAtStartPar
Name of the input filename

\sphinxlineitem{folder\_config: str}
\sphinxAtStartPar
Default is “”. Name of folder for filename

\sphinxlineitem{subfolder: str}
\sphinxAtStartPar
Name of subfolder to add in the path

\end{description}

\end{fulllineitems}



\subsubsection{pymusepipe.util\_image module}
\label{\detokenize{api/pymusepipe:module-pymusepipe.util_image}}\label{\detokenize{api/pymusepipe:pymusepipe-util-image-module}}\index{module@\spxentry{module}!pymusepipe.util\_image@\spxentry{pymusepipe.util\_image}}\index{pymusepipe.util\_image@\spxentry{pymusepipe.util\_image}!module@\spxentry{module}}
\sphinxAtStartPar
Utility functions for images in pymusepipe
\index{CircleZone (class in pymusepipe.util\_image)@\spxentry{CircleZone}\spxextra{class in pymusepipe.util\_image}}

\begin{fulllineitems}
\phantomsection\label{\detokenize{api/pymusepipe:pymusepipe.util_image.CircleZone}}
\pysigstartsignatures
\pysigline{\sphinxbfcode{\sphinxupquote{class\DUrole{w,w}{  }}}\sphinxcode{\sphinxupquote{pymusepipe.util\_image.}}\sphinxbfcode{\sphinxupquote{CircleZone}}}
\pysigstopsignatures
\sphinxAtStartPar
Bases: {\hyperref[\detokenize{api/pymusepipe:pymusepipe.util_image.SelectionZone}]{\sphinxcrossref{\sphinxcode{\sphinxupquote{SelectionZone}}}}}

\sphinxAtStartPar
Define a Circular zone, defined by
a center and a radius
\index{select() (pymusepipe.util\_image.CircleZone method)@\spxentry{select()}\spxextra{pymusepipe.util\_image.CircleZone method}}

\begin{fulllineitems}
\phantomsection\label{\detokenize{api/pymusepipe:pymusepipe.util_image.CircleZone.select}}
\pysigstartsignatures
\pysiglinewithargsret{\sphinxbfcode{\sphinxupquote{select}}}{\sphinxparam{\DUrole{n,n}{xin}}, \sphinxparam{\DUrole{n,n}{yin}}}{}
\pysigstopsignatures
\sphinxAtStartPar
Define a selection within a circle


\paragraph{Input}
\label{\detokenize{api/pymusepipe:id121}}\begin{description}
\sphinxlineitem{xin, yin: 2d arrays}
\sphinxAtStartPar
Input positions for the spaxels

\end{description}

\end{fulllineitems}


\end{fulllineitems}

\index{PointingTable (class in pymusepipe.util\_image)@\spxentry{PointingTable}\spxextra{class in pymusepipe.util\_image}}

\begin{fulllineitems}
\phantomsection\label{\detokenize{api/pymusepipe:pymusepipe.util_image.PointingTable}}
\pysigstartsignatures
\pysiglinewithargsret{\sphinxbfcode{\sphinxupquote{class\DUrole{w,w}{  }}}\sphinxcode{\sphinxupquote{pymusepipe.util\_image.}}\sphinxbfcode{\sphinxupquote{PointingTable}}}{\sphinxparam{\DUrole{n,n}{input\_table}\DUrole{o,o}{=}\DUrole{default_value}{None}}, \sphinxparam{\DUrole{o,o}{**}\DUrole{n,n}{kwargs}}}{}
\pysigstopsignatures
\sphinxAtStartPar
Bases: \sphinxhref{https://docs.python.org/3.10/library/functions.html\#object}{\sphinxcode{\sphinxupquote{object}}}
\index{assign\_pointings() (pymusepipe.util\_image.PointingTable method)@\spxentry{assign\_pointings()}\spxextra{pymusepipe.util\_image.PointingTable method}}

\begin{fulllineitems}
\phantomsection\label{\detokenize{api/pymusepipe:pymusepipe.util_image.PointingTable.assign_pointings}}
\pysigstartsignatures
\pysiglinewithargsret{\sphinxbfcode{\sphinxupquote{assign\_pointings}}}{\sphinxparam{\DUrole{o,o}{**}\DUrole{n,n}{kwargs}}}{}
\pysigstopsignatures
\sphinxAtStartPar
Assign pointing according to distance rules. Will also update the centre values.
\begin{description}
\sphinxlineitem{{\color{red}\bfseries{}**}kwargs: additional keywords including}
\sphinxAtStartPar
verbose: bool default=self.verbose
overwrite: bool default=False
\begin{quote}

\sphinxAtStartPar
overwrite the pointing
\end{quote}

\end{description}


\paragraph{Updates}
\label{\detokenize{api/pymusepipe:updates}}
\sphinxAtStartPar
The values of the ‘pointing’ column for the selected filenames (‘select’==1)

\end{fulllineitems}

\index{dict\_allnames\_in\_datasets (pymusepipe.util\_image.PointingTable property)@\spxentry{dict\_allnames\_in\_datasets}\spxextra{pymusepipe.util\_image.PointingTable property}}

\begin{fulllineitems}
\phantomsection\label{\detokenize{api/pymusepipe:pymusepipe.util_image.PointingTable.dict_allnames_in_datasets}}
\pysigstartsignatures
\pysigline{\sphinxbfcode{\sphinxupquote{property\DUrole{w,w}{  }}}\sphinxbfcode{\sphinxupquote{dict\_allnames\_in\_datasets}}}
\pysigstopsignatures
\sphinxAtStartPar
Dictionary of the names per dataset

\end{fulllineitems}

\index{dict\_allnames\_in\_pointings (pymusepipe.util\_image.PointingTable property)@\spxentry{dict\_allnames\_in\_pointings}\spxextra{pymusepipe.util\_image.PointingTable property}}

\begin{fulllineitems}
\phantomsection\label{\detokenize{api/pymusepipe:pymusepipe.util_image.PointingTable.dict_allnames_in_pointings}}
\pysigstartsignatures
\pysigline{\sphinxbfcode{\sphinxupquote{property\DUrole{w,w}{  }}}\sphinxbfcode{\sphinxupquote{dict\_allnames\_in\_pointings}}}
\pysigstopsignatures
\sphinxAtStartPar
Dictionary of the names per pointing

\end{fulllineitems}

\index{dict\_names\_in\_datasets (pymusepipe.util\_image.PointingTable property)@\spxentry{dict\_names\_in\_datasets}\spxextra{pymusepipe.util\_image.PointingTable property}}

\begin{fulllineitems}
\phantomsection\label{\detokenize{api/pymusepipe:pymusepipe.util_image.PointingTable.dict_names_in_datasets}}
\pysigstartsignatures
\pysigline{\sphinxbfcode{\sphinxupquote{property\DUrole{w,w}{  }}}\sphinxbfcode{\sphinxupquote{dict\_names\_in\_datasets}}}
\pysigstopsignatures
\sphinxAtStartPar
Dictionary of the names per dataset

\end{fulllineitems}

\index{dict\_names\_in\_pointings (pymusepipe.util\_image.PointingTable property)@\spxentry{dict\_names\_in\_pointings}\spxextra{pymusepipe.util\_image.PointingTable property}}

\begin{fulllineitems}
\phantomsection\label{\detokenize{api/pymusepipe:pymusepipe.util_image.PointingTable.dict_names_in_pointings}}
\pysigstartsignatures
\pysigline{\sphinxbfcode{\sphinxupquote{property\DUrole{w,w}{  }}}\sphinxbfcode{\sphinxupquote{dict\_names\_in\_pointings}}}
\pysigstopsignatures
\sphinxAtStartPar
Dictionary of the names per pointing

\end{fulllineitems}

\index{dict\_tplexpo\_per\_dataset (pymusepipe.util\_image.PointingTable property)@\spxentry{dict\_tplexpo\_per\_dataset}\spxextra{pymusepipe.util\_image.PointingTable property}}

\begin{fulllineitems}
\phantomsection\label{\detokenize{api/pymusepipe:pymusepipe.util_image.PointingTable.dict_tplexpo_per_dataset}}
\pysigstartsignatures
\pysigline{\sphinxbfcode{\sphinxupquote{property\DUrole{w,w}{  }}}\sphinxbfcode{\sphinxupquote{dict\_tplexpo\_per\_dataset}}}
\pysigstopsignatures
\end{fulllineitems}

\index{dict\_tplexpo\_per\_pointing (pymusepipe.util\_image.PointingTable property)@\spxentry{dict\_tplexpo\_per\_pointing}\spxextra{pymusepipe.util\_image.PointingTable property}}

\begin{fulllineitems}
\phantomsection\label{\detokenize{api/pymusepipe:pymusepipe.util_image.PointingTable.dict_tplexpo_per_pointing}}
\pysigstartsignatures
\pysigline{\sphinxbfcode{\sphinxupquote{property\DUrole{w,w}{  }}}\sphinxbfcode{\sphinxupquote{dict\_tplexpo\_per\_pointing}}}
\pysigstopsignatures
\end{fulllineitems}

\index{fullnameout (pymusepipe.util\_image.PointingTable property)@\spxentry{fullnameout}\spxextra{pymusepipe.util\_image.PointingTable property}}

\begin{fulllineitems}
\phantomsection\label{\detokenize{api/pymusepipe:pymusepipe.util_image.PointingTable.fullnameout}}
\pysigstartsignatures
\pysigline{\sphinxbfcode{\sphinxupquote{property\DUrole{w,w}{  }}}\sphinxbfcode{\sphinxupquote{fullnameout}}}
\pysigstopsignatures
\end{fulllineitems}

\index{fulltablename (pymusepipe.util\_image.PointingTable property)@\spxentry{fulltablename}\spxextra{pymusepipe.util\_image.PointingTable property}}

\begin{fulllineitems}
\phantomsection\label{\detokenize{api/pymusepipe:pymusepipe.util_image.PointingTable.fulltablename}}
\pysigstartsignatures
\pysigline{\sphinxbfcode{\sphinxupquote{property\DUrole{w,w}{  }}}\sphinxbfcode{\sphinxupquote{fulltablename}}}
\pysigstopsignatures
\end{fulllineitems}

\index{list\_colnames\_ptable (pymusepipe.util\_image.PointingTable attribute)@\spxentry{list\_colnames\_ptable}\spxextra{pymusepipe.util\_image.PointingTable attribute}}

\begin{fulllineitems}
\phantomsection\label{\detokenize{api/pymusepipe:pymusepipe.util_image.PointingTable.list_colnames_ptable}}
\pysigstartsignatures
\pysigline{\sphinxbfcode{\sphinxupquote{list\_colnames\_ptable}}\sphinxbfcode{\sphinxupquote{\DUrole{w,w}{  }\DUrole{p,p}{=}\DUrole{w,w}{  }{[}\textquotesingle{}filename\textquotesingle{}, \textquotesingle{}dataset\textquotesingle{}, \textquotesingle{}tpls\textquotesingle{}, \textquotesingle{}expo\textquotesingle{}{]}}}}
\pysigstopsignatures
\end{fulllineitems}

\index{list\_datasets (pymusepipe.util\_image.PointingTable property)@\spxentry{list\_datasets}\spxextra{pymusepipe.util\_image.PointingTable property}}

\begin{fulllineitems}
\phantomsection\label{\detokenize{api/pymusepipe:pymusepipe.util_image.PointingTable.list_datasets}}
\pysigstartsignatures
\pysigline{\sphinxbfcode{\sphinxupquote{property\DUrole{w,w}{  }}}\sphinxbfcode{\sphinxupquote{list\_datasets}}}
\pysigstopsignatures
\sphinxAtStartPar
List of unique datasets in the pointing table

\end{fulllineitems}

\index{list\_pointings (pymusepipe.util\_image.PointingTable property)@\spxentry{list\_pointings}\spxextra{pymusepipe.util\_image.PointingTable property}}

\begin{fulllineitems}
\phantomsection\label{\detokenize{api/pymusepipe:pymusepipe.util_image.PointingTable.list_pointings}}
\pysigstartsignatures
\pysigline{\sphinxbfcode{\sphinxupquote{property\DUrole{w,w}{  }}}\sphinxbfcode{\sphinxupquote{list\_pointings}}}
\pysigstopsignatures
\sphinxAtStartPar
List of unique pointings in the pointing table

\end{fulllineitems}

\index{read() (pymusepipe.util\_image.PointingTable method)@\spxentry{read()}\spxextra{pymusepipe.util\_image.PointingTable method}}

\begin{fulllineitems}
\phantomsection\label{\detokenize{api/pymusepipe:pymusepipe.util_image.PointingTable.read}}
\pysigstartsignatures
\pysiglinewithargsret{\sphinxbfcode{\sphinxupquote{read}}}{\sphinxparam{\DUrole{o,o}{**}\DUrole{n,n}{kwargs}}}{}
\pysigstopsignatures
\sphinxAtStartPar
Read the input filename in given folder assuming a given format.


\paragraph{Input}
\label{\detokenize{api/pymusepipe:id124}}\begin{description}
\sphinxlineitem{filename: str, optional}
\sphinxAtStartPar
Name of the filename

\sphinxlineitem{folder: str default=’’, optional}
\sphinxAtStartPar
Name of the folder where to find the filename

\end{description}

\sphinxAtStartPar
table\_format: str default=’ascii’
\begin{quote}\begin{description}
\sphinxlineitem{rtype}
\sphinxAtStartPar
self.qtable with the content of the file

\end{description}\end{quote}

\end{fulllineitems}

\index{scan\_folder() (pymusepipe.util\_image.PointingTable method)@\spxentry{scan\_folder()}\spxextra{pymusepipe.util\_image.PointingTable method}}

\begin{fulllineitems}
\phantomsection\label{\detokenize{api/pymusepipe:pymusepipe.util_image.PointingTable.scan_folder}}
\pysigstartsignatures
\pysiglinewithargsret{\sphinxbfcode{\sphinxupquote{scan\_folder}}}{\sphinxparam{\DUrole{n,n}{folder}\DUrole{o,o}{=}\DUrole{default_value}{None}}, \sphinxparam{\DUrole{o,o}{**}\DUrole{n,n}{kwargs}}}{}
\pysigstopsignatures
\sphinxAtStartPar
Scan a folder to create a full pointing table


\paragraph{Input}
\label{\detokenize{api/pymusepipe:id125}}\begin{description}
\sphinxlineitem{folder: str default to None}
\sphinxAtStartPar
If not provided, will use the default self.folder

\sphinxlineitem{{\color{red}\bfseries{}**}kwargs:}
\sphinxAtStartPar
Other keywords are passed to create\_pointing\_table\_from\_folder

\end{description}


\paragraph{Creates}
\label{\detokenize{api/pymusepipe:id128}}
\sphinxAtStartPar
Attribute qtable

\end{fulllineitems}

\index{select\_datasets() (pymusepipe.util\_image.PointingTable method)@\spxentry{select\_datasets()}\spxextra{pymusepipe.util\_image.PointingTable method}}

\begin{fulllineitems}
\phantomsection\label{\detokenize{api/pymusepipe:pymusepipe.util_image.PointingTable.select_datasets}}
\pysigstartsignatures
\pysiglinewithargsret{\sphinxbfcode{\sphinxupquote{select\_datasets}}}{\sphinxparam{\DUrole{o,o}{**}\DUrole{n,n}{kwargs}}}{}
\pysigstopsignatures
\sphinxAtStartPar
Select all filenames with a given list of datasets


\paragraph{Input}
\label{\detokenize{api/pymusepipe:id129}}
\sphinxAtStartPar
list\_datasets: list of int, optional


\paragraph{Updates}
\label{\detokenize{api/pymusepipe:id130}}
\sphinxAtStartPar
‘select’ values in the astropy pointing table according to the list of datasets

\end{fulllineitems}

\index{select\_filename() (pymusepipe.util\_image.PointingTable method)@\spxentry{select\_filename()}\spxextra{pymusepipe.util\_image.PointingTable method}}

\begin{fulllineitems}
\phantomsection\label{\detokenize{api/pymusepipe:pymusepipe.util_image.PointingTable.select_filename}}
\pysigstartsignatures
\pysiglinewithargsret{\sphinxbfcode{\sphinxupquote{select\_filename}}}{\sphinxparam{\DUrole{n,n}{filename}}, \sphinxparam{\DUrole{n,n}{verbose}\DUrole{o,o}{=}\DUrole{default_value}{False}}}{}
\pysigstopsignatures
\sphinxAtStartPar
Select the filename as provided, by putting the value of column ‘select’ to 1


\paragraph{Input}
\label{\detokenize{api/pymusepipe:id131}}\begin{description}
\sphinxlineitem{filename: str}
\sphinxAtStartPar
Name of the file (see column ‘filename’)

\end{description}

\sphinxAtStartPar
Put the right value of the ‘select’ column to 1

\end{fulllineitems}

\index{select\_pointings() (pymusepipe.util\_image.PointingTable method)@\spxentry{select\_pointings()}\spxextra{pymusepipe.util\_image.PointingTable method}}

\begin{fulllineitems}
\phantomsection\label{\detokenize{api/pymusepipe:pymusepipe.util_image.PointingTable.select_pointings}}
\pysigstartsignatures
\pysiglinewithargsret{\sphinxbfcode{\sphinxupquote{select\_pointings}}}{\sphinxparam{\DUrole{o,o}{**}\DUrole{n,n}{kwargs}}}{}
\pysigstopsignatures
\sphinxAtStartPar
Select all filenames with pointings in the pointing list


\paragraph{Input}
\label{\detokenize{api/pymusepipe:id132}}
\sphinxAtStartPar
list\_pointings: list of int, optional


\paragraph{Updates}
\label{\detokenize{api/pymusepipe:id133}}
\sphinxAtStartPar
‘select’ values in the astropy pointing table according to the list of pointings

\end{fulllineitems}

\index{select\_pointings\_and\_datasets() (pymusepipe.util\_image.PointingTable method)@\spxentry{select\_pointings\_and\_datasets()}\spxextra{pymusepipe.util\_image.PointingTable method}}

\begin{fulllineitems}
\phantomsection\label{\detokenize{api/pymusepipe:pymusepipe.util_image.PointingTable.select_pointings_and_datasets}}
\pysigstartsignatures
\pysiglinewithargsret{\sphinxbfcode{\sphinxupquote{select\_pointings\_and\_datasets}}}{\sphinxparam{\DUrole{o,o}{**}\DUrole{n,n}{kwargs}}}{}
\pysigstopsignatures
\sphinxAtStartPar
Select all filenames with that pointing or dataset number.


\paragraph{Input}
\label{\detokenize{api/pymusepipe:id134}}
\sphinxAtStartPar
list\_pointings: list of int, optional
list\_datasets: list of int, optional


\paragraph{Updates}
\label{\detokenize{api/pymusepipe:id135}}
\sphinxAtStartPar
‘select’ values in the astropy pointing table according to the list of pointings and
datasets

\end{fulllineitems}

\index{selected\_filenames (pymusepipe.util\_image.PointingTable property)@\spxentry{selected\_filenames}\spxextra{pymusepipe.util\_image.PointingTable property}}

\begin{fulllineitems}
\phantomsection\label{\detokenize{api/pymusepipe:pymusepipe.util_image.PointingTable.selected_filenames}}
\pysigstartsignatures
\pysigline{\sphinxbfcode{\sphinxupquote{property\DUrole{w,w}{  }}}\sphinxbfcode{\sphinxupquote{selected\_filenames}}}
\pysigstopsignatures
\sphinxAtStartPar
Return the list of filenames following the selection
\begin{quote}\begin{description}
\sphinxlineitem{Return type}
\sphinxAtStartPar
list\_filename

\end{description}\end{quote}

\end{fulllineitems}

\index{set\_select\_value() (pymusepipe.util\_image.PointingTable method)@\spxentry{set\_select\_value()}\spxextra{pymusepipe.util\_image.PointingTable method}}

\begin{fulllineitems}
\phantomsection\label{\detokenize{api/pymusepipe:pymusepipe.util_image.PointingTable.set_select_value}}
\pysigstartsignatures
\pysiglinewithargsret{\sphinxbfcode{\sphinxupquote{set\_select\_value}}}{\sphinxparam{\DUrole{n,n}{filename}}, \sphinxparam{\DUrole{n,n}{value}\DUrole{o,o}{=}\DUrole{default_value}{1}}, \sphinxparam{\DUrole{n,n}{verbose}\DUrole{o,o}{=}\DUrole{default_value}{False}}}{}
\pysigstopsignatures
\sphinxAtStartPar
Set the value of the select column to 1, according to a given filename

\sphinxAtStartPar
filename: str
value: int default 1

\end{fulllineitems}

\index{unselect\_filename() (pymusepipe.util\_image.PointingTable method)@\spxentry{unselect\_filename()}\spxextra{pymusepipe.util\_image.PointingTable method}}

\begin{fulllineitems}
\phantomsection\label{\detokenize{api/pymusepipe:pymusepipe.util_image.PointingTable.unselect_filename}}
\pysigstartsignatures
\pysiglinewithargsret{\sphinxbfcode{\sphinxupquote{unselect\_filename}}}{\sphinxparam{\DUrole{n,n}{filename}}, \sphinxparam{\DUrole{n,n}{verbose}\DUrole{o,o}{=}\DUrole{default_value}{False}}}{}
\pysigstopsignatures
\sphinxAtStartPar
Select the filename as provided, by putting the value of column ‘select’ to 1


\paragraph{Input}
\label{\detokenize{api/pymusepipe:id136}}\begin{description}
\sphinxlineitem{filename: str}
\sphinxAtStartPar
Name of the file (see column ‘filename’)

\end{description}

\sphinxAtStartPar
Put the right value of the ‘select’ column to 1

\end{fulllineitems}

\index{write() (pymusepipe.util\_image.PointingTable method)@\spxentry{write()}\spxextra{pymusepipe.util\_image.PointingTable method}}

\begin{fulllineitems}
\phantomsection\label{\detokenize{api/pymusepipe:pymusepipe.util_image.PointingTable.write}}
\pysigstartsignatures
\pysiglinewithargsret{\sphinxbfcode{\sphinxupquote{write}}}{\sphinxparam{\DUrole{n,n}{overwrite}\DUrole{o,o}{=}\DUrole{default_value}{False}}, \sphinxparam{\DUrole{o,o}{**}\DUrole{n,n}{kwargs}}}{}
\pysigstopsignatures
\sphinxAtStartPar
Write out the table on disk, using the nameout and provided folder.
\begin{quote}\begin{description}
\sphinxlineitem{Parameters}\begin{itemize}
\item {} 
\sphinxAtStartPar
\sphinxstyleliteralstrong{\sphinxupquote{overwrite}} (\sphinxstyleliteralemphasis{\sphinxupquote{bool default=False}}) \textendash{} 

\item {} 
\sphinxAtStartPar
\sphinxstyleliteralstrong{\sphinxupquote{**kwargs}} \textendash{} Valid keywords are
folder: str
nameout: str
Extra keywords are passed to the astropy QTable.write() function

\item {} 
\sphinxAtStartPar
\sphinxstyleliteralstrong{\sphinxupquote{disk}} (\sphinxstyleliteralemphasis{\sphinxupquote{Writes the pointing table on}}) \textendash{} 

\end{itemize}

\end{description}\end{quote}

\end{fulllineitems}


\end{fulllineitems}

\index{RectangleZone (class in pymusepipe.util\_image)@\spxentry{RectangleZone}\spxextra{class in pymusepipe.util\_image}}

\begin{fulllineitems}
\phantomsection\label{\detokenize{api/pymusepipe:pymusepipe.util_image.RectangleZone}}
\pysigstartsignatures
\pysigline{\sphinxbfcode{\sphinxupquote{class\DUrole{w,w}{  }}}\sphinxcode{\sphinxupquote{pymusepipe.util\_image.}}\sphinxbfcode{\sphinxupquote{RectangleZone}}}
\pysigstopsignatures
\sphinxAtStartPar
Bases: {\hyperref[\detokenize{api/pymusepipe:pymusepipe.util_image.SelectionZone}]{\sphinxcrossref{\sphinxcode{\sphinxupquote{SelectionZone}}}}}

\sphinxAtStartPar
Define a rectangular zone, given by
a center, a length, a width and an angle
\index{select() (pymusepipe.util\_image.RectangleZone method)@\spxentry{select()}\spxextra{pymusepipe.util\_image.RectangleZone method}}

\begin{fulllineitems}
\phantomsection\label{\detokenize{api/pymusepipe:pymusepipe.util_image.RectangleZone.select}}
\pysigstartsignatures
\pysiglinewithargsret{\sphinxbfcode{\sphinxupquote{select}}}{\sphinxparam{\DUrole{n,n}{xin}}, \sphinxparam{\DUrole{n,n}{yin}}}{}
\pysigstopsignatures\begin{description}
\sphinxlineitem{Define a selection within a rectangle}
\sphinxAtStartPar
It can be rotated by an angle theta (in degrees)

\end{description}


\paragraph{Input}
\label{\detokenize{api/pymusepipe:id137}}\begin{description}
\sphinxlineitem{xin, yin: 2d arrays}
\sphinxAtStartPar
Input positions for the spaxels

\end{description}

\end{fulllineitems}


\end{fulllineitems}

\index{SelectionZone (class in pymusepipe.util\_image)@\spxentry{SelectionZone}\spxextra{class in pymusepipe.util\_image}}

\begin{fulllineitems}
\phantomsection\label{\detokenize{api/pymusepipe:pymusepipe.util_image.SelectionZone}}
\pysigstartsignatures
\pysiglinewithargsret{\sphinxbfcode{\sphinxupquote{class\DUrole{w,w}{  }}}\sphinxcode{\sphinxupquote{pymusepipe.util\_image.}}\sphinxbfcode{\sphinxupquote{SelectionZone}}}{\sphinxparam{\DUrole{n,n}{params}\DUrole{o,o}{=}\DUrole{default_value}{None}}}{}
\pysigstopsignatures
\sphinxAtStartPar
Bases: \sphinxhref{https://docs.python.org/3.10/library/functions.html\#object}{\sphinxcode{\sphinxupquote{object}}}

\sphinxAtStartPar
Parent class for Rectangle\_Zone and Circle\_Zone


\paragraph{Input}
\label{\detokenize{api/pymusepipe:id138}}\begin{description}
\sphinxlineitem{params: list of floats}
\sphinxAtStartPar
List of parameters for the selection zone

\end{description}

\end{fulllineitems}

\index{TrailZone (class in pymusepipe.util\_image)@\spxentry{TrailZone}\spxextra{class in pymusepipe.util\_image}}

\begin{fulllineitems}
\phantomsection\label{\detokenize{api/pymusepipe:pymusepipe.util_image.TrailZone}}
\pysigstartsignatures
\pysigline{\sphinxbfcode{\sphinxupquote{class\DUrole{w,w}{  }}}\sphinxcode{\sphinxupquote{pymusepipe.util\_image.}}\sphinxbfcode{\sphinxupquote{TrailZone}}}
\pysigstopsignatures
\sphinxAtStartPar
Bases: {\hyperref[\detokenize{api/pymusepipe:pymusepipe.util_image.SelectionZone}]{\sphinxcrossref{\sphinxcode{\sphinxupquote{SelectionZone}}}}}

\sphinxAtStartPar
Define a Trail zone, defined by
two points and a width
\index{select() (pymusepipe.util\_image.TrailZone method)@\spxentry{select()}\spxextra{pymusepipe.util\_image.TrailZone method}}

\begin{fulllineitems}
\phantomsection\label{\detokenize{api/pymusepipe:pymusepipe.util_image.TrailZone.select}}
\pysigstartsignatures
\pysiglinewithargsret{\sphinxbfcode{\sphinxupquote{select}}}{\sphinxparam{\DUrole{n,n}{xin}}, \sphinxparam{\DUrole{n,n}{yin}}}{}
\pysigstopsignatures
\sphinxAtStartPar
Define a selection within trail


\paragraph{Input}
\label{\detokenize{api/pymusepipe:id139}}\begin{description}
\sphinxlineitem{xin, yin: 2d arrays}
\sphinxAtStartPar
Input positions for the spaxels

\end{description}

\end{fulllineitems}


\end{fulllineitems}

\index{check\_column\_set() (in module pymusepipe.util\_image)@\spxentry{check\_column\_set()}\spxextra{in module pymusepipe.util\_image}}

\begin{fulllineitems}
\phantomsection\label{\detokenize{api/pymusepipe:pymusepipe.util_image.check_column_set}}
\pysigstartsignatures
\pysiglinewithargsret{\sphinxcode{\sphinxupquote{pymusepipe.util\_image.}}\sphinxbfcode{\sphinxupquote{check\_column\_set}}}{\sphinxparam{\DUrole{n,n}{input\_table}}}{}
\pysigstopsignatures
\sphinxAtStartPar
Check the minimum column set for the Pointing table


\paragraph{Input}
\label{\detokenize{api/pymusepipe:id140}}
\sphinxAtStartPar
input\_table: astropy Table
\begin{quote}\begin{description}
\sphinxlineitem{returns}
\sphinxAtStartPar
\sphinxstylestrong{bool}

\sphinxlineitem{rtype}
\sphinxAtStartPar
True if all names are in the table, False otherwise

\end{description}\end{quote}

\end{fulllineitems}

\index{chunk\_stats() (in module pymusepipe.util\_image)@\spxentry{chunk\_stats()}\spxextra{in module pymusepipe.util\_image}}

\begin{fulllineitems}
\phantomsection\label{\detokenize{api/pymusepipe:pymusepipe.util_image.chunk_stats}}
\pysigstartsignatures
\pysiglinewithargsret{\sphinxcode{\sphinxupquote{pymusepipe.util\_image.}}\sphinxbfcode{\sphinxupquote{chunk\_stats}}}{\sphinxparam{\DUrole{n,n}{list\_arrays}}, \sphinxparam{\DUrole{n,n}{chunk\_size}\DUrole{o,o}{=}\DUrole{default_value}{15}}}{}
\pysigstopsignatures
\sphinxAtStartPar
Cut the datasets in 2d chunks and take the median
Return the set of medians for all chunks.
\begin{quote}\begin{description}
\sphinxlineitem{Parameters}\begin{itemize}
\item {} 
\sphinxAtStartPar
\sphinxstyleliteralstrong{\sphinxupquote{list\_arrays}} (\sphinxhref{https://docs.python.org/3.10/library/stdtypes.html\#list}{\sphinxstyleliteralemphasis{\sphinxupquote{list}}}\sphinxstyleliteralemphasis{\sphinxupquote{ of }}\sphinxstyleliteralemphasis{\sphinxupquote{np.arrays}}) \textendash{} List of arrays with the same sizes/shapes

\item {} 
\sphinxAtStartPar
\sphinxstyleliteralstrong{\sphinxupquote{chunk\_size}} (\sphinxhref{https://docs.python.org/3.10/library/functions.html\#int}{\sphinxstyleliteralemphasis{\sphinxupquote{int}}}) \textendash{} number of pixel (one D of a 2D chunk)
of the chunk to consider (Default value = 15)

\end{itemize}

\sphinxlineitem{Returns}
\sphinxAtStartPar
\sphinxstylestrong{median, standard} \textendash{} for the given datasets analysed in chunks.

\sphinxlineitem{Return type}
\sphinxAtStartPar
2 arrays of the medians and standard deviations

\end{description}\end{quote}

\end{fulllineitems}

\index{compute\_diagnostics() (in module pymusepipe.util\_image)@\spxentry{compute\_diagnostics()}\spxextra{in module pymusepipe.util\_image}}

\begin{fulllineitems}
\phantomsection\label{\detokenize{api/pymusepipe:pymusepipe.util_image.compute_diagnostics}}
\pysigstartsignatures
\pysiglinewithargsret{\sphinxcode{\sphinxupquote{pymusepipe.util\_image.}}\sphinxbfcode{\sphinxupquote{compute\_diagnostics}}}{\sphinxparam{\DUrole{n,n}{pointing\_dict}}, \sphinxparam{\DUrole{n,n}{center\_dict}}}{}
\pysigstopsignatures
\sphinxAtStartPar
Compute the average and std of the distance between the exposures belonging
to the same pointing.


\paragraph{Input}
\label{\detokenize{api/pymusepipe:id141}}\begin{description}
\sphinxlineitem{pointing\_dict: dict}
\sphinxAtStartPar
Dictionary for the pointings

\sphinxlineitem{center\_dict: dict}
\sphinxAtStartPar
dictionary of the files to be used

\end{description}
\begin{quote}\begin{description}
\sphinxlineitem{returns}
\sphinxAtStartPar
\sphinxstylestrong{Diagnostic} \textendash{} Each pointing key has its {[}mean, std{]} as value of the distionary

\sphinxlineitem{rtype}
\sphinxAtStartPar
dict

\end{description}\end{quote}

\end{fulllineitems}

\index{create\_offset\_table() (in module pymusepipe.util\_image)@\spxentry{create\_offset\_table()}\spxextra{in module pymusepipe.util\_image}}

\begin{fulllineitems}
\phantomsection\label{\detokenize{api/pymusepipe:pymusepipe.util_image.create_offset_table}}
\pysigstartsignatures
\pysiglinewithargsret{\sphinxcode{\sphinxupquote{pymusepipe.util\_image.}}\sphinxbfcode{\sphinxupquote{create\_offset\_table}}}{\sphinxparam{\DUrole{n,n}{image\_names}}, \sphinxparam{\DUrole{n,n}{table\_folder}\DUrole{o,o}{=}\DUrole{default_value}{\textquotesingle{}\textquotesingle{}}}, \sphinxparam{\DUrole{n,n}{table\_name}\DUrole{o,o}{=}\DUrole{default_value}{\textquotesingle{}dummy\_offset\_table.fits\textquotesingle{}}}, \sphinxparam{\DUrole{n,n}{overwrite}\DUrole{o,o}{=}\DUrole{default_value}{False}}}{}
\pysigstopsignatures
\sphinxAtStartPar
Create an offset list table from a given set of images. It will use
the MJD and DATE as read from the descriptors of the images. The names for
these keywords is stored in the dictionary default\_offset\_table from
config\_pipe.py
\begin{quote}\begin{description}
\sphinxlineitem{Parameters}\begin{itemize}
\item {} 
\sphinxAtStartPar
\sphinxstyleliteralstrong{\sphinxupquote{image\_names}} (\sphinxhref{https://docs.python.org/3.10/library/stdtypes.html\#list}{\sphinxstyleliteralemphasis{\sphinxupquote{list}}}\sphinxstyleliteralemphasis{\sphinxupquote{ of }}\sphinxhref{https://docs.python.org/3.10/library/stdtypes.html\#str}{\sphinxstyleliteralemphasis{\sphinxupquote{str}}}) \textendash{} List of image names to be considered. (Default value = {[}{]})

\item {} 
\sphinxAtStartPar
\sphinxstyleliteralstrong{\sphinxupquote{table\_folder}} (\sphinxhref{https://docs.python.org/3.10/library/stdtypes.html\#str}{\sphinxstyleliteralemphasis{\sphinxupquote{str}}}) \textendash{} folder of the table (Default value = “”)

\item {} 
\sphinxAtStartPar
\sphinxstyleliteralstrong{\sphinxupquote{table\_name}} (\sphinxhref{https://docs.python.org/3.10/library/stdtypes.html\#str}{\sphinxstyleliteralemphasis{\sphinxupquote{str}}}) \textendash{} name of the table to save {[}‘dummy\_offset\_table.fits’{]}
(Default value = “dummy\_offset\_table.fits”)

\item {} 
\sphinxAtStartPar
\sphinxstyleliteralstrong{\sphinxupquote{overwrite}} (\sphinxhref{https://docs.python.org/3.10/library/functions.html\#bool}{\sphinxstyleliteralemphasis{\sphinxupquote{bool}}}) \textendash{} if the table exists, it will be overwritten if set
to True only. (Default value = False)

\item {} 
\sphinxAtStartPar
\sphinxstyleliteralstrong{\sphinxupquote{overwrite}} \textendash{} if the table exists, it will be overwritten if set
to True only. (Default value = False)

\end{itemize}

\sphinxlineitem{Return type}
\sphinxAtStartPar
A fits table with the output given name. (Default value = False)

\end{description}\end{quote}

\end{fulllineitems}

\index{crop\_data() (in module pymusepipe.util\_image)@\spxentry{crop\_data()}\spxextra{in module pymusepipe.util\_image}}

\begin{fulllineitems}
\phantomsection\label{\detokenize{api/pymusepipe:pymusepipe.util_image.crop_data}}
\pysigstartsignatures
\pysiglinewithargsret{\sphinxcode{\sphinxupquote{pymusepipe.util\_image.}}\sphinxbfcode{\sphinxupquote{crop\_data}}}{\sphinxparam{\DUrole{n,n}{data}}, \sphinxparam{\DUrole{n,n}{border}\DUrole{o,o}{=}\DUrole{default_value}{10}}}{}
\pysigstopsignatures
\sphinxAtStartPar
Crop a 2D data and return it cropped after a border
has been removed (number of pixels) from each edge
(borderx2 pixels are removed from each dimension)


\paragraph{Input}
\label{\detokenize{api/pymusepipe:id142}}\begin{description}
\sphinxlineitem{data: 2d array}
\sphinxAtStartPar
Array which has the signal to be cropped

\sphinxlineitem{border: int}
\sphinxAtStartPar
Number of pixels to be cropped at each edge

\end{description}
\begin{quote}\begin{description}
\sphinxlineitem{returns}
\sphinxAtStartPar
\sphinxstylestrong{cdata} \textendash{} Cropped data array

\sphinxlineitem{rtype}
\sphinxAtStartPar
2d array

\end{description}\end{quote}

\end{fulllineitems}

\index{filter\_list\_with\_pointingtable() (in module pymusepipe.util\_image)@\spxentry{filter\_list\_with\_pointingtable()}\spxextra{in module pymusepipe.util\_image}}

\begin{fulllineitems}
\phantomsection\label{\detokenize{api/pymusepipe:pymusepipe.util_image.filter_list_with_pointingtable}}
\pysigstartsignatures
\pysiglinewithargsret{\sphinxcode{\sphinxupquote{pymusepipe.util\_image.}}\sphinxbfcode{\sphinxupquote{filter\_list\_with\_pointingtable}}}{\sphinxparam{\DUrole{n,n}{input\_list}}, \sphinxparam{\DUrole{n,n}{pointing\_table}\DUrole{o,o}{=}\DUrole{default_value}{None}}, \sphinxparam{\DUrole{n,n}{verbose}\DUrole{o,o}{=}\DUrole{default_value}{True}}, \sphinxparam{\DUrole{n,n}{str\_dataset}\DUrole{o,o}{=}\DUrole{default_value}{\textquotesingle{}OB\textquotesingle{}}}, \sphinxparam{\DUrole{n,n}{ndigits}\DUrole{o,o}{=}\DUrole{default_value}{3}}, \sphinxparam{\DUrole{n,n}{list\_pointings}\DUrole{o,o}{=}\DUrole{default_value}{None}}, \sphinxparam{\DUrole{n,n}{filtername}\DUrole{o,o}{=}\DUrole{default_value}{None}}}{}
\pysigstopsignatures

\paragraph{Input}
\label{\detokenize{api/pymusepipe:id143}}\begin{description}
\sphinxlineitem{input\_list: list of str}
\sphinxAtStartPar
Input list of filenames to filter

\end{description}

\sphinxAtStartPar
pointing\_table: PointingTable or QTable or Table
str\_dataset: str default=default\_str\_dataset
ndigits: int default=default\_ndigits
filtername: str default=None
verbose: bool default=True

\end{fulllineitems}

\index{filtermed\_image() (in module pymusepipe.util\_image)@\spxentry{filtermed\_image()}\spxextra{in module pymusepipe.util\_image}}

\begin{fulllineitems}
\phantomsection\label{\detokenize{api/pymusepipe:pymusepipe.util_image.filtermed_image}}
\pysigstartsignatures
\pysiglinewithargsret{\sphinxcode{\sphinxupquote{pymusepipe.util\_image.}}\sphinxbfcode{\sphinxupquote{filtermed\_image}}}{\sphinxparam{\DUrole{n,n}{data}}, \sphinxparam{\DUrole{n,n}{border}\DUrole{o,o}{=}\DUrole{default_value}{0}}, \sphinxparam{\DUrole{n,n}{filter\_size}\DUrole{o,o}{=}\DUrole{default_value}{2}}, \sphinxparam{\DUrole{n,n}{keepnan}\DUrole{o,o}{=}\DUrole{default_value}{False}}}{}
\pysigstopsignatures
\sphinxAtStartPar
Process image by removing the borders
and filtering it via a median filter


\paragraph{Input}
\label{\detokenize{api/pymusepipe:id144}}\begin{description}
\sphinxlineitem{data: 2d array}
\sphinxAtStartPar
Array to be processed

\sphinxlineitem{border: int}
\sphinxAtStartPar
Number of pixels to remove at each edge

\sphinxlineitem{filter\_size: float}
\sphinxAtStartPar
Size of the filtering (median)

\end{description}
\begin{quote}\begin{description}
\sphinxlineitem{returns}
\sphinxAtStartPar
\sphinxstylestrong{cdata} \textendash{} Processed array

\sphinxlineitem{rtype}
\sphinxAtStartPar
2d array

\end{description}\end{quote}

\end{fulllineitems}

\index{flatclean\_image() (in module pymusepipe.util\_image)@\spxentry{flatclean\_image()}\spxextra{in module pymusepipe.util\_image}}

\begin{fulllineitems}
\phantomsection\label{\detokenize{api/pymusepipe:pymusepipe.util_image.flatclean_image}}
\pysigstartsignatures
\pysiglinewithargsret{\sphinxcode{\sphinxupquote{pymusepipe.util\_image.}}\sphinxbfcode{\sphinxupquote{flatclean\_image}}}{\sphinxparam{\DUrole{n,n}{data}}, \sphinxparam{\DUrole{n,n}{border}\DUrole{o,o}{=}\DUrole{default_value}{10}}, \sphinxparam{\DUrole{n,n}{dynamic\_range}\DUrole{o,o}{=}\DUrole{default_value}{10}}, \sphinxparam{\DUrole{n,n}{median\_window}\DUrole{o,o}{=}\DUrole{default_value}{10}}, \sphinxparam{\DUrole{n,n}{threshold}\DUrole{o,o}{=}\DUrole{default_value}{0.0}}, \sphinxparam{\DUrole{n,n}{squeeze}\DUrole{o,o}{=}\DUrole{default_value}{True}}, \sphinxparam{\DUrole{n,n}{remove\_bkg}\DUrole{o,o}{=}\DUrole{default_value}{True}}}{}
\pysigstopsignatures
\sphinxAtStartPar
Process image by squeezing the range, removing
the borders and filtering it. The image is first filtered,
then it is cropped. All values below a given minimum are
set to 0 and all Nan set to 0 or infinity accordingly.


\paragraph{Input}
\label{\detokenize{api/pymusepipe:id145}}\begin{description}
\sphinxlineitem{data: 2d ndarray}
\sphinxAtStartPar
Input array to process

\sphinxlineitem{dynamic\_range: float {[}10{]}}
\sphinxAtStartPar
Dynamic range used to squash the bright pixels down

\sphinxlineitem{median\_window: int {[}10{]}}
\sphinxAtStartPar
Size of the window used for the median filtering.

\sphinxlineitem{threshold: float {[}0{]}}
\sphinxAtStartPar
Value of the minimum value allowed.

\sphinxlineitem{squeeze: bool}
\sphinxAtStartPar
Squeeze the dynamic range by using the dynamic\_range variable

\sphinxlineitem{crop: bool}
\sphinxAtStartPar
Crop the borders using border as the variable

\end{description}

\sphinxAtStartPar
remove\_bkg: remove the filter\_medianed background
\begin{quote}\begin{description}
\sphinxlineitem{returns}
\sphinxAtStartPar
\sphinxstylestrong{flatcleaned\_array}

\sphinxlineitem{rtype}
\sphinxAtStartPar
2d ndarray

\end{description}\end{quote}

\end{fulllineitems}

\index{get\_centre\_from\_image\_or\_cube() (in module pymusepipe.util\_image)@\spxentry{get\_centre\_from\_image\_or\_cube()}\spxextra{in module pymusepipe.util\_image}}

\begin{fulllineitems}
\phantomsection\label{\detokenize{api/pymusepipe:pymusepipe.util_image.get_centre_from_image_or_cube}}
\pysigstartsignatures
\pysiglinewithargsret{\sphinxcode{\sphinxupquote{pymusepipe.util\_image.}}\sphinxbfcode{\sphinxupquote{get\_centre\_from\_image\_or\_cube}}}{\sphinxparam{\DUrole{n,n}{filename}}, \sphinxparam{\DUrole{n,n}{ext}\DUrole{o,o}{=}\DUrole{default_value}{1}}, \sphinxparam{\DUrole{n,n}{dtype}\DUrole{o,o}{=}\DUrole{default_value}{\textquotesingle{}image\textquotesingle{}}}}{}
\pysigstopsignatures
\sphinxAtStartPar
Compute the coordinate of the center of the FOV from an image. Only pixels with
actual signal are considered.


\paragraph{Input}
\label{\detokenize{api/pymusepipe:id146}}\begin{description}
\sphinxlineitem{file\_name: st}
\sphinxAtStartPar
name of the file to analyse

\sphinxlineitem{ext: int default=1, optional}
\sphinxAtStartPar
extension where the data and WCS info are located. Defaults to 1.

\sphinxlineitem{dtype: str default=’image’, optional}
\sphinxAtStartPar
type of file to be analyzed. It can be either image or cube. Defaults to image.

\end{description}
\begin{quote}\begin{description}
\sphinxlineitem{returns}
\sphinxAtStartPar
\sphinxstylestrong{SkyCoord} \textendash{} Coordinate of the center of the FOV

\sphinxlineitem{rtype}
\sphinxAtStartPar
astropy.coordinates.SkyCoord

\end{description}\end{quote}

\end{fulllineitems}

\index{get\_centre\_from\_pixtable() (in module pymusepipe.util\_image)@\spxentry{get\_centre\_from\_pixtable()}\spxextra{in module pymusepipe.util\_image}}

\begin{fulllineitems}
\phantomsection\label{\detokenize{api/pymusepipe:pymusepipe.util_image.get_centre_from_pixtable}}
\pysigstartsignatures
\pysiglinewithargsret{\sphinxcode{\sphinxupquote{pymusepipe.util\_image.}}\sphinxbfcode{\sphinxupquote{get\_centre\_from\_pixtable}}}{\sphinxparam{\DUrole{n,n}{pixtable\_name}}}{}
\pysigstopsignatures
\sphinxAtStartPar
Get the center of the FOV from pixtables


\paragraph{Input}
\label{\detokenize{api/pymusepipe:id147}}\begin{description}
\sphinxlineitem{pixtable\_name: str}
\sphinxAtStartPar
name of the pixeltable

\end{description}
\begin{quote}\begin{description}
\sphinxlineitem{returns}
\sphinxAtStartPar
\sphinxstylestrong{SkyCoord} \textendash{} Coordinates of the center of the field

\sphinxlineitem{rtype}
\sphinxAtStartPar
astropy.coordinates.Skycoord

\end{description}\end{quote}

\end{fulllineitems}

\index{get\_flux\_range() (in module pymusepipe.util\_image)@\spxentry{get\_flux\_range()}\spxextra{in module pymusepipe.util\_image}}

\begin{fulllineitems}
\phantomsection\label{\detokenize{api/pymusepipe:pymusepipe.util_image.get_flux_range}}
\pysigstartsignatures
\pysiglinewithargsret{\sphinxcode{\sphinxupquote{pymusepipe.util\_image.}}\sphinxbfcode{\sphinxupquote{get\_flux\_range}}}{\sphinxparam{\DUrole{n,n}{data}}, \sphinxparam{\DUrole{n,n}{border}\DUrole{o,o}{=}\DUrole{default_value}{15}}, \sphinxparam{\DUrole{n,n}{low}\DUrole{o,o}{=}\DUrole{default_value}{2}}, \sphinxparam{\DUrole{n,n}{high}\DUrole{o,o}{=}\DUrole{default_value}{98}}}{}
\pysigstopsignatures
\sphinxAtStartPar
Get the range of fluxes within the array
by looking at percentiles.


\paragraph{Input}
\label{\detokenize{api/pymusepipe:id148}}\begin{description}
\sphinxlineitem{data: 2d array}
\sphinxAtStartPar
Input array with signal to process

\sphinxlineitem{low, high: two floats (10, 99)}
\sphinxAtStartPar
Percentiles to consider to filter

\end{description}
\begin{quote}\begin{description}
\sphinxlineitem{returns}
\sphinxAtStartPar
\sphinxstylestrong{lperc, hperc} \textendash{} Low and high percentiles

\sphinxlineitem{rtype}
\sphinxAtStartPar
2 floats

\end{description}\end{quote}

\end{fulllineitems}

\index{get\_normfactor() (in module pymusepipe.util\_image)@\spxentry{get\_normfactor()}\spxextra{in module pymusepipe.util\_image}}

\begin{fulllineitems}
\phantomsection\label{\detokenize{api/pymusepipe:pymusepipe.util_image.get_normfactor}}
\pysigstartsignatures
\pysiglinewithargsret{\sphinxcode{\sphinxupquote{pymusepipe.util\_image.}}\sphinxbfcode{\sphinxupquote{get\_normfactor}}}{\sphinxparam{\DUrole{n,n}{array1}}, \sphinxparam{\DUrole{n,n}{array2}}, \sphinxparam{\DUrole{n,n}{median\_filter}\DUrole{o,o}{=}\DUrole{default_value}{True}}, \sphinxparam{\DUrole{n,n}{border}\DUrole{o,o}{=}\DUrole{default_value}{0}}, \sphinxparam{\DUrole{n,n}{convolve\_data1}\DUrole{o,o}{=}\DUrole{default_value}{0.0}}, \sphinxparam{\DUrole{n,n}{convolve\_data2}\DUrole{o,o}{=}\DUrole{default_value}{0.0}}, \sphinxparam{\DUrole{n,n}{chunk\_size}\DUrole{o,o}{=}\DUrole{default_value}{10}}, \sphinxparam{\DUrole{n,n}{threshold}\DUrole{o,o}{=}\DUrole{default_value}{0.0}}, \sphinxparam{\DUrole{n,n}{add\_background1}\DUrole{o,o}{=}\DUrole{default_value}{0}}}{}
\pysigstopsignatures
\sphinxAtStartPar
Get the normalisation factor for shifted and projected images. This function
only consider the input images given by their data (numpy) arrays.


\paragraph{Input}
\label{\detokenize{api/pymusepipe:id149}}
\sphinxAtStartPar
array1: 2d np.array
array2: 2d np.array
\begin{quote}

\sphinxAtStartPar
Input arrays. Should be the same size
\end{quote}
\begin{description}
\sphinxlineitem{median\_filter: bool}
\sphinxAtStartPar
If True, will median filter

\sphinxlineitem{convolve\_muse: float {[}0{]}}
\sphinxAtStartPar
Will convolve the image with index nima
with a gaussian with that sigma. 0 means no convolution

\sphinxlineitem{convolve\_reference: float {[}0{]}}
\sphinxAtStartPar
Will convolve the reference image
with a gaussian with that sigma. 0 means no convolution

\sphinxlineitem{border: int}
\sphinxAtStartPar
Number of pixels to crop

\sphinxlineitem{threshold: float {[}None{]}}
\sphinxAtStartPar
Threshold for the input image flux to consider

\end{description}
\begin{quote}\begin{description}
\sphinxlineitem{returns}\begin{itemize}
\item {} 
\sphinxAtStartPar
\sphinxstylestrong{data} (\sphinxstyleemphasis{2d array})

\item {} 
\sphinxAtStartPar
\sphinxstylestrong{refdata} (\sphinxstyleemphasis{2d array}) \textendash{} The 2 arrays (input, reference) after processing

\item {} 
\sphinxAtStartPar
\sphinxstylestrong{polypar} (\sphinxstyleemphasis{the result of an ODR regression})

\end{itemize}

\end{description}\end{quote}

\end{fulllineitems}

\index{get\_polynorm() (in module pymusepipe.util\_image)@\spxentry{get\_polynorm()}\spxextra{in module pymusepipe.util\_image}}

\begin{fulllineitems}
\phantomsection\label{\detokenize{api/pymusepipe:pymusepipe.util_image.get_polynorm}}
\pysigstartsignatures
\pysiglinewithargsret{\sphinxcode{\sphinxupquote{pymusepipe.util\_image.}}\sphinxbfcode{\sphinxupquote{get\_polynorm}}}{\sphinxparam{\DUrole{n,n}{array1}}, \sphinxparam{\DUrole{n,n}{array2}}, \sphinxparam{\DUrole{n,n}{chunk\_size}\DUrole{o,o}{=}\DUrole{default_value}{15}}, \sphinxparam{\DUrole{n,n}{threshold1}\DUrole{o,o}{=}\DUrole{default_value}{0.0}}, \sphinxparam{\DUrole{n,n}{threshold2}\DUrole{o,o}{=}\DUrole{default_value}{0}}, \sphinxparam{\DUrole{n,n}{percentiles}\DUrole{o,o}{=}\DUrole{default_value}{(0.0, 100.0)}}, \sphinxparam{\DUrole{n,n}{sigclip}\DUrole{o,o}{=}\DUrole{default_value}{0}}}{}
\pysigstopsignatures
\sphinxAtStartPar
Find the normalisation factor between two arrays.

\sphinxAtStartPar
Including the background and slope. This uses the function
regress\_odr which is included in align\_pipe.py and itself
makes use of ODR in scipy.odr.ODR.
\begin{quote}\begin{description}
\sphinxlineitem{Parameters}\begin{itemize}
\item {} 
\sphinxAtStartPar
\sphinxstyleliteralstrong{\sphinxupquote{array1}} (\sphinxstyleliteralemphasis{\sphinxupquote{2D np.array}}) \textendash{} 

\item {} 
\sphinxAtStartPar
\sphinxstyleliteralstrong{\sphinxupquote{array2}} (\sphinxstyleliteralemphasis{\sphinxupquote{2D np.array}}) \textendash{} 2 arrays (2D) of identical shapes

\item {} 
\sphinxAtStartPar
\sphinxstyleliteralstrong{\sphinxupquote{chunk\_size}} (\sphinxhref{https://docs.python.org/3.10/library/functions.html\#int}{\sphinxstyleliteralemphasis{\sphinxupquote{int}}}) \textendash{} Default value = 15

\item {} 
\sphinxAtStartPar
\sphinxstyleliteralstrong{\sphinxupquote{threshold1}} (\sphinxhref{https://docs.python.org/3.10/library/functions.html\#float}{\sphinxstyleliteralemphasis{\sphinxupquote{float}}}) \textendash{} Lower threshold for array1 (Default value = 0.)

\item {} 
\sphinxAtStartPar
\sphinxstyleliteralstrong{\sphinxupquote{threshold2}} (\sphinxhref{https://docs.python.org/3.10/library/functions.html\#float}{\sphinxstyleliteralemphasis{\sphinxupquote{float}}}) \textendash{} Lower threshold for array2 (Default value = 0)

\item {} 
\sphinxAtStartPar
\sphinxstyleliteralstrong{\sphinxupquote{percentiles}} (\sphinxhref{https://docs.python.org/3.10/library/stdtypes.html\#list}{\sphinxstyleliteralemphasis{\sphinxupquote{list}}}\sphinxstyleliteralemphasis{\sphinxupquote{ of }}\sphinxstyleliteralemphasis{\sphinxupquote{2 floats}}) \textendash{} Percentiles (Default value = {[}0., 100.{]})

\item {} 
\sphinxAtStartPar
\sphinxstyleliteralstrong{\sphinxupquote{sigclip}} (\sphinxhref{https://docs.python.org/3.10/library/functions.html\#float}{\sphinxstyleliteralemphasis{\sphinxupquote{float}}}) \textendash{} Sigma clipping factor (Default value = 0)

\end{itemize}

\sphinxlineitem{Returns}
\sphinxAtStartPar
\sphinxstylestrong{result} \textendash{} Result of the regression (ODR)

\sphinxlineitem{Return type}
\sphinxAtStartPar
python structure

\end{description}\end{quote}

\end{fulllineitems}

\index{group\_exposures\_per\_pointing() (in module pymusepipe.util\_image)@\spxentry{group\_exposures\_per\_pointing()}\spxextra{in module pymusepipe.util\_image}}

\begin{fulllineitems}
\phantomsection\label{\detokenize{api/pymusepipe:pymusepipe.util_image.group_exposures_per_pointing}}
\pysigstartsignatures
\pysiglinewithargsret{\sphinxcode{\sphinxupquote{pymusepipe.util\_image.}}\sphinxbfcode{\sphinxupquote{group\_exposures\_per\_pointing}}}{\sphinxparam{\DUrole{n,n}{list\_files}}, \sphinxparam{\DUrole{n,n}{target\_path}\DUrole{o,o}{=}\DUrole{default_value}{\textquotesingle{}\textquotesingle{}}}, \sphinxparam{\DUrole{n,n}{limit}\DUrole{o,o}{=}\DUrole{default_value}{10.0}}, \sphinxparam{\DUrole{n,n}{unit}\DUrole{o,o}{=}\DUrole{default_value}{Unit(\textquotesingle{}arcsec\textquotesingle{})}}, \sphinxparam{\DUrole{n,n}{ext}\DUrole{o,o}{=}\DUrole{default_value}{1}}, \sphinxparam{\DUrole{n,n}{dtype}\DUrole{o,o}{=}\DUrole{default_value}{\textquotesingle{}image\textquotesingle{}}}}{}
\pysigstopsignatures
\sphinxAtStartPar
Separate a list of files in pointings based on their proximity.

\sphinxAtStartPar
This function assign each file to a pointing. Pointings are defined as groups of exposures
whose distance between the centers falls within a certain limit. Once the groups of exposures
have been defined, they are sorted, and then a pointing number starting from 1 is assigned to
all of them. Some info on the average std of the eparation between exposures can be optionally
computed.


\paragraph{Input}
\label{\detokenize{api/pymusepipe:id150}}\begin{description}
\sphinxlineitem{list\_files: list}
\sphinxAtStartPar
list of files to be reorganized in pointings

\sphinxlineitem{target\_path: str default=’’}
\sphinxAtStartPar
path of the target files

\sphinxlineitem{limit: float default=10}
\sphinxAtStartPar
maximum separation for files to belong to the same pointing

\sphinxlineitem{unit: astropy unit default=u.arcsec}
\sphinxAtStartPar
Unit of spatial distance (e.g., astropy.unit.arcsec)

\sphinxlineitem{ext: int default=1, optional}
\sphinxAtStartPar
header extension where the WCS information is located.

\sphinxlineitem{dtype: str default ‘image’, optional}
\sphinxAtStartPar
type of file to be analyzed. It can be pixtable, image or cube. Defaults to image.

\end{description}
\begin{quote}\begin{description}
\sphinxlineitem{returns}\begin{itemize}
\item {} 
\sphinxAtStartPar
\sphinxstylestrong{Pointing dictionary} (\sphinxstyleemphasis{Dict of {[}int, list{]}}) \textendash{} Dictionary grouping the input files by pointing. The keys of the dictionary are the
pointing numbers, and to each pointing a list of filenames is associated.

\item {} 
\sphinxAtStartPar
\sphinxstylestrong{Diagnostic dictionary} (\sphinxstyleemphasis{Dict of {[}int, list{]}, only if diagnostics}) \textendash{} Dictionary containing basic information on the distance between exposures belonging to
the same pointing. For each pointing, the mean and std of the distance with respect to
a reference exposure is reported.

\end{itemize}

\end{description}\end{quote}

\end{fulllineitems}

\index{group\_xy\_per\_fieldofview() (in module pymusepipe.util\_image)@\spxentry{group\_xy\_per\_fieldofview()}\spxextra{in module pymusepipe.util\_image}}

\begin{fulllineitems}
\phantomsection\label{\detokenize{api/pymusepipe:pymusepipe.util_image.group_xy_per_fieldofview}}
\pysigstartsignatures
\pysiglinewithargsret{\sphinxcode{\sphinxupquote{pymusepipe.util\_image.}}\sphinxbfcode{\sphinxupquote{group\_xy\_per\_fieldofview}}}{\sphinxparam{\DUrole{n,n}{center\_dict}}, \sphinxparam{\DUrole{n,n}{limit=\textless{}Quantity 10. arcsec\textgreater{}}}}{}
\pysigstopsignatures
\sphinxAtStartPar
Group exposures in pointings based on their proximity.


\paragraph{Input}
\label{\detokenize{api/pymusepipe:id151}}\begin{description}
\sphinxlineitem{center\_dict: dict}
\sphinxAtStartPar
Dictionary containing a list of filenames and their coordinates.

\sphinxlineitem{limit: Quantity default=10*.u.arcsec, optional}
\sphinxAtStartPar
maximum separation for files to belong to the same pointing. Defaults to 10*u.arcsec.

\end{description}

\sphinxAtStartPar
Returns:
Dict: {[}int, list{]}
\begin{quote}

\sphinxAtStartPar
Dictionary grouping the input files by pointing. The keys of the dictionary are the
pointing numbers, and to each pointing a list of filenames is associated.
\end{quote}

\end{fulllineitems}

\index{mask\_point\_sources() (in module pymusepipe.util\_image)@\spxentry{mask\_point\_sources()}\spxextra{in module pymusepipe.util\_image}}

\begin{fulllineitems}
\phantomsection\label{\detokenize{api/pymusepipe:pymusepipe.util_image.mask_point_sources}}
\pysigstartsignatures
\pysiglinewithargsret{\sphinxcode{\sphinxupquote{pymusepipe.util\_image.}}\sphinxbfcode{\sphinxupquote{mask\_point\_sources}}}{\sphinxparam{\DUrole{n,n}{ima}}, \sphinxparam{\DUrole{n,n}{fwhm}\DUrole{o,o}{=}\DUrole{default_value}{3}}, \sphinxparam{\DUrole{n,n}{mask\_radius}\DUrole{o,o}{=}\DUrole{default_value}{30.0}}, \sphinxparam{\DUrole{n,n}{brightest}\DUrole{o,o}{=}\DUrole{default_value}{5}}, \sphinxparam{\DUrole{n,n}{sigma}\DUrole{o,o}{=}\DUrole{default_value}{3.0}}, \sphinxparam{\DUrole{n,n}{verbose}\DUrole{o,o}{=}\DUrole{default_value}{False}}}{}
\pysigstopsignatures
\sphinxAtStartPar
Find and mask point sources in an image by adding NaN


\paragraph{Input}
\label{\detokenize{api/pymusepipe:id152}}\begin{description}
\sphinxlineitem{ima: ndarray}
\sphinxAtStartPar
Image to mask

\sphinxlineitem{fwhm: float}
\sphinxAtStartPar
guess for the FWHM in pixel of the PSF. Defaults to 3.

\sphinxlineitem{mask\_radius: float}
\sphinxAtStartPar
Radius in pixels to mask around sources

\sphinxlineitem{brightest: int}
\sphinxAtStartPar
Maximum number of bright stars to mask. Defaults to 5.

\sphinxlineitem{sigma: float}
\sphinxAtStartPar
Sigma to clip the image

\end{description}

\sphinxAtStartPar
verbose: bool
\begin{quote}\begin{description}
\sphinxlineitem{returns}
\sphinxAtStartPar
\sphinxstylestrong{ima} \textendash{} with NaN where the mask applied

\sphinxlineitem{rtype}
\sphinxAtStartPar
np.array

\end{description}\end{quote}

\end{fulllineitems}

\index{my\_linear\_model() (in module pymusepipe.util\_image)@\spxentry{my\_linear\_model()}\spxextra{in module pymusepipe.util\_image}}

\begin{fulllineitems}
\phantomsection\label{\detokenize{api/pymusepipe:pymusepipe.util_image.my_linear_model}}
\pysigstartsignatures
\pysiglinewithargsret{\sphinxcode{\sphinxupquote{pymusepipe.util\_image.}}\sphinxbfcode{\sphinxupquote{my\_linear\_model}}}{\sphinxparam{\DUrole{n,n}{b}}, \sphinxparam{\DUrole{n,n}{x}}}{}
\pysigstopsignatures
\sphinxAtStartPar
Linear function for the regression.


\paragraph{Input}
\label{\detokenize{api/pymusepipe:id153}}\begin{description}
\sphinxlineitem{b}{[}1D np.array of 2 floats{]}
\sphinxAtStartPar
Input 1D polynomial parameters (0=constant, 1=slope)

\sphinxlineitem{x}{[}np.array{]}
\sphinxAtStartPar
Array which will be multiplied by the polynomial

\end{description}
\begin{quote}\begin{description}
\sphinxlineitem{rtype}
\sphinxAtStartPar
An array = b{[}1{]} * (x + b{[}0{]})

\end{description}\end{quote}

\end{fulllineitems}

\index{prepare\_image() (in module pymusepipe.util\_image)@\spxentry{prepare\_image()}\spxextra{in module pymusepipe.util\_image}}

\begin{fulllineitems}
\phantomsection\label{\detokenize{api/pymusepipe:pymusepipe.util_image.prepare_image}}
\pysigstartsignatures
\pysiglinewithargsret{\sphinxcode{\sphinxupquote{pymusepipe.util\_image.}}\sphinxbfcode{\sphinxupquote{prepare\_image}}}{\sphinxparam{\DUrole{n,n}{data}}, \sphinxparam{\DUrole{n,n}{median\_filter}\DUrole{o,o}{=}\DUrole{default_value}{True}}, \sphinxparam{\DUrole{n,n}{sigma}\DUrole{o,o}{=}\DUrole{default_value}{0.0}}, \sphinxparam{\DUrole{n,n}{border}\DUrole{o,o}{=}\DUrole{default_value}{0}}}{}
\pysigstopsignatures
\sphinxAtStartPar
Median filter plus convolve the input image


\paragraph{Input}
\label{\detokenize{api/pymusepipe:id154}}\begin{description}
\sphinxlineitem{data: 2D np.array}
\sphinxAtStartPar
Data to process

\sphinxlineitem{median\_filter: bool}
\sphinxAtStartPar
If True, will median filter

\sphinxlineitem{convolve float {[}0{]}}
\sphinxAtStartPar
Will convolve the data with this gaussian width (sigma)
0 means no convolution

\end{description}
\begin{quote}\begin{description}
\sphinxlineitem{returns}
\sphinxAtStartPar
\sphinxstylestrong{data}

\sphinxlineitem{rtype}
\sphinxAtStartPar
2d array

\end{description}\end{quote}

\end{fulllineitems}

\index{regress\_odr() (in module pymusepipe.util\_image)@\spxentry{regress\_odr()}\spxextra{in module pymusepipe.util\_image}}

\begin{fulllineitems}
\phantomsection\label{\detokenize{api/pymusepipe:pymusepipe.util_image.regress_odr}}
\pysigstartsignatures
\pysiglinewithargsret{\sphinxcode{\sphinxupquote{pymusepipe.util\_image.}}\sphinxbfcode{\sphinxupquote{regress\_odr}}}{\sphinxparam{\DUrole{n,n}{x}}, \sphinxparam{\DUrole{n,n}{y}}, \sphinxparam{\DUrole{n,n}{sx}}, \sphinxparam{\DUrole{n,n}{sy}}, \sphinxparam{\DUrole{n,n}{beta0}\DUrole{o,o}{=}\DUrole{default_value}{(0.0, 1.0)}}, \sphinxparam{\DUrole{n,n}{percentiles}\DUrole{o,o}{=}\DUrole{default_value}{(0.0, 100.0)}}, \sphinxparam{\DUrole{n,n}{sigclip}\DUrole{o,o}{=}\DUrole{default_value}{0.0}}}{}
\pysigstopsignatures
\sphinxAtStartPar
Return an ODR linear regression using scipy.odr.ODR
\begin{quote}\begin{description}
\sphinxlineitem{Parameters}\begin{itemize}
\item {} 
\sphinxAtStartPar
\sphinxstyleliteralstrong{\sphinxupquote{x}} \textendash{} numpy.array

\item {} 
\sphinxAtStartPar
\sphinxstyleliteralstrong{\sphinxupquote{y}} \textendash{} numpy.array
Input array with signal

\item {} 
\sphinxAtStartPar
\sphinxstyleliteralstrong{\sphinxupquote{sx}} \textendash{} numpy.array

\item {} 
\sphinxAtStartPar
\sphinxstyleliteralstrong{\sphinxupquote{sy}} \textendash{} numpy.array
Input array (as x,y) with standard deviations

\item {} 
\sphinxAtStartPar
\sphinxstyleliteralstrong{\sphinxupquote{beta0}} \textendash{} list or tuple of 2 floats
Initial guess for the constant and slope

\item {} 
\sphinxAtStartPar
\sphinxstyleliteralstrong{\sphinxupquote{percentiles}} \textendash{} tuple or list of 2 floats
Two numbers providing the min and max percentiles

\item {} 
\sphinxAtStartPar
\sphinxstyleliteralstrong{\sphinxupquote{sigclip}} \textendash{} float
sigma factor for sigma clipping. If 0, no sigma clipping
is performed

\end{itemize}

\sphinxlineitem{Returns}
\sphinxAtStartPar
result of the ODR analysis

\sphinxlineitem{Return type}
\sphinxAtStartPar
result

\end{description}\end{quote}

\end{fulllineitems}

\index{scan\_filenames\_from\_folder() (in module pymusepipe.util\_image)@\spxentry{scan\_filenames\_from\_folder()}\spxextra{in module pymusepipe.util\_image}}

\begin{fulllineitems}
\phantomsection\label{\detokenize{api/pymusepipe:pymusepipe.util_image.scan_filenames_from_folder}}
\pysigstartsignatures
\pysiglinewithargsret{\sphinxcode{\sphinxupquote{pymusepipe.util\_image.}}\sphinxbfcode{\sphinxupquote{scan\_filenames\_from\_folder}}}{\sphinxparam{\DUrole{n,n}{folder}\DUrole{o,o}{=}\DUrole{default_value}{\textquotesingle{}\textquotesingle{}}}, \sphinxparam{\DUrole{n,n}{prefix}\DUrole{o,o}{=}\DUrole{default_value}{\textquotesingle{}\textquotesingle{}}}, \sphinxparam{\DUrole{n,n}{suffix}\DUrole{o,o}{=}\DUrole{default_value}{\textquotesingle{}\textquotesingle{}}}, \sphinxparam{\DUrole{n,n}{ext}\DUrole{o,o}{=}\DUrole{default_value}{\textquotesingle{}fits\textquotesingle{}}}, \sphinxparam{\DUrole{o,o}{**}\DUrole{n,n}{kwargs}}}{}
\pysigstopsignatures
\sphinxAtStartPar
Scan a given folder and look for a set of filenames that could enter a pointing
table. Those names are decrypted following a given scheme (extracting tpls, expo, dataset)


\paragraph{Input}
\label{\detokenize{api/pymusepipe:id155}}\begin{description}
\sphinxlineitem{folder: str default=’’}
\sphinxAtStartPar
Name of the folder to scan

\sphinxlineitem{prefix: str default=’’}
\sphinxAtStartPar
Prefix to be used to filter the filenames

\sphinxlineitem{suffix: str default=’’}
\sphinxAtStartPar
End of the word before the stem (extension)

\sphinxlineitem{ext: str default=’fits’}
\sphinxAtStartPar
Extension

\end{description}
\begin{quote}\begin{description}
\sphinxlineitem{returns}
\sphinxAtStartPar
\sphinxstylestrong{filename\_table}

\sphinxlineitem{rtype}
\sphinxAtStartPar
astropy QTable with columns ‘filename’ ‘tpls’ ‘dataset’ ‘expo’

\end{description}\end{quote}

\end{fulllineitems}

\index{scan\_filenames\_from\_list() (in module pymusepipe.util\_image)@\spxentry{scan\_filenames\_from\_list()}\spxextra{in module pymusepipe.util\_image}}

\begin{fulllineitems}
\phantomsection\label{\detokenize{api/pymusepipe:pymusepipe.util_image.scan_filenames_from_list}}
\pysigstartsignatures
\pysiglinewithargsret{\sphinxcode{\sphinxupquote{pymusepipe.util\_image.}}\sphinxbfcode{\sphinxupquote{scan\_filenames\_from\_list}}}{\sphinxparam{\DUrole{n,n}{list\_files}}, \sphinxparam{\DUrole{o,o}{**}\DUrole{n,n}{kwargs}}}{}
\pysigstopsignatures
\sphinxAtStartPar
Extract values of dataset, tpls, expo from a list of names


\paragraph{Input}
\label{\detokenize{api/pymusepipe:id156}}
\sphinxAtStartPar
list\_files: list of str
kwargs: additional keywords including
\begin{quote}

\sphinxAtStartPar
str\_dataset: str
ndigits: int
filtername: str
\end{quote}
\begin{quote}\begin{description}
\sphinxlineitem{rtype}
\sphinxAtStartPar
QTable including filenames, tpls, dataset, expo

\end{description}\end{quote}

\end{fulllineitems}

\index{select\_spaxels() (in module pymusepipe.util\_image)@\spxentry{select\_spaxels()}\spxextra{in module pymusepipe.util\_image}}

\begin{fulllineitems}
\phantomsection\label{\detokenize{api/pymusepipe:pymusepipe.util_image.select_spaxels}}
\pysigstartsignatures
\pysiglinewithargsret{\sphinxcode{\sphinxupquote{pymusepipe.util\_image.}}\sphinxbfcode{\sphinxupquote{select\_spaxels}}}{\sphinxparam{\DUrole{n,n}{maskdict}}, \sphinxparam{\DUrole{n,n}{maskname}}, \sphinxparam{\DUrole{n,n}{x}}, \sphinxparam{\DUrole{n,n}{y}}}{}
\pysigstopsignatures
\sphinxAtStartPar
Selecting spaxels defined by their coordinates
using the masks defined by Circle or Rectangle Zones

\end{fulllineitems}



\subsubsection{pymusepipe.util\_pipe module}
\label{\detokenize{api/pymusepipe:module-pymusepipe.util_pipe}}\label{\detokenize{api/pymusepipe:pymusepipe-util-pipe-module}}\index{module@\spxentry{module}!pymusepipe.util\_pipe@\spxentry{pymusepipe.util\_pipe}}\index{pymusepipe.util\_pipe@\spxentry{pymusepipe.util\_pipe}!module@\spxentry{module}}
\sphinxAtStartPar
MUSE\sphinxhyphen{}PHANGS utility functions for pymusepipe
\index{ExposureInfo (class in pymusepipe.util\_pipe)@\spxentry{ExposureInfo}\spxextra{class in pymusepipe.util\_pipe}}

\begin{fulllineitems}
\phantomsection\label{\detokenize{api/pymusepipe:pymusepipe.util_pipe.ExposureInfo}}
\pysigstartsignatures
\pysiglinewithargsret{\sphinxbfcode{\sphinxupquote{class\DUrole{w,w}{  }}}\sphinxcode{\sphinxupquote{pymusepipe.util\_pipe.}}\sphinxbfcode{\sphinxupquote{ExposureInfo}}}{\sphinxparam{\DUrole{n,n}{targetname}}, \sphinxparam{\DUrole{n,n}{dataset}}, \sphinxparam{\DUrole{n,n}{tpl}}, \sphinxparam{\DUrole{n,n}{nexpo}}}{}
\pysigstopsignatures
\sphinxAtStartPar
Bases: \sphinxhref{https://docs.python.org/3.10/library/functions.html\#object}{\sphinxcode{\sphinxupquote{object}}}

\end{fulllineitems}

\index{TimeStampDict (class in pymusepipe.util\_pipe)@\spxentry{TimeStampDict}\spxextra{class in pymusepipe.util\_pipe}}

\begin{fulllineitems}
\phantomsection\label{\detokenize{api/pymusepipe:pymusepipe.util_pipe.TimeStampDict}}
\pysigstartsignatures
\pysiglinewithargsret{\sphinxbfcode{\sphinxupquote{class\DUrole{w,w}{  }}}\sphinxcode{\sphinxupquote{pymusepipe.util\_pipe.}}\sphinxbfcode{\sphinxupquote{TimeStampDict}}}{\sphinxparam{\DUrole{n,n}{description}\DUrole{o,o}{=}\DUrole{default_value}{\textquotesingle{}\textquotesingle{}}}, \sphinxparam{\DUrole{n,n}{myobject}\DUrole{o,o}{=}\DUrole{default_value}{None}}}{}
\pysigstopsignatures
\sphinxAtStartPar
Bases: \sphinxhref{https://docs.python.org/3.10/library/collections.html\#collections.OrderedDict}{\sphinxcode{\sphinxupquote{OrderedDict}}}

\sphinxAtStartPar
Class which builds a time stamp driven
dictionary of objects
\index{create\_new\_timestamp() (pymusepipe.util\_pipe.TimeStampDict method)@\spxentry{create\_new\_timestamp()}\spxextra{pymusepipe.util\_pipe.TimeStampDict method}}

\begin{fulllineitems}
\phantomsection\label{\detokenize{api/pymusepipe:pymusepipe.util_pipe.TimeStampDict.create_new_timestamp}}
\pysigstartsignatures
\pysiglinewithargsret{\sphinxbfcode{\sphinxupquote{create\_new\_timestamp}}}{\sphinxparam{\DUrole{n,n}{myobject}\DUrole{o,o}{=}\DUrole{default_value}{None}}}{}
\pysigstopsignatures
\sphinxAtStartPar
Create a new item in dictionary
using a time stamp

\end{fulllineitems}

\index{delete\_timestamp() (pymusepipe.util\_pipe.TimeStampDict method)@\spxentry{delete\_timestamp()}\spxextra{pymusepipe.util\_pipe.TimeStampDict method}}

\begin{fulllineitems}
\phantomsection\label{\detokenize{api/pymusepipe:pymusepipe.util_pipe.TimeStampDict.delete_timestamp}}
\pysigstartsignatures
\pysiglinewithargsret{\sphinxbfcode{\sphinxupquote{delete\_timestamp}}}{\sphinxparam{\DUrole{n,n}{tstamp}\DUrole{o,o}{=}\DUrole{default_value}{None}}}{}
\pysigstopsignatures
\sphinxAtStartPar
Delete a key in the dictionary

\end{fulllineitems}


\end{fulllineitems}

\index{abspath() (in module pymusepipe.util\_pipe)@\spxentry{abspath()}\spxextra{in module pymusepipe.util\_pipe}}

\begin{fulllineitems}
\phantomsection\label{\detokenize{api/pymusepipe:pymusepipe.util_pipe.abspath}}
\pysigstartsignatures
\pysiglinewithargsret{\sphinxcode{\sphinxupquote{pymusepipe.util\_pipe.}}\sphinxbfcode{\sphinxupquote{abspath}}}{\sphinxparam{\DUrole{n,n}{path}}}{}
\pysigstopsignatures
\sphinxAtStartPar
Normalise the path to get it short but absolute

\end{fulllineitems}

\index{add\_key\_dataset\_expo() (in module pymusepipe.util\_pipe)@\spxentry{add\_key\_dataset\_expo()}\spxextra{in module pymusepipe.util\_pipe}}

\begin{fulllineitems}
\phantomsection\label{\detokenize{api/pymusepipe:pymusepipe.util_pipe.add_key_dataset_expo}}
\pysigstartsignatures
\pysiglinewithargsret{\sphinxcode{\sphinxupquote{pymusepipe.util\_pipe.}}\sphinxbfcode{\sphinxupquote{add\_key\_dataset\_expo}}}{\sphinxparam{\DUrole{n,n}{imaname}}, \sphinxparam{\DUrole{n,n}{iexpo}}, \sphinxparam{\DUrole{n,n}{dataset}}}{}
\pysigstopsignatures
\sphinxAtStartPar
Add dataset and expo number to image


\paragraph{Input}
\label{\detokenize{api/pymusepipe:id157}}
\sphinxAtStartPar
imaname: str
iexpo: int
dataset: int

\end{fulllineitems}

\index{add\_string() (in module pymusepipe.util\_pipe)@\spxentry{add\_string()}\spxextra{in module pymusepipe.util\_pipe}}

\begin{fulllineitems}
\phantomsection\label{\detokenize{api/pymusepipe:pymusepipe.util_pipe.add_string}}
\pysigstartsignatures
\pysiglinewithargsret{\sphinxcode{\sphinxupquote{pymusepipe.util\_pipe.}}\sphinxbfcode{\sphinxupquote{add\_string}}}{\sphinxparam{\DUrole{n,n}{text}}, \sphinxparam{\DUrole{n,n}{word}\DUrole{o,o}{=}\DUrole{default_value}{\textquotesingle{}\_\textquotesingle{}}}, \sphinxparam{\DUrole{n,n}{loc}\DUrole{o,o}{=}\DUrole{default_value}{0}}}{}
\pysigstopsignatures
\sphinxAtStartPar
Adding string at given location
Default is underscore for string which are not empty.


\paragraph{Input}
\label{\detokenize{api/pymusepipe:id158}}
\sphinxAtStartPar
text (str): input text
word (str): input word to be added
loc (int): location in ‘text’. {[}Default is 0=start{]}
\begin{quote}

\sphinxAtStartPar
If None, will be added at the end.
\end{quote}
\begin{quote}\begin{description}
\sphinxlineitem{rtype}
\sphinxAtStartPar
Updated text

\end{description}\end{quote}

\end{fulllineitems}

\index{analyse\_musemode() (in module pymusepipe.util\_pipe)@\spxentry{analyse\_musemode()}\spxextra{in module pymusepipe.util\_pipe}}

\begin{fulllineitems}
\phantomsection\label{\detokenize{api/pymusepipe:pymusepipe.util_pipe.analyse_musemode}}
\pysigstartsignatures
\pysiglinewithargsret{\sphinxcode{\sphinxupquote{pymusepipe.util\_pipe.}}\sphinxbfcode{\sphinxupquote{analyse\_musemode}}}{\sphinxparam{\DUrole{n,n}{musemode}}, \sphinxparam{\DUrole{n,n}{field}}, \sphinxparam{\DUrole{n,n}{delimiter}\DUrole{o,o}{=}\DUrole{default_value}{\textquotesingle{}\sphinxhyphen{}\textquotesingle{}}}}{}
\pysigstopsignatures
\sphinxAtStartPar
Extract the named field from the musemode


\paragraph{Input}
\label{\detokenize{api/pymusepipe:id159}}\begin{description}
\sphinxlineitem{musemode: str}
\sphinxAtStartPar
Mode of the MUSE data to be analysed

\sphinxlineitem{field: str}
\sphinxAtStartPar
Field to analyse (‘ao’, ‘field’, ‘lambda\_range’)

\sphinxlineitem{delimiter: str}
\sphinxAtStartPar
Character to delimit the fields to analyse

\end{description}
\begin{quote}\begin{description}
\sphinxlineitem{returns}
\sphinxAtStartPar
\sphinxstylestrong{val} \textendash{} Value of the field which was analysed (e.g., ‘AO’ or ‘NOAO’)

\sphinxlineitem{rtype}
\sphinxAtStartPar
str

\end{description}\end{quote}

\end{fulllineitems}

\index{append\_file() (in module pymusepipe.util\_pipe)@\spxentry{append\_file()}\spxextra{in module pymusepipe.util\_pipe}}

\begin{fulllineitems}
\phantomsection\label{\detokenize{api/pymusepipe:pymusepipe.util_pipe.append_file}}
\pysigstartsignatures
\pysiglinewithargsret{\sphinxcode{\sphinxupquote{pymusepipe.util\_pipe.}}\sphinxbfcode{\sphinxupquote{append\_file}}}{\sphinxparam{\DUrole{n,n}{filename}}, \sphinxparam{\DUrole{n,n}{content}}}{}
\pysigstopsignatures
\sphinxAtStartPar
Append in ascii file

\end{fulllineitems}

\index{append\_value\_to\_dict() (in module pymusepipe.util\_pipe)@\spxentry{append\_value\_to\_dict()}\spxextra{in module pymusepipe.util\_pipe}}

\begin{fulllineitems}
\phantomsection\label{\detokenize{api/pymusepipe:pymusepipe.util_pipe.append_value_to_dict}}
\pysigstartsignatures
\pysiglinewithargsret{\sphinxcode{\sphinxupquote{pymusepipe.util\_pipe.}}\sphinxbfcode{\sphinxupquote{append\_value\_to\_dict}}}{\sphinxparam{\DUrole{n,n}{mydict}}, \sphinxparam{\DUrole{n,n}{key}}, \sphinxparam{\DUrole{n,n}{value}}}{}
\pysigstopsignatures
\sphinxAtStartPar
Append a value to key within a given dictionary. If the key does not exist it creates
a list of 1 element for that key


\paragraph{Input}
\label{\detokenize{api/pymusepipe:id160}}
\sphinxAtStartPar
mydict: dict
key:
value:
\begin{quote}\begin{description}
\sphinxlineitem{rtype}
\sphinxAtStartPar
Updated dictionary

\end{description}\end{quote}

\end{fulllineitems}

\index{build\_dict\_datasets() (in module pymusepipe.util\_pipe)@\spxentry{build\_dict\_datasets()}\spxextra{in module pymusepipe.util\_pipe}}

\begin{fulllineitems}
\phantomsection\label{\detokenize{api/pymusepipe:pymusepipe.util_pipe.build_dict_datasets}}
\pysigstartsignatures
\pysiglinewithargsret{\sphinxcode{\sphinxupquote{pymusepipe.util\_pipe.}}\sphinxbfcode{\sphinxupquote{build\_dict\_datasets}}}{\sphinxparam{\DUrole{n,n}{data\_path}\DUrole{o,o}{=}\DUrole{default_value}{\textquotesingle{}\textquotesingle{}}}, \sphinxparam{\DUrole{n,n}{str\_dataset}\DUrole{o,o}{=}\DUrole{default_value}{\textquotesingle{}OB\textquotesingle{}}}, \sphinxparam{\DUrole{n,n}{ndigits}\DUrole{o,o}{=}\DUrole{default_value}{3}}}{}
\pysigstopsignatures
\sphinxAtStartPar
Build a dictionary of datasets for each target in the sample


\paragraph{Input}
\label{\detokenize{api/pymusepipe:id161}}\begin{description}
\sphinxlineitem{data\_path: str}
\sphinxAtStartPar
Path of the target data

\sphinxlineitem{str\_dataset: str default=default\_str\_dataset}
\sphinxAtStartPar
Prefix string for datasets (see config\_pipe.py)

\sphinxlineitem{ndigits: int default=default\_ndigits}
\sphinxAtStartPar
Number of digits to format the name of the dataset (see config\_pipe.py)

\end{description}
\begin{quote}\begin{description}
\sphinxlineitem{returns}
\sphinxAtStartPar
\sphinxstylestrong{dict\_dataset}

\sphinxlineitem{rtype}
\sphinxAtStartPar
dict

\end{description}\end{quote}

\end{fulllineitems}

\index{build\_dict\_exposures() (in module pymusepipe.util\_pipe)@\spxentry{build\_dict\_exposures()}\spxextra{in module pymusepipe.util\_pipe}}

\begin{fulllineitems}
\phantomsection\label{\detokenize{api/pymusepipe:pymusepipe.util_pipe.build_dict_exposures}}
\pysigstartsignatures
\pysiglinewithargsret{\sphinxcode{\sphinxupquote{pymusepipe.util\_pipe.}}\sphinxbfcode{\sphinxupquote{build\_dict\_exposures}}}{\sphinxparam{\DUrole{n,n}{target\_path}\DUrole{o,o}{=}\DUrole{default_value}{\textquotesingle{}\textquotesingle{}}}, \sphinxparam{\DUrole{n,n}{str\_dataset}\DUrole{o,o}{=}\DUrole{default_value}{\textquotesingle{}OB\textquotesingle{}}}, \sphinxparam{\DUrole{n,n}{ndigits}\DUrole{o,o}{=}\DUrole{default_value}{3}}, \sphinxparam{\DUrole{n,n}{show\_pointings}\DUrole{o,o}{=}\DUrole{default_value}{False}}}{}
\pysigstopsignatures
\sphinxAtStartPar
Build a dictionary of exposures using the list of datasets found for the
given dataset path


\paragraph{Input}
\label{\detokenize{api/pymusepipe:id162}}\begin{description}
\sphinxlineitem{target\_path: str}
\sphinxAtStartPar
Path of the target data

\sphinxlineitem{str\_dataset: str}
\sphinxAtStartPar
Prefix string for datasets

\sphinxlineitem{ndigits: int}
\sphinxAtStartPar
Number of digits to format the name of the dataset

\end{description}
\begin{quote}\begin{description}
\sphinxlineitem{returns}
\sphinxAtStartPar
\sphinxstylestrong{dict\_expo} \textendash{} Dictionary of exposures in each dataset

\sphinxlineitem{rtype}
\sphinxAtStartPar
dict

\end{description}\end{quote}

\end{fulllineitems}

\index{check\_filter\_list() (in module pymusepipe.util\_pipe)@\spxentry{check\_filter\_list()}\spxextra{in module pymusepipe.util\_pipe}}

\begin{fulllineitems}
\phantomsection\label{\detokenize{api/pymusepipe:pymusepipe.util_pipe.check_filter_list}}
\pysigstartsignatures
\pysiglinewithargsret{\sphinxcode{\sphinxupquote{pymusepipe.util\_pipe.}}\sphinxbfcode{\sphinxupquote{check\_filter\_list}}}{\sphinxparam{\DUrole{n,n}{filter\_list}}}{}
\pysigstopsignatures
\end{fulllineitems}

\index{create\_time\_name() (in module pymusepipe.util\_pipe)@\spxentry{create\_time\_name()}\spxextra{in module pymusepipe.util\_pipe}}

\begin{fulllineitems}
\phantomsection\label{\detokenize{api/pymusepipe:pymusepipe.util_pipe.create_time_name}}
\pysigstartsignatures
\pysiglinewithargsret{\sphinxcode{\sphinxupquote{pymusepipe.util\_pipe.}}\sphinxbfcode{\sphinxupquote{create\_time\_name}}}{}{}
\pysigstopsignatures
\sphinxAtStartPar
Create a time\sphinxhyphen{}link name for file saving purposes

\sphinxAtStartPar
Return: a string including the YearMonthDay\_HourMinSec

\end{fulllineitems}

\index{filter\_list\_to\_str() (in module pymusepipe.util\_pipe)@\spxentry{filter\_list\_to\_str()}\spxextra{in module pymusepipe.util\_pipe}}

\begin{fulllineitems}
\phantomsection\label{\detokenize{api/pymusepipe:pymusepipe.util_pipe.filter_list_to_str}}
\pysigstartsignatures
\pysiglinewithargsret{\sphinxcode{\sphinxupquote{pymusepipe.util\_pipe.}}\sphinxbfcode{\sphinxupquote{filter\_list\_to\_str}}}{\sphinxparam{\DUrole{n,n}{filter\_list}}}{}
\pysigstopsignatures
\end{fulllineitems}

\index{filter\_list\_with\_suffix\_list() (in module pymusepipe.util\_pipe)@\spxentry{filter\_list\_with\_suffix\_list()}\spxextra{in module pymusepipe.util\_pipe}}

\begin{fulllineitems}
\phantomsection\label{\detokenize{api/pymusepipe:pymusepipe.util_pipe.filter_list_with_suffix_list}}
\pysigstartsignatures
\pysiglinewithargsret{\sphinxcode{\sphinxupquote{pymusepipe.util\_pipe.}}\sphinxbfcode{\sphinxupquote{filter\_list\_with\_suffix\_list}}}{\sphinxparam{\DUrole{n,n}{list\_names}}, \sphinxparam{\DUrole{n,n}{included\_suffix\_list}\DUrole{o,o}{=}\DUrole{default_value}{{[}{]}}}, \sphinxparam{\DUrole{n,n}{excluded\_suffix\_list}\DUrole{o,o}{=}\DUrole{default_value}{{[}{]}}}, \sphinxparam{\DUrole{n,n}{name\_list}\DUrole{o,o}{=}\DUrole{default_value}{\textquotesingle{}\textquotesingle{}}}}{}
\pysigstopsignatures
\sphinxAtStartPar
Filter a list using suffixes (to exclude or include)


\paragraph{Input}
\label{\detokenize{api/pymusepipe:id163}}
\sphinxAtStartPar
list\_names: list of str
included\_suffix\_list: list of str
excluded\_suffix\_list: list of str
name\_list: str default=””

\end{fulllineitems}

\index{formatted\_time() (in module pymusepipe.util\_pipe)@\spxentry{formatted\_time()}\spxextra{in module pymusepipe.util\_pipe}}

\begin{fulllineitems}
\phantomsection\label{\detokenize{api/pymusepipe:pymusepipe.util_pipe.formatted_time}}
\pysigstartsignatures
\pysiglinewithargsret{\sphinxcode{\sphinxupquote{pymusepipe.util\_pipe.}}\sphinxbfcode{\sphinxupquote{formatted\_time}}}{}{}
\pysigstopsignatures
\sphinxAtStartPar
Return: a string including the formatted time

\end{fulllineitems}

\index{get\_dataset\_name() (in module pymusepipe.util\_pipe)@\spxentry{get\_dataset\_name()}\spxextra{in module pymusepipe.util\_pipe}}

\begin{fulllineitems}
\phantomsection\label{\detokenize{api/pymusepipe:pymusepipe.util_pipe.get_dataset_name}}
\pysigstartsignatures
\pysiglinewithargsret{\sphinxcode{\sphinxupquote{pymusepipe.util\_pipe.}}\sphinxbfcode{\sphinxupquote{get\_dataset\_name}}}{\sphinxparam{\DUrole{n,n}{dataset}\DUrole{o,o}{=}\DUrole{default_value}{1}}, \sphinxparam{\DUrole{n,n}{str\_dataset}\DUrole{o,o}{=}\DUrole{default_value}{\textquotesingle{}OB\textquotesingle{}}}, \sphinxparam{\DUrole{n,n}{ndigits}\DUrole{o,o}{=}\DUrole{default_value}{3}}}{}
\pysigstopsignatures
\sphinxAtStartPar
Formatting for the dataset/pointing names using the number and
the number of digits and prefix string


\paragraph{Input}
\label{\detokenize{api/pymusepipe:id164}}\begin{description}
\sphinxlineitem{dataset: int}
\sphinxAtStartPar
Dataset (or Pointing) number

\sphinxlineitem{str\_dataset: str}
\sphinxAtStartPar
Prefix representing the dataset (or pointing)

\sphinxlineitem{ndigits: int}
\sphinxAtStartPar
Number of digits to be used for formatting

\end{description}
\begin{quote}\begin{description}
\sphinxlineitem{rtype}
\sphinxAtStartPar
string for the dataset/pointing name prefix

\end{description}\end{quote}

\end{fulllineitems}

\index{get\_dataset\_tpl\_nexpo() (in module pymusepipe.util\_pipe)@\spxentry{get\_dataset\_tpl\_nexpo()}\spxextra{in module pymusepipe.util\_pipe}}

\begin{fulllineitems}
\phantomsection\label{\detokenize{api/pymusepipe:pymusepipe.util_pipe.get_dataset_tpl_nexpo}}
\pysigstartsignatures
\pysiglinewithargsret{\sphinxcode{\sphinxupquote{pymusepipe.util\_pipe.}}\sphinxbfcode{\sphinxupquote{get\_dataset\_tpl\_nexpo}}}{\sphinxparam{\DUrole{n,n}{filename}}, \sphinxparam{\DUrole{n,n}{str\_dataset}\DUrole{o,o}{=}\DUrole{default_value}{\textquotesingle{}OB\textquotesingle{}}}, \sphinxparam{\DUrole{n,n}{ndigits}\DUrole{o,o}{=}\DUrole{default_value}{3}}, \sphinxparam{\DUrole{n,n}{filtername}\DUrole{o,o}{=}\DUrole{default_value}{None}}}{}
\pysigstopsignatures
\sphinxAtStartPar
Get the tpl and nexpo from a filename assuming it is at the end
of the filename


\paragraph{Input}
\label{\detokenize{api/pymusepipe:id165}}\begin{description}
\sphinxlineitem{filename: str}
\sphinxAtStartPar
Input filename

\end{description}
\begin{quote}\begin{description}
\sphinxlineitem{returns}
\sphinxAtStartPar
\sphinxstylestrong{tpl, nexpo}

\sphinxlineitem{rtype}
\sphinxAtStartPar
str, int

\end{description}\end{quote}

\end{fulllineitems}

\index{get\_list\_datasets() (in module pymusepipe.util\_pipe)@\spxentry{get\_list\_datasets()}\spxextra{in module pymusepipe.util\_pipe}}

\begin{fulllineitems}
\phantomsection\label{\detokenize{api/pymusepipe:pymusepipe.util_pipe.get_list_datasets}}
\pysigstartsignatures
\pysiglinewithargsret{\sphinxcode{\sphinxupquote{pymusepipe.util\_pipe.}}\sphinxbfcode{\sphinxupquote{get\_list\_datasets}}}{\sphinxparam{\DUrole{n,n}{target\_path}\DUrole{o,o}{=}\DUrole{default_value}{\textquotesingle{}\textquotesingle{}}}, \sphinxparam{\DUrole{n,n}{str\_dataset}\DUrole{o,o}{=}\DUrole{default_value}{\textquotesingle{}OB\textquotesingle{}}}, \sphinxparam{\DUrole{n,n}{ndigits}\DUrole{o,o}{=}\DUrole{default_value}{3}}, \sphinxparam{\DUrole{n,n}{verbose}\DUrole{o,o}{=}\DUrole{default_value}{False}}}{}
\pysigstopsignatures
\sphinxAtStartPar
Getting the list of existing datasets for a given target path


\paragraph{Input}
\label{\detokenize{api/pymusepipe:id166}}\begin{description}
\sphinxlineitem{target\_path: str}
\sphinxAtStartPar
Path of the target data

\sphinxlineitem{str\_dataset: str}
\sphinxAtStartPar
Prefix string for datasets

\sphinxlineitem{ndigits: int}
\sphinxAtStartPar
Number of digits to format the name of the dataset

\end{description}
\begin{quote}\begin{description}
\sphinxlineitem{returns}
\sphinxAtStartPar
\sphinxstylestrong{list\_datasets}

\sphinxlineitem{rtype}
\sphinxAtStartPar
list of int

\end{description}\end{quote}

\end{fulllineitems}

\index{get\_list\_exposures() (in module pymusepipe.util\_pipe)@\spxentry{get\_list\_exposures()}\spxextra{in module pymusepipe.util\_pipe}}

\begin{fulllineitems}
\phantomsection\label{\detokenize{api/pymusepipe:pymusepipe.util_pipe.get_list_exposures}}
\pysigstartsignatures
\pysiglinewithargsret{\sphinxcode{\sphinxupquote{pymusepipe.util\_pipe.}}\sphinxbfcode{\sphinxupquote{get\_list\_exposures}}}{\sphinxparam{\DUrole{n,n}{dataset\_path}\DUrole{o,o}{=}\DUrole{default_value}{\textquotesingle{}\textquotesingle{}}}, \sphinxparam{\DUrole{n,n}{object\_folder}\DUrole{o,o}{=}\DUrole{default_value}{\textquotesingle{}Object/\textquotesingle{}}}}{}
\pysigstopsignatures
\sphinxAtStartPar
Getting a list of exposures from a given path


\paragraph{Input}
\label{\detokenize{api/pymusepipe:id167}}\begin{description}
\sphinxlineitem{dataset\_path: str}
\sphinxAtStartPar
Folder name where the dataset is

\end{description}
\begin{quote}\begin{description}
\sphinxlineitem{returns}
\sphinxAtStartPar
\sphinxstylestrong{list\_expos}

\sphinxlineitem{rtype}
\sphinxAtStartPar
list of int

\end{description}\end{quote}

\end{fulllineitems}

\index{get\_list\_reduced\_pixtables() (in module pymusepipe.util\_pipe)@\spxentry{get\_list\_reduced\_pixtables()}\spxextra{in module pymusepipe.util\_pipe}}

\begin{fulllineitems}
\phantomsection\label{\detokenize{api/pymusepipe:pymusepipe.util_pipe.get_list_reduced_pixtables}}
\pysigstartsignatures
\pysiglinewithargsret{\sphinxcode{\sphinxupquote{pymusepipe.util\_pipe.}}\sphinxbfcode{\sphinxupquote{get\_list\_reduced\_pixtables}}}{\sphinxparam{\DUrole{n,n}{target\_path}\DUrole{o,o}{=}\DUrole{default_value}{\textquotesingle{}\textquotesingle{}}}, \sphinxparam{\DUrole{n,n}{list\_datasets}\DUrole{o,o}{=}\DUrole{default_value}{None}}, \sphinxparam{\DUrole{n,n}{suffix}\DUrole{o,o}{=}\DUrole{default_value}{\textquotesingle{}\textquotesingle{}}}, \sphinxparam{\DUrole{n,n}{str\_dataset}\DUrole{o,o}{=}\DUrole{default_value}{\textquotesingle{}OB\textquotesingle{}}}, \sphinxparam{\DUrole{n,n}{ndigits}\DUrole{o,o}{=}\DUrole{default_value}{3}}, \sphinxparam{\DUrole{o,o}{**}\DUrole{n,n}{kwargs}}}{}
\pysigstopsignatures
\sphinxAtStartPar
Provide a list of reduced pixtables


\paragraph{Input}
\label{\detokenize{api/pymusepipe:id168}}\begin{description}
\sphinxlineitem{target\_path: str}
\sphinxAtStartPar
Path for the target folder

\sphinxlineitem{list\_datasets: list of int}
\sphinxAtStartPar
List of integers, providing the list of datasets to consider

\sphinxlineitem{suffix: str}
\sphinxAtStartPar
Additional suffix, if needed, for the names of the PixTables.

\end{description}

\end{fulllineitems}

\index{get\_list\_targets() (in module pymusepipe.util\_pipe)@\spxentry{get\_list\_targets()}\spxextra{in module pymusepipe.util\_pipe}}

\begin{fulllineitems}
\phantomsection\label{\detokenize{api/pymusepipe:pymusepipe.util_pipe.get_list_targets}}
\pysigstartsignatures
\pysiglinewithargsret{\sphinxcode{\sphinxupquote{pymusepipe.util\_pipe.}}\sphinxbfcode{\sphinxupquote{get\_list\_targets}}}{\sphinxparam{\DUrole{n,n}{folder}\DUrole{o,o}{=}\DUrole{default_value}{\textquotesingle{}\textquotesingle{}}}}{}
\pysigstopsignatures
\sphinxAtStartPar
Getting a list of existing targets given path. This is done by simply listing the existing
folders. This may need to be filtered.


\paragraph{Input}
\label{\detokenize{api/pymusepipe:id169}}\begin{description}
\sphinxlineitem{folder: str}
\sphinxAtStartPar
Folder name where the targets are

\end{description}
\begin{quote}\begin{description}
\sphinxlineitem{returns}
\sphinxAtStartPar
\sphinxstylestrong{list\_targets}

\sphinxlineitem{rtype}
\sphinxAtStartPar
list of str

\end{description}\end{quote}

\end{fulllineitems}

\index{get\_pointing\_name() (in module pymusepipe.util\_pipe)@\spxentry{get\_pointing\_name()}\spxextra{in module pymusepipe.util\_pipe}}

\begin{fulllineitems}
\phantomsection\label{\detokenize{api/pymusepipe:pymusepipe.util_pipe.get_pointing_name}}
\pysigstartsignatures
\pysiglinewithargsret{\sphinxcode{\sphinxupquote{pymusepipe.util\_pipe.}}\sphinxbfcode{\sphinxupquote{get\_pointing\_name}}}{\sphinxparam{\DUrole{n,n}{pointing}\DUrole{o,o}{=}\DUrole{default_value}{1}}, \sphinxparam{\DUrole{n,n}{str\_pointing}\DUrole{o,o}{=}\DUrole{default_value}{\textquotesingle{}P\textquotesingle{}}}, \sphinxparam{\DUrole{n,n}{ndigits}\DUrole{o,o}{=}\DUrole{default_value}{3}}}{}
\pysigstopsignatures
\sphinxAtStartPar
Formatting for the names using the number and
the number of digits and prefix string


\paragraph{Input}
\label{\detokenize{api/pymusepipe:id170}}\begin{description}
\sphinxlineitem{pointing: int}
\sphinxAtStartPar
Pointing number

\sphinxlineitem{str\_pointing: str}
\sphinxAtStartPar
Prefix representing the pointing

\sphinxlineitem{ndigits: int}
\sphinxAtStartPar
Number of digits to be used for formatting

\end{description}
\begin{quote}\begin{description}
\sphinxlineitem{rtype}
\sphinxAtStartPar
string for the dataset/pointing name prefix

\end{description}\end{quote}

\end{fulllineitems}

\index{get\_tpl\_nexpo() (in module pymusepipe.util\_pipe)@\spxentry{get\_tpl\_nexpo()}\spxextra{in module pymusepipe.util\_pipe}}

\begin{fulllineitems}
\phantomsection\label{\detokenize{api/pymusepipe:pymusepipe.util_pipe.get_tpl_nexpo}}
\pysigstartsignatures
\pysiglinewithargsret{\sphinxcode{\sphinxupquote{pymusepipe.util\_pipe.}}\sphinxbfcode{\sphinxupquote{get\_tpl\_nexpo}}}{\sphinxparam{\DUrole{n,n}{filename}}}{}
\pysigstopsignatures
\sphinxAtStartPar
Get the tpl and nexpo from a filename assuming it is at the end
of the filename


\paragraph{Input}
\label{\detokenize{api/pymusepipe:id171}}\begin{description}
\sphinxlineitem{filename: str}
\sphinxAtStartPar
Input filename

\end{description}
\begin{quote}\begin{description}
\sphinxlineitem{returns}
\sphinxAtStartPar
\sphinxstylestrong{tpl, nexpo}

\sphinxlineitem{rtype}
\sphinxAtStartPar
str, int

\end{description}\end{quote}

\end{fulllineitems}

\index{lower\_allbutfirst\_letter() (in module pymusepipe.util\_pipe)@\spxentry{lower\_allbutfirst\_letter()}\spxextra{in module pymusepipe.util\_pipe}}

\begin{fulllineitems}
\phantomsection\label{\detokenize{api/pymusepipe:pymusepipe.util_pipe.lower_allbutfirst_letter}}
\pysigstartsignatures
\pysiglinewithargsret{\sphinxcode{\sphinxupquote{pymusepipe.util\_pipe.}}\sphinxbfcode{\sphinxupquote{lower\_allbutfirst\_letter}}}{\sphinxparam{\DUrole{n,n}{mystring}}}{}
\pysigstopsignatures
\sphinxAtStartPar
Lowercase all letters except the first one

\end{fulllineitems}

\index{lower\_rep() (in module pymusepipe.util\_pipe)@\spxentry{lower\_rep()}\spxextra{in module pymusepipe.util\_pipe}}

\begin{fulllineitems}
\phantomsection\label{\detokenize{api/pymusepipe:pymusepipe.util_pipe.lower_rep}}
\pysigstartsignatures
\pysiglinewithargsret{\sphinxcode{\sphinxupquote{pymusepipe.util\_pipe.}}\sphinxbfcode{\sphinxupquote{lower\_rep}}}{\sphinxparam{\DUrole{n,n}{text}}}{}
\pysigstopsignatures
\sphinxAtStartPar
Lower the text and return it after removing all underscores
\begin{quote}\begin{description}
\sphinxlineitem{Parameters}
\sphinxAtStartPar
\sphinxstyleliteralstrong{\sphinxupquote{text}} (\sphinxhref{https://docs.python.org/3.10/library/stdtypes.html\#str}{\sphinxstyleliteralemphasis{\sphinxupquote{str}}}) \textendash{} text to treat

\sphinxlineitem{Returns}
\sphinxAtStartPar
updated text (with removed underscores and lower\sphinxhyphen{}cased)

\end{description}\end{quote}

\end{fulllineitems}

\index{merge\_dict() (in module pymusepipe.util\_pipe)@\spxentry{merge\_dict()}\spxextra{in module pymusepipe.util\_pipe}}

\begin{fulllineitems}
\phantomsection\label{\detokenize{api/pymusepipe:pymusepipe.util_pipe.merge_dict}}
\pysigstartsignatures
\pysiglinewithargsret{\sphinxcode{\sphinxupquote{pymusepipe.util\_pipe.}}\sphinxbfcode{\sphinxupquote{merge\_dict}}}{\sphinxparam{\DUrole{n,n}{dict1}}, \sphinxparam{\DUrole{n,n}{dict2}}}{}
\pysigstopsignatures
\sphinxAtStartPar
Merging two dictionaries by appending
keys which are duplicated


\paragraph{Input}
\label{\detokenize{api/pymusepipe:id172}}
\sphinxAtStartPar
dict1: dict
dict2: dict
\begin{quote}\begin{description}
\sphinxlineitem{returns}
\sphinxAtStartPar
\sphinxstylestrong{dict1} \textendash{} merged dictionary

\sphinxlineitem{rtype}
\sphinxAtStartPar
dict

\end{description}\end{quote}

\end{fulllineitems}

\index{normpath() (in module pymusepipe.util\_pipe)@\spxentry{normpath()}\spxextra{in module pymusepipe.util\_pipe}}

\begin{fulllineitems}
\phantomsection\label{\detokenize{api/pymusepipe:pymusepipe.util_pipe.normpath}}
\pysigstartsignatures
\pysiglinewithargsret{\sphinxcode{\sphinxupquote{pymusepipe.util\_pipe.}}\sphinxbfcode{\sphinxupquote{normpath}}}{\sphinxparam{\DUrole{n,n}{path}}}{}
\pysigstopsignatures
\sphinxAtStartPar
Normalise the path to get it short

\end{fulllineitems}

\index{print\_debug() (in module pymusepipe.util\_pipe)@\spxentry{print\_debug()}\spxextra{in module pymusepipe.util\_pipe}}

\begin{fulllineitems}
\phantomsection\label{\detokenize{api/pymusepipe:pymusepipe.util_pipe.print_debug}}
\pysigstartsignatures
\pysiglinewithargsret{\sphinxcode{\sphinxupquote{pymusepipe.util\_pipe.}}\sphinxbfcode{\sphinxupquote{print\_debug}}}{\sphinxparam{\DUrole{n,n}{text}}, \sphinxparam{\DUrole{o,o}{**}\DUrole{n,n}{kwargs}}}{}
\pysigstopsignatures
\sphinxAtStartPar
Print debugging information


\paragraph{Input}
\label{\detokenize{api/pymusepipe:id173}}
\sphinxAtStartPar
text: str
pipe: musepipe {[}None{]}
\begin{quote}

\sphinxAtStartPar
If provided, will print the text in the logfile
\end{quote}

\end{fulllineitems}

\index{print\_endline() (in module pymusepipe.util\_pipe)@\spxentry{print\_endline()}\spxextra{in module pymusepipe.util\_pipe}}

\begin{fulllineitems}
\phantomsection\label{\detokenize{api/pymusepipe:pymusepipe.util_pipe.print_endline}}
\pysigstartsignatures
\pysiglinewithargsret{\sphinxcode{\sphinxupquote{pymusepipe.util\_pipe.}}\sphinxbfcode{\sphinxupquote{print\_endline}}}{\sphinxparam{\DUrole{n,n}{text}}, \sphinxparam{\DUrole{o,o}{**}\DUrole{n,n}{kwargs}}}{}
\pysigstopsignatures
\end{fulllineitems}

\index{print\_error() (in module pymusepipe.util\_pipe)@\spxentry{print\_error()}\spxextra{in module pymusepipe.util\_pipe}}

\begin{fulllineitems}
\phantomsection\label{\detokenize{api/pymusepipe:pymusepipe.util_pipe.print_error}}
\pysigstartsignatures
\pysiglinewithargsret{\sphinxcode{\sphinxupquote{pymusepipe.util\_pipe.}}\sphinxbfcode{\sphinxupquote{print\_error}}}{\sphinxparam{\DUrole{n,n}{text}}, \sphinxparam{\DUrole{o,o}{**}\DUrole{n,n}{kwargs}}}{}
\pysigstopsignatures
\sphinxAtStartPar
Print error information


\paragraph{Input}
\label{\detokenize{api/pymusepipe:id174}}
\sphinxAtStartPar
text: str
pipe: musepipe {[}None{]}
\begin{quote}

\sphinxAtStartPar
If provided, will print the text in the logfile
\end{quote}

\end{fulllineitems}

\index{print\_info() (in module pymusepipe.util\_pipe)@\spxentry{print\_info()}\spxextra{in module pymusepipe.util\_pipe}}

\begin{fulllineitems}
\phantomsection\label{\detokenize{api/pymusepipe:pymusepipe.util_pipe.print_info}}
\pysigstartsignatures
\pysiglinewithargsret{\sphinxcode{\sphinxupquote{pymusepipe.util\_pipe.}}\sphinxbfcode{\sphinxupquote{print\_info}}}{\sphinxparam{\DUrole{n,n}{text}}, \sphinxparam{\DUrole{o,o}{**}\DUrole{n,n}{kwargs}}}{}
\pysigstopsignatures
\sphinxAtStartPar
Print processing information


\paragraph{Input}
\label{\detokenize{api/pymusepipe:id175}}
\sphinxAtStartPar
text: str
pipe: musepipe {[}None{]}
\begin{quote}

\sphinxAtStartPar
If provided, will print the text in the logfile
\end{quote}

\end{fulllineitems}

\index{print\_warning() (in module pymusepipe.util\_pipe)@\spxentry{print\_warning()}\spxextra{in module pymusepipe.util\_pipe}}

\begin{fulllineitems}
\phantomsection\label{\detokenize{api/pymusepipe:pymusepipe.util_pipe.print_warning}}
\pysigstartsignatures
\pysiglinewithargsret{\sphinxcode{\sphinxupquote{pymusepipe.util\_pipe.}}\sphinxbfcode{\sphinxupquote{print\_warning}}}{\sphinxparam{\DUrole{n,n}{text}}, \sphinxparam{\DUrole{o,o}{**}\DUrole{n,n}{kwargs}}}{}
\pysigstopsignatures
\end{fulllineitems}

\index{reconstruct\_filter\_images() (in module pymusepipe.util\_pipe)@\spxentry{reconstruct\_filter\_images()}\spxextra{in module pymusepipe.util\_pipe}}

\begin{fulllineitems}
\phantomsection\label{\detokenize{api/pymusepipe:pymusepipe.util_pipe.reconstruct_filter_images}}
\pysigstartsignatures
\pysiglinewithargsret{\sphinxcode{\sphinxupquote{pymusepipe.util\_pipe.}}\sphinxbfcode{\sphinxupquote{reconstruct\_filter\_images}}}{\sphinxparam{\DUrole{n,n}{cubename}}, \sphinxparam{\DUrole{n,n}{filter\_list}\DUrole{o,o}{=}\DUrole{default_value}{\textquotesingle{}white,Johnson\_B,Johnson\_V,Cousins\_R,SDSS\_g,SDSS\_r,SDSS\_i\textquotesingle{}}}, \sphinxparam{\DUrole{n,n}{filter\_fits\_file}\DUrole{o,o}{=}\DUrole{default_value}{\textquotesingle{}filter\_list.fits\textquotesingle{}}}}{}
\pysigstopsignatures
\sphinxAtStartPar
Reconstruct all images in a list of Filters
cubename: str
\begin{quote}

\sphinxAtStartPar
Name of the cube
\end{quote}
\begin{description}
\sphinxlineitem{filter\_list: str}
\sphinxAtStartPar
List of filters, e.g., “Cousins\_R,Johnson\_I”
By default, the default\_filter\_list from pymusepipe.config\_pipe

\sphinxlineitem{filter\_fits\_file: str}
\sphinxAtStartPar
Name of the fits file containing all the filter characteristics
Usually in filter\_list.fits (MUSE default)

\end{description}

\end{fulllineitems}

\index{safely\_create\_folder() (in module pymusepipe.util\_pipe)@\spxentry{safely\_create\_folder()}\spxextra{in module pymusepipe.util\_pipe}}

\begin{fulllineitems}
\phantomsection\label{\detokenize{api/pymusepipe:pymusepipe.util_pipe.safely_create_folder}}
\pysigstartsignatures
\pysiglinewithargsret{\sphinxcode{\sphinxupquote{pymusepipe.util\_pipe.}}\sphinxbfcode{\sphinxupquote{safely\_create\_folder}}}{\sphinxparam{\DUrole{n,n}{path}}, \sphinxparam{\DUrole{n,n}{verbose}\DUrole{o,o}{=}\DUrole{default_value}{True}}}{}
\pysigstopsignatures
\sphinxAtStartPar
Create a folder given by the input path This small function tries to create it
and if it fails it checks whether the reason is that it is not a path and then warn the user


\paragraph{Input}
\label{\detokenize{api/pymusepipe:id176}}
\sphinxAtStartPar
path: str
verbose: bool


\paragraph{Creates}
\label{\detokenize{api/pymusepipe:id177}}
\sphinxAtStartPar
A new folder if the folder does not yet exist

\end{fulllineitems}



\subsubsection{pymusepipe.version module}
\label{\detokenize{api/pymusepipe:module-pymusepipe.version}}\label{\detokenize{api/pymusepipe:pymusepipe-version-module}}\index{module@\spxentry{module}!pymusepipe.version@\spxentry{pymusepipe.version}}\index{pymusepipe.version@\spxentry{pymusepipe.version}!module@\spxentry{module}}
\sphinxAtStartPar
Copyright (c) 2016\sphinxhyphen{}2019 Eric Emsellem \textless{}\sphinxhref{mailto:eric.emsellem@eso.org}{eric.emsellem@eso.org}\textgreater{}

\sphinxAtStartPar
All rights reserved.

\sphinxAtStartPar
Redistribution and use in source and binary forms, with or without
modification, are permitted provided that the following conditions are met:
\begin{enumerate}
\sphinxsetlistlabels{\arabic}{enumi}{enumii}{}{.}%
\item {} 
\sphinxAtStartPar
Redistributions of source code must retain the above copyright notice, this
list of conditions and the following disclaimer.

\item {} 
\sphinxAtStartPar
Redistributions in binary form must reproduce the above copyright notice,
this list of conditions and the following disclaimer in the documentation
and/or other materials provided with the distribution.

\item {} 
\sphinxAtStartPar
Neither the name of the copyright holder nor the names of its contributors
may be used to endorse or promote products derived from this software
without specific prior written permission.

\end{enumerate}

\sphinxAtStartPar
THIS SOFTWARE IS PROVIDED BY THE COPYRIGHT HOLDERS AND CONTRIBUTORS “AS IS” AND
ANY EXPRESS OR IMPLIED WARRANTIES, INCLUDING, BUT NOT LIMITED TO, THE IMPLIED
WARRANTIES OF MERCHANTABILITY AND FITNESS FOR A PARTICULAR PURPOSE ARE
DISCLAIMED. IN NO EVENT SHALL THE COPYRIGHT HOLDER OR CONTRIBUTORS BE LIABLE
FOR ANY DIRECT, INDIRECT, INCIDENTAL, SPECIAL, EXEMPLARY, OR CONSEQUENTIAL
DAMAGES (INCLUDING, BUT NOT LIMITED TO, PROCUREMENT OF SUBSTITUTE GOODS OR
SERVICES; LOSS OF USE, DATA, OR PROFITS; OR BUSINESS INTERRUPTION) HOWEVER
CAUSED AND ON ANY THEORY OF LIABILITY, WHETHER IN CONTRACT, STRICT LIABILITY,
OR TORT (INCLUDING NEGLIGENCE OR OTHERWISE) ARISING IN ANY WAY OUT OF THE USE
OF THIS SOFTWARE, EVEN IF ADVISED OF THE POSSIBILITY OF SUCH DAMAGE.


\subsubsection{Module contents}
\label{\detokenize{api/pymusepipe:module-pymusepipe}}\label{\detokenize{api/pymusepipe:module-contents}}\index{module@\spxentry{module}!pymusepipe@\spxentry{pymusepipe}}\index{pymusepipe@\spxentry{pymusepipe}!module@\spxentry{module}}
\sphinxAtStartPar
Copyright (C) 2017 ESO/Centre de Recherche Astronomique de Lyon (CRAL)
print pymusepipe.\_\_LICENSE\_\_  for the terms of use

\sphinxAtStartPar
This package is a wrapper around the MUSE pipeline commands to reduce
muse raw data frames. It includes modules for aligning and convolving the frames.
It also has some basic routines wrapped around mpdaf, the excellent
python package built around the MUSE PIXTABLES and reduced data.


\chapter{Indices and tables}
\label{\detokenize{index:indices-and-tables}}\begin{itemize}
\item {} 
\sphinxAtStartPar
\DUrole{xref,std,std-ref}{genindex}

\item {} 
\sphinxAtStartPar
\DUrole{xref,std,std-ref}{modindex}

\item {} 
\sphinxAtStartPar
\DUrole{xref,std,std-ref}{search}

\end{itemize}


\renewcommand{\indexname}{Python Module Index}
\begin{sphinxtheindex}
\let\bigletter\sphinxstyleindexlettergroup
\bigletter{p}
\item\relax\sphinxstyleindexentry{pymusepipe}\sphinxstyleindexpageref{api/pymusepipe:\detokenize{module-pymusepipe}}
\item\relax\sphinxstyleindexentry{pymusepipe.align\_pipe}\sphinxstyleindexpageref{api/pymusepipe:\detokenize{module-pymusepipe.align_pipe}}
\item\relax\sphinxstyleindexentry{pymusepipe.check\_pipe}\sphinxstyleindexpageref{api/pymusepipe:\detokenize{module-pymusepipe.check_pipe}}
\item\relax\sphinxstyleindexentry{pymusepipe.combine}\sphinxstyleindexpageref{api/pymusepipe:\detokenize{module-pymusepipe.combine}}
\item\relax\sphinxstyleindexentry{pymusepipe.config\_pipe}\sphinxstyleindexpageref{api/pymusepipe:\detokenize{module-pymusepipe.config_pipe}}
\item\relax\sphinxstyleindexentry{pymusepipe.create\_sof}\sphinxstyleindexpageref{api/pymusepipe:\detokenize{module-pymusepipe.create_sof}}
\item\relax\sphinxstyleindexentry{pymusepipe.cube\_convolve}\sphinxstyleindexpageref{api/pymusepipe:\detokenize{module-pymusepipe.cube_convolve}}
\item\relax\sphinxstyleindexentry{pymusepipe.emission\_lines}\sphinxstyleindexpageref{api/pymusepipe:\detokenize{module-pymusepipe.emission_lines}}
\item\relax\sphinxstyleindexentry{pymusepipe.graph\_pipe}\sphinxstyleindexpageref{api/pymusepipe:\detokenize{module-pymusepipe.graph_pipe}}
\item\relax\sphinxstyleindexentry{pymusepipe.init\_musepipe}\sphinxstyleindexpageref{api/pymusepipe:\detokenize{module-pymusepipe.init_musepipe}}
\item\relax\sphinxstyleindexentry{pymusepipe.mpdaf\_pipe}\sphinxstyleindexpageref{api/pymusepipe:\detokenize{module-pymusepipe.mpdaf_pipe}}
\item\relax\sphinxstyleindexentry{pymusepipe.musepipe}\sphinxstyleindexpageref{api/pymusepipe:\detokenize{module-pymusepipe.musepipe}}
\item\relax\sphinxstyleindexentry{pymusepipe.prep\_recipes\_pipe}\sphinxstyleindexpageref{api/pymusepipe:\detokenize{module-pymusepipe.prep_recipes_pipe}}
\item\relax\sphinxstyleindexentry{pymusepipe.recipes\_pipe}\sphinxstyleindexpageref{api/pymusepipe:\detokenize{module-pymusepipe.recipes_pipe}}
\item\relax\sphinxstyleindexentry{pymusepipe.target\_sample}\sphinxstyleindexpageref{api/pymusepipe:\detokenize{module-pymusepipe.target_sample}}
\item\relax\sphinxstyleindexentry{pymusepipe.util\_image}\sphinxstyleindexpageref{api/pymusepipe:\detokenize{module-pymusepipe.util_image}}
\item\relax\sphinxstyleindexentry{pymusepipe.util\_pipe}\sphinxstyleindexpageref{api/pymusepipe:\detokenize{module-pymusepipe.util_pipe}}
\item\relax\sphinxstyleindexentry{pymusepipe.version}\sphinxstyleindexpageref{api/pymusepipe:\detokenize{module-pymusepipe.version}}
\end{sphinxtheindex}

\renewcommand{\indexname}{Index}
\printindex
\end{document}